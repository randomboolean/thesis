\subsection{Neural Networks}

\subsubsection{Description}

We denote by $I_f$ the \textit{domain of definition} of a function $f$ ("I" for "input") and by $O_f = f(I_f)$ its \textit{image} ("O" for "output"), and we represent it as $I_f~\xrightarrow{f}~O_f$.

\begin{definition}\textbf{Neural network (simply connected)}\\
Let $F$ be a function such that $I_f$ and $O_f$ are vector or tensor spaces.\\
$F$ is a \emph{functional formulation} of a \emph{simply connected neural network} if there are a series of linear or affine functions $(g_k)_{k=1,2,..,L}$ and a series of non-linear derivable univariate functions $(h_k)_{k=1,2,..,L}$ such that:
\begin{gather*}
\left\{
  \begin{array}{l}
    \forall k \in \{1, 2, \ldots, L\}, f_k = h_k \circ g_k, \\
    I_F = I_{f_1} \xrightarrow{f_1} O_{f_1} \cong I_{f_2} \xrightarrow{f_2} \dots \xrightarrow{f_L} O_{f_L} = O_F, \\
    F = f_{L} \circ ... \circ f_{2} \circ f_1
  \end{array}
\right.
\end{gather*}
The couple $(g_k, h_k)$ is called the \emph{$k$-th layer} of the neural network.
For $x \in I_f$, we denote by $x_k = f_k \circ ... \circ f_{2} \circ f_1 (x)$ the \emph{activations} of the $k$-th layer.
\label{def:nn}
\end{definition}

\todo{introduce what is their purpose ie classifiers, why is training, and make a little plan of what follows}
\todo{remarks on universal approximators and refs, overfitting, generalization}

\begin{definition}\textbf{Activation function}\\
Let a layer $(g,h)$. $h$ is called the \emph{activation function} of the layer. It is non-linear, derivable and univariate. Of common use for univariate functions is the functional notation $h(v)[i_1, i_2, \ldots, i_r] = h(v[i_1, i_2, \ldots, i_r])$.
\end{definition}

\begin{remark}\textbf{Example of activation functions}\\
\todo{blabla: refs activation functions}
\end{remark}

\begin{definition}\textbf{Layer}\\
A couple $(g,h)$, where $g$ is an affine or linear function, and $h$ is an activation function is called a \emph{layer}. The set of layers is denoted~$\cl$.
\end{definition}

\begin{remark}\textbf{Bias}\\
Affine functions $\widetilde{g}$ can be written as a sum between a linear function $g$ and a constant vector $b$ which is called the \emph{bias}. Its role is to augment the expressivity of the neural network's family of functions. For notational conveniency, we will omit the biases in the rest of this section and thus only consider linear functions.
\end{remark}

\begin{definition}\textbf{Connectivity matrix}\\
Let $g$ a linear function. Without loss of generality subject to a flattening, let's suppose $I_g$ and $O_g$ are vector spaces. Then there exists a \emph{connectivity matrix}~$W_g$, such that:
\begin{gather*}
\forall x \in I_g, g(x) = W_g x
\end{gather*}
\end{definition}
We denote $W_k$ the connectivity matrix of the $k$-th layer.

\begin{remark}\textbf{Biological inspiration}\\
A \emph{(computational) neuron} is a computational unit that is biologically inspired. Each neuron should be capable of:
\begin{enumerate}
\item receiving modulated signals from other neurons and aggregate them,
\item applying to the result a derivable activation,
\item passing the signal to other neurons.
\end{enumerate}
That is to say, each domain $\{I_{f_k}\}$ and $O_F$ can be interpreted as a layer of neurons, with one neuron for each dimension. The connectivity matrices $\{W_k\}$ describe the connexions between each successive layers.
%The parameters of $\Theta_g$ are the modulation weights that characterize the connections.
A neuron is illustrated on \figref{fig:neuron}.
\end{remark}

\begin{figure}[H]
\centering
\begin{tikzpicture}
\draw (0,0) -- (4,0) -- (4,4) -- (0,4) -- (0,0);
\node at (2,2){placeholder};
\end{tikzpicture}
\caption{A neuron}
\label{fig:neuron}
\end{figure}

The former neural networks are said to be \emph{simply connected} because each layer only takes as input the output of the previous one. We'll give a more general definition after first defining branching operations.

\begin{definition}\textbf{Branching}\\
A \emph{binary branching operation} of a neural network is an operation between two activations, $x_{k_1} \Join x_{k_2}$, that outputs, subject to shape compatibility, either their addition, either their concatenation along a rank, or their concatenation as a list.

A \emph{branching operation} of a neural network between $n$ activations, $x_{k_1} \Join x_{k_2} \Join \cdots \Join x_{k_n}$, is a composition of binary branching operations, or is the identity function $Id$ if $n = 1$.
\end{definition}

\begin{definition}\textbf{Neural network (generic definition)}\\
The set of \emph{neural network functions} $\cn$ is defined inductively as follow
%\begin{gather*}
\begin{enumerate}
%\left\{
  %\begin{array}{l}
  \item $Id \in \cn$
  \item $f \in \cn \wedge (g,h) \in \cl \wedge O_f \subset I_g \Rightarrow h \circ g \circ f \in \cn$
  %\text{for all shape compatible branching operation:}\\
  \item for all shape compatible branching operations:\\
  $\quad f_1, f_2, \ldots, f_n \in \cn \Rightarrow  f_1 \Join f_2 \Join \cdots \Join f_n \in \cn$
  %\end{array}
%\right.
\end{enumerate}
%\end{gather*}
\label{def:nn2}
\end{definition}

\todo{blabla: residual connections, skip connections,branching layers}

\subsubsection{Training}

\begin{definition}\textbf{Weights}\\
Let consider the $k$-th layer of a neural networks. We define its weights as coordinates of a vector $\theta_k$, called the \emph{weight kernel}, such that:
\begin{gather*}
  \forall (i,j),
    \begin{cases}
      \exists p, W_k[i,j] := \theta_k[p] \\
      \text{ or } W_k[i,j] = 0
    \end{cases}
\end{gather*}
\end{definition}
A weight $p$ that appears multiple times in $W_k$ is said to be \emph{shared}. Two parameters of $W_k$ that share a same weight $p$ are said to be \emph{tied}. The number of weights of the $k$-th layer is $n_1^{(\theta_k)}$.

\begin{remark}\textbf{Learning}\\
A \emph{loss} function $\mathcal{L}$ penalizes the output $x_L = F(x)$ relatively to what can be expected. Gradient w.r.t.~$\theta_k$, denoted $\vec{\bigtriangledown}_{\theta_k}$, is used to update the weights via an optimization algorithm based on gradient descent and a learning rate $\alpha$, that is:
\begin{gather}
\theta_k^{(\text{new})} = \theta_k^{(\text{old})} - \alpha \cdot \vec{\bigtriangledown}_{\theta_k} \left( \mathcal{L}\left( x_L, \theta_k^{(\text{old})} \right) + R\left( \theta_k^{(\text{old})} \right) \right)
\end{gather}
where $\alpha$~can be a scalar or a vector, $\cdot$~can denote outer or pointwise product, and $R$~is a regularizer. They depend on the optimization algorithm.
\end{remark}

\todo{examples of optimizations}

\begin{remark}\textbf{Linear complexity}\\
{The complexity of computing the gradients is linear with the number of weights.}
\begin{proof}
Without loss of generality, we assume that the neural network is simply connected. Thanks to the chain rule, $\vec{\bigtriangledown}_{\theta_k}$ can be computed using gradients that are w.r.t. $x_k$, denoted $\vec{\bigtriangledown}_{x_k}$, which in turn can be computed using gradients w.r.t. outputs of the next layer $k+1$, up to the gradients given on the output layer.

That is:
\begin{align}
  \vec{\bigtriangledown}_{\theta_k} & = J_{\theta_k}(x_k) \vec{\bigtriangledown}_{x_k} \\
  \begin{split}
  \vec{\bigtriangledown}_{x_k} & = J_{x_k}(x_{k+1}) \vec{\bigtriangledown}_{x_{k+1}} \\
  \vec{\bigtriangledown}_{x_{k+1}} & = J_{x_{k+1}}(x_{k+2}) \vec{\bigtriangledown}_{x_{k+2}} \\
  \quad \quad \ldots\\
  \vec{\bigtriangledown}_{x_{L-1}} & = J_{x_{L-1}}(x_{L}) \vec{\bigtriangledown}_{x_{L}}
  \label{eq:bp}
  \end{split}
\end{align}
Obtaining,
\begin{align}
  \vec{\bigtriangledown}_{\theta_k} = J_{\theta_k}(x_k) (\prod_{p=k}^{L-1} J_{x_p}(x_{p+1})) \vec{\bigtriangledown}_{x_L}
\end{align}
where $J_{\text{wrt}}(.)$ are the respective jacobians which can be determined with the layer's expressions and the $\{x_k\}$; and $\vec{\bigtriangledown}_{x_L}$ can be determined using $\mathcal{L}$, $R$ and $x_L$.
\end{proof}
This allows to compute the gradients with a complexity that is linear with the number of weights (only one computation of the activations), instead of being quadratic if it were done with the difference quotient expression of the derivatives (one more computation of the activations for each weight).
\end{remark}

\begin{remark}\textbf{Back propagation}\\
We can remark that \eqref{eq:bp} rewrites as
\begin{align}
  \begin{split}
  \vec{\bigtriangledown}_{x_k} & = J_{x_k}(x_{k+1}) \vec{\bigtriangledown}_{x_{k+1}} \\ 
                               & = J_{x'_k}(h(x'_k)) J_{x_k}(W_k x_k) \vec{\bigtriangledown}_{x_{k+1}}
  \end{split}
\end{align}
where $x'_k = W_k x_k$, and these jacobians can be expressed as:
\begin{align}
  \begin{split}
  J_{x'_k}(h(x'_k)) & [i,j] = \delta_i^j h'(x'_k[i])\\
  J_{x'_k}(h(x'_k)) & = I \hspace{2pt} h'(x'_k)
  \end{split}\\
  J_{x_k}(W_k x_k) & = W_k^T
\end{align}
That means that we can write $\vec{\bigtriangledown}_{x_k} = (\widetilde{h}_k \circ \widetilde{g}_k)(\vec{\bigtriangledown}_{x_{k+1}})$ such that the connectivity matrix $\widetilde{W}_k$ is obtained by transposition. This can be interpreted as gradient calculation being a \emph{back-propagation} on the same neural network, in opposition of the \emph{forward-propagation} done to compute the output.
\end{remark}

\subsubsection{Example of layers}

\begin{definition}\textbf{Connections}\\
The set of \emph{connections} of a layer $(g,h)$, denoted $C_g$, is defined as:
\begin{gather*}
  C_g = \{(i,j), \exists p, W_g[i,j] := \theta_g[p]\}
\end{gather*}
We have $0 \leq |C_g| \leq n_1^{(W_g)} n_2^{(W_g)}$.
\end{definition}

\begin{definition}\textbf{Dense layer}\\
A \emph{dense layer} $(g,h)$ is a layer such that $|C_g| = n_1^{(W_g)} n_2^{(W_g)}$, \ie all possible connections exist. The map $(i,j) \mapsto p$ is usually a bijection, meaning that there is no weight sharing.
\end{definition}

\begin{definition}\textbf{Partially connected layer}\\
A \emph{partially connected layer} $(g,h)$ is a layer such that $|C_g| < n_1^{(W_g)} n_2^{(W_g)}$.

A \emph{sparsely connected layer} $(g,h)$ is a layer such that $|C_g| \ll n_1^{(W_g)} n_2^{(W_g)}$.
\end{definition}

\begin{definition}\textbf{Convolutional layer}\\
A \emph{$n$-dimensional convolutional layer} $(g,h)$ is such that the weight kernel~$\theta_g$ can be reshaped into a tensor $w$ of rank $n+2$, and such that
$$
\left\{
\begin{array}{l}
  I_g \mbox{ and } O_g \mbox{ are tensor spaces of rank }n+1 \\
  \forall x \in I_g, g(x) = (g(x)_q = \sum\limits_p{x_p \ast_n w_{p,q}})_{\forall q}
\end{array}
\right.
$$
where $p$ and $q$ index slices along the last ranks. The slices $g(x)_q$ are typically called \textit{feature maps}.
\end{definition}

\begin{remark}\textbf{Geometric shape}
\todo{blabla}
\end{remark}

\begin{definition}\textbf{Padding}\\
A layer $(g, h)$ with padding is such that $\exists (g_\text{pad}, g'), g = g_\text{pad} \circ g'$ where $g_\text{pad}$ is a padding operation.

A convolutional layer with padding $(g_\text{pad} \circ g',h)$ is such that $g_\text{pad}$ \todo{}
\todo{rewrite}
\end{definition}

\todo{blabla padding}

\begin{proposition}\textbf{Connectivity matrix of a convolution with padding}\\
A convolutional layer with padding $(g, h)$ is equivalently defined as $W_g$ being a $n_{n+1}^{(I_g)} \times n_{n+1}^{(O_g)}$ block matrix such that its blocks are Toeplitz matrices.
\end{proposition}

\begin{proof}
Let's consider the slices indexed by $p$ and $q$, and to simplify the notations, let's drop the subscripts $\hspace{0pt}_{p,q}$. We recall from \defref{convdef} that
\begin{align*}
  y &= (x \ast_n w)[j_1, \ldots, j_n] \\
 &= \displaystyle \sum_{k_1=1}^{n_1^{(w)}} \cdots \sum_{k_n=1}^{n_n^{(w)}}
    x[j_1 + n_1^{(w)} - k_1, \ldots, j_n + n_n^{(w)} - k_n] \hspace{2pt} w[k_1, \ldots, k_n] \\
 &= \displaystyle \sum_{i_1=j_1}^{j_1 + n_1^{(w)} - 1} \cdots \sum_{i_n=j_n}^{j_n + n_n^{(w)} - 1}
    x[i_1, \ldots, i_n] \hspace{2pt} w[j_1 + n_1^{(w)} - i_1, \ldots, j_n + n_n^{(w)} - i_n] \\
 &= \displaystyle \sum_{i_1=1}^{n_1^{(x)}} \cdots \sum_{i_n=1}^{n_n^{(x)}}
    x[i_1, \ldots, i_n] \hspace{2pt} \widetilde{w}[i_1, j_1, \ldots, i_n, j_n] \\
 & \text{ where } \widetilde{w}[i_1, j_1, \ldots, i_n, j_n] = \\
 & \quad \quad
 \begin{cases}
   w[j_1 + n_1^{(w)} - i_1, \ldots, j_n + n_n^{(w)} - i_n] & \text{if } \forall t, 0 \le i_t - j_t \le n_t^{(w)} - 1 \\
   0 & \text{otherwise}
 \end{cases}
\end{align*}
Using Einstein summation convention as in~\eqref{indices} and permuting indices, we recognize the following tensor contraction
\begin{align}
y_{j_1 \cdots j_n} = x_{i_1 \cdots i_n} \widetilde{w} \hspace{1pt}^{i_1 \cdots i_n} \hspace{0pt}_{j_1 \cdots j_n} \label{eq:toep1}
\end{align}
Following \remref{rq:matprodeq}, we reshape~\eqref{eq:toep1} as a matrix product. To reshape $y \mapsto Y$, we use the row major order bijections $g_j$ as in~\eqref{rowmajor} defined onto $\{(j_1, \ldots, j_n), \forall t, 1 \le j_t \le n_t^{(y)}\}$. To reshape $x \mapsto X$, we use the same row major order bijection $g_j$, however defined on the indices that support non zero-padded values, so that zero-padded values are lost after reshaping. That is, we use a bijection $g_i$ such that $g_i(i_1, i_2, \ldots, i_n) = g_j(i_1 - o_1, i_2 - o_2, \ldots, i_n - o_n)$ defined if and only if $\forall t, 1 + o_t \le i_t \le n_t^{(y)}$, where the $\{o_t\}$ are the starting offsets of the non zero-padded values. $\widetilde{w} \mapsto W$ is reshaped by using $g_j$ for its covariant indices, and $g_i$ for its contravariant indices. The entries lost by using $g_i$ do not matter because they would have been nullified by the resulting matrix product. We remark that $W$ is exactly the block $(p,q)$ of $W_g$ (and not of $W_{g'}$). Now let's prove that it is a Toeplitz matrix.

Thanks to the linearity of the expression~\eqref{rowmajor} of $g_j$, by denoting $i'_t = i_t - o_t$, we obtain
\begin{gather}
  g_i(i_1, i_2, \ldots, i_n) - g_j(j_1, j_2, \ldots, j_n) = g_j(i'_1 - j_1, i'_2 - j_2, \ldots, i'_n - j_n)
\label{eq:toep2}
\end{gather}

To simplify the notations, let's drop the arguments of $g_i$ and $g_j$. By bijectivity of $g_j$, \eqref{eq:toep2} tells us that $g_i - g_j$ remains constant if and only if $i'_t - j_t$ remains constant for all $t$. Recall that 
\begin{gather}
  W[g_i,g_j] =
 \begin{cases}
   w[j_1 + n_1^{(w)} - i'_1, \ldots, j_n + n_n^{(w)} - i'_n] & \text{if } \forall t, 0 \le i'_t - j_t \le n_t^{(w)} - 1 \\
   0 & \text{otherwise}
 \end{cases}
\label{eq:toep3}
\end{gather}
Hence, on a diagonal of $W$, $g_i - g_j$ remaining constant means that $W[g_i,g_j]$ also remains constants. So $W$ is a Toeplitz matrix.

The converse is also true as we used invertible functions in the index spaces through the proof.
\end{proof}

\todo{remark when no padding}

\todo{below}

\begin{definition}\textbf{Convolutional layer with stride}\\
Let $g$ the linear part of a convolution layer with \emph{stride} $s_p > 1$ along the $p$-th rank, and $\widetilde{g}$ the same linear part if it was a regular convolutional layer as defined above. Then $g$ is defined as
\begin{gather*}
  \forall i_p \in \{ 1, \lfloor \frac{n_p^{(g(x))}}{s_p} \rfloor \}, \forall x \in I_g, g(x)_{i_p} = \widetilde{g}(x)_{s_p i_p}
\end{gather*}
where $i_p$ and $s_p i_p$ index slices along the $p$-th rank.
\end{definition}

\todo{below}

\begin{definition}\textbf{Pooling}\\
A layer with \textit{pooling} $(g,h)$ is such that $g = g_1 \circ g_2$, where $(g_1,h)$ is a layer and $g_2$ is a pooling operation.

\end{definition}

A layer with \textit{dropout} $(g,h)$ is such that $h = h_1 \circ h_2$, where $(g,h_2)$ is a layer and $h_1$ is a dropout operation~\citep{srivastava2014dropout}. When dropout is used, a certain number of neurons are randomly set to zero during the training phase, compensated at test time by scaling down the whole layer. This is done to prevent overfitting.

\subsubsection{Example of architectures}

\todo{rephrase}

A multilayer perceptron (MLP)~\citep{hornik1989multilayer} is a neural network composed of only dense layers.
A convolutional neural network (CNN)~\citep{lecun1998gradient} is a neural network composed of convolutional layers.

Neural networks are commonly used for machine learning tasks. For example, to perform supervised classification, we usually add a dense output layer $s=(g_{L+1},h_{L+1})$ with as many neurons as classes. We measure the error between an output and its expected output with a discriminative loss function $\mathcal{L}$. During the training phase, the weights of the network are adapted for the classification task based on the errors that are back-propagated~\citep{hornik1989multilayer} via the chain rule and according to a chosen optimization algorithm (\eg~\cite{bottou2010large}).
