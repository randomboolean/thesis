\subsection{Tensors}

\todo{add ref tensor}

Intuitively, tensors in the field of deep learning are defined as a generalization of vectors and matrices, as if vectors were tensors of rank $1$ and matrices were tensors of rank $2$. That is, they are objects in a vector space and their dimensions are indexed using as many indices as their rank, so that they can be represented by multidimensional arrays. In mathematics, a tensor is usually defined as a special class of multilinear functions~\citep{bass1968cours, williamson2015tensor}. As such, a mathematical tensor is entirely defined on the cartesian product of the canonical bases onto which its outputs can be represented by a multidimensional array. In that sense, both definitions rejoin on their representation, but the underlying objects are different. In particular, mathematical tensors enjoy a more intrinsic definition as they neither depend on their representation nor on a change of basis, whereas in our domain, they are confounded with their representation.

Our definition of tensors is such that they are a bit more than multidimensional arrays but not as much as mathematical tensors, for that they are embedded in a vector space and any deep learning object can be later define rigorously.

\begin{definition}\textbf{Tensor space}\\
We define a \emph{tensor space} $\bbt$ of rank $r$ as a vector space such that its canonical basis is a cartesian product of the canonical bases of $r$ other vector spaces.

We use the \emph{tensor product} notation $\bbt = \displaystyle \bigotimes_{k=1}^r \bbv_k$, and denote its shape by $n_1 \times n_2 \times \cdots \times n_r$, where the $\{n_k\}$ are the dimensions of the vector spaces.
\end{definition}

\begin{remark}
This definition is indeed equivalent with the definition by the tensor product~\citep{hackbusch2012tensor}. We make the choice to define a tensor space with its canonical basis because the coordinates of its objects are then naturally represented by a multidimensional array.
\end{remark}

\begin{remark}
For naming conveniency, from now on, we will distinguish between the terms \emph{linear space} and \emph{vector space} \ie we will abusively use the term \emph{vector space} only to refer to a linear space that can be seen as a tensor space of rank $1$. If there is no notion of rank defined, we rather use the term \emph{linear space}.
We also make a clear distinction between the term \emph{dimension} (that is, for a tensor space it is equal to $\displaystyle \prod_{k=1}^r n_k$) and the term \emph{rank} (equal to $r$). Note that some authors use the term \emph{order} instead of \emph{rank} (\eg \cite{hackbusch2012tensor}).
\end{remark}

\begin{definition}\textbf{Tensor}\\
A \emph{tensor} $t$ is an object of a tensor space. The \emph{shape} of $t$, which is the same as the shape of the tensor space it belongs to, is denoted $n_1^{(t)} \times n_2^{(t)} \times \cdots \times n_r^{(t)}$.
\end{definition}

\begin{definition}\textbf{Indexing}\\
An \emph{entry} of a tensor $t$ is one of its scalar coordinates in the canonical basis, denoted $t[i_1, i_2, \ldots, i_r]$.

A \emph{subtensor} $t'$ is a tensor of same rank composed of entries of $t$ that are contiguous in the indexing, with at least one entry per rank. We denote $t' = t[l_1{:}u_1, l_2{:}u_2, \ldots, l_r{:}u_r]$, where the $\{l_p\}$ and the $\{u_p\}$ are the lower and upper bounds of the indices used by the entries that compose~$t'$.

The cartesian product $\bbi=\displaystyle \prod_{k=1}^r \{1, \ldots, n_k \}$ is called the \emph{index space} of the tensor space.
\end{definition}

\begin{remark}
When using an index $i_k$ for an entry of a tensor $t$, we implicitly assume that $i_k \in \{1, 2, \cdots, n_k^{(t)}\}$ if nothing is specified.
For subtensor indexings, we don't necessarily write the lower bound index if it is equal to $1$, neither the upper bound index if it is equal to $n_p^{(t)}$.
\end{remark}

\begin{definition}\textbf{Slicing}\\
A \emph{slice} operation, along the last ranks $\{r_1, r_2, \ldots, r_s\}$, and indexed by $(i_{r_1}, i_{r_2}, \ldots, i_{r_s})$, is a morphism $s: \bbt = \displaystyle \bigotimes_{k=1}^r \bbv_k \rightarrow \displaystyle \bigotimes_{k=1}^{r-s} \bbv_k$, such that:
\begin{align*}
s(t)[i'_1, i'_2, \ldots, i'_{r-s}] &= t[i'_1, i'_2, \ldots, i'_{r-s}, i_{r_1}, i_{r_2}, \ldots, i_{r_s}] \\
\text{ \ie } \quad s(t) :&= t[:,:, \ldots, :, i_{r_1}, i_{r_2}, \ldots, i_{r_s}]
\end{align*}
where $:=$ means that entries of the right operand are assigned to the left operand.
We denote $t_{i_{r_1}, i_{r_2}, \ldots i_{r_s}}$ and call it the \emph{slice} of $t$. 
Slicing along a subset of ranks that are not the lasts is defined similarly.
$s(\bbt)$ is called a \emph{slice subspace}.
\end{definition}

\begin{definition}\textbf{Flattening}\\
A \emph{flatten} operation is an isomorphism $f: \bbt \rightarrow \bbv$, between a tensor space $\bbt$ of rank~$r$ and an $n$-dimensional vector space $\bbv$, where $n =\displaystyle \prod_{k=1}^r n_k$. It is characterized by a bijection in the index spaces $g: \displaystyle \prod_{k=1}^r \{1, \ldots, n_k \} \rightarrow\{1, \ldots, n \}$ such that
\begin{gather*}
  \forall t \in \bbt, f(t)[g(i_1, i_2, \ldots, i_r)] = f(t[i_1, i_2, \ldots, i_r])
\end{gather*}

An inverse operation is called a \emph{de-flatten} operation.
\end{definition}

\begin{remark}\textbf{Row major ordering}\\
The choice of $g$ determines in which order the indexing is made. $g$ is reminescent of how data of multidimensional arrays or tensors are stored internally by programming languages. In most tensor manipulation languages, incrementing the memory adress (\ie the output of $g$) will first increment the last index $i_r$ if $i_r < n_r$ (and if else $i_r = n_r$, then $i_r := 1$ and ranks are ordered in reverse lexicographic order to decide what indices are incremented). This is called \emph{row major ordering}, as opposed to \emph{column major ordering}. That is, in row major, $g$ is defined as
\begin{align}
  g(i_1, i_2, \ldots, i_r) = \displaystyle \sum_{p=1}^r \left( \prod_{k=p+1}^r n_k \right) i_p \label{rowmajor}
\end{align}
\end{remark}

\begin{definition}\textbf{Reshaping}\\
A \emph{reshape} operation is an isomorphism defined on a tensor space $\bbt = \displaystyle \bigotimes_{k=1}^r \bbv_k$ such that some of its basis vector spaces $\{\bbv_k\}$ are de-flattened and some of its slice subspaces are flattened.
\end{definition}

\begin{definition}\textbf{Contraction}\\
A \emph{tensor contraction} between two tensors, along ranks of same dimensions, is defined by natural extension of the dot product operation to tensors.

More precisely, let $\bbt_1$ a tensor space of shape $n_1^{(1)} \times n_2^{(1)} \times \cdots \times n_{r_1}^{(1)}$, and $\bbt_2$ a tensor space of shape $n_1^{(2)} \times n_2^{(2)} \times \cdots \times n_{r_2}^{(2)}$, such that $\forall k \in \{1, 2, \ldots, s\}, n_{r_1-(s-k)}^{(1)} = n_k^{(2)}$, then the tensor contraction between $t_1 \in \bbt_1$ and $t_2 \in \bbt_2$ is defined as:
\begin{gather*}
\left\{
  \begin{array}{l}
    t_1 \otimes t_2 = t_3 \in \bbt_3 \text{ of shape } n_1^{(1)} \times \cdots \times n_{r_1-s}^{(1)} \times n_{s+1}^{(2)} \times \cdots \times n_{r_2}^{(2)}
    \text{ where} \\
    t_3[i_1^{(1)}, \ldots, i_{r_1-s}^{(1)}, i_{s+1}^{(2)}, \ldots, i_{r_2}^{(2)}] = \\
    %\displaystyle \sum_{k_1, \ldots, k_s}
    \displaystyle \sum_{k_1=1}^{n_1^{(2)}} \cdots \sum_{k_s=1}^{n_s^{(2)}}
    t_1[i_1^{(1)}, \ldots, i_{r_1-s}^{(1)}, k_1, \ldots, k_s] \hspace{2pt}
    t_2[k_1, \ldots, k_s, i_{s+1}^{(2)}, \ldots, i_{r_2}^{(2)}]
  \end{array}
\right.
\end{gather*}
\end{definition}

For the sake of simplicity, we omit the case where the contracted ranks are not the last ones for $t_1$ and the first ones for $t_2$. But this definition still holds in the general case subject to a permutation of the indices.

\begin{definition}\textbf{Covariant and contravariant indices}\\
Given a tensor contraction $t_1 \otimes t_2$, indices of the left hand operand $t_1$ that are not contracted are called \emph{covariant} indices. Those that are contracted are called \emph{contravariant} indices. For the right operand $t_2$, the naming convention is the opposite. 
The set of covariant and contravariant indices of both operands are called the \emph{transformation laws} of the tensor contraction.
\end{definition}

\begin{remark}\textbf{Transformation law independency}\\
Contrary to mathematical definitions, tensors in deep learning are independent of any transformation law, so that they must be specified for tensor contractions.
\end{remark}

\begin{remark}\textbf{Einstein summation convention}\\
Using subscript notation for covariant indices and superscript notation for contravariant indices, the previous tensor contraction can be written using the Einstein summation convention as:
\begin{gather}
t_1 \hspace{0pt}_{i_1^{(1)} \cdots i_{r_1-s}^{(1)} } \hspace{0pt}^{ k_1 \cdots k_s} 
t_2 \hspace{0pt}_{ k_1^{\phantom{(}} \cdots k_s^{\phantom{(}}} \hspace{0pt}^{i_{s+1}^{(2)} \cdots i_{r_2}^{(2)}} =
t_3 \hspace{0pt}_ {i_1^{(1)} \cdots i_{r_1-s}^{(1)} } \hspace{0pt}^{i_{s+1}^{(2)} \cdots i_{r_2}^{(2)}}
\label{indices}
\end{gather}
Dot product $u_k v^k = \lambda $ and matrix product $A_i\hspace{0pt}^k B_k\hspace{0pt}^j = C_i\hspace{0pt}^j$ are common examples of tensor contractions.
\end{remark}

%Maybe prove it as a proposition
\begin{remark}\textbf{Matrix product equivalence}\\
Using reshapings that groups all covariant indices into a single index and all contravariant indices into another single index, any tensor contraction can be rewritten as a matrix product.
\label{rq:matprodeq}
\end{remark}
\begin{proof}
Using notation of \eqref{indices}, with the reshapings $t_1 \mapsto T_1$, $t_2 \mapsto T_2$ and $t_3 \mapsto T_3$ defined as suggested in the remark, we can rewrite
$$
T_1 \hspace{0pt}_{g_i(i_1^{(1)}, \ldots, i_{r_1-s}^{(1)})} \hspace{0pt}^{g_k(k_1, \ldots, k_s)} 
T_2 \hspace{0pt}_{g_k(k_1^{\phantom{(}}, \ldots, k_s^{\phantom{(}})} \hspace{0pt}^{g_j(i_{s+1}^{(2)}, \ldots, i_{r_2}^{(2)})} =
T_3 \hspace{0pt}_ {g_i(i_1^{(1)}, \ldots, i_{r_1-s}^{(1)})} \hspace{0pt}^{g_j(i_{s+1}^{(2)}, \ldots, i_{r_2}^{(2)})}
$$
where $g_i$, $g_k$ and $g_j$ are bijections defined similarly as in \eqref{rowmajor}.
\end{proof}

\begin{definition}\textbf{Convolution}\\
The \emph{$n$-dimensional convolution}, denoted $\ast^n$, between $t_1 \in \bbt_1$ and $t_2 \in \bbt_2$, where $\bbt_1$ and $\bbt_2$ are of the same rank $n$ such that $\forall p \in \{1, 2, \ldots, n\}, n_p^{(1)} \ge n_p^{(2)}$, is defined as:
\begin{gather*}
\left\{
  \begin{array}{l}
    t_1 \ast^n t_2 = t_3 \in  \bbt_3 \text{ of shape } n_1^{(3)} \times \cdots \times n_n^{(3)}
    \text{ where} \\
    \forall p \in \{1, 2, \ldots, n\}, n_p^{(3)} = n_p^{(1)} - n_p^{(2)} + 1 \\
    t_3[i_1, \ldots, i_n] =
    \displaystyle \sum_{k_1=1}^{n_1^{(2)}} \cdots \sum_{k_n=1}^{n_n^{(2)}}
    t_1[i_1 + n_1^{(2)} - k_1, \ldots, i_n + n_n^{(2)} - k_n] \hspace{2pt} t_2[k_1, \ldots, k_n] \\
  \end{array}
\right.
\end{gather*}
\label{def:convdef}
\end{definition}

\begin{definition}\textbf{Strided convolution}\\
The $n$-dimensional \emph{strided} convolution, with strides $s = (s_1, s_2, \ldots, s_n)$, denoted $\ast^n_s$, between $t_1 \in \bbt_1$ and $t_2 \in \bbt_2$, where $\bbt_1$ and $\bbt_2$ are of the same rank $n$ such that $\forall p \in \{1, 2, \ldots, n\}, n_p^{(1)} \ge n_p^{(2)}$, is defined as:
\begin{gather*}
\left\{
  \begin{array}{l}
    t_1 \ast^n_s t_2 = t_4 \in  \bbt_4 \text{ of shape } n_1^{(4)} \times \cdots \times n_n^{(4)}
    \text{ where} \\
    \forall p \in \{1, 2, \ldots, n\}, n_p^{(4)} = \lfloor\frac{n_p^{(1)} - n_p^{(2)} + 1}{s_p}\rfloor \\
    t_4[i_1, \ldots, i_n] = (t_1 \ast^n t_2)[(i_1 - 1)s_n + 1, \ldots, (i_n - 1)s_n + 1]\\
  \end{array}
\right.
\end{gather*}
\end{definition}

\begin{remark}
Unformally, a strided convolution is defined as if it were a regular subsampling of a convolution. They match if $s = (1,1,\ldots,1)$.
\end{remark}

\begin{definition}\textbf{Pooling}\\
\todo{}
\end{definition}