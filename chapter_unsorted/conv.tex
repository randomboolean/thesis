\documentclass{article}
\usepackage[utf8]{inputenc}
\usepackage[english]{babel}
\usepackage[T1]{fontenc}

\usepackage{amsfonts}
\usepackage{amsmath}
\usepackage{amssymb}
\usepackage{tikz}
\usepackage{graphicx}
\usepackage{color}
\usepackage{enumitem}
\usetikzlibrary{arrows}
\usepackage{framed}
\usepackage{arydshln}
\usepackage{multirow}

% commands
\newtheorem{definition}{Definition}
\newcommand{\domain}{\mathcal{D}}
\newcommand{\image}{\mathcal{I}}
\newcommand{\real}{\mathbb{R}}


\begin{document}

\subsection*{Basic naming conventions}

Let's start with the naming conventions of basic notions.

A \emph{function} $f$, from the set $E$ to $F$, denoted $f: E \rightarrow F$ maps objects $x \in E$ to objects $y \in F$, as $y = f(x)$.\\
Its \emph{domain} $\domain_f = E$ is the set of objects onto which it is defined.\\
Objects of its domain $\domain_f$ are mapped to objects of its \emph{codomain} $\domain_f^c= F$.\\
We say that $f$ is \emph{taking values} in its codomain.\\
The \emph{image per $f$} of the subset $U \subset E$, denoted $f(U)$, is $\{y \in F, \exists x \in E, y = f(x)\}$.\\
The \emph{image of $f$} is the image of its domain. We denote $\image_f$.\\
The \emph{fiber} of the object $y \in \image_f$ is the object $x \in E$ such that $y = f(x)$.\\
The \emph{inverse image per $f$} of the subset $V \subset F$, denoted $f^{-1}(V)$ is $\{x \in E, \exists y \in F, y = f(x)\}$.

A finite dimensional vector space $E$ is defined as $\real^n$, and is equipped with pointwise addition and scalar multiplication. % TODO reword

\subsection*{Signals}

A \emph{signal} $s$ is a function taking values in a finite dimensional vector space.
A set is said to be \emph{static} if all its signals are defined on the same domain, it is said to be \emph{non-static} otherwise.
In experimental setups, we will often refer to the word \emph{dataset} instead of \emph{set}.

For examples, images are signals defined on a set of pixels. Typically, an image $s$ in RGB representation is a mapping from pixels $p$ to a 3d vector space, as $s: p \mapsto (r,g,b)$. Image datasets used in practice are usually static. Non-static would mean that there are images of different sizes and/or different scales; in which case, they are usually rescaled and/or padded with zeros.

\begin{figure}

\end{figure}

A \emph{graph} $G = (V, E)$ is defined as a set of vertices $V$, and a set of edges $E \subseteq\binom{V}{2}$.
A \emph{graph signal} is a signal defined on the vertices of a graph.


\subsection*{Regular and irregular domains}

In this subsection, we are going to redefine the notion of \emph{regularity} of a function's domain relatively to the context of deep learning and convolutional representations.



\begin{definition}{Regular Domain}\\
%A regular domain is a lattice
\end{definition}


\begin{definition}{Irregular Domain}\\

\end{definition}


\subsection*{Invariance}

\begin{definition}

\end{definition}

In order to be observed, invariances must be defined relatively to an observation. Let's give a formal definition to support our discussion.



\begin{definition}{A function $f: $}

\end{definition}







\end{document}