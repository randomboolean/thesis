\section{The convolution challenge on graphs (draft)}

Defining a convolution on graphs is a challenging problem. Obviously, the underlying structure determined by a graph is not necessarily isomorphic to a set onto which the convolution is already defined. 

Related works: moura, spectral convolution with laplacian.

A convolution may comprise the following properties: bilinear, equivariant with respect to a certain class of isomorphism.

We shall first study classes of graphs onto which the convolution can be naturally defined before generalizing.

*Convolution on grids

*Convolution on lattice-regular graphs

*Convolution on product graphs

*Convolution on linear combination of circulant graphs

\subsection{Convolution on grids}

We first consider a grid graph $G = \langle V,E \rangle$ agnostically of its edges \ie $G \cong \bbz^2$.

\begin{definition}\textbf{Translation on $\cs(\bbz^2)$}\\
A translation on $\bbz^2$ is defined as a function $t: \bbz^2 \rightarrow \bbz^2$ such that
\begin{gather*}
\exists (a,b) \in \bbz^2, \forall (x,y) \in \bbz^2, t(x,y) = (x+a,y+b)
\end{gather*}
Its definition is naturally extended to signal functions $t: \cs(\bbz^2) \rightarrow \cs(\bbz^2)$ with the notation 
\begin{gather*}
\forall s \in \cs(\bbz^2), t(s)[x,y] = s[x-a, y-b]
\end{gather*}
For any set $E$, we denote by $\ct(E)$ its translations.
\end{definition}

\begin{proposition}\textbf{Equivariance to translations}\\
On signals over $\bbz^2$, the class of linear functions that are equivariant to translations is exactly the class of convolutive operations on signals \ie
\begin{gather*}
\begin{cases}
 f \in \cl(\cs(\bbz^2))\\
 \forall t \in \ct(\cs(\bbz^2)), f \circ t = t \circ f
\end{cases}
 \Leftrightarrow \exists w \in \cs(\bbz^2), \forall s \in \cs(\bbz^2), f(s) = w \ast s
\end{gather*}
\end{proposition}

\begin{proof}
\todo{proof without fourier transform}
\end{proof}

\begin{remark}
CNN vs MLP, expressive efficiency
\end{remark}




\subsection{Special classes of graphs}

% \begin{definition}\textbf{Infinite graph}\\
% An \emph{infinite graph} is defined by natural extension of the notion of graph $G=\langle V,E \rangle$ where $V$ and $E$ can be infinite. We denote $\order{G} = \infty$.
% \end{definition}

\begin{definition}\textbf{Graph automorphisms}\\
A graph automorphism of a graph $G = \langle V,E \rangle$ is a bijection in the vertex domain $\phi: V \rightarrow V$ such that $\{u,v\} \in E \Leftrightarrow \{\phi(u), \phi(v)\} \in E$. We denote $\ca(G)$ the group of automorphism on $G$.

We denote by $\ce(\phi)$ the set of input-output mapping of $\phi$, defined as $\ce(\phi) = \{ (x,y) \in V^2, \phi(x) = y \}$.

A graph automorphism $\phi$ is said to be \emph{edge-constrained} (EC) if $\ce(\phi) \subseteq E$. We denote $\ca_{\EC}(G)$ the set of edge-constrained automorphism on $G$.
\end{definition}

\begin{definition}\textbf{Orthogonality}\\
Two graph automorphisms $\phi_1$ and $\phi_2$ are said to be orthogonal, if and only if $\ce(\phi_1) \cap \ce(\phi_2) = \emptyset$, denoted $\phi_1 \bot \phi_2$. They are said to be aligned otherwise.

Similarly, we define orthogonality of $r$ automophisms as $\phi_1 \bot \cdots \bot \phi_r \Leftrightarrow \ce(\phi_1) \cap \cdots \cap \ce(\phi_r) = \emptyset$
\end{definition}


\subsection{Lattice-regular graph}


\begin{definition}\textbf{Lattice-regular graph}\\
A lattice-regular graph is a regular graph that admits $r$ orthogonal edge-constrained automorphisms, where $r$ is its degree.
\end{definition}


%\subsubsection{Grids}

%\subsubsection{Lattices}

%\subsubsection{Spatial graphs}

%\subsubsection{Projections of spatial graphs}