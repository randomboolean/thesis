\section{From groups to groupoids}

\subsection{Motivation}

One possible limitation coming from searching for Cayley subgraphs is that they are order-regular \ie the in- and the out-degree $d = |\cu|$ of each vertex is the same. That is, for a general graph $G$, the size of the weight kernel $w$ of an (EC*) convolution operator $f_w$ supported on $\cu$ is bounded by $d$, which in turn is bounded by twice the mininimal degree of $G$ (twice because $G$ is undirected and $\cu$ can contain every inverse).

There are a lot of possible strategies to overcome this limitation. For example:
\begin{enumerate}
  \item connecting each vertex with its $k$-hop neighbors, with $k > 1$,
  \item artificially creating new connections for less connected vertices,
  \item allowing the supporting set $\cn$ to exceed $\cu$ \ie dropping * in (EC*).
\end{enumerate}

These strategies require to concede that the topological structure supported by $G$ is not the best one to support an (EC*) convolution on it, which breeds the following question:
\begin{itemize}
  \item What can we relax in the previous (EC*) contruction in order to unbound the supporting set, and still preserve the equivariance characterization?
\end{itemize}

The latter constraint is a consequence that every vertex of the Cayley subgraph $\vec{G}$ must be composable with every generator from $\cu$. Therefore, an answer consists in considering groupoids~\citep{brandt1927verallgemeinerung} instead of groups. Roughly speaking, a groupoid is almost a group except that its composition law needs not be defined everywhere. \cite{weinstein1996groupoids}, unveiled the benefits to base convolutions on groupoids instead of groups in order to exploit partial symmetries.

\subsection{Definition of notions related to groupoids}

\begin{definition}\textbf{Groupoid}\\
A \emph{groupoid} $\Upsilon$ is a set equipped with a partial composition law with domain $\cd \subset \Upsilon \times \Upsilon$, called \emph{composition rule}, that is
\begin{enumerate}
  \item\label{enum:g1} closed into $\Upsilon$ \ie $\forall (g, h) \in \cd, gh \in \Upsilon$
  \item\label{enum:g2} associative \ie
    $\forall f,g,h \in \Upsilon,
      \begin{cases}
        (f, g), (g, h) \in \cd \Leftrightarrow (fg,h), (f, gh) \in \cd\\
        (f, g), (fg, h) \in \cd \Leftrightarrow (g,h), (f, gh) \in \cd\\
        \text{when defined, } (fg)h = f(gh)
      \end{cases}$ \label{enum:2}
  \item\label{enum:g3} invertible \ie
    $\forall g \in \Upsilon, \exists! g^{-1} \in \Upsilon \quad\st
      \begin{cases}
        (g,g^{-1}), (g^{-1},g)  \in \cd\\
        (g,h) \in \cd \Rightarrow g^{-1}gh = h\\
        (h,g) \in \cd \Rightarrow hgg^{-1} = h
      \end{cases}$
\end{enumerate}
Optionally, it can be \emph{domain-symmetric} \ie $(g,h) \in \cd \Leftrightarrow (h,g) \in \cd$, and \emph{abelian} \ie domain-symmetric with $gh = hg$.
\end{definition}

\begin{remark}
Note that left and right inverses are necessarily equal (because $(gg^{-1})g=g(g^{-1}g)$). Also note we can define a right identity element $e^r_g = g^{-1}g$, and a left one $e^l_g = gg^{-1}$, but they are not necessarily equal and depend on $g$.
\end{remark}

Most definitions related to groups can be adapted to groupoids. In particular, let's adapt a few notions.

\begin{definition}\textbf{Groupoid partial action}\\
A partial \emph{action} of a groupoid $\Upsilon$ on a set $V$, is a function $L$, with domain $\cd_L \subset \Upsilon \times V$ and valued in $V$, such that the map $g \mapsto L_g$ is a groupoid homomorphism.
\end{definition}

\begin{remark}
As usual, we will confound $L_g$ and $g$ when there is no possible confusion, and we denote $\cd_{L_g} = \cd_g = \{ v \in V, (g,v) \in \cd_L \}$.
\end{remark}

\begin{definition}\textbf{Partial equivariant map}\\
A map $\varphi$ from a groupoid $\Upsilon$ partially acting on the destination set $V$ is said to be a \emph{partial equivariant map} if
\begin{gather*}
\forall g,h \in \Upsilon,
  \begin{cases}
    \varphi(h) \in \cd_g \Leftrightarrow (g,h) \in \cd\\
    g(\varphi(h)) = \varphi(gh)
  \end{cases}
\end{gather*}
\end{definition}

Also, $\varphi$-equivalence between a subgroupoid and a set is defined similarly with $\varphi$ being a bijective \emph{partial} equivariant map between them.

\begin{definition}\textbf{Partial transformations groupoid}\\
The \emph{partial transformations groupoid} $\Psi^*(V)$, is the set of invertible partial transformations, equipped with the functional composition law with domain~$\cd$ such that
\begin{gather*}
\begin{cases}
\cd_{gh} = h(\cd_h) \cap \cd_g\\
(g,h) \in \cd \Leftrightarrow \cd_{gh} \neq \emptyset
\end{cases}
\end{gather*}
\end{definition}

\begin{remark}Note that a subgroupoid $\Upsilon \subset \Psi^*(V)$ is domain-symmetric when $\exists v \in V, g(v) \in \cd_h \Leftrightarrow \exists u \in V, h(u) \in \cd_g$
\end{remark}

\subsection{Construction of partial convolutions}

The expression of the convolution we constructed in the previous section cannot be applied as is. We first need to extend the algebraic objects we work with. 
Extending a partial transformation $g$ on the signal space $\cs(V)$ (and thus the convolutions) is a bit tricky, because only the signal entries corresponding to $\cd_g$ are moved. A convenient way to do this is to consider the groupoid closure obtained with the addition of an absorbing element.

\begin{definition}\textbf{Zero-closure}\\
The \emph{zero-closure} of a groupoid $\Upsilon$, denoted $\Upsilon^0$, is the set $\Upsilon \cup {0}$, such that the groupoid axioms \ref{enum:g1}, \ref{enum:g2} and \ref{enum:g3}, and the domain $\cd$ are left unchanged, and
\begin{enumerate}\setcounter{enumi}{3}
\item the composition law is extended to $\Upsilon^0 \times \Upsilon^0$ with $\forall (g,h) \notin \cd, gh = 0$
\end{enumerate}
\end{definition}

\begin{remark}Note that this is coherent as the properties \ref{enum:g2} and \ref{enum:g3} are still partially defined on the original domain $\cd$.
\end{remark}

Now, we will also extend every other algebraic object used in the expression of the $\varphi$-convolution and the $\M$-convolution, so that we can directly apply our previous constructions.

\begin{lemma}\textbf{Extension of $\varphi$ on $V^0$}\\
Let a partial equivariant map $\varphi: \Upsilon \rightarrow V$. It can be extended to a (total) equivariant map $\varphi: \Upsilon^0 \rightarrow V^0 = V \cup \varphi(0)$, such that $\varphi(0) \notin V$, that we denote $0_V = \varphi(0)$, and such that
\begin{gather*}
	\forall g \in \Upsilon^0, \forall v \in V^0, g(v) = \begin{cases} \varphi(gg_v) &\text{ if } g_v \in \cd_g\\ 0_V &\text{ else}\end{cases}
\end{gather*}
\end{lemma}
\begin{proof}
We have $\varphi(0) \notin V$ because $\varphi$ is bijective. Additionaly, we must have $\forall (g,h) \notin \cd, g(\varphi(h)) = \varphi(gh) = \varphi(0) = 0_V$.
\end{proof}

\begin{remark} Note that for notational conveniency, we may use the same symbol~$0$ for~$0_{\Upsilon}$, $0_V$ and~$0_{\bbr}$.
\end{remark}

Similarly to $\Phi^*(V)$, $\Psi^*(V)$ can also move signals of $\cs(V)$.

\begin{lemma}\textbf{Extension of injective partial transformations to $\cs(V)$}\\
Let $g \in \Psi^*(V)$. Its extension is done in two steps:
\begin{enumerate}
  \item $g$ is extended to $V^0 = V \cup \{0_V\}$ as $g(v) = 0_V \Leftrightarrow v \notin \cd_g$.
  \item Under the convention $\forall s \in \cs(V), s[0_V] = 0_\bbr$, $g$ is extended via linear extension to $\cs(V)$, and we have
  \begin{gather*}
  \forall s \in \cs(V), \forall v \in V, g(s)[v] = s[g^{-1}(v)]
  \end{gather*}
\end{enumerate}
\end{lemma}
\begin{proof}
Straightforward.
\end{proof}

With these extensions, we can obtain the partial $\varphi$- and $\M$-convolutions related to~$\Upsilon$ almost by substituting~$\Upsilon^0$ to~$\Gamma$ in \defref{def:convv} and \defref{def:convm}.

\begin{definition}\textbf{Partial convolution}\\
Let a subgroupoid $\Upsilon \subset \Psi^*(V)$, such that $\Upsilon \overset\varphi\equiv V$. The partial $\varphi$- and $\M$-convolutions, based on~$\Upsilon$, are defined on its zero-closure, with the same expression as if~$\Upsilon^0$ were a subgroup, and by extension of~$\varphi$ and of the groupoid partial actions \ie
\begin{enumerate}[label=(\roman*)]
  \item $\forall s, w \in \cs(V),
  			s \ast_\varphi w = \displaystyle\sum_{v \in V} s[v] \h{2} g_v(w) = \displaystyle\sum_{g \in \Upsilon} s[\varphi(g)] \h{2} g(w)$ \label{enum:i}
  \item $\forall (w,s) \in \cs(\Upsilon) \times \cs(V), w \ast_{\M} s  = \displaystyle\sum_{g \in \Upsilon} w[g] \h{2} g(s)$ \label{enum:ii}
\end{enumerate}
\end{definition}

\paragraph{Symmetrical expressions}
Note that, as $\forall r, r[0] = 0$, the partial convolutions can also be expressed on the domain~$\cd$ with a convenient symmetrical expression:
\begin{enumerate}[label=(\roman*)]
  \item $\forall u \in V, (s \ast_\varphi w)[u] =  \displaystyle\sum_{\substack{(g_a,g_b) \in \cd\\ \st g_ag_b = g_u}} s[a] \h{2} w[b]$
  \item $\forall u \in V, (w \ast_{\M} s)[u]  = \displaystyle\sum_{\substack{v \in \cd_g\\ \st g(v) = u}} w[g] \h{2} s[v]$
\end{enumerate}

We obtain an equivariance characterization similar to \propref{prop:equiG} and \corref{cor:equiM}.

\begin{proposition}\textbf{Characterization by equivariance to $\Upsilon$}\\
Let a subgroupoid $\Upsilon \subset \Psi^*(V)$, such that~$\Upsilon \overset\varphi\equiv V$, with $\ast$ based on~$\Upsilon$.
\begin{enumerate}
\item Then,
\begin{enumerate}[label=(\roman*)]
  \item partial $\varphi$-convolution right-operators are equivariant to~$\Upsilon$,\label{enum:i}
  \item if $\Upsilon$ is abelian, partial $\M$-convolution left-operators are equiv to~$\Upsilon$.\label{enum:ii}
\end{enumerate}
\item Conversely, \label{enum:2}
\begin{enumerate}[label=(\roman*)]
  \item if $\Upsilon$ is domain-symmetric, linear transformations of $\cs(V)$ that are equivariant to~$\Upsilon$ are partial $\varphi$-convolution right-operators,
  \item if $\Upsilon$ is abelian, they are also partial $\M$-convolution left-operators.
\end{enumerate}
\end{enumerate}
\label{prop:equiP}
\end{proposition}
\begin{proof}
\begin{enumerate}[label=(\roman*)]
   \item
  \begin{enumerate}
    \item Direct sense:\\
    Using the symmetrical expressions, and the fact that $\forall r, r[0] = 0$, we have
    \begin{align*}
    (f_w \circ g (s))[u] & = \displaystyle\sum_{\substack{(g_a,g_b) \in \cd\\ \st g_ag_b = g_u}} g(s)[a] \h{2} w[b]\\
    				& = \displaystyle\sum_{\substack{(g_a,g_b) \in \cd\\ \st g_ag_b = g_u}} s[g^{-1}(a)] \h{2} w[b]\\
    				& = \displaystyle\sum_{\substack{(g_a,g_b) \in \cd\\ \st (g, g_a) \in \cd\\ \st gg_ag_b = g_u}} s[a] \h{2} w[b]\\
    				& = \displaystyle\sum_{\substack{(g_a,g_b) \in \cd\\ \st (g, g_a) \in \cd\\ \st g_ag_b = g^{-1}g_u = g_{\varphi(g^{-1}g_u)} = g_{g^{-1}(u)} }} s[a] \h{2} w[b]\\
    				& = f_w (s)[g^{-1}(u)]\\
    				& = (g \circ f_w (s))[u]
    \end{align*}

    \item Converse:\\
    Let $v \in V$. Denote $e^r_{g_v} = g_v^{-1}g_v$ the right identity element of $g_v$, and $e^r_v = \varphi(e^r_{g_v})$. We have that
    \begin{gather*}
    g_v(e^r_v) = v\\
    \text{So, } \delta_v = g_v(\delta_{e^r_v})
    \end{gather*}

    Let $f \in \cl(\cs(V))$ that is equivariant to $\Upsilon$, and $s \in \cs(V)$. Thanks to the previous remark we obtain that
    \begin{align}
    f(s) & = \displaystyle\sum_{v \in V} s[v] \h{2} f(\delta_v)\nonumber\\
         & = \displaystyle\sum_{v \in V} s[v] \h{2} f(g_v(\delta_{e^r_v}))\nonumber\\
         & = \displaystyle\sum_{v \in V} s[v] \h{2} g_v(f(\delta_{e^r_v}))\nonumber\\
         & = \displaystyle\sum_{v \in V} s[v] \h{2} g_v(w_v) \label{eq:last}
    \end{align}
    where $w_v = f(\delta_{e^r_v})$. In order to finish the proof, we need to find $w$ such that $\forall v \in V, g_v(w) = g_v(w_v)$.

    Let's consider the equivalence relation $\ccr$ defined on $V \times V$ such that:
    \begin{align}
    a \ccr b & \Leftrightarrow    w_a = w_b\nonumber\\
            & \Leftrightarrow e^r_a = e^r_b\nonumber\\
            & \Leftrightarrow g_a^{-1}g_a = g_b^{-1}g_b\nonumber\\
            & \Leftrightarrow (g_b, g_a^{-1}) \in \cd\nonumber\\
            & \Leftrightarrow (g_a^{-1}, g_b) \in \cd \label{eq:eq}
    \end{align}
    with \eqref{eq:eq} owing to the fact that $\Upsilon$ is domain-symmetric.

    Given $x\in V$, denote its equivalence class $\ccr(x)$. Under the hypothesis of the axiom of choice~\citep{zermelo1904beweis} (if $V$ is infinite), define the set $\aleph$ that contains exactly one representative per equivalence class. Let $w = \sum_{n \in \aleph} w_n$. Then $V$ is the disjoint union $V = \displaystyle\cup_{n \in \aleph} \ccr(n)$ and \eqref{eq:last} rewrites:
    \begin{align}
    \forall u \in V, f(s)[u] & = \displaystyle\sum_{n \in \aleph}\sum_{v \in \ccr(n)} s[v] \h{2} g_v(w_n)[u]\nonumber\\
    	 				  & = \displaystyle\sum_{n \in \aleph}\sum_{v \in \ccr(n)} s[v] \h{2} w_n[g_v^{-1}(u)]\nonumber\\
    	 				  & = \displaystyle\sum_{n \in \aleph}\sum_{v \in \ccr(n)} s[v] \h{2} w[g_v^{-1}(u)]\label{eq:eqq}\\
    	 				  & = (s \ast_\varphi w)[u]\nonumber
    \end{align}
    where \eqref{eq:eqq} is obtained thanks to \eqref{eq:eq}.
  \end{enumerate}

  \item With symmetrical expressions, it is clear that the convolution is abelian, if and only if, $\Upsilon$ is abelian. Then \ref{enum:i} concludes.
\end{enumerate}
\end{proof}

% \subsection{Edge-constrained groupoid convolutions}

% \todo{Irrelevant because associativity and equivariant map property clash and we still have the same constraints as before}

% \todo{Erase and rewrite this subsection}

% Similarly to the construction with Cayley graphs of \ref{sec:cayley}, we start by adapting the related notions.

% \begin{definition}\textbf{Groupoid generating set}\\
% A set~$\cu$, equipped with a partial composition law of domain~$\cd_0$, is a \emph{generating set} of a groupoid~$\Upsilon$, if every object of~$\Upsilon$ can be expressed as a composition of objects of~$\cu$ and their inverses. In this case, the domain $\cd$ of the partial composition law of~$\Upsilon$ can be deducted inductively from~$\cd_0$ with the associativity and invertibility axioms of~$\Upsilon$ (and eventually also with the domain-symmetric axiom if~$\Upsilon$ is domain-symmetric).
% \end{definition}

% A partial Cayley graph is defined similarly as the (total) Cayley graph.

% \begin{definition}\textbf{Partial Cayley graph}\\
% Let a groupoid $\Upsilon = \langle \cu, \cd_0 \rangle$. The \emph{partial Cayley graph} generated by $\cu$ and $\cd_0$, is the digraph $\vgve$ such that $V = \Upsilon$ and $E$ is such that:
% \begin{gather*}
% u \rightarrow v \Leftrightarrow \exists g \in \cu, ga = b
% \end{gather*}
% We call \emph{partial Cayley subgraph}, a subgraph that is isomorph to a partial Cayley graph.
% \end{definition}

% \begin{remark}
% The characterization by a partial Cayley subgraph, of graphs admitting an (EC) convolution based on a groupoid, similar to \propref{prop:chc} also holds for the groupoid representation.% However, it is somewhat trivial as every graph admits a partial Cayley subgraph (for example by labelling every edge in a neighbour with a transformation).
% \end{remark}

% In the case of the groupoid representation, what is more interesting is that an (EC) groupoid convolution can be characterized by a generating set of the groupoid it is based on.

% \begin{proposition}\textbf{Characterization by generating set}\\
% Let a graph $\gve$ such that it admits an (EC) convolution based on a groupoid~$\Upsilon$. Then,~$G$~contains a sugraph~$\vec{G}$ that is isomorph to a partial Cayley subgraph~$\vec{G_c}$, of generating set $\cu \subset \Psi^*_{\EC}(G)$, such that the underlying equivariant map~$\varphi$ of the (EC) convolution is also a graph isomorphism between~$\vec{G_c}$ and~$\vec{G}$.
% \end{proposition}

% The proof is omitted because it would be highly similar with the one of \corref{cor:cayley}.

% \paragraph{Strictly edge-constrained groupoid convolution}

% Thanks to the previous construction, in order to define a meaningful convolution operator on a general graph~$\gve$, all we need is a set~$\cu$ of (EC) partial transformations \emph{that need not to be defined on every vertex}, and a composition rule~$\cd_0$. From the set~$\cu$, we can choose a supporting set $\cn \subset \cu$ (from which we eventually need to derive the local patches $\ck_u$), and obtain the (EC*) convolution operator~$f_w$, without the need of fully determining $\Upsilon = \langle \cu \rangle$. We can choose between three levels of constraints, which are inherited from~$\cu$ and~$\cd_0$:
% \begin{enumerate}
%   \item $\Upsilon$ is unconstrained, then $f_w$ is an (EC*) $\varphi$-convolution right-operator that is equivariant to $\Upsilon$, but the converse doesn't hold,
%   \item $\Upsilon$ is domain-symmetric, then $f_w$ is an (EC*) $\varphi$-convolution right operator and the equivariance characterization holds,
%   \item $\Upsilon$ is abelian, then $f_w$ is an (EC*) $\M$-convolution operator, the equivariance characterization holds, and we don't need to compute the local patches $\ck_u$.
% \end{enumerate}

% In the next chapter, we will encounter examples of convolutions defined on graphs, that can be expressed under group representations or under groupoid representations.

% \todo{Limitation: border effect because $g$ goes in implies $g$ goes out}

\paragraph{Inclusion of (EC)}
Similarly to the construction in \secref{sec:edges}, partial convolutions can define (EC) and (EC*) counterparts with a characterization of admissibility by groupoid Cayley subgraph isomorphism.

\paragraph{Limitation of partial convolutions}
However, because of the groupoid associativity, if $g \in \Psi_{\EC}^*(G)$, then, any $v \in V$ \st $g(u) = v$ would be constrained to allow to be acted by every $h$ \st $(h,g) \in \cd$, which fails at unbounding the supporting set of a partial (EC*)~convolutions.
