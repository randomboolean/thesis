\section{From groups to groupoids}
\label{sec:2.4}

\subsection{Motivation}
\label{sec:motiv}

One possible limitation coming from searching for Cayley subgraphs is that they are degree-regular \ie the in- and the out-degree $d = |\cu|$ of each vertex is the same. That is, for a general graph $G$, the size of the weight kernel $w$ of an EC* convolution operator $f_w$ supported on $\cu$ is bounded by $d$, which in turn is bounded by twice the mininimal degree of $G$ (twice because $G$ is undirected and $\cu$ can contain every inverse).

There are a lot of possible strategies to overcome this limitation. For example:
\begin{enumerate}
  \item connecting each vertex with its $k$-hop neighbors, with $k > 1$,
  \item artificially creating new connections for less connected vertices,
  \item ignoring less connected vertices,
  \item allowing the supporting set $\cm$ to exceed $\cu$ \ie dropping * in EC*.
\end{enumerate}

These strategies require to concede that the topological structure supported by $G$ is not the best one to support an EC* convolution on it, which breeds the following question:
\begin{itemize}
  \item What can we relax in the previous EC* contruction in order to unbound the supporting set, and still preserve the equivariance characterization?
\end{itemize}

The latter constraint is a consequence that every vertex of the Cayley subgraph $\vec{G}$ must be composable with every generator from $\cu$. Therefore, an answer consists in considering groupoids~\citep{brandt1927verallgemeinerung} instead of groups. Roughly speaking, a groupoid is almost a group except that its composition law needs not be defined everywhere. \cite{weinstein1996groupoids}, unveiled the benefits to base convolutions on groupoids instead of groups in order to exploit partial symmetries.

\subsection{Definition of notions related to groupoids}

\begin{definition}\textbf{Groupoid}\\
A \emph{groupoid} $\Upsilon$ is a set equipped with a partial composition law with domain $\cd \subset \Upsilon \times \Upsilon$, called \emph{composition rule}, that is
\begin{enumerate}
  \item\label{enum:g1} closed into $\Upsilon$ \ie $\forall (g, h) \in \cd, gh \in \Upsilon$
  \item\label{enum:g2} associative \ie
    $\forall f,g,h \in \Upsilon,
      \begin{cases}
        (f, g), (g, h) \in \cd \Leftrightarrow (fg,h), (f, gh) \in \cd\\
        (f, g), (fg, h) \in \cd \Leftrightarrow (g,h), (f, gh) \in \cd\\
        \text{when defined, } (fg)h = f(gh)
      \end{cases}$ \label{enum:2}
  \item\label{enum:g3} invertible \ie
    $\forall g \in \Upsilon, \exists! g^{-1} \in \Upsilon \quad\st
      \begin{cases}
        (g,g^{-1}), (g^{-1},g)  \in \cd\\
        (g,h) \in \cd \Rightarrow g^{-1}gh = h\\
        (h,g) \in \cd \Rightarrow hgg^{-1} = h
      \end{cases}$
\end{enumerate}
Optionally, it can be \emph{domain-symmetric} \ie $(g,h) \in \cd \Leftrightarrow (h,g) \in \cd$, and \emph{abelian} \ie domain-symmetric with $gh = hg$.
\end{definition}

\begin{remark}
Note that left and right inverses are necessarily equal (because $(gg^{-1})g=g(g^{-1}g)$). Also note we can define a right identity element $e^r_g = g^{-1}g$, and a left one $e^l_g = gg^{-1}$, but they are not necessarily equal and depend on $g$.
\end{remark}

Most definitions related to groups can be adapted to groupoids. In particular, let's adapt a few notions.


\begin{definition}\textbf{Partial transformations groupoid}\\
The \emph{partial transformations groupoid} $\Psi^*(V)$, is the set of invertible partial transformations, equipped with the functional composition law with domain~$\cd$ such that
\begin{gather*}
\begin{cases}
\cd_{gh} = h(\cd_h) \cap \cd_g\\
(g,h) \in \cd \Leftrightarrow \cd_{gh} \neq \emptyset
\end{cases}
\end{gather*}
\end{definition}

\begin{remark}Note that a subgroupoid $\Upsilon \subset \Psi^*(V)$ is domain-symmetric when $\exists v \in V, g(v) \in \cd_h \Leftrightarrow \exists u \in V, h(u) \in \cd_g$
\end{remark}

\begin{definition}\textbf{Groupoid partial action}\\
A partial \emph{action} of a groupoid $\Upsilon$ on a set $V$ is a groupoid homomorphism $L: \Upsilon \to \Psi^*(V)$.
\end{definition}

\begin{remark}
As usual, we will confound $L_g$ and $g(.)$ when there is no possible confusion, and we denote $\cd_{L_g} = \cd_g = \{ v \in V, (g,v) \in \cd_L \}$.
\end{remark}

\begin{definition}\textbf{Partial equivariant map}\\
A map $\varphi$ from a groupoid $\Upsilon$ partially acting on the destination set $V$ is said to be a \emph{partial equivariant map} if
\begin{gather*}
\forall g,h \in \Upsilon,
  \begin{cases}
    \varphi(h) \in \cd_g \Leftrightarrow (g,h) \in \cd\\
    g(\varphi(h)) = \varphi(g(h))
  \end{cases}
\end{gather*}
\end{definition}

Also, $\varphi$-equivalence between a groupoid and a set is defined similarly than for groups, with also left multiplicative partial action being implicit.

\subsection{Construction of partial convolutions}

The expression of the convolution we constructed in the previous section cannot be applied as is. We first need to extend the algebraic objects we work with. 
Extending a partial transformation $g$ on the signal space $\cs(V)$ (and thus the convolutions) is a bit tricky, because only the signal entries corresponding to $\cd_g$ are moved. A convenient way to do this is to consider the groupoid closure obtained with the addition of an absorbing element.

\begin{definition}\textbf{Zero-closure}\\
The \emph{zero-closure} of a groupoid $\Upsilon$, denoted $\Upsilon^0$, is the set $\Upsilon \cup {0}$, such that the groupoid axioms \ref{enum:g1}, \ref{enum:g2} and \ref{enum:g3}, and the domain $\cd$ are left unchanged, and
\begin{enumerate}\setcounter{enumi}{3}
\item the composition law is extended to $\Upsilon^0 \times \Upsilon^0$ with $\forall (g,h) \notin \cd, gh = 0$
\end{enumerate}
\end{definition}

\begin{remark}Note that this is coherent as the properties \ref{enum:g2} and \ref{enum:g3} are still partially defined on the original domain $\cd$.
\end{remark}

Now, we will also extend every other algebraic object used in the expression of the $\varphi$-convolution and the $\M$-convolution, so that we can directly apply our previous constructions.

\begin{lemma}\textbf{Zero-extension of $\varphi$ on $V^0$}\\
Let a partial equivariant map $\varphi: \Upsilon \rightarrow V$. It can be extended to a (total) equivariant map $\varphi: \Upsilon^0 \rightarrow V^0 = V \cup \varphi(0)$, such that $\varphi(0) \notin V$, that we denote $0_V = \varphi(0)$, and such that
\begin{gather*}
	\forall g \in \Upsilon^0, \forall v \in V^0, g(v) = \begin{cases} \varphi(gg_v) &\text{ if } g_v \in \cd_g\\ 0_V &\text{ else}\end{cases}
\end{gather*}
\end{lemma}
\begin{proof}
We have $\varphi(0) \notin V$ because $\varphi$ is bijective. Additionaly, we must have $\forall (g,h) \notin \cd, g(\varphi(h)) = \varphi(gh) = \varphi(0) = 0_V$.
\end{proof}

\begin{remark} Note that for notational conveniency, we may use the same symbol~$0$ for~$0_{\Upsilon}$, $0_V$ and~$0_{\bbr}$.
\end{remark}

Similarly to $\Phi^*(V)$, $\Psi^*(V)$ can also move signals of $\cs(V)$.

\begin{lemma}\textbf{Zero-extension of partial actions to $\cs(V)$}\\
Let $g \in \Psi^*(V)$. Its extension is done in two steps:
\begin{enumerate}
  \item $g$ is extended to $V^0 = V \cup \{0_V\}$ as $g(v) = 0_V \Leftrightarrow v \notin \cd_g$.
  \item Under the convention $\forall s \in \cs(V), s[0_V] = 0_\bbr$, $g$ is extended via linear extension to $\cs(V)$, and we have
  \begin{gather*}
  \forall s \in \cs(V), \forall v \in V, g(s)[v] = s[g^{-1}(v)]
  \end{gather*}
\end{enumerate}
\end{lemma}
\begin{proof}
Straightforward.
\end{proof}

With these extensions, we can obtain the partial $\varphi$- and $\M$-convolutions related to~$\Upsilon$ almost by substituting~$\Upsilon^0$ to~$\Gamma$ in \defref{def:convv} and \defref{def:convm}.

\begin{definition}\textbf{Partial convolutions}\\
Let a groupoid $\Upsilon$ acting on $V^0$. %, such that $\Upsilon \overset\varphi\equiv V$.
The partial $\varphi$- and $\M$-convolutions extend their definition on groups to groupoids using the appropriate zero-extensions \ie
\begin{enumerate}[label=(\roman*)]
  \item $\forall s, w \in \cs(V),
  			s \ast_\varphi w = \displaystyle\sum_{v \in V} s[v] \h{2} g_v(w) = \displaystyle\sum_{g \in \Upsilon} s[\varphi(g)] \h{2} g(w)$ \st $\Upsilon \overset\varphi\equiv V$\label{enum:i}
  \item $\forall (w,s) \in \cs(\Upsilon) \times \cs(V), w \ast_{\M} s  = \displaystyle\sum_{g \in \Upsilon} w[g] \h{2} g(s)$ \st $\Upsilon$ is abelian\label{enum:ii}
\end{enumerate}
\end{definition}

\paragraph{Symmetrical expressions}
Note that, as $\forall r, r[0] = 0$, the partial convolutions can also be expressed on the domain~$\cd$ with a convenient symmetrical expression:
\begin{enumerate}[label=(\roman*)]
  \item $\forall u \in V, (s \ast_\varphi w)[u] =  \displaystyle\sum_{\substack{(g_a,g_b) \in \cd\\ \st g_ag_b = g_u}} s[a] \h{2} w[b]$
  \item $\forall u \in V, (w \ast_{\M} s)[u]  = \displaystyle\sum_{\substack{v \in \cd_g\\ \st g(v) = u}} w[g] \h{2} s[v]$
\end{enumerate}

We obtain an equivariance characterization similar to \thref{prop:equiG} and \thref{cor:equiM}, under the mild condition that $\Upsilon$ is at least domain-symmetric if it is not abelian.

\begin{theorem}\textbf{Characterization by equivariance to $\Upsilon$}\\
Let a groupoid $\Upsilon$ acting on $V^0$. Supposedly for \ref{enum:ci}, $\Upsilon \overset\varphi\equiv V$, and for \ref{enum:cii}, $\Upsilon$ is abelian.
\begin{enumerate}[nolistsep, noitemsep]
\item Then,
\begin{enumerate}[nolistsep, noitemsep,label=(\roman*)]
  \item partial $\varphi$-convolution right operators are equivariant to~$\Upsilon$,\label{enum:ci}
  \item partial $\M$-convolution left operators are equiv to~$\Upsilon$.\label{enum:cii}
\end{enumerate}
\item Conversely, \label{enum:2}
\begin{enumerate}[label=(\roman*)]
  \item if $\Upsilon$ is domain-symmetric, linear transformations of $\cs(V)$ that are equivariant to~$\Upsilon$ are partial $\varphi$-convolution right operators,
  \item linear transformations of $\cs(V)$ that are equivariant to~$\Upsilon$ are partial $\M$-convolution left operators.
\end{enumerate}
\end{enumerate}
\label{prop:equiP}
\end{theorem}
\begin{proof}
\begin{enumerate}[label=(\roman*)]
   \item
  \begin{enumerate}
    \item Direct sense:\\
    Using the symmetrical expressions, and the fact that $\forall r, r[0] = 0$, we have
    \begin{align*}
    (f_w \circ g (s))[u] & = \displaystyle\sum_{\substack{(g_a,g_b) \in \cd\\ \st g_ag_b = g_u}} g(s)[a] \h{2} w[b]\\
    				& = \displaystyle\sum_{\substack{(g_a,g_b) \in \cd\\ \st g_ag_b = g_u}} s[g^{-1}(a)] \h{2} w[b]\\
    				& = \displaystyle\sum_{\substack{(g_a,g_b) \in \cd\\ \st (g, g_a) \in \cd\\ \st gg_ag_b = g_u}} s[a] \h{2} w[b]\\
    				& = \displaystyle\sum_{\substack{(g_a,g_b) \in \cd\\ \st (g, g_a) \in \cd\\ \st g_ag_b = g^{-1}g_u = g_{\varphi(g^{-1}g_u)} = g_{g^{-1}(u)} }} s[a] \h{2} w[b]\\
    				& = f_w (s)[g^{-1}(u)]\\
    				& = (g \circ f_w (s))[u]
    \end{align*}

    \item Converse:\\
    Let $v \in V$. Denote $e^r_{g_v} = g_v^{-1}g_v$ the right identity element of $g_v$, and $e^r_v = \varphi(e^r_{g_v})$. We have that
    \begin{gather*}
    g_v(e^r_v) = v\\
    \text{So, } \delta_v = g_v(\delta_{e^r_v})
    \end{gather*}

    Let $f \in \cl(\cs(V))$ that is equivariant to $\Upsilon$, and $s \in \cs(V)$. Thanks to the previous remark we obtain that
    \begin{align}
    f(s) & = \displaystyle\sum_{v \in V} s[v] \h{2} f(\delta_v)\nonumber\\
         & = \displaystyle\sum_{v \in V} s[v] \h{2} f(g_v(\delta_{e^r_v}))\nonumber\\
         & = \displaystyle\sum_{v \in V} s[v] \h{2} g_v(f(\delta_{e^r_v}))\nonumber\\
         & = \displaystyle\sum_{v \in V} s[v] \h{2} g_v(w_v) \label{eq:last}
    \end{align}
    where $w_v = f(\delta_{e^r_v})$. In order to finish the proof, we need to find $w$ such that $\forall v \in V, g_v(w) = g_v(w_v)$.

    Let's consider the equivalence relation $\ccr$ defined on $V \times V$ such that:
    \begin{align}
    a \ccr b & \Leftrightarrow    w_a = w_b\nonumber\\
            & \Leftrightarrow e^r_a = e^r_b\nonumber\\
            & \Leftrightarrow g_a^{-1}g_a = g_b^{-1}g_b\nonumber\\
            & \Leftrightarrow (g_b, g_a^{-1}) \in \cd\nonumber\\
            & \Leftrightarrow (g_a^{-1}, g_b) \in \cd \label{eq:eq}
    \end{align}
    with \eqref{eq:eq} owing to the fact that $\Upsilon$ is domain-symmetric.

    Given $x\in V$, denote its equivalence class $\ccr(x)$. Under the hypothesis of the axiom of choice~\citep{zermelo1904beweis} (if $V$ is infinite), define the set $\aleph$ that contains exactly one representative per equivalence class. Let $w = \sum_{n \in \aleph} w_n$. Then $V$ is the disjoint union $V = \displaystyle\cup_{n \in \aleph} \ccr(n)$ and \eqref{eq:last} rewrites:
    \begin{align}
    \forall u \in V, f(s)[u] & = \displaystyle\sum_{n \in \aleph}\sum_{v \in \ccr(n)} s[v] \h{2} g_v(w_n)[u]\nonumber\\
    	 				  & = \displaystyle\sum_{n \in \aleph}\sum_{v \in \ccr(n)} s[v] \h{2} w_n[g_v^{-1}(u)]\nonumber\\
    	 				  & = \displaystyle\sum_{n \in \aleph}\sum_{v \in \ccr(n)} s[v] \h{2} w[g_v^{-1}(u)]\label{eq:eqq}\\
    	 				  & = (s \ast_\varphi w)[u]\nonumber
    \end{align}
    where \eqref{eq:eqq} is obtained thanks to \eqref{eq:eq}.
  \end{enumerate}

  \item With symmetrical expressions, it is clear that the convolution is abelian, if and only if, $\Upsilon$ is abelian. Then \ref{enum:i} concludes.
\end{enumerate}
\end{proof}

\paragraph{Inclusion of EC and LP}
Similarly to the construction in \secref{sec:edges}, partial convolutions can define EC, EC*, LP and LP* counterparts with the same characterizations by Cayley subgraphs whose vertex sets are zero-closure of groupoids, and other similar results.

\paragraph{Limitation of partial convolutions}
However, because of the groupoid associativity, if $g \in \Psi_{\EC}^*(G)$, then, any $v \in V$ \st $g(u) = v$ would be constrained to allow to be acted by every $h$ \st $(h,g) \in \cd$. That is, unless partial transformations that we want to allow to be in each other domains are restricted to the same connected component of the graph, the supporting set of a partial EC*~convolutions would still be bounded by the minimal degree.%complicated to explain..

\subsection{Construction of path convolutions}
\label{sec:path}

To answer the limitation of partial convolutions, given $\cu \subset \Psi_{\EC}^*(G)$, we explore the idea of proceeding with a foliation of each $g \in \cu$ into pieces, each piece corresponding to an edge $e \in E$, and together generating another groupoid with a different associativity law, as follows.

\begin{definition}\textbf{Path groupoid}\\
Let $\cu \subset \Psi_{\EC}^*(G)$. The \emph{path groupoid} generated from $\cu$, denoted $\cu \ltimes G$, with composition rule $\cdl$, is the groupoid obtained inductively with:
\begin{enumerate}
  \item $\cu \ltimes_1 G = \{(g,v) \in \cu \times V, v \in \cd_g \} \subset \cu \ltimes G$
  \item $((g_n,v_n) \cdots (g_1,v_1) , (h_m,u_m) \cdots (h_1,u_1)) \in \cdl \Leftrightarrow h_m(u_m) = v_1$
  \item $((g_n,v_n) \cdots (g_1,v_1))^{-1} = (g_1^{-1}, g_1(v_1)) \cdots (g_n^{-1}, g_n(v_n))$
\end{enumerate}
We call path its objects. Given a length $l \in \bbn$, we denote $\cu \ltimes_l G$ the subset composed of the paths that are the composition of exactly~$l$ paths of $\cu~\ltimes_1~G$.
\end{definition}

\begin{remark}
This groupoid construction is inspired from the field of operator algebra where partial action groupoids have been extensively studied, \eg \cite{nica1994groupoid,exel1998partial,li2016partial}.
\end{remark}

Such groupoids usually come equipped with source and target maps. We also define the path map.

\begin{definition}\textbf{Source, target and path maps}\\
Let a path groupoid $\cu \ltimes G$. We define on it the \emph{source map}~$\alpha$ the \emph{target map}~$\beta$ and the \emph{path map}~$\gamma$ as:
\begin{gather*}
\begin{cases}
  \alpha: (g_n,v_n) \cdots (g_1,v_1) \mapsto v_1 \in V\\
  \beta: (g_n,v_n) \cdots (g_1,v_1) \mapsto g_n(v_n) \in V\\
  \gamma: (g_n,v_n) \cdots (g_1,v_1) \mapsto g_ng_{n-1}\ldots g_1 \in \Psi^*(V^0)\\
  %\lambda: (g_1,v_1) \cdots (g_n,v_n) \mapsto n \in \bbn^*
\end{cases}
\end{gather*}
\end{definition}

\begin{remark}%Note that the path groupoid can also be obtained by derivation of the partial transformation groupoid (\ie $p \in \cu \ltimes G$ can be seen as a derivative of $\gamma(p)$ \wrt $\alpha(p)$), and can thus be seen as the local structure of it.
Note that the path groupoid can be seen as the local structure of the partial transformation groupoid.
\end{remark}

\begin{lemma}\h{0}\\
Note the following properties:
\begin{enumerate}
  \item $(p,q) \in \cdl \Leftrightarrow \alpha(p) = \beta(q)$
  \item $\alpha(p) = \beta(p^{-1})$
  \item $e^l_p=pp^{-1} = (\id, \beta(p))$ and $e^r_p=p^{-1}p = (\id, \alpha(p))$
  \item $\gamma$ is a groupoid partial action. We will denote $\gamma_p$ instead of $\gamma(p)$. \label{enum:3}
  %\item $\forall (p,q) \in \cdl, \gamma_{pq} = \gamma_{p}\gamma_{q}$
  %\item $\beta$ is a partial equivariant map for the groupoid partial action $\gamma$ on $V$.
\end{enumerate}
\end{lemma}

\begin{remark}
Note that this time we won't use the notation $p(v)$ for $\gamma_p(v)$ for clarity.
\end{remark}

One of the key object of our contruction is the use of $\varphi$-equivalence in order to transform a sum over a group(oid) of (partial) transformations, into a sum over the vertex set. With the current notion of path groupoid, searching for something similar amounts to searching for a graph traversal.

\begin{definition}\textbf{Traversal set}\\
Let a graph $\gve$ that is connected. A \emph{traversal set} is a pair $(\cu, \ct)$ of EC partial transformations subsets $\subset \Psi_{\EC}^*(G)$, such that
\begin{enumerate}
  \item $\cu$ is \emph{edge-deterministic}, in the sense that an edge can only correspond to a unique $g$,
    \ie $\forall g,h \in \cu : \exists v \in V, g(v) = h(v) \Rightarrow g=h$
  \item The EC partial transformations of $\ct$ are restrictions of those of $\cu$,\\
    \ie $\forall g \in \cu, \exists! h \in \ct, \begin{cases}\cd_h \subset \cd_g\\ \forall v \in \cd_h, h(v) = g(v)\end{cases}$\\
    (equivalently, $\ct \ltimes G$ is a path subgroupoid of $\cu \ltimes G$ \st $|\ct|=|\cu|$)
  % \item The edges corresponding to $\ct$ are included in those of $\cu$,\\
  %   \ie $\ct \ltimes_1 G \subset \cu \ltimes_1 G$
  \item The subgraph $G_{\ct} = \langle V, \ct \ltimes_1 G \rangle$ is a spanning tree of $G$.
\end{enumerate}
We denote $(\cu, \ct) \in \tree(G)$, and denote by $r$ the root of $G_{\ct}$.\\
For $p \in \ct \ltimes G \subset \cu \ltimes G$, we denote $\gamma_p^{\ct \ltimes G}$ and $\gamma_p^{\cu \ltimes G}$ its path maps.
\end{definition}

\begin{remark}The assumption that the graph $G$ is connected does not lose generality as the construction can be replicated to each connected component in the general case.
\end{remark}

A traversal set $(\cu, \ct)$ defines a $\varphi$-equivalence between the $\alpha$-fiber of the root~$r$ and the vertex set~$V$ as follows.

\begin{lemma}\textbf{Path $\varphi$-Equivalence}\\
Let $(\cu, \ct) \in \tree(G)$. Given $v \in V$, there exists a unique $p_v \in \ct \ltimes G$ such that $\alpha(p_v) = r$ and $\beta(p_v) = v$. Denote $\ct \ltimes^r G = \alpha^{-1}_{\ct \ltimes G}\{r\}$. We can do the following construction:
\begin{enumerate}
  \item Define $\varphi: p_v \mapsto v$.
  \item Define $(p_v, p_u) \mapsto p_v^u \in \cu^0 \ltimes^r G$ such that the sequence of partial transformations of $p_v^u$ and $p_v$ are the same (\ie $\gamma_{p_v^u}^{\cu^0 \ltimes G} = \gamma_{p_v}^{\cu \ltimes G}$), and the source of $p_v^u$ is the target of $p_u$ (\ie $\alpha(p_v^u) = \beta(p_u) = u$)
  \item Define the external composition $p_vp_u = p_v^up_u \in \cu^0 \ltimes^r G$.
\end{enumerate}
Then $\varphi: \alpha^{-1}_{\ct \ltimes G}\{r\} \rightarrow V$ is a bijective partial equivariant map.
\end{lemma}

\begin{proof}
Bijectivity is a consequence of the spanning tree structure of $\ct$.
Equivariance because $\gamma_{p_v}(u) = \gamma_{p_v}\gamma_{p_u}(r) = \gamma_{p_vp_u}(r) = \varphi(p_vp_u)$.
\end{proof}

We can now define the convolution that is based on a path groupoid.

\begin{definition}\textbf{Path convolution}\\
Let $(\cu, \ct) \in \tree(G)$. The \emph{path convolution} is the partial convolution based on the path subgroupoid $\ct \ltimes G$, which uses the groupoid partial action $\gamma := \gamma^{\cu^0 \ltimes G}$ of the embedding groupoid zero-closure $\cu^0 \ltimes G$.
\begin{enumerate}[label=(\roman*)]
  \item In what follows are the three expressions of the path $\varphi$-convolution for signals $s_1, s_2 \in \cs(V)$, and $u \in V$:
\begin{align*}
(s \ast_\varphi w) & = \displaystyle\sum_{v \in V} s[v] \h{2} \gamma_{p_v}(w)\\
                   & = \displaystyle\sum_{\substack{p \in \ct \ltimes G\\ \st \alpha(p) = r}} s[\varphi(p)] \h{2} \gamma_{p}(w)\\
(s \ast_\varphi w)[u] & = \displaystyle\sum_{\substack{(a,b) \in V\\ \st \gamma_{p_a}(b)=u}} s[a] \h{2} w[b]
\end{align*}
  \item The mixed formulations with $w \in \cs(\ct \ltimes G)$ are:
\begin{align*}
(w \ast_{\M} s) & = \displaystyle\sum_{\substack{p \in \ct \ltimes G\\ \st \alpha(p) = r}} w[p] \h{2} \gamma_{p}(s)\\
(w \ast_{\M} s)[u] & = \displaystyle\sum_{\substack{(p,v) \in \ct \ltimes G \times V\\ \st \alpha(p) = r\\ \st \gamma_{p}(v)=u}} w[p] \h{2} s[v]
\end{align*}
\end{enumerate}
\end{definition}

\begin{remark}The role of $\ct$ is to provide a $\varphi$-equivalence. The role of $\cu$ is to extend every partial transformation $\gamma^{\ct \ltimes G}_g$ to the domain of its unrestricted counterpart $\gamma^{\cu \ltimes G}_g$.
\end{remark}

\thref{prop:equiP} also holds for path groupoids, except that the domain-symmetric condition of 2.(i) is not needed.

\begin{theorem}\textbf{Characterization by equivariance to $\cu \ltimes G$'s action}\\
Let $(\cu, \ct) \in \tree(G)$.
\begin{enumerate}[label=(\roman*)]
\item The class of linear transformations of $\cs(V)$ that are equivariant to the path actions of $\cu \ltimes G$ is exactly the path $\varphi$-convolution right-operators;
\item in the abelian case, they are also exactly the $\M$-convolution left-operators.
\end{enumerate}
\label{th:equiU}
\end{theorem}

\begin{proof}
Instead of the domain-symmetric condition that was used in the proof of the converse of \thref{prop:equiP} (2.(i)), we use the fact that any vertex can be reached with an action from the root of the spanning tree of the traversal set. Indeed, given $v \in V$, as we have $\gamma_{p_v}(r)=v$, then $\gamma_{p_v}(\delta_r) = \delta_v$. Therefore, by developping a linear transformation $f(s)$ on the dirac family, and commuting $f$ with $\gamma_{p_v}$, we obtain that $f(s) = s \ast_\varphi w$, where $w = f(\delta_r)$. The rest of the proof is similar to that of \thref{prop:equiP}.
\end{proof}

\begin{remark}
Note that $\cu \ltimes V$'s action is almost the same as the groupoid partial action of $\Upsilon = \langle \cu \rangle$ (only "almost" because not all combinations of partial transformations might exist in the paths).
\end{remark}

Even though $\cu \ltimes V$ alleviates the limitations of $\Upsilon$ we discussed earlier, it introduces another one: the zero-extension of the partial actions $p^u_v$ can be harsh for some graphs with $p^u_v$ often being equal to the groupoid zero (since the edge sequence of $p_u$ must be walkable from $v$). However, there would be less degenerated scenarios for EC* operators counterparts supported on paths of small length. On the other hand, it can be preferable to avoid the degenerated cases and stick to partial actions or strategies discussed in \secref{sec:motiv}.

% \paragraph{Edge convolution operators}
% The counterparts of EC* convolution operators for path convolutions, are indeed path convolution operators obtained with supporting set $\cn \subset \ct \ltimes_1 G$ which any graph can admit. By extrapolation, we can coin them \emph{edge convolution operators}. As shown by this section, to construct one, all we need is a traversal set of partial transformations $(\cu, \ct)$.

