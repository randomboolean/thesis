\section{Inclusion of the edge set in the construction}
\label{sec:edges}

%\subsection{Introduction}

The constructions from the previous section involve the vertex set $V$ and depend on $\Gamma$, a subgroup of the set of invertible transformations on $V$. Therefore, it looks natural to try to relate the edge set and $\Gamma$.

There are two approaches. Either $\Gamma$ describes an underlying graph structure $\gve$, either $G$ can be used to define a relevant subgroup $\Gamma$ to which the produced convolutive operators will be equivariant. Both approaches will help characterize classes of graphs that can support natural definitions of convolutions.

\todo{introduction to be developped more}

%%
%% Draft
%%

%%% What follows is to be reworded and put in the neural network section !
% However we must keep in mind that the resulting convolution would be meant to breed a class of operators that are equivariant to $\Gamma$. In an application where this property is not relevant, such operators would produce layers that are less expressive than dense counterparts. So it is not always possible to leverage a graph structure coming with a dataset if one cannot build meaningful transformations from it. 

\subsection{Convolution on Cayley subgraphs}

In this subsection, we are trying to answser the following question:
\begin{itemize}
	\item What graphs admit a $\varphi$-convolution, or an $\M$-convolution (in the sense that they can be defined with the characterization), under the condition that $\Gamma$ is generated by a set of edge-constrained transformations ?
\end{itemize}

\begin{definition}\textbf{Edge-constrained transformation}\\
An \emph{edge-constrained} (EC) transformation on a graph $\gve$ is a transformation $f: V \mapsto V$ such that
\begin{gather*}
\forall u,v \in V, f(u) = v \Rightarrow u \overset{E}{\sim} v
\end{gather*}
\end{definition}

We denote $\Phi_{\EC}(G)$ and $\Phi^*_{\EC}(G)$ the sets of (EC) and invertible (EC) transformations. When a convolution is defined as a sum over a set that is in one-to-one correspondence with a group that is generated from a set of (EC) transformations, we call it an (EC) convolution.

\begin{remark}
Note that $\Phi^*_{\EC}(G)$ is not a group, thus why we are interested in groups and their generating sets.
\end{remark}

This leads us to consider Cayley graphs~\citep{cayley1878desiderata,wiki:cayley}.

\begin{definition}\textbf{Cayley graph}\\
Let a group $\Gamma$ and one of its generating set $\cu$. The \emph{Cayley graph} generated by $\cu$, is the digraph $\vgve$ such that $V = \Gamma$ and $E$ is such that:
\begin{gather*}
u \rightarrow v \Leftrightarrow \exists g \in \cu, ga = b
\end{gather*}
Also, if $\Gamma$ is abelian, we call it an \emph{abelian Cayley graph}. We call Cayley subgraph, a subgraph that is isomorph to a Cayley graph.
\end{definition}

\begin{remark}
Note that for compatibility with the functional notation that we use, we define Cayley graphs with $ga = b$ instead of $ag = b$.
\end{remark}

\paragraph{Convolution on Cayley graphs}
In the case of Cayley graphs, it is clear that $\cu \subseteq \Phi^*_{\EC}$ and $\Phi^* \supseteq \langle \cu \rangle \equiv V$. So that they admit (EC) $\varphi$-convolutions, and (EC) $\M$-convolutions in the abelian case.

More precisely, we obtain the following characterization:

\begin{proposition}\textbf{Characterization by Cayley subgraph isomorphism}\\
Let a graph $\gve$, then:
\begin{enumerate}[label=(\roman*)]
\item $G$ admits an (EC) $\varphi$-convolution if and only if it contains a subgraph isomorph to a Cayley graph
\item $G$ admits an (EC) $\M$-convolution if and only if it contains a subgraph isomorph to an abelian Cayley graph
\end{enumerate}
\end{proposition}

\begin{proof}
We show the result only in the general case as the proof for the abelian case is similar.
\begin{enumerate}
	\item From left to right: as a direct application of the definitions.

	\item From right to left:\\
Let a graph $\gve$. We suppose it contains a subgraph $\vec{G_s} = \langle V_s, E_s \rangle$ that is graph-isomorph to a Cayley graph $\vec{G_c} = \langle V_c, E_c \rangle$, generated by $\cu$. Let $\psi$ be a graph isomorphism from $G_s$ to $G_c$. To obtain the proof, we need to find a group of invertible transformations $\Gamma$ of $V_s$ generated by a set of (EC) transformations, such that $\Gamma \equiv V_s$.

Let's define the group action $L : V_c \times V_s \rightarrow V_s$ inductively as follows:
\begin{enumerate}[label=(\alph*)]
  \item $\forall g \in \cu, L_g(u) = v \Leftrightarrow g\psi(u) = \psi(v)$ \label{enum:a}
  \item Whenever $L_g$ and $L_h$ are defined, the action of $gh$ is defined by homomorphism as $L_{gh}= L_g \circ L_h$ \label{enum:b}
  \item Whenever $L_g$ is defined, the action of $g^{-1}$ is defined by homomorphism as $L_{g^{-1}}=L_g^{-1}$ \ie $L_{g^{-1}}(u) = v \Leftrightarrow \psi(u) = g\psi(v)$ \label{enum:c}
\end{enumerate}

Note that the induction transfers the property \ref{enum:a} to all $g \in V_c$ in a transitive manner because
\begin{gather*}
L_{gh}(u) = L_g(L_h(u)) = w \Leftrightarrow \exists v \in V_s
\begin{cases}
L_h(u) = v\\
L_g(v) = w
\end{cases}
\end{gather*}
and
\begin{gather*}
\exists v \in V_s
\begin{cases}
h\psi(u) = \psi(v)\\
g\psi(v) = \psi(w)
\end{cases}
\Leftrightarrow gh\psi(u) = \psi(w)
\end{gather*}

We must also verify that this construction is well-defined, \ie whenever we define an action with \ref{enum:b} or \ref{enum:c}, if the action was already defined, then they must be equal. This is the case because the homomorphism $g \mapsto L_g$ on $V_c$ is in fact an isomorphism as
\begin{align*}
L_g = L_h & \Leftrightarrow \forall u \in V, L_g(u) = L_h(u)\\
 & \Leftrightarrow \forall u \in V, g\psi(u) = h\psi(u)\\
 & \Leftrightarrow g = h
\end{align*}

Also note that \ref{enum:c} is needed only in case that $V_c$ is infinite.

Denote the set $L_{\cu} = \{L_g, g \in \cu \}$ and $\Gamma = \langle L_{\cu} \rangle \cong V_c$. Let's define the map $\varphi$ as:
\begin{align*}
\Gamma & \rightarrow V_s\\
\varphi: L_g & \mapsto L_g(\psi^{-1}(\id))
\end{align*}
$\varphi$ is bijective because $\forall g \in V_c, \varphi(L_g) = \psi^{-1}(g)$ thanks to \ref{enum:a}.

Additionally, we have:
\begin{align*}
L_h(\varphi(L_g) & = L_h(L_g(\psi^{-1}(\id)))\\
 & = L_h \circ L_g(\psi^{-1}(\id))\\
 & = L_{hg}(\psi^{-1}(\id))\\
 & = \varphi(L_{hg})\\
 & = \varphi(L_h \circ L_g)
\end{align*}
That is, $\varphi$ is a bijective equivariant map and $ \langle L_{\cu} \rangle = \Gamma \overset{\varphi}{\equiv} V_s$. Moreover, $L_{\cu}$ is a set of (EC) transformations thanks to \ref{enum:a}. Therefore, $G$ admits an (EC)~$\varphi$-convolution.
\end{enumerate}
\end{proof}

\begin{corrolary}\textbf{Characterization by $\varphi$}\\
Let a graph $\gve$, and a set $\cu \subset \Phi^*_{\EC}$ \st
\begin{gather*}
\langle \cu \rangle \cong \Gamma \equiv V' \subset V
\end{gather*}
$G$ admits an (EC) $\varphi$-convolution, if and only if, $\varphi$ is a graph isomorphism between the Cayley graph generated by $\cu$ and the subgraph induced by~$V'$.
\label{cor:cayley}
\end{corrolary}

% \begin{proof}
% \todo{or not to do ?}
% \end{proof}

% \paragraph{Locality and weight sharing also hold}
% Let $f_w$ be a $\varphi$-or-$\M$-convolution operator on a Cayley graph (or Cayley subgraph). If the support of $w$ is included in, or equal to $\cu$, then $f_w(s)[v]$ amounts to a dot product between a weight vector $\widetilde{w}$ made of the non-zero entries of $w$ and another one $s_v$ made of entries of $s$ that are located on vertices connected by an edge to $v$. The same weight vector $\widetilde{w}$ is shared for the calculation of every other $f_w(s)[u]$.

\subsection{Intrinsic properties}

\begin{itemize}
  \item Obviously the contructed convolutions are linear. But do they also preserve the locality and weight sharing properties ?
\end{itemize}

Let $\vgve$ be a Cayley subgraph, generated by $\cu$, of some graph $G$. Recall that its (EC) $\varphi$-convolution operator is a right operator, and can be expressed as
\begin{align}
\forall s \in \cs(V) &, \forall u \in V,\nonumber\\
f_w(s)[u] & = (s \ast_{\varphi} w) [u]\nonumber\\
          & = \displaystyle\sum_{v \in V} s[v] \h{2} w[g_v^{-1}(u)] \label{eq:sum}
\end{align}

From this expression, it is not obvious that $f_w$ is a local operator. To see this, we can show for example the following proposition.

\begin{proposition}\textbf{Locality}\\
When the support of $w$ is a compact (in the sense that its induced subgraph in $G$ is connected), of diameter $d$, the same holds for the support of the sum $\Sigma$ in~\eqref{eq:sum}. More precisely, the subgraph induced by the support of $\Sigma$ is isomorphic to the transpose of the subgraph induced by the support of $w$.
\end{proposition}

\begin{proof}
Without loss of generality subject to growing $\cu$, let's suppose that $w$ has a support $\cm = \varphi(\cn)$, such that $\cn \subset \cu$. $\cn$ and $\cm$ are obviously compacts of diameter $2$. Thanks to \eqref{eq:P}, we have 
\begin{align*}
g_v^{-1}(u) \in \cm & \Leftrightarrow u \in g_v(\cm) = g_v(\varphi(\cn)) = \varphi(g_v\cn)\\
& \Leftrightarrow g_u \in g_v\cn\\
& \Leftrightarrow g_v^{-1} \in \cn g_u^{-1}\\
& \Leftrightarrow g_v \in g_u \cn^{-1}\\
& \Leftrightarrow v \in g_u(\varphi(\cn^{-1}))\\
\end{align*}
where $\cn^{-1}$ reverses the edges of $\cn$. Let's denote $\ck_u = g_u(\varphi(\cn^{-1})) \subset V$.

By composing edge reversal and graph isomorphisms (as $\varphi$ and its inverse are graph isomorphisms by \propref{cor:cayley}), the compactness and diameter of~$\cm$ is preserved for $\ck_u$. More preceisely, the transposed subgraph structure is also preserved.
\end{proof}

Let's define $\cm$, $\cn$ and $\ck_u$ as in the previous proof.
\begin{definition}\textbf{Supporting set}\\
The \emph{supporting set} of an (EC) convolution operator $f_w$, is a set $\cn \subset \Phi^*_{\EC}$, such that
\begin{enumerate}[label=(\roman*)]
  \item when $\ast \text{ is } \ast_\varphi$: $0 \notin w[\cm]$, where $\cm= \varphi(\cn)$ \label{enum:i}
  \item when $\ast \text{ is } \ast_{\M}$: $0 \notin w[\cn]$ \label{enum:ii}
\end{enumerate}
\end{definition}

\begin{definition}\textbf{Local patch for $\ast_\varphi$}\\
The \emph{local patch} at $u \in V$ of an (EC) $\varphi$-convolution operator $f_w$ is defined as $\ck_u = g_u(\varphi(\cn^{-1}))$.
\end{definition}

To see that the weights are tied in the general case \ref{enum:i}, we can show the following proposition.

\begin{proposition}\textbf{Weight sharing}\\
$\forall a, \alpha \in V, \forall b \in \ck_a: \exists \beta \in \ck_{\alpha} \Leftrightarrow g_{\beta}^{-1}(\alpha) = g_{b}^{-1}(a)$
\end{proposition}
\begin{proof}
By using \eqref{eq:P},
\begin{align*}
g_{\ck_\alpha}^{-1}(\alpha) = g_{\ck_a}^{-1}(a)
	& \Leftrightarrow  g_{\alpha}^{-1}g_{\ck_{\alpha}} = g_a^{-1}g_{\ck_a}\\
	& \Leftrightarrow  \ck_{\alpha} = g_{\alpha}g_a^{-1}(\ck_a) = g_{\alpha}g_a^{-1}g_a(\varphi(\cn^{-1}))\\
	& \Leftrightarrow  \ck_{\alpha} = g_{\alpha}(\varphi(\cn^{-1}))
\end{align*}
\end{proof}

\paragraph{Operators supported on the generating set $\cu$ of a Cayley subgraph}

We make the disctinction between general (EC) convolution operators and those for which the weight kernel $w$ is smaller and is supported only on (EC) transformations of $\cu$.

\begin{definition}\textbf{Strictly (EC) convolution operator}\\
A \emph{strictly} edge-constrained (EC*) convolution operator $f_w$, is an (EC) convolution operator such that its supporting set $\cn \subset \cu$.
\end{definition}

Let $f_w$ be an (EC*) convolutional operator. In the general case~\ref{enum:i}, $w \in \cs(V)$, so its support is $\cm = \varphi(\cn)$ such that $\cn \subseteq \cu$. In the abelian case~\ref{enum:ii}, we use instead $w \in \cs(\Gamma)$, and thus its support is directly $\cn$.
Therefore, we can rewrite the expressions of the convolution operator as:
\begin{enumerate}[label=(\roman*)]
  \item $\forall s \in \cs(V), \forall u \in V, f_w(s)[u]
          \overset{(\varphi)}= \displaystyle\sum_{v \in \ck_u} s[v] \h{2} w[g_v^{-1}(u)]$ \label{enum:i}
  \item $\forall s \in \cs(V), f_w(s) \overset{(\M)}= \displaystyle\sum_{g \in \cn} w[g] \h{2} g(s)$ \label{enum:ii}
\end{enumerate}

\begin{remark}
Note that in the abelian case, we can see from \ref{enum:ii} that a definition of a local patch would coincide with the supporting set, so that locality and weight sharing is straightforward.
\end{remark}

From these expressions, it is clear that $\Gamma$ needs not to be fully determined to calculate $f_w(s)[u]$. The case \ref{enum:ii} is the simplest as the only requirement is a supporting set $\cn$ of (EC) invertible transformations. In the case \ref{enum:i}, we only need to determine $\ck_u$.

\paragraph{What exactly is the local patch $\ck_u$ ?}

\todo{}

%\subsection{On the choice of the supporting set $\cn$}

%\paragraph{Exploiting subgraph symmetries}

% \begin{proposition}
% Let a Cayley graph $\vec{G} = \langle \Gamma = \langle \cu \rangle, E \rangle$, it bears the following interesting graph symmetries:
% \begin{enumerate}
% 	\item it is vertex-transitive \ie $\forall g, h \in \Gamma, \exists \varphi \in 
%   \item it is order-regular
% \end{enumerate}
% \end{proposition}

%\todo{}

\subsection{Groupoid convolutions}

One possible limitation coming from searching for Cayley subgraphs is that they are order-regular \ie the in- and the out-degree $d = |\cu|$ of each vertex is the same. That is, for a general graph $G$, the size of the weight kernel $w$ of an (EC*) convolution operator $f_w$ supported on $\cu$ is bounded by $d$, which in turn is bounded by twice the mininimal degree of $G$ (twice because $G$ is undirected and $\cu$ can contain every inverse).

There are a lot of possible strategies to overcome this limitation. For example:
\begin{enumerate}
  \item connecting each vertex with its $k$-hop neighbors, with $k > 1$,
  \item artificially creating new connections for less connected vertices,
  \item allowing the supporting set $\cn$ to exceed $\cu$ \ie dropping * in (EC*).
\end{enumerate}

These strategies require to concede that the topological structure supported by $G$ is not the best one to support an (EC*) convolution on it, which breeds the following question:
\begin{itemize}
  \item What can we relax in the previous (EC*) contruction in order to unbound the supporting set, and still preserve the equivariance characterization?
\end{itemize}

The latter constraint is a consequence that every vertex of the Cayley subgraph $\vec{G}$ must be composable with every generator from $\cu$. Therefore, the answer consists in considering groupoids~\citep{brandt1927verallgemeinerung} instead of groups. Roughly speaking, a groupoid is almost a group except that its composition law needs not be defined everywhere. \cite{weinstein1996groupoids}, unveiled the benefits to base convolutions on groupoids instead of groups in order to exploit partial symmetries.

\begin{definition}\textbf{Groupoid}\\
A groupoid $\Upsilon$ is a set equipped with a partial composition law with domain $\cd \subset \Upsilon \times \Upsilon$ that is
\begin{enumerate}
	\item closed into $\Upsilon$ \ie $\forall (g, h) \in \cd, gh \in \Upsilon$
	\item associative \ie
    $\forall f,g,h \in \Upsilon,
      \begin{cases}
        (f, g), (g, h) \in \cd \Leftrightarrow (fg,h), (f, gh) \in \cd\\
        (f, g), (fg, h) \in \cd \Leftrightarrow (g,h), (f, gh) \in \cd\\
        \text{when defined, } (fg)h = f(gh)
      \end{cases}$ \label{enum:2}
	\item invertible \ie
		$\forall g \in \Upsilon, \exists! g^{-1} \in \Upsilon \quad\st
			\begin{cases}
				(g,g^{-1}), (g^{-1},g)  \in \cd\\
				(g,h) \in \cd \Rightarrow g^{-1}gh = h\\
				(h,g) \in \cd \Rightarrow hgg^{-1} = h
			\end{cases}$
\end{enumerate}
\end{definition}

\begin{remark}
Note that left and right inverses are necessarily equal (because $(gg^{-1})g=g(g^{-1}g)$). Also note we can define a right identity element $e^r_g = g^{-1}g$, and a left one $e^l_g = gg^{-1}$, but they are not necessarily equal and depend on $g$.
\end{remark}

Every definition related to groups can be adapted to groupoids. In particular, let's adapt a few notions.

\begin{definition}\textbf{Groupoid partial action}\\
A partial \emph{action} of a groupoid $\Upsilon$ on a set $V$, is a function $L$, with domain $\cd_L \subset \Upsilon \times V$ and valued in $V$, such that the map $g \mapsto L_g$ is a groupoid homomorphism.
\end{definition}

\begin{remark}
As usual, we will confound $L_g$ and $g$ when there is no possible confusion, and we denote $\cd_{L_g} = \cd_g = \{ v \in V, (g,v) \in \cd_L \}$.
\end{remark}

\begin{definition}\textbf{Partial equivariant map}\\
A map $\varphi$ from a groupoid $\Upsilon$ partially acting on the destination set $V$ is said to be a \emph{partial equivariant map} if
\begin{gather*}
\forall g,h \in \Upsilon,
  \begin{cases}
    \varphi(h) \in \cd_g \Leftrightarrow (g,h) \in \cd\\
    g(\varphi(h)) = \varphi(gh)
  \end{cases}
\end{gather*}
\end{definition}

Also, $\varphi$-equivalence between a subgroupoid and a set is defined similarly with $\varphi$ being a bijective \emph{partial} equivariant map between them.

\paragraph{Partial transformations groupoid}
We denote by $\Psi(V)$ the set of partially defined transformations on $v$, and $\Psi^*(V)$ the groupoid of its invertible ones (\ie injective ones).

Extending a partial transformation $g$ on the signal space $\cs(V)$ is a bit tricky, because only the signal entries corresponding to $\cd_g$ are moved. A convenient way to do this is to consider the groupoid closure obtained with the zero of an embedding semigroup. We explain this construction with the following definitions.

\begin{definition}\textbf{Semigroup with zero}\\
A semigroup $\cg$ is a set equipped with a closed and associative composition law. The zero $0$ of $\cg$, if it exists, is the unique absorbing element \ie
\begin{gather*}
\forall g \in \cg, 0g = g0 = 0
\end{gather*}
\end{definition}

\begin{remark}
A group is a semigroup for which every element is invertible. However a group cannot have a zero.
\end{remark}

\begin{definition}\textbf{Zero-closure}\\
The \emph{zero-closure} of a groupoid $\Upsilon$, denoted $\Upsilon^0$, is the semigroup $\Upsilon \cup {0}$, such that
\begin{gather*}
\forall (g,h) \notin \cd, gh = 0
\end{gather*}
\end{definition}

\begin{lemma}\textbf{Zero-extension of $\varphi$}\\
Let a partial equivariant map $\varphi: \Upsilon \rightarrow V$. It can be extended to a (total) equivariant map $\varphi: \Upsilon^0 \rightarrow V \cup \varphi(0)$ and imposes $L_0 = \varphi(0)$. Additionaly, it also imposes that $\varphi(0) \notin V$, and so we can denote $0 =\varphi(0)$.
\end{lemma}
\begin{proof}
We have $\varphi(0) \notin V$ because $\varphi$ is bijective. Additionaly, we must have $\forall (g,h) \notin \cd, g(\varphi(h)) = \varphi(gh) = \varphi(0) = 0$.
\end{proof}

\begin{remark} Note that for notational conveniency, we use the same symbol $0$ for $0_{\bbr}$, $0_{\Psi^*(V)^0}$ and $0_V$.
\end{remark}

\begin{lemma}\textbf{Zero-extension of $\cs(V)$}\\
Let a subgroupoid $\Upsilon \subset \Psi^*(V)$ such that $\Upsilon \overset\varphi\equiv V$. Under the convention that $s[0] = 0$, a transformation $g \in \Upsilon^0$ can be extended to the signal space $\cs(V)$ as usual:
\begin{gather*}
\forall s \in \cs(V), \forall v \in V, g(s)[v] =
  \begin{cases}
    0 &\text{ if } g = 0\\
    s[g^{-1}(v)]% = s[\varphi(g^{-1}g_v)]
      &\text{ else}
  \end{cases}
\end{gather*}
\end{lemma}
\begin{proof}
Straightforward with \eqref{eq:P}.
\end{proof}

We can now define partial convolutions as follows.

\todo{}

% \begin{definition}\textbf{Partial convolution}\\
% Let a subgroupoid $\Upsilon \subset \Psi*(V)$, such that $\Upsilon \overset\varphi\equiv V$. The partial convolution between two signals $s_1$ and $s_2 \in \cs(V)$ is defined as:
% \end{definition}

