\section{Construction with the edge set}

At this point, there are a two drawbacks: (D1) that $V \cong \Gamma$, (D2) the converse of the characterization doesn't hold. Let's now consider the edge set in our construction and keep these drawbacks in mind.

\subsection{Construction on Cayley graphs}

We now consider the class of Cayley (di)graphs~\citep{cayley1878desiderata,wiki:cayley} because the $\varphi$-equivalence property \eqref{eq:P} naturally holds on them.

\todo{rewrite and obtain: On a Cayley graph, there is $\varphi$ such that the subgroupoid of transformation that are EC is $\varphi$-equivalent (proof on $\cu$ then extended)}

\begin{definition}\textbf{Cayley graph}\\
Let a group $\Gamma$ and one of its generating set $\cu$. The \emph{Cayley graph} generated by $\cu$, is the digraph $\vgve$ such that $V \cong \Gamma$ through $u \overset{\h{9}\varphi^{-1}}{\mapsto} g_u$, and $E$ is such that:
\begin{gather*}
a \sim b \Leftrightarrow \exists u \in \varphi(\cu) \subset V, g_ug_a = g_b
\end{gather*}
\end{definition}

\begin{proposition}
A Cayley graph of group $\Gamma \overset{\varphi}{\cong} V$ is $\varphi$-equivalent.
\end{proposition}

\begin{proof}
Let $\cu \subset \Gamma$ the generating set of the Cayley graph.



\todo{}
\end{proof}

We can alleviate (D1) by summing onto the generating set $\cu$ instead of onto~$\Gamma$, which also makes the convolution on Cayley graphs edge-constrained.

\todo{not equivariant unless abelian group}

\begin{definition}\textbf{Cayley graph convolution}\\
\begin{align*}
\forall u \in V, (s_1 \ast s_2) [u] & = \displaystyle \sum_{g \in \cu} s_1[g] \h{2} g(s_2)
\end{align*}
\end{definition}

Conversely, let a graph $\gve$ and a $\varphi$-equivalent subgroup $\Gamma \subseteq \Phi^{*}(V)$. Then we can generate a Cayley graphs with a generating set of $\Gamma$, and thus define a Cayley graph convolution. In particular, when transformations of~$\Gamma$ are edge-constrained, that would also be the case for this convolution on $G$, which leads us to study edge-constrained transformations.

\todo{operator and characterization}
\todo{which graph is a Cayley graph ?}

\subsection{Construction on graph groupoids}

\todo{work in progress}

On graphs, we notice that the property \eqref{eq:P} can be realized by transformations acting on edges. However, unless the graph is complete, these actions can't be composed everywhere to form another edge constrained action. The algebraic structure that posesses the same kind of properties than a group except that its composition law is not defined everywhere is called a groupoid. The following definitions clarify our discussion.

\begin{definition}\textbf{Groupoid}\\
A groupoid is a set equiped with a closed partial composition law, a unique identity element, and every unique inverses.
\end{definition}

\begin{remark}We use the convention than left and right inverses must be the same.
\end{remark}

\begin{definition}\textbf{Graph groupoid}\\
The \emph{groupoid} $\cp(G)$ of a graph $\gve$ is the set of its paths equiped with:
\begin{enumerate}
\item two maps $\psi$ and $\varphi$ that respectively map a path to its first and last element,
\item a closed partial composition law $gh$ defined if and only if $\psi(g) = \varphi(h)$, which concatenates $g$ behind $h$ and \sout{removes adjacent duplicate vertices} \textcolor{red}{to rewrite},
\item an inverse operator $\h{0}^{-1}$ which maps a path to its reverse,
\item an identity element $\id$ which is the path of length $0$.
\end{enumerate}
\end{definition}

\begin{remark}Recall from \defref{def:path} that a path can't contain adjacent duplicates.
\end{remark}

\begin{remark}Note that even though the composite path $gh$ has elements of $h$ before those of $g$ we write $gh$ instead of $hg$ because we'll need the left operand to act on the right one through functional notation $g(h)$.
\end{remark}

\begin{definition}\textbf{Graph $k$-groupoid}\\
The \emph{$k$-groupoid} $\cp_k(G)$ of a graph $\gve$, for $k \in \bbn^*$, is the groupoid obtained by restricting $\cp(G)$ to paths of length at most $k$ (the definition domain of its composition law is also further restricted by the length of the resulting paths in $\cp(G)$).
\end{definition}

\begin{definition}\textbf{$k$-Groupoid convolution}\\
Let a graph $\gve$. Let a subgroupoid $\Gamma \subseteq \cp_k(G)$. The $k$-groupoid convolution between two signals $s_1$ and $s_2 \in \cs(\Gamma)$ is defined as:
\begin{align*}
\forall h \in \Gamma, (s_1 \ast s_2) [h] & = \displaystyle \sum_{\substack{(a,b) \in \Gamma^2 \\ \st ab=h }} s_1[a] \h{2} s_2[b] \\
& = \displaystyle \sum_{\substack{g \in \Gamma\\ \st \varphi(g) = \varphi(h)}} s_1[g] \h{2} s_2[g^{-1}h]\\
& = \displaystyle \sum_{\substack{g \in \Gamma\\ \st \psi(g) = \psi(h)}} s_1[hg^{-1}] \h{2} s_2[g]
\end{align*}
\label{def:pconv}
\end{definition}

\begin{claim}\textbf{Path transformation}\\
Let a graph $\gve$. By identifying vertices with paths of length $1$, a path $g \in \cp(G)$ can act as a transformation on $v \in V$ through the composition law of $\cp(G)$. Also note that $g(v) = g(v^{-1})$.
\end{claim}

We can nom define the $k$-Groupoid convolution operator on $\cs(G)$ by restriction of the second operand from $\cs(\Gamma)$ to paths of length $1$:

\begin{definition}\textbf{$k$-Groupoid convolution operator}\\
Let a graph $\gve$. Let a subgroupoid $\Gamma \subseteq \cp_k(G)$. The $k$-groupoid convolution operator $f_w$ with parameter $w \in \cs(\Gamma)$ is defined as:
\begin{gather*}
\forall s \in \cs(\Gamma), \forall h \in \Gamma, f_w(s)[h] = (s \ast w)[h]
\end{gather*}
And when restricted to $\cs(G)$ it is defined as:
\begin{gather*}
\forall s \in \cs(G), \forall v \in V, f_w(s)[v] = \displaystyle \sum_{\substack{g \in \Gamma\\ \st \psi(g) = v}} s[g(v)] \h{2} w[g]\\
\forall s \in \cs(G), \forall v \in V, f_w(s)[v] = \displaystyle \sum_{\substack{g \in \Gamma\\ \st \varphi(g) = v}} s[g] \h{2} w[g^{-1}(v)]
\end{gather*}
\end{definition}

\begin{proposition}\textbf{Groupoid equivariance to $\Gamma$}\\
$k$-Groupoid convolution operators on $\cs(G)$ are groupoid equivariant to $\Gamma$ \ie
\begin{gather*}
\exists w \in \cs(\Gamma), f = w \ast . \Rightarrow
\forall v \in V, \forall g \in \Gamma \st \psi(g^{-1}) = v,
f \circ g [v]= g \circ f [v]
\end{gather*}
\label{prop:equi}
\end{proposition}

\begin{gather}
g(h(v)) maybe false
\end{gather}




Mini patron of todo:
\begin{itemize}
\item Equivariance to $\Gamma$ holds, proof
\item Converse of characterization does not hold yet, except on orbits
\item property for it to hold
\item relaxing one-to-one correspondence constraint but keeping other properties
\item other avenue instead of property: should make use of edges to build a group structure
\item ideal graph (lattice-regular)
\item if group is too much then just groupoid structure from edges is enough
\end{itemize}

\todo{finish this section}

\subsection{To rename}

% \begin{definition}\textbf{Infinite graph}\\
% An \emph{infinite graph} is defined by natural extension of the notion of graph $G=\langle V,E \rangle$ where $V$ and $E$ can be infinite. We denote $\order{G} = \infty$.
% \end{definition}

\begin{definition}\textbf{Graph automorphisms}\\
A graph automorphism of a graph $\gve$ is a bijection in the vertex domain $\phi: V \rightarrow V$ such that $\{u,v\} \in E \Leftrightarrow \{\phi(u), \phi(v)\} \in E$. We denote $\ca(G)$ the group of automorphism on $G$.

We denote by $\ce(\phi)$ the set of input-output mapping of $\phi$, defined as $\ce(\phi) = \{ (x,y) \in V^2, \phi(x) = y \}$.

A graph automorphism $\phi$ is said to be \emph{edge-constrained} (EC) if $\ce(\phi) \subseteq E$. We denote $\ca_{\EC}(G)$ the set of edge-constrained automorphism on $G$.
\end{definition}

\begin{definition}\textbf{Orthogonality}\\
Two graph automorphisms $\phi_1$ and $\phi_2$ are said to be orthogonal, if and only if $\ce(\phi_1) \cap \ce(\phi_2) = \emptyset$, denoted $\phi_1 \bot \phi_2$. They are said to be aligned otherwise.

Similarly, we define orthogonality of $r$ automophisms as $\phi_1 \bot \cdots \bot \phi_r \Leftrightarrow \ce(\phi_1) \cap \cdots \cap \ce(\phi_r) = \emptyset$
\end{definition}


\subsection{Lattice-regular graph}

\begin{definition}\textbf{Lattice-regular graph}\\
A lattice-regular graph is a regular graph that admits $r$ orthogonal edge-constrained automorphisms, where $r$ is its degree.
\end{definition}


%\subsubsection{Grids}{}

%\subsubsection{Lattices}

%\subsubsection{Spatial graphs}

%\subsubsection{Projections of spatial graphs}