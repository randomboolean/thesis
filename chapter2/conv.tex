\section{Analysis of the classical convolution}

In this section, we are exposing a few properties of the classical convolution that a generalization to graphs would likely try to preserve. For now let's consider a graph $G$ agnostically of its edges \ie $G \cong V$ is just the set of its vertices.

\subsection{Properties of the convolution}

Consider an edge-less grid graph \ie $G \cong \bbz^2$. By restriction to compactly supported signals, this case encompass the case of images.

\begin{definition}\textbf{Convolution on $\cs(\bbz^2)$}\\
Recall that the convolution between two signals $s_1$ and $s_2$ over $\bbz^2$ is a binary operator in $\cs(\bbz^2)$ defined as:
\begin{align*}
\forall (a,b) \in \bbz^2, (s_1 * s_2) [a,b] & = \displaystyle \sum_i \sum_j s_1[i,j] \h{2} s_2[a-i, b-j]
\end{align*}
\end{definition}

\todo{list some properties}

\subsection{Characterization on grid graphs}

Let's recall first what is a transformation, and how it acts on signals.

\begin{definition}\textbf{Transformation}\\
A \emph{transformation} $f: V \rightarrow V$ is a function with same domain and codomain. The set of transformations is denoted $\Phi(V)$. The set of bijective transformations is denoted $\Phi^*(V) \subset \Phi(V)$.

In case $f \in \Phi^*(V)$, it can act on $\cs(V)$ through the linear operator $L_f \in \cl(\cs(V))$ defined as:
\begin{gather*}
\forall s \in \cs(V), \forall v \in V, f(s)[v] := L_f s[v] = s[f^{-1}(v)]
\end{gather*}
\ie an entry of a transformed signal is obtained by doing a lookup of the entry of the original signal.

In case $f \notin \Phi^*(V)$, we can still define $L_f \in \cl(\cs(V))$, however we need to linearly aggregate the entries on the fibers:
\begin{gather*}
\forall s \in \cs(V), \forall v \in V, f(s)[v] := L_f s[v] = \agg\{s[u], u \in f^{-1}\{v\}\}
\end{gather*}
where $\agg$ can be for example the sum, the average, or the max, and $\agg(\emptyset) = 0$.
\end{definition}

We also recall the formalism of translations.

\begin{definition}\textbf{Translation on $\cs(\bbz^2)$}\\
A translation on $\bbz^2$ is defined as a transformation $t \in \Phi^*(\bbz^2)$ such that
\begin{gather*}
\exists (a,b) \in \bbz^2, \forall (x,y) \in \bbz^2, t(x,y) = (x+a,y+b)
\end{gather*}
It also acts on $\cs(\bbz^2)$ with the notation $t_{a,b}$ \ie
\begin{gather*}
\forall s \in \cs(\bbz^2), \forall (x,y) \in \bbz^2, t_{a,b}(s)[x,y] = s[x-a, y-b]
\end{gather*}
For any set $E$, we denote by $\ct(E)$ its translations if they are defined.
\end{definition}

The next proposition fully characterizes convolution operators with their translational equivariance property. This can be seen as a discretization of a classic result from the theory of distributions.

\begin{proposition}\textbf{Characterization of convolution operators on $\cs(\bbz^2)$}\\
On real-valued signals over $\bbz^2$, the class of linear transformations that are equivariant to translations is exactly the class of convolutive operations \ie
\begin{gather*}
\exists w \in \cs(\bbz^2), f = . \ast w \Leftrightarrow
\begin{cases}
 f \in \cl(\cs(\bbz^2))\\
 \forall t \in \ct(\cs(\bbz^2)), f \circ t = t \circ f
\end{cases}
\end{gather*}
\label{prop:equi}
\end{proposition}

\begin{proof}
The result from left to right is a direct consequence of the definitions:
\begin{align}
\forall s \in \cs(\bbz^2), \forall s' \in \cs(\bbz^2), & \forall (\alpha, \beta) \in \bbr^2,\forall (a,b) \in \bbz^2,\nonumber\\
 %f_w(s)[a,b] & = \displaystyle \sum_i \sum_j s[i,j] \h{2} w[a-i, b-j] \tag{definition}\\
 f_w(\alpha s + \beta s')[a,b] & = \displaystyle \sum_i \sum_j (\alpha s + \beta s')[i,j] \h{2} w[a-i, b-j]\nonumber\\
 & = \alpha f_w(s)[a,b] + \beta f_w(s')[a,b] \tag{linearity}\\
\forall s \in \cs(\bbz^2), \forall (\alpha, \beta) \in \bbz^2, & \forall (a,b) \in \bbz^2,\nonumber\\
f_w \circ t_{\alpha,\beta} (s)[a,b] & = \displaystyle \sum_i \sum_j t_{\alpha,\beta}(s)[i,j] \h{2} w[a-i, b-j]\nonumber\\
 & = \displaystyle \sum_i \sum_j s[i - \alpha,j - \beta] \h{2} w[a-i, b-j]\nonumber\\
 & = \displaystyle \sum_{i'} \sum_{j'} s[i',j'] \h{2} w[a - i' - \alpha, b - j'- \beta]\label{eq:bij}\\
 & = f_w (s)[a - \alpha,b - \beta]\nonumber\\
 & = t_{\alpha,\beta} \circ f_w (s)[a,b] \tag{equivariance}
\end{align}
Now let's prove the result from right to left .

Let $f \in \cl(\cs(\bbz^2))$, $s \in \cs(\bbz^2)$. We suppose that $f$ commutes with translations. Recall that $s$ can be linearly decomposed on the infinite family of dirac signals:
\begin{gather*}
s = \displaystyle \sum_i \sum_j s[i,j] \h{2} \delta_{i,j} \text{, where }
\delta_{i,j}[x,y] = \begin{cases} 1 & \text{if } (x,y) = (i,j)\\ 0 & \text{otherwise} \end{cases}
\end{gather*}
By linearity of $f$ and then equivariance to translations:
\begin{align*}
f(s) & = \displaystyle \sum_i \sum_j s[i,j] \h{2} f(\delta_{i,j})\\
 & = \displaystyle \sum_i \sum_j s[i,j] \h{2} f \circ t_{i,j} (\delta_{0,0})\\
 & = \displaystyle \sum_i \sum_j s[i,j] \h{2} t_{i,j} \circ f (\delta_{0,0})
\end{align*}
By denoting $w = f (\delta_{0,0}) \in \cs(\bbz^2)$, we obtain:
\begin{align}
\forall (a,b) \in \bbz^2, f(s)[a,b] & = \displaystyle \sum_i \sum_j s[i,j] \h{2} t_{i,j}(w)[a,b] \label{eq:conv}\\
 & = \displaystyle \sum_i \sum_j s[i,j] \h{2} w[a-i, b-j] \nonumber\\
\text{\ie } f(s) & = s \ast w \nonumber
\end{align}
\end{proof}

\subsection{Usefulness of convolutions in deep learning}

\begin{remark}\textbf{Equivariance property of CNNs}\\
In deep learning, an important argument in favor of CNNs is that convolutional layers are equivariant to translations. Intuitively, that means that a detail of an object in an image should produce the same features independently of its position in the image.
\end{remark}

\begin{remark}\textbf{Lossless superiority of CNNs over MLPs}\\
The converse result, as a consequence of~\propref{prop:equi}, is never mentioned in deep learning literature. However it is also a strong one: it means that layers of CNNs have every translational equivariant functions in their search space, so it implies that the reduction of parameters from an MLP to a CNN is done with strictly no loss of expressivity (provided the objective function is know to bear this property). Besides, it helps the training to search in a much more confined space.
\end{remark}

Hence, in our construction, we will try to preserve the characterization from \propref{prop:equi} as it is mostly the reason why they are succesful in deep leraning. Note that the reduction of parameters compared to a dense layer is also a consequence of this characterization.

\subsection{Related work generalising the discrete convolution on graphs}

\todo{}