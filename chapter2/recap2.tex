\section{Conclusion}

In this chapter, we constructed the convolution on graph domains.

\begin{enumerate}
\item We first saw that classical convolutions are in fact the class of linear transformations of the signal space that are equivariant to translations. For signals defined on graph domains, there is no natural definition of translations.
\item Therefore, we adopted a more abstract standpoint and considered in the first place any kind of transformation of the vertex set~$V$. Hence, given a subgroup of transformation~$\Gamma$, we constructed the class of linear transformations of the signal space that are equivariant to it. This provided us with an expression of a convolution based on this subgroup, and a bijective equivariant map between~$\Gamma$ and~$V$, in order to transport a sum over~$\Gamma$ into a sum over~$V$. We also proposed a simpler expression in the abelian case.
\item Then, we introduced the role of the edge set~$E$, and we constrained~$\Gamma$ by it. This allows us to obtain a characterization of admissibility of convolutions by Cayley subgraph isomorphism, and to analyze intrinsic properties of the constructed convolution operator, namely locality and weight sharing. We also discussed operators with a smaller kernel, in particular those that are strictly edge-constrained, as they are simpler to construct.
\item Finally, we overcome the limitation that some graphs only have trivials or low order Cayley subgraphs. In this case, we rebased our construction on groupoids of partial transformations~$\Upsilon$ as a first iteration, but this one didn't overcome fully the above-mentioned limitation. As a last iteration, we broke down the previous construction into elementary partial actions onto the edges, recomposed into path groupoids~$\cu \ltimes G$. Similarly, equivariance characterization and intrinsic properties hold, and the simpler (EC*) construction is also possible.
\end{enumerate}

\paragraph{Summary of practical (EC*) convolution operators}
\begin{enumerate}
  \setcounter{enumi}{2}
\item For graphs that are quite regular, in the sense that they contain an above-low-order Cayley subgraph (order $k \geq 4)$, we saw in \secref{sec:ec} that all we need to construct an (EC*) convolution operator is a generating set $\cu$ of transformations, without the need of composing its elements, and optionally (in the non-abelian case) to move a local patch $\ck_{\id}$ over the graph domain.
\item For a general graph, we saw in \secref{sec:path} that all we need to construct an (EC*) path convolution operator is a traversal set $(\cu, \ct)$ of partial transformations, without the need to compose the paths.
\end{enumerate}

In the next chapter, we will encounter examples of (EC) and (EC*) convolution operators defined on graphs, that can be expressed under group representations or under path groupoid representations.
