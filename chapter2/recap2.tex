\section{Conclusion}
\label{sec:2.5}

In this chapter, we constructed convolutions on graph domains.
% \begin{enumerate}[nolistsep, noitemsep]
% \item We first saw that classical convolutions are in fact the class of linear transformations of the signal space that are equivariant to translations. For signals defined on graph domains, there is no natural definition of translations.
% \item Therefore, we adopted a more abstract standpoint and considered in the first place any kind of transformation of the vertex set~$V$. Hence, given a group~$\Gamma$ acting on~$V$, we constructed the class of linear transformations of the signal space that are equivariant to it. This provided us with an expression of a convolution based on this subgroup, and a bijective equivariant map between~$\Gamma$ and~$V$, in order to transport a sum over~$\Gamma$ into a sum over~$V$. We also proposed a simpler expression in the abelian case.
% \item Then, we introduced the role of the edge set~$E$, and we constrained~$\Gamma$ by it. This leads us to define two types of properties. One is about contraining the transformation to follow the edges, and the other one is about preserving locality through the group actions. For each property, we obtain characterizations of graphs that admit convolutions that bear them. We analyzed intrinsic properties of the constructed convolution operator, namely locality and weight sharing. We also discussed operators with a smaller kernel, in particular those that are EC*, as they are simpler to construct.
% \item Finally, we explored avenues to overcome the limitation that groups aren't well representative of symmetries on some graphs. So we extended the previous construction with two types of groupoids, and obtained construction for which the characterization by equivariance also holds. We saw that they can handle degenerated cases, but the expression of the obtained convolution can be degenerated as well.
% \end{enumerate}
We first saw that classical convolutions are in fact the class of linear transformations of the signal space that are equivariant to translations. For signals defined on graph domains, there is no natural definition of translations.
Therefore, we adopted a more abstract standpoint and considered in the first place any kind of transformation of the vertex set~$V$. Hence, given a group~$\Gamma$ acting on~$V$, we constructed the class of linear transformations of the signal space that are equivariant to it. This provided us with an expression of a convolution based on this subgroup, and a bijective equivariant map between~$\Gamma$ and~$V$, in order to transport a sum over~$\Gamma$ into a sum over~$V$. We also proposed a simpler expression in the abelian case.
Then, we introduced the role of the edge set~$E$, and we constrained~$\Gamma$ by it. This leads us to define two types of properties. One is about constraining the transformations to follow the edges, and the other one is about preserving locality through them. For each property, we obtain characterizations of graphs that admit convolutions that bear them. We analyzed intrinsic properties of the constructed convolution operator, namely locality and weight sharing. We also discussed operators with a smaller kernel, as they are more practical and simpler to construct.
Finally, we explored avenues to overcome the limitation that groups aren't well representative of symmetries on some graphs. So we extended the previous construction with two types of groupoids, and we obtained constructions for which the characterization by equivariance also holds. We saw that they can handle degenerated cases, but the expressions of the obtained convolutios can be degenerated as well.

% \paragraph{Summary of practical EC* convolution operators}
% \begin{enumerate}
%   \setcounter{enumi}{2}
% \item For graphs that are quite regular, in the sense that they contain an above-low-degree Cayley subgraph (degree $d \geq 4)$, we saw in \secref{sec:ec} that all we need to construct an EC* convolution operator is a generating set $\cu$ of transformations, without the need of composing its elements, and optionally (in the non-abelian case) to move a local patch $\ck_{\id}$ over the graph domain.
% \item For a general graph, we saw in \secref{sec:path} that all we need to construct an EC* path convolution operator is a traversal set $(\cu, \ct)$ of partial transformations, without the need to compose the paths.
% \end{enumerate}

% In the next chapter, we will encounter examples of EC and EC* convolution operators defined on graphs, that can be expressed under group representations or under path groupoid representations.
