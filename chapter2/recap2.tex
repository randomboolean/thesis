\section{Conclusion}

\todo{WIP}



% \subsection{Edge-constrained groupoid convolutions}

% \todo{Irrelevant because associativity and equivariant map property clash and we still have the same constraints as before}

% \todo{Erase and rewrite this subsection}

% Similarly to the construction with Cayley graphs of \ref{sec:cayley}, we start by adapting the related notions.

% \begin{definition}\textbf{Groupoid generating set}\\
% A set~$\cu$, equipped with a partial composition law of domain~$\cd_0$, is a \emph{generating set} of a groupoid~$\Upsilon$, if every object of~$\Upsilon$ can be expressed as a composition of objects of~$\cu$ and their inverses. In this case, the domain $\cd$ of the partial composition law of~$\Upsilon$ can be deducted inductively from~$\cd_0$ with the associativity and invertibility axioms of~$\Upsilon$ (and eventually also with the domain-symmetric axiom if~$\Upsilon$ is domain-symmetric).
% \end{definition}

% A partial Cayley graph is defined similarly as the (total) Cayley graph.

% \begin{definition}\textbf{Partial Cayley graph}\\
% Let a groupoid $\Upsilon = \langle \cu, \cd_0 \rangle$. The \emph{partial Cayley graph} generated by $\cu$ and $\cd_0$, is the digraph $\vgve$ such that $V = \Upsilon$ and $E$ is such that:
% \begin{gather*}
% u \rightarrow v \Leftrightarrow \exists g \in \cu, ga = b
% \end{gather*}
% We call \emph{partial Cayley subgraph}, a subgraph that is isomorph to a partial Cayley graph.
% \end{definition}

% \begin{remark}
% The characterization by a partial Cayley subgraph, of graphs admitting an (EC) convolution based on a groupoid, similar to \propref{prop:chc} also holds for the groupoid representation.% However, it is somewhat trivial as every graph admits a partial Cayley subgraph (for example by labelling every edge in a neighbour with a transformation).
% \end{remark}

% In the case of the groupoid representation, what is more interesting is that an (EC) groupoid convolution can be characterized by a generating set of the groupoid it is based on.

% \begin{proposition}\textbf{Characterization by generating set}\\
% Let a graph $\gve$ such that it admits an (EC) convolution based on a groupoid~$\Upsilon$. Then,~$G$~contains a sugraph~$\vec{G}$ that is isomorph to a partial Cayley subgraph~$\vec{G_c}$, of generating set $\cu \subset \Psi^*_{\EC}(G)$, such that the underlying equivariant map~$\varphi$ of the (EC) convolution is also a graph isomorphism between~$\vec{G_c}$ and~$\vec{G}$.
% \end{proposition}

% The proof is omitted because it would be highly similar with the one of \corref{cor:cayley}.

% \paragraph{Strictly edge-constrained groupoid convolution}

% Thanks to the previous construction, in order to define a meaningful convolution operator on a general graph~$\gve$, all we need is a set~$\cu$ of (EC) partial transformations \emph{that need not to be defined on every vertex}, and a composition rule~$\cd_0$. From the set~$\cu$, we can choose a supporting set $\cn \subset \cu$ (from which we eventually need to derive the local patches $\ck_u$), and obtain the (EC*) convolution operator~$f_w$, without the need of fully determining $\Upsilon = \langle \cu \rangle$. We can choose between three levels of constraints, which are inherited from~$\cu$ and~$\cd_0$:
% \begin{enumerate}
%   \item $\Upsilon$ is unconstrained, then $f_w$ is an (EC*) $\varphi$-convolution right-operator that is equivariant to $\Upsilon$, but the converse doesn't hold,
%   \item $\Upsilon$ is domain-symmetric, then $f_w$ is an (EC*) $\varphi$-convolution right operator and the equivariance characterization holds,
%   \item $\Upsilon$ is abelian, then $f_w$ is an (EC*) $\M$-convolution operator, the equivariance characterization holds, and we don't need to compute the local patches $\ck_u$.
% \end{enumerate}

% In the next chapter, we will encounter examples of convolutions defined on graphs, that can be expressed under group representations or under groupoid representations.

% \todo{Limitation: border effect because $g$ goes in implies $g$ goes out}