\todo{}

\paragraph{TODO}
Use groupoids and $\Psi^*$.
Show it is useless.
But (ER) from $\Psi^*$, not $\Phi^*$


\subsection{Construction of path convolutions}

\todo{work in progress}
% %% OK must redo and sum over Psi X V


% The groupoid convolution may be expressed as the restriction of the group convolution to the composition rule $\cd$, \ie:

% \begin{definition}\textbf{Groupoid convolution}\\
% Let a groupoid $\Upsilon$, the groupoid convolution between two signals $s_1,s_2 \in \cs(\Upsilon)$ is defined as:
% \begin{gather*}
% \forall h \in \Upsilon, (s_1 \ast s_2) [h] = \displaystyle\sum_{\substack{f \in \Upsilon ~\st\\ (f^{-1}, h) \in \cd}} s_1[f] \h{2} s_2[f^{-1}h]
% \end{gather*}
% \end{definition}

% \paragraph{Symmetrical expression}
% Note that a convenient change of variables gives the following symmetrical expression:
% \begin{gather*}
% \forall h \in \Upsilon, (s_1 \ast s_2) [h] = \displaystyle\sum_{\substack{(f,g) \in \cd\\ \st gf=h}} s_1[f] \h{2} s_2[g]
% \end{gather*}

% Let a graph $\gve$. As~$V$ is not a groupoid, we cannot apply the groupoid convolution on~$\cs(V)$. So we need to find a meaningful groupoid.

% \subsection{A first attempt}

% First, let's consider the obvious generalization of the previous construction with the following groupoid:

% \begin{definition}\textbf{Partial transformations groupoid}\\
% The \emph{partial transformations groupoid} $\Psi^*(V)$, is the set of invertible partial transformations, equipped with the functional composition law with domain~$\cd$ such that
% \begin{gather*}
% (g,h) \in \cd \Leftrightarrow \cd_{gh} = h(\cd_h) \cap \cd_g \neq \emptyset
% \end{gather*}
% \end{definition}

% Let's suppose we have extended rigorously every notions from the previous construction. Then, because of the associativity, if $g \in \Psi_{\EC}^*(V)$, then, any $v \in V, g(u) = v$ would be constrained to allow to be acted by every $h$ \st $(h,g) \in \cd$, which fails at unbounding the supporting set. 

%\subsection{Edge-restricted convolutions}

In what follows, our aim is to restrict the construction of the convolution we made on groups to a composition rule~$\cd$ dicted by the edge set~$E$.

\begin{definition}\textbf{Edge-restricted transformation}\\
Let a graph $\gve$. An \emph{edge-restricted} (ER) transformation $g_{\ER}$ is the restriction of a transformation $g \in \Phi^*(V)$ to the domain
\begin{gather*}
\cd_g = \{v \in V, v \sim g(v)\}
\end{gather*}
(or $v \rightarrow g(v)$ in the case of a digraph $\vgve$).
\end{definition}

The set of (ER) transformations, and injective (ER) transformations are denoted $\Phi_{\ER}(G)$, and $\Phi_{\ER}^*(G)$ respectively.

\paragraph{Extension to $\cs(V)$ by (ER) group action}
As a restriction of the symmetric group of $V$, $\Phi_{\ER}^*(G)$ can also move signals of $\cs(V)$. Let $g \in \Phi_{\ER}^*(G)$. Its extension is done in two steps:
\begin{enumerate}
  \item $g$ is extended to $V^0 = V \cup \{0_V\}$ as $g(v) = 0_V \Leftrightarrow v \notin \cd_g$.
  \item Under the convention $\forall s \in \cs(V), s[0_V] = 0_\bbr$, $g$ is extended via linear extension to $\cs(V)$, and we have
  \begin{gather*}
  \forall s \in \cs(V), \forall v \in V, g(s)[v] = s[g^{-1}(v)]
  \end{gather*}
  similarly to \lemref{lem:ext}.
\end{enumerate}

Note that even though $\Phi_{\ER}^*(G)$ can generate a groupoid, we won't use it to define the convolution of the graph~$G$ as its composition rule would be too narrow. Instead, we form another groupoid from the actions of a semigroup of (ER) transformations.

\begin{definition}\textbf{Path Groupoid}\\
Let $\cu \subset \Phi^*(V)$. The \emph{path groupoid} generated from $\cuer$, denoted $\cuer \ltimes V$, with composition rule $\cd$, is the groupoid obtained inductively as:
\begin{enumerate}
  \item $\cuer \ltimes_0 V = \{(g,v) \in \cuer \times V, v \in \cd_g \} \subset \cu \ltimes V$
  \item $((g_1,v_1) \cdots (g_n,v_n) , (h_1,u_1) \cdots (h_m,u_m)) \in \cd \Leftrightarrow g_n(v_n) = u_1$
  \item $(g_1,v_1) \cdots (g_n,v_n) \in \cuer \ltimes V \Rightarrow (g_n^{-1}, g_n(v_n)) \cdots (g_1^{-1}, g_1(v_1)) \in \cuer \ltimes V$
\end{enumerate}
\end{definition}

\begin{remark}
This groupoid construction is inspired from the field of operator algebra where partial action groupoids have been extensively studied, \eg \cite{nica1994groupoid,exel1998partial,li2016partial}.
\end{remark}

\begin{definition}\textbf{Source, target, and path maps}\\
Let a path groupoid $\cuer \ltimes V$. We define on it the \emph{source map}~$\alpha$, the \emph{target map}~$\beta$, and the \emph{path map}~$\gamma$, such that:
\begin{gather*}
\begin{cases}
  \alpha: (g_1,v_1) \cdots (g_n,v_n) \mapsto v_1 \in V\\
  \beta: (g_1,v_1) \cdots (g_n,v_n) \mapsto g_n(v_n) \in V\\
  \gamma: (g_1,v_1) \cdots (g_n,v_n) \mapsto g_ng_{n-1}\ldots g_1 \in \Phi^*(V)
\end{cases}
\end{gather*}
Note that $(p,q) \in \cd \Leftrightarrow \beta(p) = \alpha(q)$
\end{definition}

% \begin{remark}Note that $k \in \cuer \ltimes V$ doesn't imply $(\alpha(k), \beta(k)) \in \cuer \ltimes V$.
% \end{remark}

\begin{remark}Note that the path groupoid 
%is a Lie groupoid~\citep{wiki:lie}, and that it 
can also be obtained by discrete derivation of the partial transformation groupoid (\eg $p \in \cuer \ltimes V$ can be seen as derivative of $\gamma(p)$ \wrt $\alpha(p)$).% As such it can be seen as a foliation of the graph domain.%, and so as a sort of Lie groupoid for graph domains instead of manifolds.
\end{remark}

\begin{lemma}\textbf{Useful properties of $\alpha$, $\beta$, and $\gamma$}%\\
\begin{enumerate}
  \item $(p,q) \in \cd \Leftrightarrow \beta(p) = \alpha(q)$.
  \item $\gamma$ is a groupoid partial action. Denote $p(v) := \gamma(p)(v)$.% for $p \in \cuer$ and $v \in V$
  \item $\beta$ is an equivariant map for the action $\gamma$ on $V$.
\end{enumerate}
\end{lemma}

We can now define the all-paths convolution based on a path groupoid, as an equivalent of a $\varphi$-convolution where $\beta$ takes the role of $\varphi$.

\begin{definition}\textbf{All-paths convolution}\\
The \emph{all-paths convolution} $\ast$ is defined for signals $s_1, s_2 \in \cs(V)$, or with a mixed expression $\ast_{\M}$ for signals $\widetilde{s_1} \in \cuer \ltimes V$ and $s_2 \in \cs(V)$:
\begin{enumerate}[label=(\roman*)]
\item $\forall u \in V, (s_1 \ast s_2) [u] = \displaystyle\sum_{\substack{p \in \cu \ltimes V\\ \st \beta(p)=u}} s_1[\beta(\gamma(p))] \h{2} s_2[\alpha(p)]$
\item $\forall u \in V, (\widetilde{s_1} \ast_{\M} s_2) [u] = \displaystyle\sum_{\substack{p \in \cu \ltimes V\\ \st \beta(p)=u}} \widetilde{s_1}[\gamma(p)] \h{2} s_2[\alpha(p)]$
\end{enumerate}
\end{definition}


