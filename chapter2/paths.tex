\subsection{Construction of path convolutions}

To answer the limitation of partial convolutions, given $g \in \langle \cu \rangle$ where $\cu \subset \Psi_{\EC}^*(G)$, the idea is to proceed with a foliation of $g$ into pieces, each corresponding to an edge $e \in E$, and together generating another groupoid with a different associativity law, as follows.

\begin{definition}\textbf{Path groupoid}\\
Let $\cu \subset \Psi_{\EC}^*(G)$. The \emph{path groupoid} generated from $\cu$, denoted $\cu \ltimes G$, with composition rule $\cdl$, is the groupoid obtained inductively as:
\begin{enumerate}
  \item $\cu \ltimes_1 G = \{(g,v) \in \cu \times V, v \in \cd_g \} \subset \cu \ltimes G$
  \item $((g_n,v_n) \cdots (g_1,v_1) , (h_m,u_m) \cdots (h_1,u_1)) \in \cdl \Leftrightarrow h_m(u_m) = v_1$
  \item $(g_n,v_n) \cdots (g_1,v_1) \in \cu \ltimes G \Rightarrow (g_1^{-1}, g_1(v_1)) \cdots (g_n^{-1}, g_n(v_n)) \in \cu \ltimes G$
\end{enumerate}
Call path its objects. Given a length $l \in \bbn^*$, denote $\cu \ltimes_l G$ the subset composed of the paths that are the composition of exactly~$l$ paths of $\cu \ltimes_1 G$.
\end{definition}

\begin{remark}
This groupoid construction is inspired from the field of operator algebra where partial action groupoids have been extensively studied, \eg \cite{nica1994groupoid,exel1998partial,li2016partial}.
\end{remark}

Such groupoids usually come equipped with source and target maps. We also define the path map.

\begin{definition}\textbf{Source, target and path maps}\\
Let a path groupoid $\cu \ltimes G$. We define on it the \emph{source map}~$\alpha$ the \emph{target map}~$\beta$ and the \emph{path map}~$\gamma$ as:
\begin{gather*}
\begin{cases}
  \alpha: (g_n,v_n) \cdots (g_1,v_1) \mapsto v_1 \in V\\
  \beta: (g_n,v_n) \cdots (g_1,v_1) \mapsto g_n(v_n) \in V\\
  \gamma: (g_n,v_n) \cdots (g_1,v_1) \mapsto g_ng_{n-1}\ldots g_1 \in \Psi^*(V^0)\\
  %\lambda: (g_1,v_1) \cdots (g_n,v_n) \mapsto n \in \bbn^*
\end{cases}
\end{gather*}
\end{definition}

\begin{remark}Note that the path groupoid can also be obtained by derivation of the partial transformation groupoid (\ie $p \in \cu \ltimes G$ can be seen as a derivative of $\gamma(p)$ \wrt $\alpha(p)$), and can thus be seen as the local structure of it.
\end{remark}

\begin{lemma}\h{0}\\
Note the following properties:
\begin{enumerate}
  \item $(p,q) \in \cdl \Leftrightarrow \beta(p) = \alpha(q)$
  \item $\alpha(p) = \beta(p^{-1})$
  \item $\gamma$ is a groupoid partial action. We will denote $\gamma_p$ instead of $\gamma(p)$. \label{enum:3}
  %\item $\forall (p,q) \in \cdl, \gamma_{pq} = \gamma_{p}\gamma_{q}$
  %\item $\beta$ is a partial equivariant map for the groupoid partial action $\gamma$ on $V$.
\end{enumerate}
\end{lemma}

\begin{remark}
Note that this time we won't use the notation $p(v)$ for $\gamma_p(v)$ in order to better differentiate between the composition laws in $\langle \cu \rangle$ and $\cu \ltimes G$.
\end{remark}

One of the key object of our contruction is the use of $\varphi$-equivalence in order to transform a sum over a group(oid) of (partial) transformations, into a sum over the vertex set. With the current notion of path groupoid, searching for something similar amounts to searching for a graph traversal.

\begin{definition}\textbf{Traversal set}\\
Let a graph $\gve$ that is connected. A \emph{traversal set} is a pair $(\cu, \ct)$ of (EC) partial transformations subsets $\subset \Psi_{\EC}^*(G)$, such that
\begin{enumerate}
  \item An edge can only correspond to a unique $g\in \cu$,\\
    \ie $\forall g,h \in \cu : \exists v \in V, g(v) = h(v) \Rightarrow g=h$
  \item The (EC) partial transformations of $\ct$ are restrictions of those of $\cu$,\\
    \ie $\forall g \in \cu, \exists! h \in \ct, \begin{cases}\cd_h \subset \cd_g\\ \forall v \in \cd_h, h(v) = g(v)\end{cases}$, \\
    (equivalently, $\ct \ltimes G$ is a subgroupoid of $\cu \ltimes G$)
  % \item The edges corresponding to $\ct$ are included in those of $\cu$,\\
  %   \ie $\ct \ltimes_1 G \subset \cu \ltimes_1 G$
  \item The subgraph $G_{\ct} = \langle V, \ct \ltimes_1 G \rangle$ is a covering tree of $G$.
\end{enumerate}
We denote $(\cu, \ct) \in \tree(G)$, and denote by $r$ the root of $G_{\ct}$.
\end{definition}

\begin{remark}The assumption that the graph $G$ is connected has been made. This doesn't lose generality as the construction can be replicated to each connected component in the general case.
\end{remark}

A traversal set $(\cu, \ct)$ defines a $\varphi$-equivalence between the $\alpha$-fiber of the root~$r$ and the vertex set~$V$ as follows.

\begin{lemma}\textbf{Path $\varphi$-Equivalence}\\
Let $(\cu, \ct) \in \tree(G)$. Given $v \in V$, there exists a unique $p_v \in \ct \ltimes G$ such that $\alpha(p_v) = r$ and $\beta(p_v) = v$. Define $\varphi: p_v \mapsto v$. Then $\varphi: \alpha^{-1}_{\ct \ltimes G}\{r\} \rightarrow V$ is a bijective partial equivariant map.
\end{lemma}
\begin{proof}
Bijectivity is a consequence of the covering tree structure of $\ct$.
Equivariance because $\gamma_{p_v}(u) = \gamma_{p_v}\gamma_{p_u}(r) = \gamma_{p_vp_u}(r) = \varphi(p_vp_u)$.
\end{proof}

We can now define the convolution that is based on a path groupoid.

\begin{definition}\textbf{Path convolution}\\
Let $(\cu, \ct) \in \tree(G)$. The \emph{path convolution} is the partial convolution based on the path subgroupoid $\ct \ltimes G$, which uses the groupoid partial action $\gamma := \gamma^{\cu \ltimes G}$ of the embedding groupoid $\cu \ltimes G$.
\begin{enumerate}[label=(\roman*)]
  \item In what follows are the three expressions of the path $\varphi$-convolution for signals $s_1, s_2 \in \cs(V)$, and $u \in V$:
\begin{align*}
(s \ast_\varphi w) & = \displaystyle\sum_{v \in V} s[v] \h{2} \gamma_{p_v}(w)\\
                   & = \displaystyle\sum_{\substack{p \in \ct \ltimes G\\ \st \alpha(p) = r}} s[\varphi(p)] \h{2} \gamma_{p}(w)\\
(s \ast_\varphi w)[u] & = \displaystyle\sum_{\substack{(a,b) \in V\\ \st \gamma_{p_a}(b)=u}} s[a] \h{2} w[b]
\end{align*}
  \item The mixed formulations with $w \in \cs(\ct \ltimes G)$ are:
\begin{align*}
(w \ast_{\M} s) & = \displaystyle\sum_{\substack{p \in \ct \ltimes G\\ \st \alpha(p) = r}} w[p] \h{2} \gamma_{p}(s)\\
(w \ast_{\M} s)[u] & = \displaystyle\sum_{\substack{(p,v) \in \ct \ltimes G \times V\\ \st \alpha(p) = r\\ \st \gamma_{p}(v)=u}} w[p] \h{2} s[v]
\end{align*}
\end{enumerate}
\end{definition}

\begin{remark}The role of $\ct$ is to provide a $\varphi$-equivalence. The role of $\cu$ is to extend every partial transformation $\gamma^{\ct \ltimes G}_g$ to the domain of its unrestricted counterpart $\gamma^{\cu \ltimes G}_g$.
\end{remark}

\propref{prop:equiP} also holds for path groupoids, except that the domain-symmetric condition of 2.(i) is not needed.

\begin{proposition}\textbf{Characterization by equivariance to $\cu \ltimes G$'s action}\\
Let $(\cu, \ct) \in \tree(G)$.
\begin{enumerate}[label=(\roman*)]
\item The class of linear transformations of $\cs(V)$ that are equivariant to the path actions of $\cu \ltimes G$ is exactly the path $\varphi$-convolution right-operators;
\item in the abelian case, they are also exactly the $\M$-convolution left-operators.
\end{enumerate}
\end{proposition}

\begin{proof}
Instead of the domain-symmetric condition that was used in the proof of the converse of \propref{prop:equiP} (2.(i)), we use the fact that any vertex can be reached with an action from the root of the covering tree of the traversal set. Indeed, given $v \in V$, as we have $\gamma_{p_v}(r)=v$, then $\gamma_{p_v}(\delta_r) = \delta_v$. Therefore, by developping a linear transformation $f(s)$ on the dirac family, and commuting $f$ with $\gamma_{p_v}$, we obtain that $f(s) = s \ast_\varphi w$, where $w = f(\delta_r)$. The rest of the proof is similar to that of \propref{prop:equiP}.
\end{proof}

\paragraph{(EC*) Path convolution operators}
The counterparts of strictly edge-constrained (EC*) convolution operators for path convolutions, are indeed path convolution operators obtained with bounded supporting set $\cn \subset \ct \ltimes_1 G$ which any graph can admit. As shown by this section, all we need to construct one is a traversal set of partial transformations $(\cu, \ct)$.
