\todo{}

\paragraph{TODO}
Use groupoids and $\Psi^*$.
Show it is useless.
But (ER) from $\Psi^*$, not $\Phi^*$


\section{Paths}

\todo{work in progress}
% %% OK must redo and sum over Psi X V

% % maybe present miniplan
\subsection{Motivation}

One possible limitation coming from searching for Cayley subgraphs is that they are order-regular \ie the in- and the out-degree $d = |\cu|$ of each vertex is the same. That is, for a general graph $G$, the size of the weight kernel $w$ of an (EC*) convolution operator $f_w$ supported on $\cu$ is bounded by $d$, which in turn is bounded by twice the mininimal degree of $G$ (twice because $G$ is undirected and $\cu$ can contain every inverse).

There are a lot of possible strategies to overcome this limitation. For example:
\begin{enumerate}
  \item connecting each vertex with its $k$-hop neighbors, with $k > 1$,
  \item artificially creating new connections for less connected vertices,
  \item allowing the supporting set $\cn$ to exceed $\cu$ \ie dropping * in (EC*).
\end{enumerate}

These strategies require to concede that the topological structure supported by $G$ is not the best one to support an (EC*) convolution on it, which breeds the following question:
\begin{itemize}
  \item What can we relax in the previous (EC*) contruction in order to unbound the supporting set, and still preserve the equivariance characterization?
\end{itemize}

The latter constraint is a consequence that every vertex of the Cayley subgraph $\vec{G}$ must be composable with every generator from $\cu$. Therefore, an answer consists in considering groupoids~\citep{brandt1927verallgemeinerung} instead of groups. Roughly speaking, a groupoid is almost a group except that its composition law needs not be defined everywhere. \cite{weinstein1996groupoids}, unveiled the benefits to base convolutions on groupoids instead of groups in order to exploit partial symmetries.

\subsection{Groupoids}

\begin{definition}\textbf{Groupoid}\\
A \emph{groupoid} $\Upsilon$ is a set equipped with a partial composition law with domain $\cd \subset \Upsilon \times \Upsilon$, called \emph{composition rule}, that is
\begin{enumerate}
	\item\label{enum:g1} closed into $\Upsilon$ \ie $\forall (g, h) \in \cd, gh \in \Upsilon$
	\item\label{enum:g2} associative \ie
    $\forall f,g,h \in \Upsilon,
      \begin{cases}
        (f, g), (g, h) \in \cd \Leftrightarrow (fg,h), (f, gh) \in \cd\\
        (f, g), (fg, h) \in \cd \Leftrightarrow (g,h), (f, gh) \in \cd\\
        \text{when defined, } (fg)h = f(gh)
      \end{cases}$ \label{enum:2}
	\item\label{enum:g3} invertible \ie
		$\forall g \in \Upsilon, \exists! g^{-1} \in \Upsilon \quad\st
			\begin{cases}
				(g,g^{-1}), (g^{-1},g)  \in \cd\\
				(g,h) \in \cd \Rightarrow g^{-1}gh = h\\
				(h,g) \in \cd \Rightarrow hgg^{-1} = h
			\end{cases}$
\end{enumerate}
Optionally, it can be \emph{domain-symmetric} \ie $(g,h) \in \cd \Leftrightarrow (h,g) \in \cd$, and \emph{abelian} \ie domain-symmetric with $gh = hg$.
\end{definition}

\begin{remark}
Note that left and right inverses are necessarily equal (because $(gg^{-1})g=g(g^{-1}g)$). Also note we can define a right identity element $e^r_g = g^{-1}g$, and a left one $e^l_g = gg^{-1}$, but they are not necessarily equal and depend on $g$.
\end{remark}

The groupoid convolution may be expressed as the restriction of the group convolution to the composition rule $\cd$, \ie:

\begin{definition}\textbf{Groupoid convolution}\\
Let a groupoid $\Upsilon$, the groupoid convolution between two signals $s_1,s_2 \in \cs(\Upsilon)$ is defined as:
\begin{gather*}
\forall h \in \Upsilon, (s_1 \ast s_2) [h] = \displaystyle\sum_{\substack{f \in \Upsilon ~\st\\ (f^{-1}, h) \in \cd}} s_1[f] \h{2} s_2[f^{-1}h]
\end{gather*}
\end{definition}

\paragraph{Symmetrical expression}
Note that a convenient change of index gives the following symmetrical expression:
\begin{gather*}
\forall h \in \Upsilon, (s_1 \ast s_2) [h] = \displaystyle\sum_{\substack{(f,g) \in \cd\\ \st gf=h}} s_1[f] \h{2} s_2[g]
\end{gather*}

Let a graph $\gve$. As~$V$ is not a groupoid, we cannot apply the groupoid convolution on~$\cs(V)$. So we need to find a meaningful groupoid.

% \subsection{A first attempt}

% First, let's consider the obvious generalization of the previous construction with the following groupoid:

% \begin{definition}\textbf{Partial transformations groupoid}\\
% The \emph{partial transformations groupoid} $\Psi^*(V)$, is the set of invertible partial transformations, equipped with the functional composition law with domain~$\cd$ such that
% \begin{gather*}
% (g,h) \in \cd \Leftrightarrow \cd_{gh} = h(\cd_h) \cap \cd_g \neq \emptyset
% \end{gather*}
% \end{definition}

% Let's suppose we have extended rigorously every notions from the previous construction. Then, because of the associativity, if $g \in \Psi_{\EC}^*(V)$, then, any $v \in V, g(u) = v$ would be constrained to allow to be acted by every $h$ \st $(h,g) \in \cd$, which fails at unbounding the supporting set. 

\subsection{Edge-restricted convolutions}

In what follows, our aim is to restrict the construction of the convolution we made on groups to a composition rule~$\cd$ dicted by the edge set~$E$.

\begin{definition}\textbf{Edge-restricted transformation}\\
Let a graph $\gve$. An \emph{edge-restricted} (ER) transformation $g_{\ER}$ is the restriction of a transformation $g \in \Phi^*(V)$ to the domain
\begin{gather*}
\cd_g = \{v \in V, v \sim g(v)\}
\end{gather*}
(or $v \rightarrow g(v)$ in the case of a digraph $\vgve$).
\end{definition}

The set of (ER) transformations, and injective (ER) transformations are denoted $\Phi_{\ER}(G)$, and $\Phi_{\ER}^*(G)$ respectively.

\paragraph{Extension to $\cs(V)$ by (ER) group action}
As a restriction of the symmetric group of $V$, $\Phi_{\ER}^*(G)$ can also move signals of $\cs(V)$. Let $g \in \Phi_{\ER}^*(G)$. Its extension is done in two steps:
\begin{enumerate}
  \item $g$ is extended to $V^0 = V \cup \{0_V\}$ as $g(v) = 0_V \Leftrightarrow v \notin \cd_g$.
  \item Under the convention $\forall s \in \cs(V), s[0_V] = 0_\bbr$, $g$ is extended via linear extension to $\cs(V)$, and we have
  \begin{gather*}
  \forall s \in \cs(V), \forall v \in V, g(s)[v] = s[g^{-1}(v)]
  \end{gather*}
  similarly to \lemref{lem:ext}.
\end{enumerate}

Note that even though $\Phi_{\ER}^*(G)$ can generate a groupoid, we won't use it to define the convolution of the graph~$G$ as its composition rule would be too narrow. Instead, we form another groupoid from the actions of a semigroup of (ER) transformations.

\begin{definition}\textbf{Edge-restricted (EC) Groupoid}\\
Let $\cu \subset \Phi^*(V)$. The \emph{(ER) groupoid} generated from $\cuer$, denoted $\cuer \ltimes V$, with composition rule $\cd$, is the groupoid obtained inductively as:
\begin{enumerate}
  \item $\cuer \ltimes_0 V = \{(g,v) \in \cuer \times V, v \in \cd_g \} \subset \cu \ltimes V$
  \item $((g_1,v_1) \cdots (g_n,v_n) , (h_1,u_1) \cdots (h_m,u_m)) \in \cd \Leftrightarrow g_n(v_n) = u_1$
  \item $(g_1,v_1) \cdots (g_n,v_n) \in \cuer \ltimes V \Rightarrow (g_n^{-1}, g_n(v_n)) \cdots (g_1^{-1}, g_1(v_1)) \in \cuer \ltimes V$
\end{enumerate}
\end{definition}

\begin{remark}
This groupoid construction is inspired from the field of operator algebra where partial action groupoids are common, \eg \cite{nica1994groupoid,exel1998partial,li2016partial}.
\end{remark}

\begin{definition}\textbf{Source, target, and path}\\
Let an (ER) groupoid $\cuer \ltimes V$. We define on it the \emph{path map}~$\alpha$, the \emph{source map}~$\beta$, and the \emph{target map}~$\gamma$, such that:
\begin{gather*}
\begin{cases}
  \alpha: (g_1,v_1) \cdots (g_n,v_n) \mapsto g_1g_2\ldots g_n \in \Phi^*(V)\\
  \beta: (g_1,v_1) \cdots (g_n,v_n) \mapsto v_1 \in V\\
  \gamma: (g_1,v_1) \cdots (g_n,v_n) \mapsto v_n \in V
\end{cases}
\end{gather*}
\end{definition}

\begin{remark}Note that $k \in \cuer \ltimes V$ doesn't imply $(\alpha(k), \beta(k)) \in \cuer \ltimes V$.
\end{remark}

We can now define the (ER) convolution based on an (ER) groupoid.

\begin{definition}\textbf{Edge-restricted (EC) Convolution}\\
\begin{enumerate}[label=(\roman*)]
\item $\forall u \in V, (s_1 \ast_\varphi s_2) [u] = \displaystyle\sum_{\substack{k \in \cu \ltimes V\\ \st \gamma(k)=u}} s_1[\varphi(\alpha(k))] \h{2} s_2[\beta(k)]$
\item $\forall u \in V, (s_1 \ast_{\M} s_2) [u] = \displaystyle\sum_{\substack{k \in \cu \ltimes V\\ \st \gamma(k)=u}} s_1[\alpha(k)] \h{2} s_2[\beta(k)]$
\end{enumerate}
\end{definition}

