\section*{Chapter overview}
\addcontentsline{toc}{section}{Chapter overview}

\h{20}
Defining a convolution of signals over graph domains is a challenging problem. If the graph is not a grid graph, there exists no natural extension of the euclidean convolution.

\h{20}
In \secref{sec:2.1}, we analyze the reasons why the euclidean convolution operator is useful in deep learning, and give a characterization. Then we will search for domains onto which a convolution with these properties can be naturally obtained.

\h{20}
This will lead us to put our interest on representation theory and convolutions defined on groups in \secref{sec:2.2}. As the euclidean convolution is just a particular case of the group convolution, it makes perfect sense to steer our construction in this direction. Hence, we will aim at transferring its representation to the vertex domain.% At first, we will do this construction agnostically of the edge set. 

\h{20}
Then, in \secref{sec:edges}, we will introduce the role of the edge set and see how it should influence it. This will provide us with some particular classes of graphs for which we will obtain a natural construction with the wanted characteristics that we exposed in the first place.

\h{20}
Finally, we will relax some aspect of the construction to adapt it to general graphs in \secref{sec:2.4}. The obtained construction is a set of general expressions that describes convolutions on graph domains and preserves some key properties.

\h{20}
We summarize our constructions in a conclusive \secref{sec:2.5}.