Defining a convolution of signals over graph domains is a challenging problem. Obviously, if the graph is not a grid graph there exists no natural definition.

We first analyse the reasons why the classical convolution operator is useful in deep learning, and give a characterization. Then we will search for domains onto which a convolution with this characteristics can be naturally obtained. This will lead us to put our interest on representation theory and convolutions defined on algebraic structures, such as groups, in order to transfer its construction on vertex domains of graphs, agnostically of the edge set. Then, we will introduce the role of the edge sets and how it should influence the contruction. This will provide us with some particular classes of graphs for which we will obtain a natural construction with respect to the wanted characteristics. Finally, we study how we can loosen them to adapt the construction to more irregular domains.