\section*{Introduction}

Defining a convolution of signals over graph domains is a challenging problem. Obviously, if the graph is not a grid graph there exists no natural definition.

We first analyse the reasons why the euclidean convolution operator is useful in deep learning, and give a characterization. Then we will search for domains onto which a convolution with these properties can be naturally obtained. This will lead us to put our interest on representation theory and convolutions defined on groups. As the euclidean convolution is just a particular case of the group convolution, it makes perfect sense to steer our construction in this direction. Hence, we will aim at transferring its representation on the vertex domain. First we will do this construction agnostically of the edge set. Then, we will introduce the role of the edge set and see how it should influence it. This will provide us with some particular classes of graphs for which we will obtain a natural construction with the wanted characteristics that we exposed in the first place. Finally, we can relax some aspect of the construction to adapt it to graphs that are not order-regular. The obtained construction is a set of general expressions that describes convolutions on graph domains, which preserve some key properties.