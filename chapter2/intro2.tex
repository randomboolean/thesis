\section*{Chapter overview}
\addcontentsline{toc}{section}{Chapter overview}

Defining a convolution of signals over graph domains is a challenging problem. If the graph is not a grid graph, there exists no natural extension of the Euclidean convolution. In \secref{sec:2.1}, we analyze the reasons why the Euclidean convolution operator is useful in deep learning. In particular, we recall a classical characterization: that convolution operators are exactly the class of linear functions that are equivariant to translations (\thref{prop:equi}). Therefore, we then search for domains onto which a convolution with these properties can be naturally obtained. This leads us to put our interest on representation theory and convolutions defined on groups. Since the Euclidean convolution is just a particular case of the group convolution, it makes perfect sense to steer our construction in this direction. In \secref{sec:2.2}, we seek to transfer the definition of the group convolution onto the vertex domain, through its symmetric group. To obtain the wanted characterization, we will see that we need to base our construction on actions of groups, rather than on their elements. We manage to obtain it should we fix an equivariant mapping between the active group and the vertex domain (\thref{prop:equiG}). Then, we propose a mixed formulation of this convolution as a binary operation between a signal defined on the vertex domain and a signal defined on the corresponding group, for which we demonstrate that the characterization also holds under abelianity (\corref{cor:equiM}). In \secref{sec:edges}, we introduce the role of the edge set and see how it influences the construction. In particular, we define a notion of edge constraint and a notion of locality preservation. For both, we obtain a characterization of graphs that admit a natural construction of convolutions with this property (\thref{th:cayleychar} and \thref{th:cayleycharLP}). We analyze the notions of locality and weight sharing in this construction, and give a formulation for small kernels. At this point, with the obtained theorems we are able to describe convolutions on any graphs, as convolutions on appropriate subgraphs. Then, in \secref{sec:2.4}, we relax some aspect of the construction to better adapt it to general graphs. We explain why a construction based on groups is less interesting for some graphs, and introduce the notion of groupoid. We extend the previous construction with groupoids of partial transformations, and prove that under a mild condition the characterization by equivariance is preserved (\thref{prop:equiP}). Finally, we extend it another time with another type of groupoid, that we call path groupoids. Path groupoids allow to tackle the more general case, and for them we obtain the characterization should we fix a way to traverse the vertex set using a subset of the edge set (\thref{th:equiU}), but at the price of allowing more degenerated cases. We summarize our constructions in a conclusive \secref{sec:2.5}. The result of this chapter is the obtention of a set of general expressions and theorems that describe convolutions of graph signals.