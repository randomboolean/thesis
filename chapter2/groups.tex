\section{Construction on the vertex set}
\label{sec:2.2}

As \thref{prop:equi} is a complete characterization of convolutions, it can be used to define them \ie convolution operators can be constructed as the set of linear transformations that are equivariant to translations. However, in the general case where $G$ is not a grid graph, translations are not defined, so that construction needs to be generalized beyond translational equivariances.

In mathematics, convolutions are more generally defined for functions defined over a group structure. The classical convolution that is used in deep learning is just a narrow case where the domain group is an Euclidean space. Therefore, constructing a convolution on graphs should start from the more general definition of convolution on groups rather than convolution on Euclidean domains.

Our construction is motivated by the following questions:
\begin{itemize}[nolistsep,noitemsep]
\item Does the equivariance property holds ? Does the characterization from \thref{prop:equi} still holds ?
\item Is it possible to extend the construction on non-group domains, or at least on mixed domains ? (\ie one signal is defined over a set, and the other is defined over its transformations).
\item Can a group domain draw an underlying graph structure ? Is the group convolution naturally defined on this class of graphs ?
\item Can we characterize graphs accepting our construction ?
\end{itemize}

In this section, we first aim at transferring the group convolution onto the vertex set. Then, in \secref{sec:edges}, we will see the implications of considering the edge set in the process.

\subsection{Preliminaries}

We first recall preliminary notions about groups and group convolutions.

\begin{definition}\textbf{Group}\\
A group $\Gamma$ is a set equipped with a closed, associative and invertible composition law that admits a unique left-right identity element $e$.
\end{definition}

When the law is commutative, the group is said to be \emph{abelian}.
The group convolution extends the notion of the classical discrete convolution and is defined as follows.

\begin{definition}\textbf{Group convolution}\\
Let a group $\Gamma$, the \emph{group convolution} between two signals $s_1$ and $s_2 \in \cs(\Gamma)$ is defined as:
\begin{align*}
\forall h \in \Gamma, (s_1 \ast_{\Gamma} s_2) [h] = \displaystyle \sum_{g \in \Gamma} s_1[g] \h{2} s_2[g^{-1}h]
\end{align*}
provided at least one of the signals has finite support if $\Gamma$ is not finite.
\label{def:conv1}
\end{definition}

We call \emph{left multiplication} the transformation $L_g: h \mapsto gh$, and \emph{right multiplication} the transformation $R_g: h \mapsto hg$. Then, thanks to \lemref{lem:extlin}, the group convolution can be rewritten with the functional formulation:
\begin{align}
s_1 \ast_{\Gamma} s_2 & = \displaystyle \sum_{g \in \Gamma} s_1[g] \h{2} L_g(s_2) \label{eq:lm}\\
                      &  = \displaystyle \sum_{g \in \Gamma} s_2[g] \h{2} R_g(s_1) \label{eq:rm}
\end{align}
which would then be commutative if, and only if, $\Gamma$ were abelian.

The group convolution preserves the characterization from \thref{prop:equi}.

\begin{theorem}\textbf{Characterization of group convolution operators}\\
Let a group $\Gamma$, let $f \in \cl(\cs(\Gamma))$,
\begin{enumerate}[nolistsep,noitemsep,label=(\roman*)]
  \item $f$ is a group convolution right operator $\Leftrightarrow$ $f$ is equivariant to left multiplications, \label{enum:p}
  \item $f$ is a group convolution left operator $\Leftrightarrow$ $f$ is equivariant to right multiplications, \label{enum:pp}
  \item $f$ is a group convolution commutative operator $\Leftrightarrow$ $f$ is equivariant to multiplications. \label{enum:ppp}%$\Leftrightarrow$ $\Gamma$ is abelian \label{enum:ppp}
\end{enumerate}
\label{th:cgco}
\end{theorem}
\begin{proof}
The proof is similar than the proof of \thref{prop:equi}.
\begin{itemize}
  \item[$\Rightarrow$]: Given $w \in \cs(\Gamma)$, let $f_R = .\ast_\Gamma w$, $f_L = w \ast_\Gamma .$, $s \in \cs(\Gamma)$, and $p \in \Gamma$. We have:
  \begin{align*}
  (L_p \circ f_R) (s) & = L_p( s \ast_\Gamma w )                                         & (R_p \circ f_L) (s) & = R_p( w \ast_\Gamma s )\\
                & = L_p\left(\displaystyle \sum_{g \in \Gamma} s[g] \h{2} L_g(w)\right)       & & = R_p\left(\displaystyle \sum_{g \in \Gamma} s[g] \h{2} R_g(w)\right)\\
                & = \displaystyle \sum_{g \in \Gamma} s[g] \h{2} (L_p \circ L_g)(w)           & & = \displaystyle \sum_{g \in \Gamma} s[g] \h{2} (R_p \circ R_g)(w)\\
                & = \displaystyle \sum_{pg \in \Gamma} s[p^{-1}pg] \h{2} L_{pg}(w)            & & = \displaystyle \sum_{gp \in \Gamma} s[gpp^{-1}] \h{2} R_{gp}(w)\\          
                & = \displaystyle \sum_{g \in \Gamma} L_p(s)[g] \h{2} L_g(w)                  & & = \displaystyle \sum_{g \in \Gamma} R_p(s)[g] \h{2} R_g(w)\\
                & = L_p(s) \ast_\Gamma w                                                      & & = w \ast_\Gamma R_p(s)\\
                & = (f_R \circ L_p) (s)                                                         & & = (f_L \circ R_p) (s)
  \end{align*}
  \item[$\Leftarrow$]: We show the converse only for \ref{enum:p} since it is similar for  \ref{enum:pp}.
  Let $f \in \cl(\cs(\Gamma))$. We suppose $\forall p \in \Gamma, L_p \circ f = f \circ L_p$. First, note that
  \begin{align*}
  \forall h \in \Gamma, L_g(\delta_g)[h] = 1 & \Leftrightarrow \delta_g[g^{-1}h] = 1\\
  & \Leftrightarrow g^{-1}h = g\\
  & \Leftrightarrow h = e\\
  \ie L_g(\delta_g) = \delta_e
  \end{align*}
  Then, we define $w = f(\delta_e)$. Let $s \in \cs(\Gamma)$, we have:
  \begin{align*}
  f(s) & = \sum_{g \in \Gamma} s[g] \h{2} f(\delta_g)\\
       & = \sum_{g \in \Gamma} s[g] \h{2} (f \circ L_g) (\delta_e)\\
       & = \sum_{g \in \Gamma} s[g] \h{2} (L_g \circ f) (\delta_e)\\
       & = \sum_{g \in \Gamma} s[g] \h{2} L_g(w)\\
       & = s \ast_\Gamma w
  \end{align*}
\end{itemize}
\ref{enum:ppp} derives from \ref{enum:p} and \ref{enum:pp}.
\end{proof}

\subsection{Steered construction from groups}

For a graph $\gve$ and a subgroup $\Gamma \subset \Phi^*(V)$ of its invertible transformations, \defref{def:conv1} is applicable for $\cs(\Gamma)$, but not for $\cs(V)$ as $V$ is not a group. Nonetheless, we will try to use the group convolution on $\cs(\Gamma)$ to construct the convolutions on $\cs(V)$. Our goal is to construct a formulation like \eqref{eq:lm} while preserving \thref{th:cgco} at the same time.

For now, let us assume that there is a subgroup $\Gamma \subset \Phi^*(V)$, which we associate through a one-to-one correspondence $\varphi$ with $V$. We denote $\Gamma \overset{\varphi}{\cong} V$ and $g_v \overset{\varphi}\mapsto v$. Then, the linear morphism $\widetilde\varphi$ from $\cs(\Gamma)$ to $\cs(V)$ defined on the Dirac bases by $\widetilde\varphi(\delta_{g_v}) = \delta_{\varphi(g_v)}$ is a linear isomorphism. Hence $\cs(V)$ inherits the same structural properties as $\cs(\Gamma)$. For the sake of notational simplicity, we will use the same symbol $\varphi$ for both $\varphi$ and $\widetilde\varphi$ (as done between $f$ and $L_f$). A commutative diagram between the sets is depicted on \figref{fig:iso}.

\begin{figure}[H]
\centering
\begin{tikzcd}%
    \Gamma \arrow{r}{\varphi}  \arrow{d}[swap]{\cs}  & V \arrow{d}{\cs}\\  
    \cs(\Gamma) \arrow{r}[swap]{\widetilde\varphi}  & \cs(V) 
\end{tikzcd}%
\caption{Commutative diagram between sets}
\label{fig:iso}
\end{figure}

We naturally obtain the following relation, which put in simpler words means that signals on $\cs(\Gamma)$ are mapped to $\cs(V)$ when $\varphi$ is simultaneously applied on both the signal space and its domain. At the same time, this relation is reminiscent of \lemref{lem:extlin}.

\begin{lemma}\textbf{Relation between $\cs(\Gamma)$ and $\cs(V)$}\\
\centerline{$\forall s \in \cs(\Gamma), \forall u \in V, \varphi(s)[u] = s[\varphi^{-1}(u)] = s[g_u]$}
\label{lem:outer}
\end{lemma}

\begin{proof}
\begin{align*}
\forall s \in \cs(\Gamma), \varphi(s) & = \varphi(\displaystyle \sum_{g \in \Gamma} s[g] \h{2} \delta_g)
 = \displaystyle \sum_{g \in \Gamma} s[g] \h{2} \varphi(\delta_g)
 = \displaystyle \sum_{g \in \Gamma} s[g] \h{2} \delta_{\varphi(g)}\\
 & = \displaystyle \sum_{v \in V} s[g_v] \h{2} \delta_v\\
 \text{So }\forall u \in V, \varphi(s) [u] & = s[g_u].
\end{align*}
\end{proof}

With this lemma, we can steer the definition of the group convolution from $\cs(\Gamma)$ to $\cs(V)$ as in the following definition.

\begin{remark}
Note that the following definition is just a step in our construction since we will see that it is not enough to obtain the wanted counterpart of \thref{prop:equi}.
\end{remark}

\begin{definition}\textbf{Vertex-group convolution}\\
Let a subgroup $\Gamma \subset \Phi^*(V)$ such that $\Gamma \overset{\varphi}{\cong} V$.
The vertex-group convolution between two signals $s_1$ and $s_2 \in \cs(V)$ is defined as:
\begin{gather*}
\forall u \in V, (s_1 \ast_{\Gamma} s_2) [u] = \displaystyle \sum_{v \in V} s_1[v] \h{2} s_2[\varphi(g_v^{-1}g_u)]
\end{gather*}
\label{def:conv2}
\end{definition}

We use the same symbol $\ast_\Gamma$ since the group and vertex-group convolution are inherently the same operation but applied to different domains.

\begin{lemma}\textbf{Relation between group and vertex-group convolutions}\\
Let a subgroup $\Gamma \subset \Phi^*(V)$ such that $\Gamma \overset{\varphi}{\cong} V$,
\begin{gather*}
\forall s_1, s_2 \in \cs(\Gamma), \forall u \in V,
(\varphi(s_1) \ast_{\Gamma} \varphi(s_2))[u] = (s_1 \ast_{\Gamma} s_2)[g_u]
\end{gather*}
\label{lem:rel12}
\end{lemma}
\begin{proof}
Using \lemref{lem:outer},
\begin{align*}
(\varphi(s_1) \ast_{\Gamma} \varphi(s_2))[u] & = \displaystyle \sum_{v \in V} \varphi(s_1)[v] \h{2} \varphi(s_2)[\varphi(g_v^{-1}g_u)]\\
 & = \displaystyle \sum_{v \in V} s_1[g_v] \h{2} s_2[g_v^{-1}g_u]\\
 & = \displaystyle \sum_{g \in \Gamma} s_1[g] \h{2} s_2[g^{-1}g_u]\\
 & = (s_1 \ast_{\Gamma} s_2)[g_u]
\end{align*}
\end{proof}

Let $f$ be a convolution right operator on $\cs(V)$, \st $f = . \ast_\Gamma w$, and $\widetilde f$ be its counterpart on $\cs(\Gamma)$, \ie $\widetilde f = . \ast_\Gamma \varphi^{-1} (w)$. Then, we have
\begin{align}
\forall s \in \cs(V), f(s) [u] &= \widetilde f (\varphi^{-1}(s)) [g_u] \tag{\lemref{lem:rel12}}\\
                               &= \varphi(\widetilde f (\varphi^{-1}(s))) [u] \tag{\lemref{lem:outer}}\\
\ie f = \varphi \circ \widetilde f \circ \varphi^{-1} \label{eq:ff}
\end{align}

This is also better depicted on a commutative diagram, see \figref{fig:comcom}.
\begin{figure}[H]
\centering
\begin{tikzcd}%
    \cs(\Gamma) \arrow{d}[swap]{\widetilde f} & \cs(V) \arrow{d}{f} \arrow{l}[swap]{\varphi^{-1}}\\
    \cs(\Gamma) \arrow{r}[swap]{\varphi} & \cs(V)
\end{tikzcd}%
\caption{Commutative diagram of \eqref{eq:ff}}
\label{fig:comcom}
\end{figure}

And so, given $p \in \Gamma$, the equivariance of $\widetilde f$ rewrites for $f$:
\begin{align}
\varphi \circ L_p \circ \widetilde f \circ \varphi^{-1} & = \varphi \circ \widetilde f \circ L_p \circ \varphi^{-1}\\
\ie (\varphi \circ L_p \circ \varphi^{-1}) \circ f & = f \circ (\varphi \circ L_p \circ \varphi^{-1}) \label{eq:fpf}
\end{align}
That is, if we paraphrase \thref{prop:equi}, the class of vertex-group convolution operators is the class of linear operators that are equivariant to $\{ \varphi \circ L_p \circ \varphi^{-1}, p \in \Gamma \}$. However, our goal is to obtain equivariance to $\Gamma$. Since $L_p$ is not exactly an object of $\Gamma$, but a realization of a group action, we are going to consider group actions next.

\subsection{Construction under group actions}
\label{sec:groupaction}

We have been previously using the notion of group action implicitly, let us define it here.

\begin{definition}\textbf{Group action}\\
An \emph{action} of a group $\Gamma$ on a set $V$ is a homomorphism $L: \Gamma \to \Phi^*(V)$.
\end{definition}

In particular, $L(\Gamma)$ is a subgroup of the symmetric group $\Phi^*(V)$.

\begin{remark}
Given $g \in \Gamma$, we denote $L_g = L(g)$, or just $g(.)$ when there is no ambiguity. $L_g$ is called the action of $g$ by $L$ on $V$. When a group~$\Gamma$ is in one-to-one correspondence with a set $V$, we will abusively attribute \emph{actions} of the objects of the group to their corresponding objects in the set.
\end{remark}

Hence, note that an group element $g \in \Gamma$ can act on both $\Gamma$ through the left multiplication~$L$ and on~$V$. This ambivalence can be seen on a commutative diagram, see \figref{fig:com}.%, where all arrows except the one labeled with \eqref{eq:P} are always true.

\begin{figure}[H]
\centering
\begin{tikzcd}
    g_u \arrow{r}{L_{g_v}}  \arrow{d}[swap]{\varphi}  & g_vg_u \arrow{d}{\varphi}\\  
    u \arrow[dashrightarrow]{r}{\eqref{eq:P}}[below]{g_v}  & \varphi(g_vg_u)
\end{tikzcd}
\caption{Commutative diagram. All arrows except for the one labeled with \eqref{eq:P} are always true.}
\label{fig:com}
\end{figure}

For \eqref{eq:P} to be true means that $\varphi$ is an equivariant map \ie whether the mapping is done before or after the action of $\Gamma$ has no impact on the result. When such $\varphi$ exists, $\Gamma$ and $V$ are said to be equivalent and we denote $\Gamma \equiv V$.

\begin{definition}\textbf{Equivariant map}\\
A map $\varphi$ from a group $\Gamma$ acting on the destination set $V$ and itself, is said to be an \emph{equivariant map} if
\begin{gather*}
\forall g, h \in \Gamma, g(\varphi(h)) = \varphi(g(h))
\end{gather*}
\label{def:eqmap}
\end{definition}
\begin{remark} For example, if $g$ acts on $\Gamma$ through left multiplication then $g(h) = L_g(h) = gh$.
\end{remark}

Suppose we have $\Gamma \overset{\varphi}{\cong} V$. If we also have that $\Gamma \equiv V$, we are interested to know if then~$\varphi$~exhibits the equivalence and under which auto-action.

\begin{definition}\textbf{$\varphi$-Equivalence}\\
A group $\Gamma$ acting on $V$ such that $\Gamma \overset{\varphi}{\cong} V$, is said to be \emph{(left) $\varphi$-equivalent} if $\varphi$ is a bijective equivariant map under left multiplication \ie:
\begin{gather}
\forall v, u \in V, g_v(u) = \varphi(g_vg_u) \label{eq:P}\\
\ie \forall g \in \Gamma, g(.) = \varphi \circ L_g \circ \varphi^{-1} \label{eq:PP}
\end{gather}
It is said to be \emph{right $\varphi$-equivalent} if $\varphi$ is a bijective equivariant map under right multiplication \ie:
\begin{gather}
\forall v, u \in V, g_v(u) = \varphi(g_ug_v) \label{eq:PPP}\\
\ie \forall g \in \Gamma, g(.) = \varphi \circ R_g \circ \varphi^{-1} \label{eq:PPPP}
\end{gather}
\end{definition}

We denote $\Gamma \overset{\varphi}{\equiv} V$. Unless stated otherwise, we will implicitly consider it is under left multiplication.

\begin{remark}
For example, translations on the grid graph, with $\varphi(t_{i,j}) = (i,j)$, are $\varphi$-equivalent since $t_{i,j}(a,b) = \varphi(t_{i,j} \circ t_{a,b})$. However, with $\varphi(t_{i,j}) = (\text{-}i,\text{-}j)$, they would not be $\varphi$-equivalent since in general $t_{i,j}(a,b) \neq t_{i,j}(\text{-}a,\text{-}b)$.
\end{remark}

We are now equipped to define the vertex-action convolution.

\begin{definition}\textbf{Vertex-action convolution}\\
Let a group $\Gamma$ acting on $V$ such that $\Gamma \overset{\varphi}{\cong} V$.
The vertex-action convolution between two signals $s_1$ and $s_2 \in \cs(V)$ is defined as:
\begin{align}
s_1 \ast_V s_2 & = \displaystyle \sum_{v \in V} s_1[v] \h{2} g_v(s_2)\label{eq:vdom}\\
& = \displaystyle \sum_{g \in \Gamma} s_1[\varphi(g)] \h{2} g(s_2) \label{eq:premix}
\end{align}
\label{def:conv3}
\end{definition}

The two expressions differ on the domain upon which the summation is done. Expression \eqref{eq:vdom} puts the emphasis on each vertex and its corresponding action, whereas expression \eqref{eq:premix} puts the emphasis on the action of~$\Gamma$.

\begin{lemma}\textbf{Relation with vertex-group convolution}\\
There is $\varphi$-equivalence if, and only if, vertex-group and vertex-action convolutions are equal, \ie
$\Gamma \overset{\varphi}{\equiv} V \Leftrightarrow \ast_{\Gamma} = \ast_{V}$
\label{lem:rel23}
\end{lemma}
\begin{proof}
\begin{align}
\forall s_1, s_2 & \in \cs(V),\nonumber\\
& s_1 \ast_{\Gamma} s_2 = s_1 \ast_{V} s_2 \nonumber\\
& \Leftrightarrow \forall u \in V,
\displaystyle \sum_{v \in V} s_1[v] \h{2} s_2[\varphi(g_v^{-1}g_u)] = \displaystyle \sum_{v \in V} s_1[v] \h{2} s_2[g_v^{-1}(u)] \label{eq:free}
\end{align}
Hence, the direct sense is obtained by applying \eqref{eq:P}.

For the converse, given $u, v \in V$, we first realize \eqref{eq:free} for $s_1 := \delta_v$, obtaining $s_2[\varphi(g_v^{-1}g_u)] = s_2[g_v^{-1}(u)]$, which we then realize for a real signal $s_2$ having no two equal entries, obtaining $\varphi(g_v^{-1}g_u) = g_v^{-1}(u)$. From the latter we finally obtain \eqref{eq:P} with the one-to-one correspondence $g_{v'} := g_v^{-1}$.
\end{proof}

\begin{remark}Note that we used the fact that the $\varphi$-equivalence is implicitly left. If it were right, we would have instead $s_1 \ast_{\Gamma} s_2 = s_2 \ast_{V} s_1$.

\end{remark}

We now coin the term $\varphi$-convolution, for when there is $\varphi$-equivalence.

\begin{definition}\textbf{$\varphi$-convolution}\\
The \emph{$\varphi$-convolution} is defined as the vertex-action convolution such that $\Gamma \overset{\varphi}{\equiv} V$, \ie given $s_1$ and $s_2 \in \cs(V)$, we have:
\begin{align*}
s_1 \ast_{\varphi} s_2 &= s_1 \ast_{V} s_2\\
&= s_1 \ast_{\Gamma} s_2
\end{align*}
\label{def:convv}
\end{definition}

This time we do obtain equivariance to $\Gamma$ as expected, and the full characterization as well.

\begin{theorem}\textbf{Characterization of $\varphi$-convolution right operators}\\
Let a group $\Gamma$ acting on $V~\st \Gamma \overset{\varphi}{\equiv} V$, let $f \in \cl(\cs(\Gamma))$, then:\\
\centerline{$f$ is a $\varphi$-convolution right operator $\Leftrightarrow$ $f$ is equivariant to $\Gamma$}
\label{prop:equiG}
\end{theorem}
\begin{proof}
As a consequence of \eqref{eq:fpf}, \eqref{eq:PP}, Lemmas \ref{lem:rel12}, and \ref{lem:rel23}, we can use bullet~\ref{enum:p} of \thref{th:cgco} to obtain the proof.
\end{proof}

\begin{corollary}\textbf{Characterization of $\varphi$-convolution operators}\\
Let a group $\Gamma$ acting on $V~\st \Gamma \overset{\varphi}{\equiv} V$, let $f \in \cl(\cs(\Gamma))$, then:\\
\begin{tabular}{rl}
  $f$ is equivariant to $\Gamma$ $\Leftrightarrow$ &
  $f$ is a $\varphi$-convolution operator \st its laterality is \\
  & opposed to the laterality of the $\varphi$-equivalence
\end{tabular}
\label{cor:equiG}
\end{corollary}
\begin{proof}
As per \thref{prop:equiG} and its counterpart for bullet \ref{enum:pp} of \thref{th:cgco}.
\end{proof}

\begin{remark}
The counterpart of bullet \ref{enum:ppp} of \thref{th:cgco} would just inform that it is equivalent to say that either the $\varphi$-equivalence is commutative, or the $\varphi$-convolution is commutative, or $\Gamma$ is abelian.
\end{remark}

% \begin{theorem}\textbf{Characterization by right-action equivariance to $\Gamma$}\\
% If $\Gamma$ is $\varphi$-equivalent, the class of linear transformations of $\cs(V)$ that are equivariant to $\Gamma$ is exactly the class of $\varphi$-convolution operators acting on the right of $\cs(V)$ \ie
% \begin{align*}
% & \text{If } \Gamma \overset{\varphi}{\equiv} V,\\
% & \exists w \in \cs(V), f = . \ast_{\varphi} w \Leftrightarrow
% \begin{cases}
% f \in \cl(\cs(V))\\
% \forall g \in \Gamma, f \circ g = g \circ f
% \end{cases}
% \end{align*}
% \label{prop:equiG}
% \end{theorem}

% \begin{proof}\begin{enumerate}\item From left to right:\\
% In the following equations, \eqref{eq:rel} is obtained by definition, \eqref{eq:left} is obtained because left multiplication in a group is bijective, and \eqref{eq:pty} is obtained because of \eqref{eq:P}.
% \begin{align}
% \forall g \in \Gamma, \forall s \in \cs(V), & \nonumber\\
% f_w \circ g (s) & = \displaystyle \sum_{h \in \Gamma} g(s)[\varphi(h)] \h{2} h(w)\label{eq:rel}\\
%  & = \displaystyle \sum_{h \in \Gamma} g(s)[\varphi(gh)] \h{2} gh(w)\label{eq:left}\\
%  & = \displaystyle \sum_{h \in \Gamma} g(s)[g(\varphi(h))] \h{2} gh(w)\label{eq:pty}\\
%  & = \displaystyle \sum_{h \in \Gamma} s[\varphi(h)] \h{2} gh(w)\nonumber\\
%  & = \displaystyle \sum_{h \in \Gamma} s[\varphi(h)] \h{2} h(w)[g^{-1}(.)]\nonumber\\
% & = f_w (s) [g^{-1}(.)]\nonumber\\
% & = g \circ f_w (s) \nonumber
% \end{align}
% Of course, we also have that $f_w$ is linear.
% \item From right to left:\\
% Let $f \in \cl(\cs(V)), s \in \cs(V)$. By linearity of $f$, we distribute $f(s)$ on the family of dirac signals:
% \begin{gather}
% f(s) = \displaystyle\sum_{v \in V} s[v]f(\delta_v) \label{eq:fdirac}
% \end{gather}
% Thanks to \eqref{eq:P}, we have that:
% \begin{align*}
% g_v(&\varphi(\id)) = \varphi(g_v\id) = v\\
% \text{So, } & v = u \Leftrightarrow  \varphi(\id) = g_v^{-1}(u)\\
% \text{So, } & \delta_v = g_v(\delta_{\varphi(\id)})
% \end{align*}
% By denoting $w = f(\delta_{\varphi(\id)})$, and using the hypothesis of equivariance, we obtain from \eqref{eq:fdirac} that:
% \begin{align*}
% f(s) & = \displaystyle\sum_{v \in V} s[v] \h{2} f \circ g_v(\delta_{\varphi(\id)})\\
%  & = \displaystyle\sum_{v \in V} s[v] \h{2} g_v \circ f(\delta_{\varphi(\id)})\\
%  & = \displaystyle\sum_{v \in V} s[v] \h{2} g_v(w)\\
%  & = s \ast_{\varphi} w
% \end{align*}
% \end{enumerate}
% \end{proof}

%\paragraph{Construction of $\varphi$-convolutions on vertex domains}
\thref{prop:equiG} tells us that in order to define a convolution on the vertex domain of a graph $\gve$, all we need is a group~$\Gamma$ acting on the vertex set~$V$ (or on a subset of~$V$), that is equivalent to it. The choice of~$\Gamma$ can be done with respect to the edge set~$E$. This is discussed in more details in \secref{sec:edges}, where we will see that in fact, we only need a generating set of~$\Gamma$.

%\paragraph{Exposure of $\varphi$}
The previous construction relies on exposing a bijective equivariant map~$\varphi$ between $\Gamma$~and~$V$. In the next subsection, we show that in case $\Gamma$ is abelian, we even need not expose $\varphi$ and the characterization still holds.

\subsection{Mixed domain formulation}

From $\eqref{eq:premix}$, we can define a mixed domain convolution \ie that is defined for $r \in \cs(\Gamma)$ and $s \in \cs(V)$, without the need of expliciting $\varphi$.

\begin{definition}\textbf{Mixed domain convolution}\\
Let a group $\Gamma$ acting on $V$.
The \emph{mixed domain convolution} between two signals $r \in \cs(\Gamma)$ and $s \in \cs(V)$ results in a signal $r \ast_\mu s \in \cs(V)$ and is defined as:
\begin{gather*}
r \ast_\mu s = \displaystyle \sum_{g \in \Gamma} r[g] \h{2} g(s)
\end{gather*}
\label{def:convm}
\end{definition}

We call it $\mu$convolution. From a practical point of view, this expression of the convolution is useful because it relegates $\varphi$ as an underpinning object.% Therefore, only $V$ and some of its transformations are enough to define a convolution.

\begin{lemma}\textbf{Relation with vertex-action convolution}\\
$\forall \varphi \in \bij(\Gamma, V), \forall (r,s) \in \cs(\Gamma) \times \cs(V),$\\
\centerline{$r \ast_\mu s = \varphi(r) \ast_{V} s$}
\label{lem:rel3m}
\end{lemma}

\begin{proof}
Let $\varphi \in \bij(\Gamma, V),(r,s) \in \cs(\Gamma) \times \cs(V)$,
\begin{align*}
r \ast_\mu s & = \displaystyle \sum_{g \in \Gamma} r[g] \h{2} g(s)
  = \displaystyle \sum_{v \in V} r[g_v] \h{2} g_v(s)
  \overset{(\diamond)}{=} \displaystyle \sum_{v \in V} \varphi(r)[v] \h{2} g_v(s)\\
& =\varphi(r) \ast_{V} s
\end{align*}
Where $\overset{(\diamond)}{=}$ comes from \lemref{lem:outer}.
\end{proof}

In other words, $\ast_\mu$ is a convenient reformulation of $\ast_{V}$.

\begin{lemma}\textbf{Relation with group, vertex-group and $\varphi$-convolutions}\\
Let $\varphi \in \bij(\Gamma, V),(r,s) \in \cs(\Gamma) \times \cs(V)$, we have:
\begin{align*}
\Gamma \overset{\varphi}{\equiv} V & \Leftrightarrow \forall v \in V, (r \ast_\mu s)[v] = (r \ast_{\Gamma} \varphi^{-1}(s))[g_v]\\
& \Leftrightarrow r \ast_\mu s = \varphi(r) \ast_{\Gamma} s\\
& \Leftrightarrow r \ast_\mu s = \varphi(r) \ast_{\varphi} s
\end{align*}
\label{lem:rel12m}
\end{lemma}

\begin{proof}
On one hand, \lemref{lem:rel3m} gives $r \ast_\mu s = \varphi(r) \ast_{V} s$. On the other hand, \lemref{lem:rel12} gives $\forall v \in V,
(r \ast_{\Gamma} \varphi^{-1}(s))[g_v] = (\varphi(r) \ast_{\Gamma} s)[v]$. Then \lemref{lem:rel23} concludes.
\end{proof}

% \begin{remark}
% The converse sense is meaningful because it justifies that when the $\M$-convolution is employed, the property $\Gamma \equiv V$ underlies, without the need of expliciting $\varphi$.
% \end{remark}

%%%%%%%%%%%% tempt 1

% \todo{rewrite below, start by stating the proposition, then discuss}

% From $\mu$convolution, a left operators can be used as a layer of a neural network since it is defined over $\cs(V)$, whereas right operators are not. Also, it is of the form $f:s \mapsto w \ast_\mu s = \sum_{g \in \Gamma} w[g] \h{2} g(s)$, so its realization on a vertex $u \in V$ can rely on the rule:
% \begin{itemize}
% \item $g(s)[u] = s[g^{-1}(u)]$, in case the underlying $\varphi$-equivalence is left,
% \item or on $g(s)[u] = s[g_u(\varphi(g^{-1}))]$, in the other case
% \end{itemize}
% Since the latter case makes resurface $\varphi$, we are here only interested in left operators with left $\varphi$-equivalence. However, \ref{cor:equiG} tells us that the equivariance characterization would be lost unless "left" also means "right", that is, unless $\Gamma$ is abelian. Therefore, we coin the following definition:

% \begin{definition}\textbf{M-convolution}\\
% The \emph{$\M$-convolution} is defined as the mixed domain convolution such that $|\Gamma| = |V|$ and $\Gamma$ is abelian.
% \end{definition}

% \begin{corollary}\textbf{Characterization of M-convolution operators}\\
% Let a subgroup $\Gamma \subset \Phi^*(V)~\st |\Gamma| = |V|$ and $\Gamma$ is abelian. Let $f \in \cl(\cs(\Gamma))$.\\
% \centerline{$f$ is a $\M$-convolution operator $\Leftrightarrow$ $f$ is equivariant to $\Gamma$}
% \end{corollary}

%%%%%%%%%%%

From $\mu$convolution, a left operator can be used as a layer of a neural network since it is defined over $\cs(V)$, whereas right operators are not. 

\begin{proposition}\textbf{Condition of equivariance for $\ast_\mu$}\\
Let a group $\Gamma$ acting on $V$. $\Gamma$ is abelian, if and only if, $\mu$convolution left operators are equivariant to~$\Gamma$.
\label{prop:equiM}
\end{proposition}
\begin{proof}
Let $w, g \in \Gamma$, and define $f_w: s \mapsto w \ast_\mu s$. In the following expressions, $\Gamma$ is abelian if and only if \eqref{eq:aba} and \eqref{eq:abb} are equal (the converse is obtained by particularizing on well chosen signals):
\begin{align}
(f_w \circ g) (s) & = \displaystyle \sum_{h \in \Gamma} w[h] \h{2} hg(s) \label{eq:aba}\\
 & = \displaystyle \sum_{h \in \Gamma} w[h] \h{2} gh(s) \label{eq:abb}\\
 & = \displaystyle g\left(\sum_{h \in \Gamma} w[h] \h{2} h(s)\right) \nn\\
 & = g(w \ast_\mu s ) \nn\\
 & = (g \circ f_w) (s) \nn
\end{align}
\end{proof}

\begin{remark}
We used this proof because it is simpler and does not require one-to-one correspondence between $\Gamma$ and $V$. However, the reason behind the abelian condition is that mixed domain convolutions are expressed as if the underlying $\varphi$-equivalence were left. Therefore it is a consequence of \corref{cor:equiG}. Note that if we used a right expression for $\ast_\mu$ (which amounts to commute the operands), we would be interested in right operators instead of left ones, which would just change the problem to "right-right" instead of "left-left" which would also require abelianity for the equivariance.
\end{remark}

We then coin the term $\M$-convolution, for which $\Gamma$ is abelian.

\begin{definition}\textbf{M-convolution}\\
Let a group $\Gamma$ acting on $V$.
The \emph{$\M$-convolution} is defined as a mixed domain convolution such that $\Gamma$~is an abelian subgroup of $\Phi^*(V)$.
\label{def:convM}
\end{definition}

\begin{corollary}\textbf{Characterization of M-convolution left operators}\\
Let a group $\Gamma$ acting on $V~\st \Gamma \cong V$, and $\Gamma$ is abelian. Let $f \in \cl(\cs(\Gamma))$, then:\\
\centerline{$f$ is a $\M$-convolution left operator $\Leftrightarrow$ $f$ is equivariant to $\Gamma$}
\label{cor:equiM}
\end{corollary}
\begin{proof}As a corollary of \corref{cor:equiG} and \propref{prop:equiM}.
\end{proof}

\begin{remark}
Note that despite abelianity, we do not call them "commutative" operators but still call "left" operators since the domain is mixed. Also, note that unlike for \propref{prop:equiM}, the converse of \corref{cor:equiM} requires $\Gamma$ and $V$ to be in one-to-one correspondence.
\end{remark}

% So they would be relevant as layers of neural networks. On the contrary, the domain of right operators is $\cs(\Gamma)$


% $\varphi$ resurface. However, the equivariance to $\Gamma$ incurring from \lemref{lem:rel3m} and \thref{prop:equiG} only holds for operators acting on the right. So we need to intertwine an abelian condition as follows. This is also a good excuse to see the influence of abelianity.

% %From $\M$-convolution, we can derive operators acting on the left of $\cs(V)$, of the form $s \mapsto w \ast_{\M} s$, parameterized by $w \in \cs(\Gamma)$. In particular, these operators would be relevant as layers of neural networks. On the contrary, derived operators acting on the right such as $r \mapsto r \ast_{\M} w$ wouldn't make sense with this formulation as they would make $\varphi$ resurface. However, the equivariance to $\Gamma$ incurring from \lemref{lem:rel3m} and \thref{prop:equiG} only holds for operators acting on the right. So we need to intertwine an abelian condition as follows. This is also a good excuse to see the influence of abelianity.

% \begin{proposition}\textbf{Equivariance to $\Gamma$ through left action}\\
% Let a subgroup $\Gamma \subset \Phi^*(V)$ such that $\Gamma \cong V$. $\Gamma$ is abelian, if and only if, $\M$-convolution operators acting on the left of $\cs(V)$ are equivariant to it \ie
% \begin{gather*}
% \forall g,h \in \Gamma, gh = hg \Leftrightarrow \forall w, g \in \Gamma, w \ast_{\M} g(.) = g \circ (w \ast_{\M} .) 
% \end{gather*}
% \end{proposition}

% \begin{proof}
% \todo{Vincent, peux-tu vérifier la preuve, ainsi que celle du théorême qui suit stp}

% Let $w, g \in \Gamma$, and define $f_w: s \mapsto w \ast_{\M} s$. In the following expressions, $\Gamma$ is abelian if and only if \eqref{eq:aba} and \eqref{eq:abb} are equal (the converse is obtained by particularizing on well chosen signals):
% \begin{align}
% f_w \circ g (s) & = \displaystyle \sum_{h \in \Gamma} w[h] \h{2} hg(s) \label{eq:aba}\\
%  & = \displaystyle \sum_{h \in \Gamma} w[h] \h{2} gh(s) \label{eq:abb}\\
%  & = \displaystyle \sum_{h \in \Gamma} w[h] \h{2} h(s)[g^{-1}(.)] \nn\\
%  & = (w \ast_{\M} s )[g^{-1}(.)] \nn\\
%  & = g \circ f_w (s) \nn
% \end{align}
% \end{proof}

% \begin{remark}Similarly, $\ast_{\varphi}$ is also equivariant to $\Gamma$ through left action if and only if $\Gamma$ is abelian, as a consequence of being commutative if and only if $\Gamma$ is abelian. On the contrary, note that commutativity of $\ast_{\M}$ does not make sense.
% \end{remark}

% \begin{theorem}\textbf{Characterization by left-action equivariance to $\Gamma$}\\
% Let $\Gamma \cong V$. If $\Gamma$ is abelian, the class of linear transformations of $\cs(V)$ that are equivariant to $\Gamma$ is exactly the class of $\M$-convolution operators acting on the left of $\cs(V)$ \ie
% \begin{align*}
% & \text{If } \Gamma \cong V \text{ and $\Gamma$ is abelian,}\\
% & \exists w \in \cs(\Gamma), f = w \ast_{\M} . \Leftrightarrow
% \begin{cases}
% f \in \cl(\cs(V))\\
% \forall g \in \Gamma, f \circ g = g \circ f
% \end{cases}
% \end{align*}
% \label{cor:equiM}
% \end{theorem}

% \begin{proof} By picking $\varphi$ such that $\Gamma \overset{\varphi}{\equiv} V$ with \lemref{lem:rel12m} and using the relation between $\ast_{\M}$~and~$\ast_{\varphi}$.
% \end{proof}

When the group domain is abelian, it is preferable to use $\ast_{\M}$ rather than $\ast_{\varphi}$ since $\varphi$ is not needed. When $\Gamma$ and $V$ are not in one-to-one correspondence, the equivariance is preserved but not the characterization. This is not a limitation on the size of the kernel $w$ of corresponding convolution operators since small kernels can be obtained by limiting the support of~$w$.

\vspace{5pt}

The obtained constructions $\varphi$- and $\M$-convolutions are independent of the edge set. In fact, they characterize more generally convolutions for signals defined on sets. Next, we will see what happens should we also consider the edge set.
