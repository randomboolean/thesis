\documentclass[12pt]{book}
\usepackage[utf8]{inputenc}
\usepackage[french, english]{babel}
\usepackage[T1]{fontenc}
\usepackage{lipsum}

\usepackage{amsfonts}
\usepackage{amsmath}
\usepackage{amssymb}
\usepackage[mathscr]{euscript}
\usepackage{stmaryrd}
\usepackage[normalem]{ulem}
\usepackage{tikz,tikz-3dplot}
\usepackage{tikz-cd}
\usepackage{pgfplots}
\usepackage{graphicx}
\usepackage{color}
\usepackage{enumitem}
\usetikzlibrary{matrix,chains,positioning,decorations.pathreplacing,arrows}
% \usetikzlibrary{external}
% \tikzexternalize[prefix=tikzext/]
% \tikzset{external/mode=graphics if exists}
\usepackage{framed}
\usepackage{arydshln}
\usepackage{multirow}
\usepackage{mathtools}
%\usepackage{float}
%\usepackage{yhmath}
\DeclareSymbolFont{yhlargesymbols}{OMX}{yhex}{m}{n}
\DeclareMathAccent{\wideparen}{\mathord}{yhlargesymbols}{"F3}
\usepackage{floatrow}
\usepackage{pdfpages}
\usepackage{minitoc}
\setcounter{secnumdepth}{3}
\setcounter{tocdepth}{3}
\setcounter{minitocdepth}{2}
%\renewcommand{\mtctitle}{~} % Empty minitoc titles
\usepackage[nobottomtitles]{titlesec}
\usepackage{etoolbox}
\makeatletter
\patchcmd{\ttlh@hang}{\parindent\z@}{\parindent\z@\leavevmode}{}{}
\patchcmd{\ttlh@hang}{\noindent}{}{}{}
\makeatother

%\usepackage{quotchap}
\usepackage{csquotes}

% Abstract
\newcommand\abstractname{Abstract}  %%% here
\makeatletter
\if@titlepage
  \newenvironment{abstract}{%
      \titlepage
      \null\vfil
      \@beginparpenalty\@lowpenalty
      \begin{center}%
        \bfseries \abstractname
        \@endparpenalty\@M
      \end{center}}%
     {\par\vfil\null\endtitlepage}
\else
  \newenvironment{abstract}{%
      \if@twocolumn
        \section*{\abstractname}%
      \else
        \small
        \begin{center}%
          {\bfseries \abstractname\vspace{-.5em}\vspace{\z@}}%
        \end{center}%
        \quotation
      \fi}
      {\if@twocolumn\else\endquotation\fi}
\fi
\makeatother

% References
\usepackage{hyperref}
\usepackage{enumitem}
\makeatletter
\def\namedlabel#1#2{\begingroup
    #2%
    \def\@currentlabel{#2}%
    \phantomsection\label{#1}\endgroup
}
\makeatother

% Bibliography
\usepackage[backend=bibtex,
            sorting=nyt, %or ynt?
            style=authoryear,
            natbib=true,
            maxcitenames=2,
            mincitenames=1,
            maxbibnames=99,
            backref=true]
            {biblatex}
\addbibresource{refs/datasets.bib}
\addbibresource{refs/dl_history.bib}
\addbibresource{refs/dl_understanding.bib}
\addbibresource{refs/dl_activations.bib}
\addbibresource{refs/dl_bastnet.bib}
\addbibresource{refs/dl_manifold.bib}
\addbibresource{refs/dl_vertex.bib}
\addbibresource{refs/dl_vertex_old.bib}
\addbibresource{refs/dl_spectral.bib}
\addbibresource{refs/gsp.bib}
\addbibresource{refs/toClassify.bib}
\addbibresource{refs/maths.bib}
\addbibresource{refs/scattering.bib}
\addbibresource{refs/prog_languages.bib}
\addbibresource{refs/online.bib}
\addbibresource{refs/wordvec.bib}

% Style
\setlength\parindent{0pt}
\newcommand{\subsubsubsection}[1]{\paragraph{#1}\mbox{}\\}
\newcommand{\subsubsubsubsection}[1]{\subparagraph{#1}\mbox{}\\}

\usepackage{setspace}
\onehalfspacing % or
%\doublespacing

% numbering lines
\usepackage[left]{lineno}
%linenumbers
%\modulolinenumbers[2]

\newcommand*\patchAmsMathEnvironmentForLineno[1]{%
  \expandafter\let\csname old#1\expandafter\endcsname\csname #1\endcsname
  \expandafter\let\csname oldend#1\expandafter\endcsname\csname end#1\endcsname
  \renewenvironment{#1}%
     {\linenomath\csname old#1\endcsname}%
     {\csname oldend#1\endcsname\endlinenomath}}% 
\newcommand*\patchBothAmsMathEnvironmentsForLineno[1]{%
  \patchAmsMathEnvironmentForLineno{#1}%
  \patchAmsMathEnvironmentForLineno{#1*}}%
\AtBeginDocument{%
\patchBothAmsMathEnvironmentsForLineno{equation}%
\patchBothAmsMathEnvironmentsForLineno{align}%
\patchBothAmsMathEnvironmentsForLineno{flalign}%
\patchBothAmsMathEnvironmentsForLineno{alignat}%
\patchBothAmsMathEnvironmentsForLineno{gather}%
\patchBothAmsMathEnvironmentsForLineno{multline}%
}

% linebreaks in math mode
%\binoppenalty=\maxdimen %700
%\relpenalty=\maxdimen %500

% chapter style
\makeatletter
\def\@makechapterhead#1{%
  %%%%\vspace*{50\p@}% %%% removed!
  \vspace*{0\p@}
  {\parindent \z@ \raggedright \normalfont
    \ifnum \c@secnumdepth >\m@ne
        \huge\bfseries \@chapapp\space \thechapter
        \par\nobreak
        \vskip 0\p@
    \fi
    \interlinepenalty\@M
    \Huge \bfseries #1\par\nobreak
    \vskip 20\p@
  }}
\def\@makeschapterhead#1{%
  %%%%%\vspace*{50\p@}% %%% removed!
  \vspace*{50\p@}
  {\parindent \z@ \raggedright
    \normalfont
    \interlinepenalty\@M
    \Huge \bfseries  #1\par\nobreak
    \vskip 40\p@
  }}
\makeatother

% Paragraphs
\makeatletter
\renewcommand\paragraph{\@startsection{paragraph}{4}{\z@}%
                                    {3.25ex \@plus1ex \@minus.2ex}%
                                    {0.01pt}%
                                    {\normalfont\normalsize\bfseries}}
\makeatother

%\setlength{\parskip}{5pt}

% Figures, Table numbering
\usepackage{chngcntr}
\counterwithout{table}{chapter}
\counterwithout{figure}{chapter}

% Theorems
\usepackage{amsthm}
\theoremstyle{definition}
\newtheorem{definition}{Definition}%[chapter]
\newtheorem{proposition}[definition]{Proposition}
\newtheorem{corrolary}[definition]{Corrolary}
\newtheorem{lemma}[definition]{Lemma}
\newtheorem{claim}[definition]{Claim}

\theoremstyle{remark}
%\newtheorem{remark}[definition]{Remark}
\newtheorem*{remark}{Remark}

\theoremstyle{plain}

\allowdisplaybreaks[1]
\usepackage{etoolbox}% http://ctan.org/pkg/etoolbox
\usepackage{needspace}% http://ctan.org/pkg/needspace
\AtBeginEnvironment{definition}{\Needspace{5\baselineskip}}% \break if fewer than 5\baselineskip is available on page
\AtBeginEnvironment{proposition}{\Needspace{5\baselineskip}}
\AtBeginEnvironment{corrolary}{\Needspace{5\baselineskip}}
\AtBeginEnvironment{lemma}{\Needspace{5\baselineskip}}
\AtBeginEnvironment{claim}{\Needspace{5\baselineskip}}
\AtBeginEnvironment{remark}{\Needspace{5\baselineskip}}

% Annotations
\newcommand{\todo}[1]{\textcolor{red}{TODO: #1\\}}
\newcommand{\etodo}{\emph{\textcolor{red}{todo}}}

% Maths
\usepackage{chngcntr}
\counterwithout{equation}{chapter}

\usepackage{mathbbol}
\newcommand{\bb}{\mathbb{b}}
%\newcommand{\cc}{\mathbb{c}}
\newcommand{\uu}{\mathbb{u}}
\newcommand{\vv}{\mathbb{v}}
\newcommand{\bbe}{\mathbb{E}}
\newcommand{\bbi}{\mathbb{I}}
\newcommand{\bbn}{\mathbb{N}}
\newcommand{\bbr}{\mathbb{R}}
\newcommand{\bbt}{\mathbb{T}}
\newcommand{\bbv}{\mathbb{V}}
\newcommand{\bbu}{\mathbb{U}}
\newcommand{\bbz}{\mathbb{Z}}

\newcommand{\ca}{\mathcal{A}}
\newcommand{\cb}{\mathcal{B}}
\newcommand{\cc}{\mathcal{C}}
\newcommand{\cd}{\mathcal{D}}
\newcommand{\ce}{\mathcal{E}}
\newcommand{\cf}{\mathcal{F}}
\newcommand{\cg}{\mathcal{G}}
\newcommand{\ch}{\mathcal{H}}
\newcommand{\ci}{\mathcal{I}}
\newcommand{\cj}{\mathcal{J}}
\newcommand{\ck}{\mathcal{K}}
\newcommand{\cl}{\mathcal{L}}
\newcommand{\cm}{\mathcal{M}}
\newcommand{\cn}{\mathcal{N}}
\newcommand{\co}{\mathcal{O}}
\newcommand{\cp}{\mathcal{P}}
\newcommand{\cq}{\mathcal{Q}}
\newcommand{\ccr}{\mathcal{R}}
\newcommand{\cs}{\mathcal{S}}
\newcommand{\ct}{\mathcal{T}}
\newcommand{\cu}{\mathcal{U}}
\newcommand{\cv}{\mathcal{V}}
\newcommand{\cW}{\mathcal{W}}
\newcommand{\cx}{\mathcal{X}}
\newcommand{\cy}{\mathcal{Y}}
\newcommand{\cz}{\mathcal{Z}}

\newcommand{\seq}[1]{\{1, 2, \ldots, #1\}}
\newcommand{\sq}[1]{\{1, \ldots, #1\}}
\newcommand{\group}{\mathcal{G}}

\DeclareMathOperator{\diag}{diag}
\DeclareMathOperator{\off}{off}
\DeclareMathOperator{\order}{order}
\DeclareMathOperator{\tree}{trav}
%\DeclareMathOperator{\deg}{deg}
%\DeclareMathOperator{\dim}{dim}
\DeclareMathOperator{\shape}{shape}
\DeclareMathOperator{\supp}{supp}
\DeclareMathOperator{\EC}{\textsc{ec}}
\DeclareMathOperator{\LRF}{\textsc{lrf}}
\DeclareMathOperator{\ER}{\textsc{er}}
\DeclareMathOperator{\OR}{\textsc{or}}
\DeclareMathOperator{\XOR}{\textsc{xor}}
\DeclareMathOperator{\AND}{\textsc{and}}
\DeclareMathOperator{\agg}{\textsc{aggregate}}
%\DeclareMathOperator{\def}{def}
\DeclareMathOperator{\id}{Id}
\DeclareMathOperator{\I}{\textsc{i}}
\DeclareMathOperator{\II}{\textsc{ii}}
\DeclareMathOperator{\III}{\textsc{iii}}
\DeclareMathOperator{\M}{\textsc{m}}
\DeclareMathOperator{\C}{\textsc{c}}
\DeclareMathOperator{\T}{\mathsf{T}}
\DeclareMathOperator{\iso}{\textsc{iso}}
\DeclareMathOperator{\bij}{\textsc{bij}}
\DeclareMathOperator{\D}{\textsc{d}}
\DeclareMathOperator{\AUT}{\textsc{aut}}

\DeclareMathOperator{\scr}{\textsc{R}}
\DeclareMathOperator{\scs}{\textsc{S}}

\DeclareMathOperator{\IN}{\textsc{in}}
\DeclareMathOperator{\OUT}{\textsc{out}}

% Acronyms
\newcommand{\iid}{\emph{i.i.d.}~}
\newcommand{\etal}{\emph{et al.}~}
\newcommand{\ie}{\emph{i.e.}~}
\newcommand{\st}{\emph{s.t.}~}
\newcommand{\eg}{\emph{e.g.}~}
\newcommand{\powth}{\text{$^\text{th}$~}}
\newcommand{\wrt}{\emph{w.r.t.}~}
\newcommand{\nn}{\nonumber}
\newcommand{\cdl}{\cd_{\ltimes}}

% References
\newcommand{\figref}[1]{Figure~\ref{#1}}
\newcommand{\chapref}[1]{Chapter~\ref{#1}}
\newcommand{\appref}[1]{Appendix~\ref{#1}}
\newcommand{\secref}[1]{Section~\ref{#1}}
\newcommand{\algref}[1]{Algorithm~\ref{#1}}
\newcommand{\thref}[1]{Theorem~\ref{#1}}
\newcommand{\propref}[1]{Proposition~\ref{#1}}
\newcommand{\remref}[1]{Remark~\ref{#1}}
\newcommand{\claref}[1]{Claim~\ref{#1}}
\newcommand{\rqref}[1]{Remark~\ref{#1}}
\newcommand{\defref}[1]{Definition~\ref{#1}}
\newcommand{\corref}[1]{Corrolary~\ref{#1}}
\newcommand{\lemref}[1]{Lemma~\ref{#1}}
\newcommand{\conjref}[1]{Conjecture~\ref{#1}}
\newcommand{\probref}[1]{Problem~\ref{#1}}
\newcommand{\quoref}[1]{Quote~\ref{#1}}
\newcommand{\tabref}[1]{Table~\ref{#1}}
%\newcommand{\eqref}[1]{(\ref{#1})} %already defined

\newcommand{\gve}{G = \langle V, E \rangle}
\newcommand{\vgve}{\vec{G} = \langle V, E \rangle}

% hspaces
\newcommand{\h}[1]{\hspace{#1pt}}

% keywords
\newcommand{\keywords}[1]{\textbf{\textit{Index terms---}} #1}

% quotes
\newcommand{\quotes}[1]{``#1''}

% For temptative plans
\newcommand{\fakechapter}[1]{%
  \par\refstepcounter{chapter}% Increase subsection counter
  \chaptermark{#1}% Add subsection mark (header)
  \addcontentsline{toc}{chapter}{\protect\numberline{\thechapter}#1}% Add subsection to ToC
}

\newcommand{\fakesection}[1]{%
  \par\refstepcounter{section}% Increase section counter
  \sectionmark{#1}% Add section mark (header)
  \addcontentsline{toc}{section}{\protect\numberline{\thesection}#1}% Add section to ToC
}

\newcommand{\fakesubsection}[1]{%
  \par\refstepcounter{subsection}% Increase section counter
  \subsectionmark{#1}% Add section mark (header)
  \addcontentsline{toc}{subsection}{\protect\numberline{\thesubsection}#1}% Add section to ToC
}

\begin{document}
\selectlanguage{english}

%
% Title
%

% ?
% (On) Deep Learning for non-regularly structured data
% ... see temptative titles chapter

%
% Table of contents
%

\dominitoc
\tableofcontents

%
% Introduction
%

%\chapter*{Introduction}
%\input{}

%
% Chapter 1
%

%\chapter{Subject disambiguation}

%
% Chapter 2
%

\setcounter{chapter}{1}
\chapter{Presentation of the field}

In this section, we present notions related to our domains of interest. In particular, for tensors we give original definitions that are more appropriate for our study. In the neural network's section, we present the concepts necessary to understand the evolution of the state of the art research in this field. In the last section, we present graphs for their usage in deep learning.

Vector spaces considered in what follows are assumed to be finite-dimensional and over the field of real numbers $\bbr$.

\vfill
\minitoc
\newpage

\subsection{Tensors}

Intuitively, tensors in the field of deep learning are defined as a generalization of vectors and matrices, as if vectors were tensors of rank $1$ and matrices were tensors of rank $2$. That is, they are objects in a linear space and their dimensions are indexed using as many indices as their rank, so that they can be represented by multidimensional arrays. In mathematics, a tensor is usually defined as a special class of multilinear functions. As such, a mathematical tensor is entirely defined on the cartesian product of the canonical bases onto which its outputs can be represented by a multidimensional array. In that sense, both definitions rejoin on their representation, but the underlying objects are different. In particular, mathematical tensors enjoy a more intrinsic definition as they neither depend on their multidimensional array representation nor on a change of basis.

Our definition of tensors is such that they are a bit more than multidimensional arrays but not as much as mathematical tensors, for that they are embedded in a linear space and any deep learning object can be later define rigorously.

\begin{definition}\textbf{Tensor space}\\
We define a \emph{tensor space} $\bbt$ of rank $r$ as a cartesian product of $r$ vector spaces under the coordinate-wise sum and the mono-linear outer product.

Its \emph{shape} is denoted $n_1 \times n_2 \times \cdots \times n_r$, where the $\{n_k\}$ are the dimensions of the vector spaces.
\end{definition}

Hence, a tensor space is a linear space. Note that for naming conveniency, we distinguish between the terms \emph{linear space} and \emph{vector space}. That is, we abusively use the term \emph{vector space} only to refer to a linear space that can be seen as a tensor space of rank $1$. If there is no notion of rank defined, we rather use the term \emph{linear space}.
We also make a clear distinction between the term \emph{dimension} (that is, for a tensor space it is equal to $\displaystyle \prod_{k=1}^r n_k$) and the term \emph{rank} (equal to $r$).

\begin{definition}\textbf{Tensor}\\
A \emph{tensor} $t$ is an object of a tensor space. The \emph{shape} of $t$, which is the same as the shape of the tensor space it belongs to, is denoted $n_1^{(t)} \times n_2^{(t)} \times \cdots \times n_r^{(t)}$.
\end{definition}

\begin{definition}\textbf{Indexing}\\
An \emph{entry} of a tensor $t$ is a scalar denoted $t[i_1, i_2, \ldots, i_r]$.

A \emph{subtensor} $t'$ is a tensor of same rank composed of entries of $t$ that are contiguous in the indexing, with at least one entry per rank. We denote $t' = t[l_1{:}u_1, l_2{:}u_2, \ldots, l_r{:}u_r]$, where the $\{l_p\}$ and the $\{u_p\}$ are the lower and upper bounds of the indices used by the entries that compose~$t'$.
\end{definition}

When using an index $i_k$ for an entry of a tensor $t$, we implicitly assume that $i_k \in \{1, 2, \cdots, n_k^{(t)}\}$ if nothing is precised.
For subtensor indexings, we don't necessarily write the lower bound index if it is equal to $1$, neither the upper bound index if it is equal to $n_p^{(t)}$.

\begin{definition}\textbf{Slicing}\\
A \emph{slice} operation, along the last ranks $\{r_1, r_2, \ldots, r_s\}$, and indexed by $(i_{r_1}, i_{r_2}, \ldots, i_{r_s})$, is a morphism $s: \bbt = \displaystyle \prod_{k=1}^r \bbv_k \rightarrow \prod_{k=1}^{r-s} \bbv_k$, such that:
\begin{align*}
s(t)[i'_1, i'_2, \ldots, i'_{r-s}] &= t[i'_1, i'_2, \ldots, i'_{r-s}, i_{r_1}, i_{r_2}, \ldots, i_{r_s}] \\
\text{ \ie } \quad s(t) :&= t[:,:, \ldots, :, i_{r_1}, i_{r_2}, \ldots, i_{r_s}]
\end{align*}
where $:=$ means that values of the right operand are assigned to the left operand.
We denote $t_{i_{r_1}, i_{r_2}, \ldots i_{r_s}}$ and call it the \emph{slice} of $t$. 
Slicing along a subset of ranks that are not the lasts is defined similarly.
$s(\bbt)$ is called a \emph{sliced subspace}.
\end{definition}

\begin{definition}\textbf{Flattening}\\
A \emph{flatten} operation is an isomorphism $f: \bbt \rightarrow \bbv$, between a tensor space $\bbt$ of rank~$r$ and an $n$-dimensional vector space $\bbv$, where $n =\displaystyle \prod_{k=1}^r n_k$. It is characterized by a bijection in the index spaces $g: \displaystyle \prod_{k=1}^r \{1, \ldots, n_k \} \rightarrow\{1, \ldots, n \}$ such that
\begin{gather*}
  \forall t \in \bbt, f(t)[g(i_1, i_2, \ldots, i_r)] = f(t[i_1, i_2, \ldots, i_r])
\end{gather*}

An inverse operation is called a \emph{de-flatten} operation.
\end{definition}

\begin{remark}\textbf{Row major ordering}\\
The choice of $g$ determines in which order the indexing is made. $g$ is reminescent of how data of multidimensional arrays or tensors are stored internally by programming languages. In most tensor manipulation languages, incrementing the memory adress (\ie the output of $g$) will increment only $i_r$ if $i_r < n_r$ (and then ranks are ordered in reverse lexicographic order to decide what index is also incremented). This is called \emph{row major ordering}, as opposed to \emph{column major ordering}. That is, in row major, $g$ is defined as
\begin{align}
  g(i_1, i_2, \ldots, i_r) = \displaystyle \sum_{p=1}^r \left( \prod_{k=p+1}^r n_k \right) i_p \label{rowmajor}
\end{align}
\end{remark}

\begin{definition}\textbf{Reshaping}\\
A \emph{reshape} operation is an isomorphism defined on a tensor space $\bbt = \displaystyle \prod_{k=1}^r \bbv_k$ such that some of its basis vector spaces $\{\bbv_k\}$ are de-flattened and some of its sliced subspaces are flattened.
\end{definition}

\begin{definition}\textbf{Contraction}\\
A \emph{tensor contraction} between two tensors, along ranks of same dimensions, is defined by natural extension of the dot product operation to tensors.

More precisely, let $\bbt_1$ a tensor space of shape $n_1^{(1)} \times n_2^{(1)} \times \cdots \times n_{r_1}^{(1)}$, and $\bbt_2$ a tensor space of shape $n_1^{(2)} \times n_2^{(2)} \times \cdots \times n_{r_2}^{(2)}$, such that $\forall k \in \{1, 2, \ldots, s\}, n_{r_1-(s-k)}^{(1)} = n_k^{(2)}$, then the tensor contraction between $t_1 \in \bbt_1$ and $t_2 \in \bbt_2$ is defined as:
\begin{gather*}
\left\{
  \begin{array}{l}
    t_1 \otimes t_2 = t_3 \in \bbt_3 \text{ of shape } n_1^{(1)} \times \cdots \times n_{r_1-s}^{(1)} \times n_{s+1}^{(2)} \times \cdots \times n_{r_2}^{(2)}
    \text{ where} \\
    t_3[i_1^{(1)}, \ldots, i_{r_1-s}^{(1)}, i_{s+1}^{(2)}, \ldots, i_{r_2}^{(2)}] = \\
    %\displaystyle \sum_{k_1, \ldots, k_s}
    \displaystyle \sum_{k_1=1}^{n_1^{(2)}} \cdots \sum_{k_s=1}^{n_s^{(2)}}
    t_1[i_1^{(1)}, \ldots, i_{r_1-s}^{(1)}, k_1, \ldots, k_s] \hspace{2pt}
    t_2[k_1, \ldots, k_s, i_{s+1}^{(2)}, \ldots, i_{r_2}^{(2)}]
  \end{array}
\right.
\end{gather*}
\end{definition}

For the sake of simplicity, we omit the case where the contracted ranks are not the last ones for $t_1$ and the first ones for $t_2$. But this definition still holds in the general case subject to a permutation of the indices.

\begin{definition}\textbf{Covariant and contravariant indices}\\
Given a tensor contraction $t_1 \otimes t_2$, indices of the left hand operand $t_1$ that are not contracted are called \emph{covariant} indices. Those that are contracted are called \emph{contravariant} indices. For the right operand $t_2$, the naming convention is the opposite. 
The set of covariant and contravariant indices of both operands are called the \emph{transformation laws} of the tensor contraction.
\end{definition}

\begin{remark}\textbf{Transformation law independency}\\
Contrary to mathematical definitions, tensors in deep learning are independent of any transformation law, so that they must be specified for tensor contractions.
\end{remark}

\begin{remark}\textbf{Einstein summation convention}\\
Using subscript notation for covariant indices and superscript notation for contravariant indices, the previous tensor contraction can be written using the Einstein summation convention as:
\begin{gather}
t_1 \hspace{0pt}_{i_1^{(1)} \cdots i_{r_1-s}^{(1)} } \hspace{0pt}^{ k_1 \cdots k_s} 
t_2 \hspace{0pt}_{ k_1^{\phantom{(}} \cdots k_s^{\phantom{(}}} \hspace{0pt}^{i_{s+1}^{(2)} \cdots i_{r_2}^{(2)}} =
t_3 \hspace{0pt}_ {i_1^{(1)} \cdots i_{r_1-s}^{(1)} } \hspace{0pt}^{i_{s+1}^{(2)} \cdots i_{r_2}^{(2)}}
\label{indices}
\end{gather}
Dot product $u_k v^k = \lambda $ and matrix product $A_i\hspace{0pt}^k B_k\hspace{0pt}^j = C_i\hspace{0pt}^j$ are common examples of tensor contractions.
\end{remark}

%Maybe prove it as a proposition
\begin{remark}\textbf{Matrix product equivalence}\\
Using reshapings that groups all covariant indices into a single index and all contravariant indices into another single index, any tensor contraction can be rewritten as a matrix product.
\label{rq:matprodeq}
\end{remark}
\begin{proof}
Using notation of \eqref{indices}, with the reshapings $t_1 \mapsto T_1$, $t_2 \mapsto T_2$ and $t_3 \mapsto T_3$ defined as suggested in the remark, we can rewrite
$$
T_1 \hspace{0pt}_{g_i(i_1^{(1)}, \ldots, i_{r_1-s}^{(1)})} \hspace{0pt}^{g_k(k_1, \ldots, k_s)} 
T_2 \hspace{0pt}_{g_k(k_1^{\phantom{(}}, \ldots, k_s^{\phantom{(}})} \hspace{0pt}^{g_j(i_{s+1}^{(2)}, \ldots, i_{r_2}^{(2)})} =
T_3 \hspace{0pt}_ {g_i(i_1^{(1)}, \ldots, i_{r_1-s}^{(1)})} \hspace{0pt}^{g_j(i_{s+1}^{(2)}, \ldots, i_{r_2}^{(2)})}
$$
where $g_i$, $g_k$ and $g_j$ are bijections defined similarly as in \eqref{rowmajor}.
\end{proof}

\begin{definition}\textbf{Convolution}\\
The \emph{$n$-dimensional convolution}, denoted $\ast_n$, between $t_1 \in \bbt_1$ and $t_2 \in \bbt_2$, where $\bbt_1$ and $\bbt_2$ are of the same rank $n$ such that $\forall p \in \{1, 2, \ldots, n\}, n_p^{(1)} \ge n_p^{(2)}$, is defined as:
\begin{gather*}
\left\{
  \begin{array}{l}
    t_1 \ast_n t_2 = t_3 \in  \bbt_3 \text{ of shape } n_1^{(3)} \times \cdots \times n_n^{(3)}
    \text{ where} \\
    \forall p \in \{1, 2, \ldots, n\}, n_p^{(3)} = n_p^{(1)} - n_p^{(2)} + 1 \\
    t_3[i_1, \ldots, i_n] =
    \displaystyle \sum_{k_1=1}^{n_1^{(2)}} \cdots \sum_{k_n=1}^{n_n^{(2)}}
    t_1[i_1 + n_1^{(2)} - k_1, \ldots, i_n + n_n^{(2)} - k_n] \hspace{2pt} t_2[k_1, \ldots, k_n] \\
  \end{array}
\right.
\end{gather*}
\label{convdef}
\end{definition}

\begin{definition}\textbf{Pooling}\\

\todo{Use indexing, maybe add stride in this subsection rather than in the next}

\end{definition}
\newpage

\subsection{Neural Networks}

A neural network could originally be formalized as a composite function chaining linear and non-linear functions~\citep{hornik1989multilayer, lecun1995convolutional}, even up until important breakthroughs in the field of computer vision \citep{krizhevsky2012imagenet, simonyan2014very}. However, in more recent advances, more complex architectures have emerged \citep{he2016deep,zoph2016neural,huang2017densely}, such that the former formalization does not suffice. We provide a definition for the first kind of neural networks (\defref{def:nn}) and use it to present its related concepts. Then we give a more generic definition (\defref{def:nn2}).

Note that in this manuscript, we will only consider neural networks that are \emph{feed forward} \citep{wiki:fnn, zell1994simulation}.

\subsubsection{Description}

We denote by $I_f$ the \textit{domain of definition} of a function $f$ ("I" stands for "input") and by $O_f = f(I_f)$ its \textit{image} ("O" stands for "output"), and we represent it as $I_f~\xrightarrow{f}~O_f$.

\begin{definition}\textbf{Neural network (simply connected)}\\
Let $F$ be a function such that $I_f$ and $O_f$ are vector or tensor spaces.\\
$F$ is a \emph{functional formulation} of a \emph{simply connected neural network} if there are a series of linear or affine functions $(g_k)_{k=1,2,..,L}$ and a series of non-linear derivable univariate functions $(h_k)_{k=1,2,..,L}$ such that:
\begin{gather*}
\left\{
  \begin{array}{l}
    \forall k \in \{1, 2, \ldots, L\}, f_k = h_k \circ g_k, \\
    I_F = I_{f_1} \xrightarrow{f_1} O_{f_1} \cong I_{f_2} \xrightarrow{f_2} \dots \xrightarrow{f_L} O_{f_L} = O_F, \\
    F = f_{L} \circ ... \circ f_{2} \circ f_1
  \end{array}
\right.
\end{gather*}
The couple $(g_k, h_k)$ is called the \emph{$k$-th layer} of the neural network.
For $x \in I_f$, we denote by $x_k = f_k \circ ... \circ f_{2} \circ f_1 (x)$ the \emph{activations} of the $k$-th layer.
\label{def:nn}
\end{definition}

\todo{introduce what is their purpose ie classifiers, why is training, and make a little plan of what follows}
\todo{remarks on universal approximators and refs, overfitting, generalization}

\begin{definition}\textbf{Activation function}\\
Let a layer $(g,h)$. $h$ is called the \emph{activation function} of the layer. It is non-linear, derivable and univariate. Of common use for univariate functions is the functional notation $h(v)[i_1, i_2, \ldots, i_r] = h(v[i_1, i_2, \ldots, i_r])$.
\end{definition}

\begin{remark}\textbf{Example of activation functions}\\
\todo{blabla: refs activation functions}
\end{remark}

\begin{definition}\textbf{Layer}\\
A couple $(g,h)$, where $g$ is an affine or linear function, and $h$ is an activation function is called a \emph{layer}. The set of layers is denoted~$\cl$.
\end{definition}

\begin{remark}\textbf{Bias}\\
Affine functions $\widetilde{g}$ can be written as a sum between a linear function $g$ and a constant vector $b$ which is called the \emph{bias}. Its role is to augment the expressivity of the neural network's family of functions. For notational conveniency, we will omit the biases in the rest of this section and thus only consider linear functions.
\end{remark}

\begin{definition}\textbf{Connectivity matrix}\\
Let $g$ a linear function. Without loss of generality subject to a flattening, let's suppose $I_g$ and $O_g$ are vector spaces. Then there exists a \emph{connectivity matrix}~$W_g$, such that:
\begin{gather*}
\forall x \in I_g, g(x) = W_g x
\end{gather*}
\end{definition}
We denote $W_k$ the connectivity matrix of the $k$-th layer.

\begin{remark}\textbf{Biological inspiration}\\
A \emph{(computational) neuron} is a computational unit that is biologically inspired. Each neuron should be capable of:
\begin{enumerate}
\item receiving modulated signals from other neurons and aggregate them,
\item applying to the result a derivable activation,
\item passing the signal to other neurons.
\end{enumerate}
That is to say, each domain $\{I_{f_k}\}$ and $O_F$ can be interpreted as a layer of neurons, with one neuron for each dimension. The connectivity matrices $\{W_k\}$ describe the connexions between each successive layers.
%The parameters of $\Theta_g$ are the modulation weights that characterize the connections.
A neuron is illustrated on \figref{fig:neuron}.
\end{remark}

\begin{figure}[H]
\centering
\begin{tikzpicture}
\draw (0,0) -- (4,0) -- (4,4) -- (0,4) -- (0,0);
\node at (2,2){placeholder};
\end{tikzpicture}
\caption{A neuron}
\label{fig:neuron}
\end{figure}

The former neural networks are said to be \emph{simply connected} because each layer only takes as input the output of the previous one. We'll give a more general definition after first defining branching operations.

\begin{definition}\textbf{Branching}\\
A \emph{binary branching operation} of a neural network is an operation between two activations, $x_{k_1} \Join x_{k_2}$, that outputs, subject to shape compatibility, either their addition, either their concatenation along a rank, or their concatenation as a list.

A \emph{branching operation} of a neural network between $n$ activations, $x_{k_1} \Join x_{k_2} \Join \cdots \Join x_{k_n}$, is a composition of binary branching operations, or is the identity function $Id$ if $n = 1$.
\end{definition}

\begin{definition}\textbf{Neural network (generic definition)}\\
The set of \emph{neural network functions} $\cn$ is defined inductively as follow
%\begin{gather*}
\begin{enumerate}
%\left\{
  %\begin{array}{l}
  \item $Id \in \cn$
  \item $f \in \cn \wedge (g,h) \in \cl \wedge O_f \subset I_g \Rightarrow h \circ g \circ f \in \cn$
  %\text{for all shape compatible branching operation:}\\
  \item for all shape compatible branching operations:\\
  $\quad f_1, f_2, \ldots, f_n \in \cn \Rightarrow  f_1 \Join f_2 \Join \cdots \Join f_n \in \cn$
  %\end{array}
%\right.
\end{enumerate}
%\end{gather*}
\label{def:nn2}
\end{definition}

\begin{remark}\textbf{Examples}\\
\todo{blabla: residual connections, skip connections,branching layers}
\label{rq:branching_ex}
\end{remark}

\subsubsection{Training}

\todo{blabla}

\begin{definition}\textbf{Weights}\\
Let consider the $k$-th layer of a neural networks. We define its weights as coordinates of a vector $\theta_k$, called the \emph{weight kernel}, such that:
\begin{gather*}
  \forall (i,j),
    \begin{cases}
      \exists p, W_k[i,j] := \theta_k[p] \\
      \text{ or } W_k[i,j] = 0
    \end{cases}
\end{gather*}
\end{definition}
A weight $p$ that appears multiple times in $W_k$ is said to be \emph{shared}. Two parameters of $W_k$ that share a same weight $p$ are said to be \emph{tied}. The number of weights of the $k$-th layer is $n_1^{(\theta_k)}$.

\begin{remark}\textbf{Learning}\\
A \emph{loss} function $\mathcal{L}$ penalizes the output $x_L = F(x)$ relatively to what can be expected. Gradient w.r.t.~$\theta_k$, denoted $\vec{\bigtriangledown}_{\theta_k}$, is used to update the weights via an optimization algorithm based on gradient descent and a learning rate $\alpha$, that is:
\begin{gather}
\theta_k^{(\text{new})} = \theta_k^{(\text{old})} - \alpha \cdot \vec{\bigtriangledown}_{\theta_k} \left( \mathcal{L}\left( x_L, \theta_k^{(\text{old})} \right) + R\left( \theta_k^{(\text{old})} \right) \right)
\end{gather}
where $\alpha$~can be a scalar or a vector, $\cdot$~can denote outer or pointwise product, and $R$~is a regularizer. They depend on the optimization algorithm.
\end{remark}

\todo{examples of optimizations}

\begin{remark}\textbf{Linear complexity}\\
{The complexity of computing the gradients is linear with the number of weights.}
\begin{proof}
Without loss of generality, we assume that the neural network is simply connected. Thanks to the chain rule, $\vec{\bigtriangledown}_{\theta_k}$ can be computed using gradients that are w.r.t. $x_k$, denoted $\vec{\bigtriangledown}_{x_k}$, which in turn can be computed using gradients w.r.t. outputs of the next layer $k+1$, up to the gradients given on the output layer.

That is:
\begin{align}
  \vec{\bigtriangledown}_{\theta_k} & = J_{\theta_k}(x_k) \vec{\bigtriangledown}_{x_k} \\
  \begin{split}
  \vec{\bigtriangledown}_{x_k} & = J_{x_k}(x_{k+1}) \vec{\bigtriangledown}_{x_{k+1}} \\
  \vec{\bigtriangledown}_{x_{k+1}} & = J_{x_{k+1}}(x_{k+2}) \vec{\bigtriangledown}_{x_{k+2}} \\
  \quad \quad \ldots\\
  \vec{\bigtriangledown}_{x_{L-1}} & = J_{x_{L-1}}(x_{L}) \vec{\bigtriangledown}_{x_{L}}
  \label{eq:bp}
  \end{split}
\end{align}
Obtaining,
\begin{align}
  \vec{\bigtriangledown}_{\theta_k} = J_{\theta_k}(x_k) (\prod_{p=k}^{L-1} J_{x_p}(x_{p+1})) \vec{\bigtriangledown}_{x_L}
\end{align}
where $J_{\text{wrt}}(.)$ are the respective jacobians which can be determined with the layer's expressions and the $\{x_k\}$; and $\vec{\bigtriangledown}_{x_L}$ can be determined using $\mathcal{L}$, $R$ and $x_L$.
\end{proof}
This allows to compute the gradients with a complexity that is linear with the number of weights (only one computation of the activations), instead of being quadratic if it were done with the difference quotient expression of the derivatives (one more computation of the activations for each weight).
\end{remark}

\begin{remark}\textbf{Back propagation}\\
We can remark that \eqref{eq:bp} rewrites as
\begin{align}
  \begin{split}
  \vec{\bigtriangledown}_{x_k} & = J_{x_k}(x_{k+1}) \vec{\bigtriangledown}_{x_{k+1}} \\ 
                               & = J_{x'_k}(h(x'_k)) J_{x_k}(W_k x_k) \vec{\bigtriangledown}_{x_{k+1}}
  \end{split}
\end{align}
where $x'_k = W_k x_k$, and these jacobians can be expressed as:
\begin{align}
  \begin{split}
  J_{x'_k}(h(x'_k)) & [i,j] = \delta_i^j h'(x'_k[i])\\
  J_{x'_k}(h(x'_k)) & = I \hspace{2pt} h'(x'_k)
  \end{split}\\
  J_{x_k}(W_k x_k) & = W_k^T
\end{align}
That means that we can write $\vec{\bigtriangledown}_{x_k} = (\widetilde{h}_k \circ \widetilde{g}_k)(\vec{\bigtriangledown}_{x_{k+1}})$ such that the connectivity matrix $\widetilde{W}_k$ is obtained by transposition. This can be interpreted as gradient calculation being a \emph{back-propagation} on the same neural network, in opposition of the \emph{forward-propagation} done to compute the output.
\end{remark}

\subsubsection{Example of layers}

\begin{definition}\textbf{Connections}\\
The set of \emph{connections} of a layer $(g,h)$, denoted $C_g$, is defined as:
\begin{gather*}
  C_g = \{(i,j), \exists p, W_g[i,j] := \theta_g[p]\}
\end{gather*}
We have $0 \leq |C_g| \leq n_1^{(W_g)} n_2^{(W_g)}$.
\end{definition}

\begin{definition}\textbf{Dense layer}\\
A \emph{dense layer} $(g,h)$ is a layer such that $|C_g| = n_1^{(W_g)} n_2^{(W_g)}$, \ie all possible connections exist. The map $(i,j) \mapsto p$ is usually a bijection, meaning that there is no weight sharing.
\end{definition}

\begin{definition}\textbf{Partially connected layer}\\
A \emph{partially connected layer} $(g,h)$ is a layer such that $|C_g| < n_1^{(W_g)} n_2^{(W_g)}$.

A \emph{sparsely connected layer} $(g,h)$ is a layer such that $|C_g| \ll n_1^{(W_g)} n_2^{(W_g)}$.
\end{definition}

\begin{definition}\textbf{Convolutional layer}\\
A \emph{$n$-dimensional convolutional layer} $(g,h)$ is such that the weight kernel~$\theta_g$ can be reshaped into a tensor $w$ of rank $n+2$, and such that
$$
\left\{
\begin{array}{l}
  I_g \mbox{ and } O_g \mbox{ are tensor spaces of rank }n+1 \\
  \forall x \in I_g, g(x) = (g(x)_q = \sum\limits_p{x_p \ast^n w_{p,q}})_{\forall q}
\end{array}
\right.
$$
where $p$ and $q$ index slices along the last ranks.
\label{def:convlayer}
\end{definition}

\begin{definition}\textbf{Feature maps and input channels}\\
The slices $g(x)_q$ are typically called \textit{feature maps}, and the slices $x_p$ are called \textit{input channels}. Let's denote by $n_o = n_{n+1}^{(O_g)}$ and $n_i =n_{n+1}^{(I_g)}$ the number of feature maps and input channels.
In other words, \defref{def:convlayer} means that for each feature maps, a convolution layer computes $n_i$ convolutions and sums them, computing a total if $n_i \times n_o$ convolutions.
\end{definition}

\begin{remark}
Note that because they are simply summed, entries of two different input channels that have the same coordinates are assumed to share some sort of relationship. For instance on images, entries of each input channel (typically corresponding to Red, Green and Blue channels) that have the same coordinates share the same pixel location.
\end{remark}

\begin{remark}
Given a tensor input $x$, the $n$-dimensional convolutions between the inputs channels $x_p$ and slices of a weight tensor $w_{p,q}$ would result in outputs $y_q$ of shape $n_1^{(x)} - n_1^{(w)} + 1 \times \ldots \times n_n^{(x)} - n_n^{(w)} + 1$. So, in order to preserve shapes, a padding operation must pad $x$ with $n_1^{(w)} - 1 \times \ldots \times n_n^{(w)} - 1$ zeros beforehand. For example, the padding function of the library \emph{tensorflow}~\citep{tensorflow2015-whitepaper} pads each rank with a balanced number of zeros on the left and right indices (except if $n_t^{(w)} - 1$ is odd then there is one more zero on the left).
\end{remark}

\begin{definition}\textbf{Padding}\\
A convolutional layer with \emph{padding} $(g, h)$ is such that $g$ can be decomposed as $g = g_\text{pad} \circ g'$, where $g'$ is the linear part of a convolution layer as in \defref{def:convlayer}, and $g_\text{pad}$ is an operation that pads zeros to its inputs such that $g$ preserves tensor shapes.
\end{definition}

\begin{remark}
One asset of padding operations is that they limit the possible loss of information on the borders of the subsequent convolutions, as well as preventing a decrease in size. Moreover, preserving shape is needed to build some neural network architectures, especially for ones with branching operations \eg \remref{rq:branching_ex}. On the other hand, they increase memory and computational footprints.
\end{remark}

\begin{proposition}\textbf{Connectivity matrix of a convolution with padding}\\
A convolutional layer with padding $(g, h)$ is equivalently defined as $W_g$ being a $n_i \times n_o$ block matrix such that its blocks are Toeplitz matrices.
\end{proposition}

\begin{proof}
Let's consider the slices indexed by $p$ and $q$, and to simplify the notations, let's drop the subscripts $\hspace{0pt}_{p,q}$. We recall from \defref{def:convdef} that
\begin{align*}
  y &= (x \ast^n w)[j_1, \ldots, j_n] \\
 &= \displaystyle \sum_{k_1=1}^{n_1^{(w)}} \cdots \sum_{k_n=1}^{n_n^{(w)}}
    x[j_1 + n_1^{(w)} - k_1, \ldots, j_n + n_n^{(w)} - k_n] \hspace{2pt} w[k_1, \ldots, k_n] \\
 &= \displaystyle \sum_{i_1=j_1}^{j_1 + n_1^{(w)} - 1} \cdots \sum_{i_n=j_n}^{j_n + n_n^{(w)} - 1}
    x[i_1, \ldots, i_n] \hspace{2pt} w[j_1 + n_1^{(w)} - i_1, \ldots, j_n + n_n^{(w)} - i_n] \\
 &= \displaystyle \sum_{i_1=1}^{n_1^{(x)}} \cdots \sum_{i_n=1}^{n_n^{(x)}}
    x[i_1, \ldots, i_n] \hspace{2pt} \widetilde{w}[i_1, j_1, \ldots, i_n, j_n] \\
 & \text{ where } \widetilde{w}[i_1, j_1, \ldots, i_n, j_n] = \\
 & \quad \quad
 \begin{cases}
   w[j_1 + n_1^{(w)} - i_1, \ldots, j_n + n_n^{(w)} - i_n] & \text{if } \forall t, 0 \le i_t - j_t \le n_t^{(w)} - 1 \\
   0 & \text{otherwise}
 \end{cases}
\end{align*}
Using Einstein summation convention as in~\eqref{indices} and permuting indices, we recognize the following tensor contraction
\begin{align}
y_{j_1 \cdots j_n} = x_{i_1 \cdots i_n} \widetilde{w} \hspace{1pt}^{i_1 \cdots i_n} \hspace{0pt}_{j_1 \cdots j_n} \label{eq:toep1}
\end{align}
Following \propref{prop:matprodeq}, we reshape~\eqref{eq:toep1} as a matrix product. To reshape $y \mapsto Y$, we use the row major order bijections $g_j$ as in~\eqref{rowmajor} defined onto $\{(j_1, \ldots, j_n), \forall t, 1 \le j_t \le n_t^{(y)}\}$. To reshape $x \mapsto X$, we use the same row major order bijection $g_j$, however defined on the indices that support non zero-padded values, so that zero-padded values are lost after reshaping. That is, we use a bijection $g_i$ such that $g_i(i_1, i_2, \ldots, i_n) = g_j(i_1 - o_1, i_2 - o_2, \ldots, i_n - o_n)$ defined if and only if $\forall t, 1 + o_t \le i_t \le n_t^{(y)}$, where the $\{o_t\}$ are the starting offsets of the non zero-padded values. $\widetilde{w} \mapsto W$ is reshaped by using $g_j$ for its covariant indices, and $g_i$ for its contravariant indices. The entries lost by using $g_i$ do not matter because they would have been nullified by the resulting matrix product. We remark that $W$ is exactly the block $(p,q)$ of $W_g$ (and not of $W_{g'}$). Now let's prove that it is a Toeplitz matrix.

Thanks to the linearity of the expression~\eqref{rowmajor} of $g_j$, by denoting $i'_t = i_t - o_t$, we obtain
\begin{gather}
  g_i(i_1, i_2, \ldots, i_n) - g_j(j_1, j_2, \ldots, j_n) = g_j(i'_1 - j_1, i'_2 - j_2, \ldots, i'_n - j_n)
\label{eq:toep2}
\end{gather}

To simplify the notations, let's drop the arguments of $g_i$ and $g_j$. By bijectivity of $g_j$, \eqref{eq:toep2} tells us that $g_i - g_j$ remains constant if and only if $i'_t - j_t$ remains constant for all $t$. Recall that 
\begin{gather}
  W[g_i,g_j] =
 \begin{cases}
   w[j_1 + n_1^{(w)} - i'_1, \ldots, j_n + n_n^{(w)} - i'_n] & \text{if } \forall t, 0 \le i'_t - j_t \le n_t^{(w)} - 1 \\
   0 & \text{otherwise}
 \end{cases}
\label{eq:toep3}
\end{gather}
Hence, on a diagonal of $W$, $g_i - g_j$ remaining constant means that $W[g_i,g_j]$ also remains constants. So $W$ is a Toeplitz matrix.

The converse is also true as we used invertible functions in the index spaces through the proof.
\end{proof}

\begin{remark}
The former proof makes clear that the result doesn't hold in case there is no padding. This is due to border effects when the index of the $n$\powth rank resets in the definition of the row-major ordering function $g_j$ that would be used. Indeed, under appropriate definitions, the matrices could be seen as almost Toeplitz.
\end{remark}

\begin{remark}
Comparitively with dense layers, convolution layers enjoy a significant decrease in the number of weights. For example, an input $2 \times 2$ convolution on images with $3$-color input channels, would breed only $12$ weights per feature maps, independently of the numbers of input neurons. On image datsets, their usage also breeds a significant boost in performance compared with dense layers~\citep{krizhevsky2012imagenet}, for they allow to take advantage of the topology of the inputs while dense layers don't~\citep{lecun1995convolutional}. A more thorough comparison and explanation of their assets will be discussed in \secref{sec:gnn}.
\end{remark}

\begin{definition}\textbf{Stride}\\
A convolutional layer with \emph{stride} is a convolutional layer that computes strided convolutions (with stride $> 1$) instead of convolutions.
\end{definition}

\begin{definition}\textbf{Pooling}\\
A layer with \textit{pooling} $(g,h)$ is such that $g$ can be decomposed as $g = g' \circ g_\text{pool}$, where $g_\text{pool}$ is a pooling operation.
\end{definition}

\begin{remark}\textbf{Downscaling}\\
Layers with stride or pooling downscale the signals that passes through the layer. These types of layers allows to compute features at a coarser level, giving the intuition that the deeper a layer is in the network, the more abstract are the infomations captured by the weights of the layer.
\end{remark}

\todo{below}

\subsubsection{Example of regularizations}

A layer with \textit{dropout} $(g,h)$ is such that $h = h_1 \circ h_2$, where $(g,h_2)$ is a layer and $h_1$ is a dropout operation~\citep{srivastava2014dropout}. When dropout is used, a certain number of neurons are randomly set to zero during the training phase, compensated at test time by scaling down the whole layer. This is done to prevent overfitting.

\subsubsection{Example of architectures}
\label{sec:nn_arch}

\todo{rephrase}

A multilayer perceptron (MLP)~\citep{hornik1989multilayer} is a neural network composed of only dense layers.
A convolutional neural network (CNN)~\citep{lecun1998gradient} is a neural network composed of convolutional layers.

Neural networks are commonly used for machine learning tasks. For example, to perform supervised classification, we usually add a dense output layer $s=(g_{L+1},h_{L+1})$ with as many neurons as classes. We measure the error between an output and its expected output with a discriminative loss function $\mathcal{L}$. During the training phase, the weights of the network are adapted for the classification task based on the errors that are back-propagated~\citep{hornik1989multilayer} via the chain rule and according to a chosen optimization algorithm (\eg \cite{bottou2010large}).

\newpage

\subsection{Graphs}

\todo{check this subsection}

A graph $G$ is defined as a couple $(V,E)$ where $V$ represents the set of nodes and $E \subseteq\binom{V}{2}$ is the set of edges connecting these nodes.

\todo{Example of figure}

We encounter the notion of graphs several times in deep learning:
\begin{itemize}
\item Connections between two layers of a deep learning model can be represented as a bipartite graph, coined \emph{connectivity graph}. It encodes how the information is propagated through a layer to another. See section~\ref{con_graph}.
\item A computation graph is used by deep learning frameworks to keep track of the dependencies between layers of a deep learning models, in order to compute forward and back-propagation. See section~\ref{comp_graph}.
\item A graph can represent the underlying structure of an object (often a vector), whose nodes represent its features. See section~\ref{inductive_graph}.
\item Datasets can also be graph-structured, where the nodes represent the objects of the dataset. See section~\ref{transductive_graph}.
\end{itemize}

\subsubsection{Connectivity graph}
\label{con_graph}

A Connectivity graph is the bipartite graph whose adjacency matrix is the connectivity matrix of a layer of neurons.
%$U = \{u_1, u_2, \ldots, u_n\}$
Formally, given a linear part of a layer, let $\textbf{x}$ and $\textbf{y}$ be the input and output signals, $n$ the size of the set of input neurons $N = \{u_1, u_2, \ldots, u_n\}$, and $m$ the size of the set of output neurons $M = \{v_1, v_2, \ldots, v_m\}$. This layer implements the equation $y = \Theta x$ where $\Theta$ is a $n \times m$ matrix.

\begin{definition}
{The \emph{connectivity graph} $G = (V,E)$ is defined such that $V = N \cup M$ and $E = \{(u_i,v_j) \in  N \times M, \Theta_{ij} \neq 0 \} $.}
\end{definition}

I.e. the connectivity graph is obtained by drawing an edge between neurons for which $\Theta_{ij} \neq 0$.
For instance, in the special case of a complete bipartite graph, we would obtain a dense layer. 
Connectivity graphs are especially useful to represent partially connected layers, for which most of the $\Theta_{ij}$ are $0$. 
For example, in the case of layers characterized by a small local receptive field, the connectivity graph would be sparse, and output neurons would be connected to a set of input neurons that corresponds to features that are close together in the input space. Figure~\ref{con_ex} depicts some examples.

\begin{figure}[h]
  \begin{center}
    \begin{tikzpicture}
      \tikzstyle{every node} = [draw, circle, thick, inner sep = 2pt]
      \foreach \y in {0,...,4}{
        \pgfmathtruncatemacro{\yplusone}{5 - \y}
        \node(a\y) at (0,.6*\y) {\footnotesize\yplusone};
      }
      \foreach \y in {0,...,4}{
        \pgfmathtruncatemacro{\yplusone}{5 - \y}
        \node(\y) at (2,.6*\y) {\footnotesize\yplusone};
      }

      \foreach \x in {0,...,4}{
        \foreach \y in {0,...,4}{
          \path[opacity=0.5] (a\x) edge (\y);
        }
      }
    \end{tikzpicture}
  \end{center}
  \caption{Examples}
  \label{con_ex}
\end{figure}

\todo{Figure~\ref{con_ex}. It's just a placeholder right now}


Connectivity graphs also allow to graphically modelize how weights are tied in a neural layer. Let's suppose the $\Theta_ij$ are taking their values only into the finite set $K = \{w_1, w_2, \ldots, w_\kappa\}$ of size $\kappa$, which we will refer to as the \emph{kernel} of \emph{weights}. Then we can define a labelling of the edges $s: E \rightarrow K$. $s$ is called the \emph{weight sharing scheme} of the layer. This layer can then be formulated as $\displaystyle \forall v \in M, y_v = \sum_{u \in N, (u,v) \in E} w_{s(u,v)} x_u$. Figure~\ref{cnn} depicts the connectivity graph of a 1-d convolution layer and its weight sharing scheme.

\begin{figure}[h]
  \begin{center}
    \begin{tikzpicture}
      \tikzstyle{every node} = [draw, circle, thick, inner sep = 2pt]
      \foreach \y in {0,...,4}{
        \pgfmathtruncatemacro{\yplusone}{5 - \y}
        \node(a\y) at (0,.6*\y) {\footnotesize\yplusone};
      }
      \foreach \y in {0,...,4}{
        \pgfmathtruncatemacro{\yplusone}{5 - \y}
        \node(\y) at (2,.6*\y) {\footnotesize\yplusone};
      }
      \path[opacity=0.5]
      (a0) edge (0);
      \path[dashed]
      (a0) edge (1);
      \path[dotted]
      (a1) edge (0);
      \path[opacity=0.5]
      (a1) edge (1);
      \path[dashed]
      (a1) edge (2);
      \path[dotted]
      (a2) edge (1);
      \path[opacity=0.5]
      (a2) edge (2);
      \path[dashed]
      (a2) edge (3);
      \path[dotted]
      (a3) edge (2);
      \path[opacity=0.5]
      (a3) edge (3);
      \path[dashed]
      (a3) edge (4);
      \path[dotted]
      (a4) edge (3);
      \path[opacity=0.5]
      (a4) edge (4);
    \end{tikzpicture}
  \end{center}
  \caption{Depiction of a 1D-convolutional layer and its weight sharing scheme.}
  \label{cnn}
\end{figure}


\todo{Add weight sharing scheme in Figure~\ref{cnn}}

\subsubsection{Computation graph}
\label{comp_graph}

\subsubsection{Underlying graph structure}
\label{inductive_graph}

\subsubsection{Graph-structured dataset}
\label{transductive_graph}

transductive vs inductive

\subsection{Special classes of graphs}

\subsubsection{Grid graphs}

\subsubsection{Spatial graphs}

\subsubsection{Projections of spatial graphs}

\newpage

%
% Chapter 3
%



%
% Chapter 4
%

%
% Chapter 5
%

%
% Chapter 6
%

%
% Bin
%

\setcounter{chapter}{-1}
\chapter{Drafts}
\textcolor{red}{TODO: Rework 1.1}

%\section{Disambiguations and definitions}

% This thesis manuscript is about deep learning on \emph{irregular domains}. So what does it mean exactly ?

%% start of <see below> comment

% The term \emph{deep learning}, as introduced in the previous chapter, refers to a family of learnable models based on deep neural networks. The inputs of these models are \emph{signals} of a specific type. Learning is made over a training dataset of such signals. Hence, the term \emph{domain} as in \emph{irregular domains} refers to the definition domain of these input signals.

%% Shoud be put later, must make disambiguation with "unstructured" as well

% In this section we recall the basic naming convention in~\ref{basic}, of some definitions in~\ref{regularity}, and categorize the models we will review by the tasks for what they are designed in~\ref{tasks}.

\subsection{Naming conventions}
\label{basic}

\subsubsection{Basic notions}

Let's recall the naming conventions of basic notions.

A \emph{function} $f: E \rightarrow F$ maps objects $x \in E$ to objects $y \in F$, as $y = f(x)$.\\
Its \emph{definition domain} $\cd_f = E$ is the set of objects onto which it is defined. We will often just use the term \emph{domain}.\\
%Objects of its domain $\cd_f$ are mapped to objects of its \emph{codomain} $\cd_f^c= F$.\\
We also say that $f$ is \emph{taking values} in its \emph{codomain} $F$.\\
The \emph{image per $f$} of the subset $U \subset E$, denoted $f(U)$, is $\{y \in F, \exists x \in E, y = f(x)\}$.\\
The \emph{image of $f$} is the image of its domain. We denote $\ci_f$.\\
% The \emph{fiber} of the object $y \in \ci_f$ is the object $x \in E$ such that $y = f(x)$.\\
% The \emph{inverse image per $f$} of the subset $V \subset F$, denoted $f^{-1}(V)$ is $\{x \in E, \exists y \in F, y = f(x)\}$.
A vector space $E$, which we will always assume to be finite-dimensional in our context, is defined as $\bbr^n$, and is equipped with pointwise addition and scalar multiplication.% TODO reword?

A \emph{signal} $s$ is a function taking values in a vector space. In other words, a signal can also be seen as a \emph{vector} with an \emph{underlying structure}, where the vector is composed from its image, and the underlying structure is defined by its \emph{domain}.\\

For example, images are signals defined on a set of pixels. Typically, an image~$s$ in RGB representation is a mapping from pixels~$p$ to a 3d vector space, as $s_p = (r,g,b)$.

\textcolor{red}{TODO?: figure}
% quadillage , arrow ->, quadrillage remplie en 3 images
\begin{figure}

\end{figure}

\subsubsection{Graphs and graph signals}

%
\textcolor{red}{TODO: more defs on grid graphs and other graphs}
% need to define covariance graph, nearest neighbour
% Need to define grid graphs, regular grids from geometry, etc ...
% A regular grid graph is a nearest neighbor graph of a regular geometric grid
%

A \emph{graph} $G = (V, E)$ is defined as a set of nodes $V$, and a set of edges $E \subseteq\binom{V}{2}$. The words \emph{node} and \emph{vertex} will be used equivalently, but we will rather use the first.

A \emph{graph signal}, or \emph{graph-structured signal} is a signal defined on the nodes of a graph, for which the underlying structure is the graph itself.
A \emph{node signal} is a signal defined on a node, in which case it is a \emph{node embedding} in a vector space.

Although this is rarely seen, a signal can also be defined on the edges of a graph, or on an edge. We then coin it respectively \emph{dual graph signal}, or \emph{edge signal} / \emph{edge embedding}.

\emph{Graph-structured data} can refer to any of these type of signals.

\subsubsection{Data and datasets}

% Adjacency matrix, laplacian, etc ...
A dataset of signals is said to be \emph{static} if all its signals share the same underlying structure, it is said to be \emph{non-static} otherwise.\\
For image datasets, being non-static would mean that the dataset contains images of different sizes or different scales. For graph signal datasets, it would mean thats the underlying graph structures of the signals are different.

The point in specifying that objects of a dataset of a machine learning task are signals is that we can hope to leverage their underlying structure.

\textcolor{red}{TODO: figure}

\subsection{Disambiguation of the subject}

This thesis is entitled \emph{Deep learning models for data without a regular structure}.
So either the data of interest in this manuscript do not have any structure, or either their structure is not regular.

\subsubsection{Irregularly structured data}

By structured data, we mean that there exists an underlying structure over which the data is defined. This kind of data are usually modelized as signals defined over a domain. These domains are then composed of objects that are related together by some sort of structural properties. For example, pixels of images can be seen as located on a grid with integer spatial coordinates (a 2d cartesian grid graph).

It then come in handy to define the notions of structure and regularity with the help of graph signals.

\begin{definition}{Structure}\\
  Let $s: D \rightarrow F$ be a signal defined over a finite domain.\\
  An \emph{underlying structure} of the signal $s$ is a graph $G$ that has the domain of $s$ for nodes.\\
  A dataset is said to be \emph{structured}, if its objects can be modelized as signals with an underlying structure.\\
  It is said to be \emph{static} if all its objects share the same underlying structure, and \emph{non-static} otherwise.
\end{definition}

In other words, we chose to define ``structured data'' as ``graph-structured data'' by some graph. Hence we need to specify for which graphs this structure would be said to be regular, and for which it would not.

\begin{definition}{Regularity}\\
An underlying structure is said to be \emph{regular}, if it is a regular grid graph.
It is said to be \emph{irregular} otherwise.\\
A dataset is said to be \emph{regularly structured}, if the underlying structures of its objects are regular.
It is said to be \emph{irregularly structured} otherwise.
\end{definition}


\textcolor{red}{TODO: examples}
%% example images , example time series, example graph signals, example manifolds

\subsubsection{Unstructured data}

Data can also be unstructured. If the data is not yet embedded into a finite dimensional vector space, then we will be interested in embedding techniques used in representation learning. In the other case, it is often possible to fall back to the case of irregularly structured data. For example, vectors can be seen as signals defined over the canonical basis of the vector space, and the vectors of this basis can be related together by their covariances through the dataset. It is typical to use the graph structure that has the canonical basis for nodes, with edges obtained by covariance thresholding.

\textcolor{red}{TODO: examples}
%% give examples of unstructured data, graphs, scramble image datasets, etc..

%%%%%%%%%%%%%%%%%%%%%%%%%%%%%%%%%%%%%%%%%%%%%%%%%%%%%%%%%% END %%%%%%%%%%%%%%%%%%%%%%%%%%%%%%%%%%%%%%%%%%%%%%%%

%
% DRAFTS
%
\textcolor{red}{What follows is a draft}


%\subsection{Theoretical results on regularity and convolutions}

% idea : regularity of a domain implies with poset
% prop: if a domain has a poset, define translations and convolutions
% proving the converse too

\subsection{Datasets}

\subsection{Tasks} %% probably too early
\label{tasks}



\subsection{Goals}

\subsection{Invariance}

In order to be observed, invariances must be defined relatively to an observation. Let's give a formal definition to support our discussion.

...

\subsection{Methods}
\label{methods}
\newpage

%
% Post body
%

% Temptative previsional plan
\addcontentsline{toc}{chapter}{Temptative previsional plan I}
\chapter*{}
\setcounter{chapter}{0}
\fakechapter{Introduction}
\fakesection{Plan, vision, etc}
\fakesection{Deep learning and history}
\fakesection{Regular deep learning}
\fakesection{Irregular deep learning}
\fakesection{Unstructured deep learning}
\fakesection{Propagational point of view}

\fakechapter{Presentation of the domain}
\fakesection{Typology of data}
\fakesection{Standardized terminology}
\fakesection{Motivation}
\fakesection{Datasets}
\fakesection{Unifying framework (tensorial product)}
\fakesection{Other Unifying frameworks}

\fakechapter{Review of models and propositions}
\fakesection{How to compare models}
\fakesection{Spectral models}
\fakesection{Non-spectral}
\fakesection{Non-convolutional}
%\fakesection{Our models I}
%\fakesection{Our models II}
\fakesection{Recap and (big) comparison table}
\fakesection{Explaining current SOA, current issues, and further work}

\fakechapter{Transposing the problem formulation: Structural learning}
\fakesection{Structural Representation}
\fakesection{Feature visualization (viz on input)}
\fakesection{Propagated Signal visualization (viz on S)}
\fakesection{Temptatives on learning S}
\fakesection{Temptatives on learning S (other)}
\fakesection{Covariance-based convolution}
\fakesection{Conclusion}

\fakechapter{Industrial applications}
\fakesection{Context}
\fakesection{The Warp 10 platform and Warpscript language}
\fakesection{Presentation of use cases: uni vs multi-variate, spatial vs geo, etc ..}
\fakesection{Review and application on regularly structured (spatial) time series}
\fakesection{Application to time series database (unstructured)}
\fakesection{Application to geo time series (unstructured)}
\fakesection{Application to visualization}
\fakesection{Market reality (what clients need, what they don't know that can be done ...)}
\fakesection{Conclusion}

\fakechapter{Conclusion}
\fakesection{Summary}
\fakesection{Lesson learned}
\fakesection{Further avenues}


% Temptative previsional plan
\addcontentsline{toc}{chapter}{Temptative previsional plan II}
\chapter*{}
\setcounter{chapter}{-1}
\fakechapter{Introduction}
\fakesection{Teaser}
\fakesection{Goals}
\fakesection{Difficulties}
\fakesection{Outline}

\fakechapter{Subject disambiguation}
\fakesection{Title}
\fakesection{Deep learning}
\fakesection{Signals, features, structure, underlying graph}
\fakesection{Regular, Irregular, Unstructured}
\fakesection{Motivation}

\fakechapter{Presentation of the field}
\fakesection{Tensors}
\fakesection{Neural netowrks}
\fakesection{Graphs}

\fakechapter{Supervised learning}
\fakesection{SotA review}
\fakesection{ours}
\fakesection{Analysis and discussion}

\fakechapter{Semi-supervised learning}
\fakesection{SotA review}
\fakesection{ours}
\fakesection{Analysis and discussion}

\fakechapter{Representation learning} %and unsupervised learning ?
\fakesection{SotA review}
\fakesection{ours}
\fakesection{Analysis and discussion}

\fakechapter{Industrial applications}
\fakesection{Context, Warp10, (Geo) Time series}
\fakesection{Supervised learning}
\fakesection{Semi-supervised learning}
\fakesection{Representation learning}
\fakesection{Market reality and perspectives}

\fakechapter{Conclusion}
\fakesection{Summary}
\fakesection{Discussion}
\fakesection{Further avenues}


% keywords and temptative titles
\begin{keywords}
Deep learning,
representation learning,
propagation learning,
visualization,
structured,
unstructured regular,
irregular,
covariant,
invariant,
equivariant,
tensor,
scheme,
weight sharing,
graphs,
manifold,
euclidean,
signal processing,
graph signal processing,
time series,
time series database,
distributed application,
spatial-time series,
geo time series,
industrial applications,
warp 10,
warpscript,
...
\end{keywords} % to be put somewhere else
\section*{Temptative titles}

\begin{itemize}
\item Learning propagational representations of irregular and unstructured data
\item Learning representations of unstructured or irregularly structured datasets
\item Propagational learning of unstructured or irregularly structured datasets
\item Learning tensorial representation of irregular and unstructured data
\item Tensorial representation of propagation in deep learning for irregular and unstructured dataset
\item Structural representation learning for irregular or unstructured data
\item Word for both ``irregularly structured'' + ``unstructured'' = ? (maybe ``unorthodox'' ?)
\item Unorthdox deep learning
\item ...
\item Deep learning of unstructured or irregularly structured datasets
\item Deep learning models for data without a regular structure
\item On structures in deep learning
\item On deep learning for when data is lacking a regular structure
\item Deep learning for non regularly structured data
\end{itemize}


% Bibliography
\printbibliography[heading=bibintoc]

\end{document}