\documentclass[12pt]{book}
\usepackage[utf8]{inputenc}
\usepackage[french, english]{babel}
\usepackage[T1]{fontenc}
\usepackage{lipsum}

\usepackage{amsfonts}
\usepackage{amsmath}
\usepackage{amssymb}
\usepackage[mathscr]{euscript}
\usepackage{stmaryrd}
\usepackage[normalem]{ulem}
\usepackage{tikz,tikz-3dplot}
\usepackage{tikz-cd}
\usepackage{pgfplots}
\usepackage{graphicx}
\usepackage{color}
\usepackage{enumitem}
\usetikzlibrary{matrix,chains,positioning,decorations.pathreplacing,arrows}
% \usetikzlibrary{external}
% \tikzexternalize[prefix=tikzext/]
% \tikzset{external/mode=graphics if exists}
\usepackage{framed}
\usepackage{arydshln}
\usepackage{multirow}
\usepackage{mathtools}
%\usepackage{float}
%\usepackage{yhmath}
\DeclareSymbolFont{yhlargesymbols}{OMX}{yhex}{m}{n}
\DeclareMathAccent{\wideparen}{\mathord}{yhlargesymbols}{"F3}
\usepackage{floatrow}
\usepackage{pdfpages}
\usepackage{minitoc}
\setcounter{secnumdepth}{3}
\setcounter{tocdepth}{3}
\setcounter{minitocdepth}{2}
%\renewcommand{\mtctitle}{~} % Empty minitoc titles
\usepackage[nobottomtitles]{titlesec}
\usepackage{etoolbox}
\makeatletter
\patchcmd{\ttlh@hang}{\parindent\z@}{\parindent\z@\leavevmode}{}{}
\patchcmd{\ttlh@hang}{\noindent}{}{}{}
\makeatother

%\usepackage{quotchap}
\usepackage{csquotes}

% Abstract
\newcommand\abstractname{Abstract}  %%% here
\makeatletter
\if@titlepage
  \newenvironment{abstract}{%
      \titlepage
      \null\vfil
      \@beginparpenalty\@lowpenalty
      \begin{center}%
        \bfseries \abstractname
        \@endparpenalty\@M
      \end{center}}%
     {\par\vfil\null\endtitlepage}
\else
  \newenvironment{abstract}{%
      \if@twocolumn
        \section*{\abstractname}%
      \else
        \small
        \begin{center}%
          {\bfseries \abstractname\vspace{-.5em}\vspace{\z@}}%
        \end{center}%
        \quotation
      \fi}
      {\if@twocolumn\else\endquotation\fi}
\fi
\makeatother

% References
\usepackage{hyperref}
\usepackage{enumitem}
\makeatletter
\def\namedlabel#1#2{\begingroup
    #2%
    \def\@currentlabel{#2}%
    \phantomsection\label{#1}\endgroup
}
\makeatother

% Bibliography
\usepackage[backend=bibtex,
            sorting=nyt, %or ynt?
            style=authoryear,
            natbib=true,
            maxcitenames=2,
            mincitenames=1,
            maxbibnames=99,
            backref=true]
            {biblatex}
\addbibresource{refs/datasets.bib}
\addbibresource{refs/dl_history.bib}
\addbibresource{refs/dl_understanding.bib}
\addbibresource{refs/dl_activations.bib}
\addbibresource{refs/dl_bastnet.bib}
\addbibresource{refs/dl_manifold.bib}
\addbibresource{refs/dl_vertex.bib}
\addbibresource{refs/dl_vertex_old.bib}
\addbibresource{refs/dl_spectral.bib}
\addbibresource{refs/gsp.bib}
\addbibresource{refs/toClassify.bib}
\addbibresource{refs/maths.bib}
\addbibresource{refs/scattering.bib}
\addbibresource{refs/prog_languages.bib}
\addbibresource{refs/online.bib}
\addbibresource{refs/wordvec.bib}

% Style
\setlength\parindent{0pt}
\newcommand{\subsubsubsection}[1]{\paragraph{#1}\mbox{}\\}
\newcommand{\subsubsubsubsection}[1]{\subparagraph{#1}\mbox{}\\}

\usepackage{setspace}
\onehalfspacing % or
%\doublespacing

% numbering lines
\usepackage[left]{lineno}
%linenumbers
%\modulolinenumbers[2]

\newcommand*\patchAmsMathEnvironmentForLineno[1]{%
  \expandafter\let\csname old#1\expandafter\endcsname\csname #1\endcsname
  \expandafter\let\csname oldend#1\expandafter\endcsname\csname end#1\endcsname
  \renewenvironment{#1}%
     {\linenomath\csname old#1\endcsname}%
     {\csname oldend#1\endcsname\endlinenomath}}% 
\newcommand*\patchBothAmsMathEnvironmentsForLineno[1]{%
  \patchAmsMathEnvironmentForLineno{#1}%
  \patchAmsMathEnvironmentForLineno{#1*}}%
\AtBeginDocument{%
\patchBothAmsMathEnvironmentsForLineno{equation}%
\patchBothAmsMathEnvironmentsForLineno{align}%
\patchBothAmsMathEnvironmentsForLineno{flalign}%
\patchBothAmsMathEnvironmentsForLineno{alignat}%
\patchBothAmsMathEnvironmentsForLineno{gather}%
\patchBothAmsMathEnvironmentsForLineno{multline}%
}

% linebreaks in math mode
%\binoppenalty=\maxdimen %700
%\relpenalty=\maxdimen %500

% chapter style
\makeatletter
\def\@makechapterhead#1{%
  %%%%\vspace*{50\p@}% %%% removed!
  \vspace*{0\p@}
  {\parindent \z@ \raggedright \normalfont
    \ifnum \c@secnumdepth >\m@ne
        \huge\bfseries \@chapapp\space \thechapter
        \par\nobreak
        \vskip 0\p@
    \fi
    \interlinepenalty\@M
    \Huge \bfseries #1\par\nobreak
    \vskip 20\p@
  }}
\def\@makeschapterhead#1{%
  %%%%%\vspace*{50\p@}% %%% removed!
  \vspace*{50\p@}
  {\parindent \z@ \raggedright
    \normalfont
    \interlinepenalty\@M
    \Huge \bfseries  #1\par\nobreak
    \vskip 40\p@
  }}
\makeatother

% Paragraphs
\makeatletter
\renewcommand\paragraph{\@startsection{paragraph}{4}{\z@}%
                                    {3.25ex \@plus1ex \@minus.2ex}%
                                    {0.01pt}%
                                    {\normalfont\normalsize\bfseries}}
\makeatother

%\setlength{\parskip}{5pt}

% Figures, Table numbering
\usepackage{chngcntr}
\counterwithout{table}{chapter}
\counterwithout{figure}{chapter}

% Theorems
\usepackage{amsthm}
\theoremstyle{definition}
\newtheorem{definition}{Definition}%[chapter]
\newtheorem{proposition}[definition]{Proposition}
\newtheorem{corrolary}[definition]{Corrolary}
\newtheorem{lemma}[definition]{Lemma}
\newtheorem{claim}[definition]{Claim}

\theoremstyle{remark}
%\newtheorem{remark}[definition]{Remark}
\newtheorem*{remark}{Remark}

\theoremstyle{plain}

\allowdisplaybreaks[1]
\usepackage{etoolbox}% http://ctan.org/pkg/etoolbox
\usepackage{needspace}% http://ctan.org/pkg/needspace
\AtBeginEnvironment{definition}{\Needspace{5\baselineskip}}% \break if fewer than 5\baselineskip is available on page
\AtBeginEnvironment{proposition}{\Needspace{5\baselineskip}}
\AtBeginEnvironment{corrolary}{\Needspace{5\baselineskip}}
\AtBeginEnvironment{lemma}{\Needspace{5\baselineskip}}
\AtBeginEnvironment{claim}{\Needspace{5\baselineskip}}
\AtBeginEnvironment{remark}{\Needspace{5\baselineskip}}

% Annotations
\newcommand{\todo}[1]{\textcolor{red}{TODO: #1\\}}
\newcommand{\etodo}{\emph{\textcolor{red}{todo}}}

% Maths
\usepackage{chngcntr}
\counterwithout{equation}{chapter}

\usepackage{mathbbol}
\newcommand{\bb}{\mathbb{b}}
%\newcommand{\cc}{\mathbb{c}}
\newcommand{\uu}{\mathbb{u}}
\newcommand{\vv}{\mathbb{v}}
\newcommand{\bbe}{\mathbb{E}}
\newcommand{\bbi}{\mathbb{I}}
\newcommand{\bbn}{\mathbb{N}}
\newcommand{\bbr}{\mathbb{R}}
\newcommand{\bbt}{\mathbb{T}}
\newcommand{\bbv}{\mathbb{V}}
\newcommand{\bbu}{\mathbb{U}}
\newcommand{\bbz}{\mathbb{Z}}

\newcommand{\ca}{\mathcal{A}}
\newcommand{\cb}{\mathcal{B}}
\newcommand{\cc}{\mathcal{C}}
\newcommand{\cd}{\mathcal{D}}
\newcommand{\ce}{\mathcal{E}}
\newcommand{\cf}{\mathcal{F}}
\newcommand{\cg}{\mathcal{G}}
\newcommand{\ch}{\mathcal{H}}
\newcommand{\ci}{\mathcal{I}}
\newcommand{\cj}{\mathcal{J}}
\newcommand{\ck}{\mathcal{K}}
\newcommand{\cl}{\mathcal{L}}
\newcommand{\cm}{\mathcal{M}}
\newcommand{\cn}{\mathcal{N}}
\newcommand{\co}{\mathcal{O}}
\newcommand{\cp}{\mathcal{P}}
\newcommand{\cq}{\mathcal{Q}}
\newcommand{\ccr}{\mathcal{R}}
\newcommand{\cs}{\mathcal{S}}
\newcommand{\ct}{\mathcal{T}}
\newcommand{\cu}{\mathcal{U}}
\newcommand{\cv}{\mathcal{V}}
\newcommand{\cW}{\mathcal{W}}
\newcommand{\cx}{\mathcal{X}}
\newcommand{\cy}{\mathcal{Y}}
\newcommand{\cz}{\mathcal{Z}}

\newcommand{\seq}[1]{\{1, 2, \ldots, #1\}}
\newcommand{\sq}[1]{\{1, \ldots, #1\}}
\newcommand{\group}{\mathcal{G}}

\DeclareMathOperator{\diag}{diag}
\DeclareMathOperator{\off}{off}
\DeclareMathOperator{\order}{order}
\DeclareMathOperator{\tree}{trav}
%\DeclareMathOperator{\deg}{deg}
%\DeclareMathOperator{\dim}{dim}
\DeclareMathOperator{\shape}{shape}
\DeclareMathOperator{\supp}{supp}
\DeclareMathOperator{\EC}{\textsc{ec}}
\DeclareMathOperator{\LRF}{\textsc{lrf}}
\DeclareMathOperator{\ER}{\textsc{er}}
\DeclareMathOperator{\OR}{\textsc{or}}
\DeclareMathOperator{\XOR}{\textsc{xor}}
\DeclareMathOperator{\AND}{\textsc{and}}
\DeclareMathOperator{\agg}{\textsc{aggregate}}
%\DeclareMathOperator{\def}{def}
\DeclareMathOperator{\id}{Id}
\DeclareMathOperator{\I}{\textsc{i}}
\DeclareMathOperator{\II}{\textsc{ii}}
\DeclareMathOperator{\III}{\textsc{iii}}
\DeclareMathOperator{\M}{\textsc{m}}
\DeclareMathOperator{\C}{\textsc{c}}
\DeclareMathOperator{\T}{\mathsf{T}}
\DeclareMathOperator{\iso}{\textsc{iso}}
\DeclareMathOperator{\bij}{\textsc{bij}}
\DeclareMathOperator{\D}{\textsc{d}}
\DeclareMathOperator{\AUT}{\textsc{aut}}

\DeclareMathOperator{\scr}{\textsc{R}}
\DeclareMathOperator{\scs}{\textsc{S}}

\DeclareMathOperator{\IN}{\textsc{in}}
\DeclareMathOperator{\OUT}{\textsc{out}}

% Acronyms
\newcommand{\iid}{\emph{i.i.d.}~}
\newcommand{\etal}{\emph{et al.}~}
\newcommand{\ie}{\emph{i.e.}~}
\newcommand{\st}{\emph{s.t.}~}
\newcommand{\eg}{\emph{e.g.}~}
\newcommand{\powth}{\text{$^\text{th}$~}}
\newcommand{\wrt}{\emph{w.r.t.}~}
\newcommand{\nn}{\nonumber}
\newcommand{\cdl}{\cd_{\ltimes}}

% References
\newcommand{\figref}[1]{Figure~\ref{#1}}
\newcommand{\chapref}[1]{Chapter~\ref{#1}}
\newcommand{\appref}[1]{Appendix~\ref{#1}}
\newcommand{\secref}[1]{Section~\ref{#1}}
\newcommand{\algref}[1]{Algorithm~\ref{#1}}
\newcommand{\thref}[1]{Theorem~\ref{#1}}
\newcommand{\propref}[1]{Proposition~\ref{#1}}
\newcommand{\remref}[1]{Remark~\ref{#1}}
\newcommand{\claref}[1]{Claim~\ref{#1}}
\newcommand{\rqref}[1]{Remark~\ref{#1}}
\newcommand{\defref}[1]{Definition~\ref{#1}}
\newcommand{\corref}[1]{Corrolary~\ref{#1}}
\newcommand{\lemref}[1]{Lemma~\ref{#1}}
\newcommand{\conjref}[1]{Conjecture~\ref{#1}}
\newcommand{\probref}[1]{Problem~\ref{#1}}
\newcommand{\quoref}[1]{Quote~\ref{#1}}
\newcommand{\tabref}[1]{Table~\ref{#1}}
%\newcommand{\eqref}[1]{(\ref{#1})} %already defined

\newcommand{\gve}{G = \langle V, E \rangle}
\newcommand{\vgve}{\vec{G} = \langle V, E \rangle}

% hspaces
\newcommand{\h}[1]{\hspace{#1pt}}

% keywords
\newcommand{\keywords}[1]{\textbf{\textit{Index terms---}} #1}

% quotes
\newcommand{\quotes}[1]{``#1''}

% For temptative plans
\newcommand{\fakechapter}[1]{%
  \par\refstepcounter{chapter}% Increase subsection counter
  \chaptermark{#1}% Add subsection mark (header)
  \addcontentsline{toc}{chapter}{\protect\numberline{\thechapter}#1}% Add subsection to ToC
}

\newcommand{\fakesection}[1]{%
  \par\refstepcounter{section}% Increase section counter
  \sectionmark{#1}% Add section mark (header)
  \addcontentsline{toc}{section}{\protect\numberline{\thesection}#1}% Add section to ToC
}

\newcommand{\fakesubsection}[1]{%
  \par\refstepcounter{subsection}% Increase section counter
  \subsectionmark{#1}% Add section mark (header)
  \addcontentsline{toc}{subsection}{\protect\numberline{\thesubsection}#1}% Add section to ToC
}

\begin{document}
\selectlanguage{english}

%
% Title
%

% ?
% (On) Deep Learning for non-regularly structured data
% ... see temptative titles chapter

%
% Table of contents
%

\dominitoc
\tableofcontents

%
% Introduction
%

%\chapter*{Introduction}
%\input{}

%
% Chapter 1
%

%\chapter{Subject disambiguation}

%
% Chapter 2
%

\setcounter{chapter}{1}
\chapter{Presentation of the field}

In this section, we present notions related to our domains of interest. In particular, for tensors we give original definitions that are more appropriate for our study. In the neural network's section, we present the concepts necessary to understand the evolution of the state of the art research in this field. In the last section, we present graphs for their usage in deep learning.

Vector spaces considered in what follows are assumed to be finite-dimensional and over the field of real numbers $\bbr$.

\vfill
\minitoc
\newpage

\section{Tensors}

Intuitively, tensors in the field of deep learning are defined as a generalization of vectors and matrices, as if vectors were tensors of rank $1$ and matrices were tensors of rank $2$. That is, they are objects in a vector space and their dimensions are indexed using as many indices as their rank, so that they can be represented by multidimensional arrays. In mathematics, a tensor can be defined as a special type of multilinear function~\citep{bass1968cours, marcus1975finite, williamson2015tensor}, which image of a basis can be represented by a multidimensional array. Alternatively, Hackbush propose a mathematical construction of a tensor space as a quotient set of the span of an appropriately defined tensor product \citep{hackbusch2012tensor}, which coordinates in a basis can also be represented by a multidimensional array. In particular in the field of mathematics, tensors enjoy an intrinsic definition that neither depend on a representation nor would change the underlying object after a change of basis, whereas in our domain, tensors are confounded with their representation.

\subsection{Definition}

Our definition of tensors is such that they are a bit more than multidimensional arrays but not as much as mathematical tensors, for that they are embedded in a vector space so that deep learning objects can be later defined rigorously.

Given canonical bases, we first define a tensor space, then we relate it to the definition of the tensor product of vector spaces.

\begin{definition}\textbf{Tensor space}\\
We define a \emph{tensor space} $\bbt$ of rank $r$ as a vector space such that its canonical basis is a cartesian product of the canonical bases of $r$ other vector spaces.

Its shape is denoted $n_1 \times n_2 \times \cdots \times n_r$, where the $\{n_k\}$ are the dimensions of the vector spaces.
\end{definition}

\begin{definition}\textbf{Tensor product of vector spaces}\\
Given $r$ vector spaces $\bbv_1, \bbv_2, \ldots, \bbv_r$, their \emph{tensor product} is the tensor space~$\bbt$ spanned by the cartesian product of their canonical bases under coordinate-wise sum and outer product.

We use the notation $\bbt = \displaystyle \bigotimes_{k=1}^r \bbv_k$.
\end{definition}

\begin{remark}
This simpler definition is indeed equivalent with the definition of the tensor product given in~(\cite{hackbusch2012tensor}, p. 51). The drawback of our definition is that it depends on the canonical bases, which at first can seem limiting as being canon implies that they are bounded to a certain system of coordinates. However this is not a concern in our domain as we need not distinguish tensors from their representation.
\end{remark}

\begin{remark}
For naming convenience, from now on, we will distinguish between the terms \emph{linear space} and \emph{vector space} \ie we will abusively use the term \emph{vector space} only to refer to a linear space that is seen as a tensor space of rank $1$. If we don't know its rank, we rather use the term \emph{linear space}.
We also make a clear distinction between the terms \emph{dimension} (that is, for a tensor space it is equal to $\displaystyle \prod_{k=1}^r n_k$) and the term \emph{rank} (equal to $r$). Note that some authors use the term \emph{order} instead of \emph{rank} (\eg \cite{hackbusch2012tensor}) as the latter is affected to another notion.
\end{remark}

\begin{definition}\textbf{Tensor}\\
A \emph{tensor} $t$ is an object of a tensor space. The \emph{shape} of $t$, which is the same as the shape of the tensor space it belongs to, is denoted $n_1^{(t)} \times n_2^{(t)} \times \cdots \times n_r^{(t)}$.
\end{definition}

%\subsection{Comparison with other definitions} % todo ?
%
\subsection{Manipulation}

In this subsection, we describe notations and operators used to manipulate data stored in tensors.

\begin{definition}\textbf{Indexing}\\
An \emph{entry} of a tensor $t \in \bbt$ is one of its scalar coordinates in the canonical basis, denoted $t[i_1, i_2, \ldots, i_r]$.

More precisely, if $\bbt = \displaystyle \bigotimes_{k=1}^r \bbv_k$, with bases $((e_k^i)_{i=1,\ldots,n_k})_{k=1,\ldots,r}$, then we have
\begin{gather*}
t =  \displaystyle \sum_{i_1=1}^{n_1} \cdots \sum_{i_r=1}^{n_r} t[i_1, i_2, \ldots, i_r] (e_1^{i_1}, \ldots, e_r^{i_r})
\end{gather*}

The cartesian product $\bbi=\displaystyle \prod_{k=1}^r \llbracket 1, n_k \rrbracket$ is called the \emph{index space} of~$\bbt$
\end{definition}

\begin{remark}
When using an index $i_k$ for an entry of a tensor $t$, we implicitly assume that $i_k \in \llbracket 1, n_k^{(t)} \rrbracket$ if nothing is specified.
\end{remark}

\begin{definition}\textbf{Subtensor}\\
A \emph{subtensor} $t'$ is a tensor of same rank composed of entries of $t$ that are contiguous in the indexing, with at least one entry per rank. We denote $t' = t[l_1{:}u_1, l_2{:}u_2, \ldots, l_r{:}u_r]$, where the $\{l_k\}$ and the $\{u_k\}$ are the lower and upper bounds of the indices used by the entries that compose~$t'$.
\end{definition}

\begin{remark}
We don't necessarily write the lower bound index if it is equal to $1$, neither the upper bound index if it is equal to $n_k^{(t)}$.
\end{remark}

\begin{definition}\textbf{Slicing}\\
A \emph{slice} operation, along the last ranks $\{r_1, r_2, \ldots, r_s\}$, and indexed by $(i_{r_1}, i_{r_2}, \ldots, i_{r_s})$, is a morphism $s: \bbt = \displaystyle \bigotimes_{k=1}^r \bbv_k \rightarrow \displaystyle \bigotimes_{k=1}^{r-s} \bbv_k$, such that:
\begin{align*}
s(t)[i'_1, i'_2, \ldots, i'_{r-s}] &= t[i'_1, i'_2, \ldots, i'_{r-s}, i_{r_1}, i_{r_2}, \ldots, i_{r_s}] \\
\text{ \ie } \quad s(t) :&= t[:,:, \ldots, :, i_{r_1}, i_{r_2}, \ldots, i_{r_s}]
\end{align*}
where $:=$ means that entries of the right operand are assigned to the left operand.
We denote $t_{i_{r_1}, i_{r_2}, \ldots i_{r_s}}$ and call it the \emph{slice} of $t$. 
Slicing along a subset of ranks that are not the lasts is defined similarly.
$s(\bbt)$ is called a \emph{slice subspace}.
\end{definition}

\begin{definition}\textbf{Flattening}\\
A \emph{flatten} operation is an isomorphism $f: \bbt \rightarrow \bbv$, between a tensor space $\bbt$ of rank~$r$ and an $n$-dimensional vector space $\bbv$, where $n =\displaystyle \prod_{k=1}^r n_k$. It is characterized by a bijection in the index spaces $g: \displaystyle \prod_{k=1}^r \llbracket 1, n_k \rrbracket \rightarrow \llbracket 1, n \rrbracket$ such that
\begin{gather*}
  \forall t \in \bbt, f(t)[g(i_1, i_2, \ldots, i_r)] = f(t[i_1, i_2, \ldots, i_r])
\end{gather*}

We call an inverse operation a \emph{de-flatten} operation.
\end{definition}

\begin{remark}\textbf{Row major ordering}\\
The choice of $g$ determines in which order the indexing is made. $g$ is reminiscent of how data of multidimensional arrays or tensors are stored internally by programming languages. In most tensor manipulation languages, incrementing the memory address (\ie the output of $g$) will first increment the last index $i_r$ if $i_r < n_r$ (and if else $i_r = n_r$, then $i_r := 1$ and ranks are ordered in reverse lexicographic order to decide what indices are incremented). This is called \emph{row major ordering}, as opposed to \emph{column major ordering}. That is, in row major, $g$ is defined as
\begin{align}
  g(i_1, i_2, \ldots, i_r) = \displaystyle \sum_{p=1}^r \left( \prod_{k=p+1}^r n_k \right) i_p \label{eq:rowmajor}
\end{align}
\end{remark}

\begin{definition}\textbf{Reshaping}\\
A \emph{reshape} operation is an isomorphism defined on a tensor space $\bbt = \displaystyle \bigotimes_{k=1}^r \bbv_k$ such that some of its basis vector spaces $\{\bbv_k\}$ are de-flattened and some of its slice subspaces are flattened.
\end{definition}

\subsection{Binary operations}

We define binary operations on tensors that we'll later have use for. In particular, we define  \emph{tensor contraction} which is sometimes called \emph{tensor multiplication}, \emph{tensor product} or \emph{tensor dotproduct} by other sources. We also define \emph{convolution} and \emph{pooling} which serve as the common building blocks of convolution neural network architectures (see \secref{sec:nn_arch}).

\begin{definition}\textbf{Contraction}\\
A \emph{tensor contraction} between two tensors, along ranks of same dimensions, is defined by natural extension of the dot product operation to tensors.

More precisely, let $\bbt_1$ a tensor space of shape $n_1^{(1)} \times n_2^{(1)} \times \cdots \times n_{r_1}^{(1)}$, and $\bbt_2$ a tensor space of shape $n_1^{(2)} \times n_2^{(2)} \times \cdots \times n_{r_2}^{(2)}$, such that $\forall k \in \llbracket 1, s \rrbracket, n_{r_1-(s-k)}^{(1)} = n_k^{(2)}$, then the tensor contraction between $t_1 \in \bbt_1$ and $t_2 \in \bbt_2$ is defined as:
\begin{gather*}
\left\{
  \begin{array}{l}
    t_1 \otimes t_2 = t_3 \in \bbt_3 \text{ of shape } n_1^{(1)} \times \cdots \times n_{r_1-s}^{(1)} \times n_{s+1}^{(2)} \times \cdots \times n_{r_2}^{(2)}
    \text{ where} \\
    t_3[i_1^{(1)}, \ldots, i_{r_1-s}^{(1)}, i_{s+1}^{(2)}, \ldots, i_{r_2}^{(2)}] = \\
    %\displaystyle \sum_{k_1, \ldots, k_s}
    \displaystyle \sum_{k_1=1}^{n_1^{(2)}} \cdots \sum_{k_s=1}^{n_s^{(2)}}
    t_1[i_1^{(1)}, \ldots, i_{r_1-s}^{(1)}, k_1, \ldots, k_s] \hspace{2pt}
    t_2[k_1, \ldots, k_s, i_{s+1}^{(2)}, \ldots, i_{r_2}^{(2)}]
  \end{array}
\right.
\end{gather*}
\end{definition}

For the sake of simplicity, we omit the case where the contracted ranks are not the last ones for $t_1$ and the first ones for $t_2$. But this definition still holds in the general case subject to a permutation of the indices.

\begin{definition}\textbf{Covariant and contravariant indices}\\
Given a tensor contraction $t_1 \otimes t_2$, indices of the left hand operand $t_1$ that are not contracted are called \emph{covariant} indices. Those that are contracted are called \emph{contravariant} indices. For the right operand $t_2$, the naming convention is the opposite. 
The set of covariant and contravariant indices of both operands are called the \emph{transformation laws} of the tensor contraction.
\end{definition}

\begin{remark}\textbf{Transformation law independency}\\
Contrary to most mathematical definitions, tensors in deep learning are independent of any transformation law, so that they must be specified for tensor contractions.
\end{remark}

\begin{remark}\textbf{Einstein summation convention}\\
Using subscript notation for covariant indices and superscript notation for contravariant indices, the previous tensor contraction can be written using the Einstein summation convention as:
\begin{gather}
t_1 \hspace{0pt}_{i_1^{(1)} \cdots i_{r_1-s}^{(1)} } \hspace{0pt}^{ k_1 \cdots k_s} 
t_2 \hspace{0pt}_{ k_1^{\phantom{(}} \cdots k_s^{\phantom{(}}} \hspace{0pt}^{i_{s+1}^{(2)} \cdots i_{r_2}^{(2)}} =
t_3 \hspace{0pt}_ {i_1^{(1)} \cdots i_{r_1-s}^{(1)} } \hspace{0pt}^{i_{s+1}^{(2)} \cdots i_{r_2}^{(2)}}
\label{eq:indices}
\end{gather}
Dot product $u_k v^k = \lambda $ and matrix product $A_i\hspace{0pt}^k B_k\hspace{0pt}^j = C_i\hspace{0pt}^j$ are common examples of tensor contractions.
\end{remark}

\begin{proposition}%\textbf{Matrix product equivalence I}\\
A contraction can be rewritten as a matrix product.
\label{prop:matprodeq}
\end{proposition}
\begin{proof}
Using notation of \eqref{eq:indices}, with the reshapings $t_1 \mapsto T_1$, $t_2 \mapsto T_2$ and $t_3 \mapsto T_3$ defined by grouping all covariant indices into a single index and all contravariant indices into another single index, we can rewrite
\begin{gather*}
T_1 \hspace{0pt}_{g_i(i_1^{(1)}, \ldots, i_{r_1-s}^{(1)})} \hspace{0pt}^{g_k(k_1, \ldots, k_s)} 
T_2 \hspace{0pt}_{g_k(k_1^{\phantom{(}}, \ldots, k_s^{\phantom{(}})} \hspace{0pt}^{g_j(i_{s+1}^{(2)}, \ldots, i_{r_2}^{(2)})} =
T_3 \hspace{0pt}_ {g_i(i_1^{(1)}, \ldots, i_{r_1-s}^{(1)})} \hspace{0pt}^{g_j(i_{s+1}^{(2)}, \ldots, i_{r_2}^{(2)})}
\end{gather*}
where $g_i$, $g_k$ and $g_j$ are bijections defined similarly as in \eqref{eq:rowmajor}.
\end{proof}

\begin{definition}\textbf{Convolution}\\
The \emph{$n$-dimensional convolution}, denoted $\ast^n$, between $t_1 \in \bbt_1$ and $t_2 \in \bbt_2$, where $\bbt_1$ and $\bbt_2$ are of the same rank $n$ such that $\forall p \in \llbracket 1, n \rrbracket, n_p^{(1)} \ge n_p^{(2)}$, is defined as:
\begin{gather*}
\left\{
  \begin{array}{l}
    t_1 \ast^n t_2 = t_3 \in  \bbt_3 \text{ of shape } n_1^{(3)} \times \cdots \times n_n^{(3)}
    \text{ where} \\
    \forall p \in \llbracket 1, n \rrbracket, n_p^{(3)} = n_p^{(1)} - n_p^{(2)} + 1 \\
    t_3[i_1, \ldots, i_n] =
    \displaystyle \sum_{k_1=1}^{n_1^{(2)}} \cdots \sum_{k_n=1}^{n_n^{(2)}}
    t_1[i_1 + n_1^{(2)} - k_1, \ldots, i_n + n_n^{(2)} - k_n] \hspace{2pt} t_2[k_1, \ldots, k_n] \\
  \end{array}
\right.
\end{gather*}
\label{def:convdef}
\end{definition}

\begin{proposition}%\textbf{Matrix product equivalence II}\\
A convolution can be rewritten as a matrix product.
\label{prop:matprodeq2}
\end{proposition}

\begin{proof}
Let $t_1 \ast^n t_2 = t_3$ defined as previously with $\bbt_1 = \displaystyle \bigotimes_{k=1}^r \bbv_k^{(1)}$, $\bbt_2 = \displaystyle \bigotimes_{k=1}^r \bbv_k^{(2)}$. Let $t'_1 \in \displaystyle \bigotimes_{k=1}^r \bbv_k^{(1)} \otimes \displaystyle \bigotimes_{k=1}^r \bbv_k^{(2)}$ such that $t'_1[i_1, \ldots, i_n, k_1, \ldots, k_n] = t_1[i_1 + n_1^{(2)} - k_1, \ldots, i_n + n_n^{(2)} - k_n]$, then
\begin{gather*}
t_3[i_1, \ldots, i_n] =
    \displaystyle \sum_{k_1=1}^{n_1^{(2)}} \cdots \sum_{k_n=1}^{n_n^{(2)}}
    t'_1[i_1, \ldots, i_n, k_1, \ldots, k_n] \hspace{2pt} t_2[k_1, \ldots, k_n]
\end{gather*}
where we recognize a tensor contraction. \propref{prop:matprodeq} concludes.
\end{proof}

The two following operations are meant to further decrease the shape of the resulting output.

\begin{definition}\textbf{Strided convolution}\\
The $n$-dimensional \emph{strided} convolution, with strides $s = (s_1, s_2, \ldots, s_n)$, denoted $\ast^n_s$, between $t_1 \in \bbt_1$ and $t_2 \in \bbt_2$, where $\bbt_1$ and $\bbt_2$ are of the same rank $n$ such that $\forall p \in \llbracket 1, n \rrbracket, n_p^{(1)} \ge n_p^{(2)}$, is defined as:
\begin{gather*}
\left\{
  \begin{array}{l}
    t_1 \ast^n_s t_2 = t_4 \in  \bbt_4 \text{ of shape } n_1^{(4)} \times \cdots \times n_n^{(4)}
    \text{ where} \\
    \forall p \in \llbracket 1, n \rrbracket, n_p^{(4)} = \lfloor\frac{n_p^{(1)} - n_p^{(2)} + 1}{s_p}\rfloor \\
    t_4[i_1, \ldots, i_n] = (t_1 \ast^n t_2)[(i_1 - 1)s_n + 1, \ldots, (i_n - 1)s_n + 1]\\
  \end{array}
\right.
\end{gather*}
\end{definition}

\begin{remark}
Informally, a strided convolution is defined as if it were a regular subsampling of a convolution. They match if $s = (1,1,\ldots,1)$.
\end{remark}

\begin{definition}\textbf{Pooling}\\
Let a real-valued function $f$ defined on all tensor spaces of any shape, \eg the \emph{max} or \emph{average} function.
An $f$-pooling operation is a mapping $t \mapsto t'$ such that each entry of $t'$ is an image by $f$ of a subtensor of $t$.
\end{definition}

\begin{remark}
Usually, the set of subtensors that are reduced by $f$ into entries of $t'$ are defined by a regular partition of the entries of $t$.
\end{remark}
\newpage

\section{Neural Networks}

A feed-forward neural network could originally be formalized as a composite function chaining linear and non-linear functions \citep{rumelhart1985learning,lecun1989backpropagation,lecun1995convolutional}, even up until the important breakthroughs that generated a surge of interest in the field \citep{hinton2012deep,krizhevsky2012imagenet,simonyan2014very}. However, in more recent advances, more complex architectures have emerged \citep{szegedy2015going,he2016deep,zoph2016neural,huang2017densely}, such that the former formalization does not suffice. We provide a definition for the first kind of neural networks (\defref{def:nn}) and use it to present its related concepts. Then we give a more generic definition (\defref{def:nn2}).

Note that in this manuscript, we only consider neural networks that are \emph{feed-forward} \citep{zell1994simulation, wiki:fnn}.

\subsection{Simple formalization}

We denote by $I_f$ the \textit{domain of definition} of a function $f$ ("I" stands for "input") and by $O_f = f(I_f)$ its \textit{image} ("O" stands for "output"), and we represent it as $I_f~\xrightarrow{f}~O_f$.

\begin{definition}\textbf{Neural network (simply connected)}\\
Let $f$ be a function such that $I_f$ and $O_f$ are vector or tensor spaces.\\
$f$ is a \emph{(simply connected) neural network function} if there are a series of affine functions $(g_k)_{k=1,2,..,L}$ and a series of non-linear derivable univariate functions $(h_k)_{k=1,2,..,L}$ such that:
\begin{gather*}
\left\{
  \begin{array}{l}
    \forall k \in \llbracket 1, L \rrbracket, f_k = h_k \circ g_k, \\
    I_f = I_{f_1} \xrightarrow{f_1} O_{f_1} \cong I_{f_2} \xrightarrow{f_2} \dots \xrightarrow{f_L} O_{f_L} = O_f, \\
    f = f_{L} \circ ... \circ f_{2} \circ f_1
  \end{array}
\right.
\end{gather*}
The couple $(g_k, h_k)$ is called the \emph{$k$-th layer} of the neural network. $L$ is its depth.
For $x \in I_f$, we denote by $x_k = f_k \circ ... \circ f_{2} \circ f_1 (x)$ the \emph{activations} of the $k$-th layer. We denote by $\cn$ the set of neural network functions.
\label{def:nn}
\end{definition}

\begin{definition}\textbf{Activation function}\\
An \emph{activation function} $h$ is a real-valued univariate function that is non-linear and derivable, that is also defined by extension on any linear space with the functional notation $h(v)[i] = h(v[i])$.
\end{definition}
%
% \begin{figure}[H]
% \centering
% \begin{tikzpicture}
% \draw (0,0) -- (4,0) -- (4,4) -- (0,4) -- (0,0);
% \node at (2,2){placeholder};
% \end{tikzpicture}
% \caption{reLU activation function}
% \label{fig:relu}
% \end{figure}

\begin{definition}\textbf{Layer}\\
A couple $(g,h)$, where $g$ is an affine or linear function, and $h$ is an activation function is called a \emph{layer}. The set of layers is denoted~$\cl$.
\end{definition}

\begin{remark}\textbf{Adoption of ReLU activations}\\
Historically, sigmoidal and tanh activations were mostly used \citep{cybenko1989approximation, lecun1989backpropagation}. However in recent practice, the \emph{rectified linear unit} (ReLU), which implements the \emph{rectifier} function $h: x \mapsto max(0,x)$ with convention $h'(0) = 0$ (first introduced as the \emph{positive part}, \cite{jarrett2009best}), is the most used activation, as it was demonstrated to be faster and to obtain better results \citep{glorot2011deep}. ReLU originated numerous variants \eg \emph{leaky rectified linear unit} \citep{maas2013rectifier}, \emph{parametric rectified linear unit} (PReLU, \cite{he2015delving}), \emph{exponential linear unit} (ELU, \cite{clevert2015fast}), \emph{scaled exponential linear unit} (SELU, \cite{Klambauer2017self}).
\end{remark}

\begin{remark}\textbf{Universal approximation}\\
Early researches have shown that neural networks with one level of depth can approximate any real-valued function defined on a compact subset of~$\bbr^n$. This result was first proved for sigmoidal activations \citep{cybenko1989approximation}, and then it was shown it did not depend on the sigmoidal activations \citep{hornik1989multilayer,hornik1991approximation}.
\end{remark}

For example, for the application of supervised learning when a neural network is trained from data (see \secref{sec:training}), this result is quite important because it brings theoretical justification that the objective exists (even though it doesn't inform whether an algorithm to approach it exists or is efficient).

\begin{remark}\textbf{Computational difficulty}\\
However, reaching such objective is a computationally difficult problem, which drove back interest from the field. Thanks to better hardware and to using better initialization schemes that speed up learning, researchers started to report more successes with deep neural networks \citep{hinton2006fast,glorot2010understanding} ; see \citep{bengio2009learning} for a review of this period. It ultimately came to a surge of interest in the field after a significant breakthrough on the imagenet dataset \citep{deng2009imagenet} with a deep convolutional architecture \citep{krizhevsky2012imagenet}, see \secref{sec:nn_arch}. The use of the fast ReLU activation function \citep{glorot2011deep} as well as leveraging graphical processing units with CUDA \citep{nickolls2008scalable} were also key factors in overcoming this computational difficulty.
\end{remark}

\begin{remark}\textbf{Expressivity and expressive efficiency}\\
The study of the \emph{expressivity} (also called \emph{representational power}) of families of neural networks is the field that is interested in the range of functions that can be realized or approximated by this family \citep{haastad1991power,pascanu2013number}. In general, given a maximal error~$\epsilon$ and an objective~$F$, the more expressive is a family $N \subset \cn$, the more likely it is to contain an approximation $f \in N$ such that $d(f,F) < \epsilon$. However, if we consider the approximation $f_{min} \in N$ that have the lowest number of neurons, it is possible that~$f_{min}$ is still too large and may be unpractical. For this reason, expressivity is often studied along the related notion of \emph{expressive efficiency} \citep{delalleau2011shallow,cohen2018boosting}.
\label{rem:expr}
\end{remark}

\begin{remark}\textbf{Rectifier neural netowrks}\\
Of particular interest for the intuition is a result stating that a simply connected neural networks with only ReLU activations (a rectifier neural network) is a piecewise linear function \citep{pascanu2013number,montufar2014number}, and that conversely any piecewise linear function is also a rectifier neural network such that an upper bound of its depth is logarithmically related to the input dimension (\cite{arora2018understanding}, th. 2.1.). Their expressive efficiency have also been demonstrated compared to neural networks using threshold or sigmoid activations~\citep{pan2016expressiveness}.
\end{remark}

\begin{remark}\textbf{Benefits of depth}\\
Expressive efficiency analysis have demonstrated the benefits of depth, \ie a shallow neural network would need an unfeasible large number of neurons to approximate the function of a deep neural network (\eg \cite{delalleau2011shallow,bianchini2014complexity,poggio2015theory,eldan2016power,poole2016exponential,raghu2016expressive,cohen2016convolutional,mhaskar2016learning,lin2017does,arora2018understanding}).
\end{remark}

\begin{remark}\textbf{Bias}\\
Note that affine functions $\widetilde{g}$ can be written as a sum between a linear function $g$ and a constant vector $b$ which is called the \emph{bias}. It augments the expressivity of the neural network's family of functions. For notational convenience, we will often omit to write down the biases in the layer's equations.
\end{remark}

\subsection{Generic formalization}

The former neural networks are said to be \emph{simply connected} because each layer only takes as input the output of the previous one. We'll give a more general definition after first defining branching operations.

\begin{definition}\textbf{Branching}\\
A \emph{binary branching operation} between two tensors, $x_{k_1} \Join x_{k_2}$, outputs, subject to shape compatibility, either their addition, either their concatenation along a rank, or their concatenation as a list.

A \emph{branching operation} between $n$ tensors, $x_{k_1} \Join x_{k_2} \Join \cdots \Join x_{k_n}$, is a composition of binary branching operations, or is the identity function $Id$ if $n = 1$.

Branching operations are also naturally defined on tensor-valued functions through their realizations.
\end{definition}

\begin{definition}\textbf{Neural network (generic definition)}\\
The set of \emph{neural network} functions $\cn$ is defined inductively as follows
\begin{enumerate}
  \item $Id \in \cn$
  \item $f \in \cn \wedge (g,h) \in \cl \wedge O_f \subset I_g \Rightarrow h \circ g \circ f \in \cn$
  \item for all shape compatible branching operations:\\
  $\quad f_1, f_2, \ldots, f_n \in \cn \Rightarrow  f_1 \Join f_2 \Join \cdots \Join f_n \in \cn$
\end{enumerate}
\label{def:nn2}
\end{definition}

\begin{remark}\textbf{Examples}\\
The neural network proposed in \citep{szegedy2015going}, called \emph{Inception}, use depth-wise concatenation of feature maps. Residual networks (ResNets, \cite{he2016deep}) make use of \emph{residual connections}, also called \emph{skip connections}, \ie an activation that is used as input in a lower level is added to another activation at an upper level. Densely connected networks (DenseNets, \cite{huang2017densely}) have their activations concatenated with all lower level activations. These neural networks had demonstrated state of the art performances on the imagenet classification challenge \citep{deng2009imagenet}, outperforming simply connected neural networks.
\label{rem:branching_ex}
\end{remark}

\begin{remark}\textbf{Benefits of branching operations}\\
Recent works have provided rationales supporting benefits of using branching operations, thus giving justifications for architectures obtained with the generic formalization. In particular, \citep{cohen2018boosting} have analyzed the impact of residual connections used in Wavenet-like architectures \citep{van2016wavenet} in terms of expressive efficiency (see \remref{rem:expr}) using tools from the field of tensor analysis ; \citep{orhan2018skip} have empirically demonstrated that skip connections can resolve some inefficiency problems inherent of fully-connected networks (dead activations, activations that are always equal, linearly dependent sets of activations).

\end{remark}

For layer indexing convenience, we still use the simple formalization in the subsequent subsections, even though the presentation would be similar with the generic formalization.

\subsection{Interpretation}

Until now, we have formally introduced a neural network as a mathematical function. As its name suggests, such function can be interpreted from a connectivity viewpoint \citep{lecun-87}.

\begin{definition}\textbf{Connectivity matrix}\\
Let $g$ a linear function. Without loss of generality subject to a flattening, let's suppose $I_g$ and $O_g$ are vector spaces. Then there exists a \emph{connectivity matrix}~$W_g$, such that:
\begin{gather*}
\forall x \in I_g, g(x) = W_g x
\end{gather*}
\end{definition}
We denote $W_k$ the connectivity matrix of the $k$-th layer.

\begin{remark}\textbf{Biological inspiration}\\
A \emph{(computational) neuron} is a computational unit that is biologically inspired \citep{mcculloch1943logical}. Each neuron is capable of:
\begin{enumerate}
\item receiving modulated signals from other neurons and aggregate them,
\item applying to the result a derivable activation,
\item passing the signal to other neurons.
\end{enumerate}
That is to say, each domain $\{I_{f_k}\}$ and $O_f$ can be interpreted as a layer of neurons, with one neuron for each dimension. The connectivity matrices $\{W_k\}$ describe the connections between each successive layers.
%The parameters of $\Theta_g$ are the modulation weights that characterize the connections.
A neuron is illustrated on \figref{fig:neuron}.
\end{remark}

\begin{figure}[H]
\centering
\begin{tikzpicture}
\draw (0,0) -- (4,0) -- (4,4) -- (0,4) -- (0,0);
\node at (2,2){placeholder};
\end{tikzpicture}
\caption{A neuron}
\label{fig:neuron}
\end{figure}

\subsection{Parameterization and training}
\label{sec:training}

Given an objective function $F$, training is the process of incrementally modifying a neural network $f$ upon obtaining a better approximation of $F$.
The most used training algorithms are based on gradient descent, as proposed in \citep{widrow1960adaptive}. These algorithms became popular since \citep{rumelhart1985learning}. Informally, $f$ is parameterized with initial weights that characterize its linear parts. These weights are modified step by step in the opposite direction of their gradient with respect to a loss. All possible realizations of $f$ through its weights draw a family $N$ which expressivity is crucial for the success of the training. The common points between $f$ and other objects of $N$ define what is called a neural network \emph{architecture}. That is

We present gradient based learning more formally in what follows.

\begin{definition}\textbf{Architecture}
Let $f \in \cn$ with weights $(\theta_k)_k \in $.
\end{definition}


\begin{remark}\textbf{Gradient descent}\\
The most used training algorithms are based on gradient descent, as proposed in \citep{widrow1960adaptive}. These algorithms became popular since \citep{rumelhart1985learning}. In order to be trained, $f$ is parameterized with initial weights that characterize its linear parts. These weights are modified step by step in the opposite direction of their gradient with respect to a loss.
\end{remark}

\begin{remark}\textbf{Architecture}\\
All possible realizations of $f$ through its weights draw a family $N$ which expressivity is crucial for the success of the training. The common points between $f$ and other objects of $N$ define what is called a neural network \emph{architecture}.
\end{remark}
%
%\todo{refactor}
%\todo{def training algorithm}
%The following notions describes notion related to gradient descent based training algorithms

\begin{definition}\textbf{Weights}\\
Let consider the $k$-th layer of a neural network $f$. We define its weights as coordinates of a vector $\theta_k$, called the \emph{weight kernel}, such that:
\begin{gather*}
  \forall (i,j),
    \begin{cases}
      \exists p, W_k[i,j] := \theta_k[p] \\
      \text{ or } W_k[i,j] = 0
    \end{cases}
\end{gather*}
\end{definition}
A weight $p$ that appears multiple times in $W_k$ is said to be \emph{shared}. Two parameters of $W_k$ that share a same weight $p$ are said to be \emph{tied}. The number of weights of the $k$-th layer is $n_1^{(\theta_k)}$.

\begin{remark}\textbf{Learning}\\
A \emph{loss} function $\mathcal{L}$ penalizes the output $x_L = f(x)$ relatively to the approximation error $|f(x) - F(x)|$. Gradient w.r.t.~$\theta_k$, denoted $\vec{\bigtriangledown}_{\theta_k}$, is used to update the weights via an optimization algorithm based on gradient descent and a learning rate $\alpha$, that is:
\begin{gather}
\theta_k^{(\text{new})} = \theta_k^{(\text{old})} - \alpha \cdot \vec{\bigtriangledown}_{\theta_k} \left( \mathcal{L}\left( x_L, \theta_k^{(\text{old})} \right) + R\left( \theta_k^{(\text{old})} \right) \right)
\end{gather}
where $\alpha$~can be a scalar or a vector, $\cdot$~can denote outer or pointwise product, and $R$~is a regularizer. They depend on the optimization algorithm.
\end{remark}

\todo{examples of optimization}

\begin{remark}\textbf{Linear complexity}\\
{The complexity of computing the gradients is linear with the number of weights.}
\begin{proof}
Without loss of generality, we assume that the neural network is simply connected. Thanks to the chain rule, $\vec{\bigtriangledown}_{\theta_k}$ can be computed using gradients that are w.r.t. $x_k$, denoted $\vec{\bigtriangledown}_{x_k}$, which in turn can be computed using gradients w.r.t. outputs of the next layer $k+1$, up to the gradients given on the output layer.

That is:
\begin{align}
  \vec{\bigtriangledown}_{\theta_k} & = J_{\theta_k}(x_k) \vec{\bigtriangledown}_{x_k} \\
  \begin{split}
  \vec{\bigtriangledown}_{x_k} & = J_{x_k}(x_{k+1}) \vec{\bigtriangledown}_{x_{k+1}} \\
  \vec{\bigtriangledown}_{x_{k+1}} & = J_{x_{k+1}}(x_{k+2}) \vec{\bigtriangledown}_{x_{k+2}} \\
  \quad \quad \ldots\\
  \vec{\bigtriangledown}_{x_{L-1}} & = J_{x_{L-1}}(x_{L}) \vec{\bigtriangledown}_{x_{L}}
  \label{eq:bp}
  \end{split}
\end{align}
Obtaining,
\begin{align}
  \vec{\bigtriangledown}_{\theta_k} = J_{\theta_k}(x_k) (\prod_{p=k}^{L-1} J_{x_p}(x_{p+1})) \vec{\bigtriangledown}_{x_L}
\end{align}
where $J_{\text{wrt}}(.)$ are the respective jacobians which can be determined with the layer's expressions and the $\{x_k\}$; and $\vec{\bigtriangledown}_{x_L}$ can be determined using $\mathcal{L}$, $R$ and $x_L$.
\end{proof}
This allows to compute the gradients with a complexity that is linear with the number of weights (only one computation of the activations), instead of being quadratic if it were done with the difference quotient expression of the derivatives (one more computation of the activations for each weight).
\end{remark}

\begin{remark}\textbf{Backpropagation}\\
We can remark that \eqref{eq:bp} rewrites as
\begin{align}
  \begin{split}
  \vec{\bigtriangledown}_{x_k} & = J_{x_k}(x_{k+1}) \vec{\bigtriangledown}_{x_{k+1}} \\ 
                               & = J_{x'_k}(h(x'_k)) J_{x_k}(W_k x_k) \vec{\bigtriangledown}_{x_{k+1}}
  \end{split}
\end{align}
where $x'_k = W_k x_k$, and these jacobians can be expressed as:
\begin{align}
  \begin{split}
  J_{x'_k}(h(x'_k)) & [i,j] = \delta_i^j h'(x'_k[i])\\
  J_{x'_k}(h(x'_k)) & = I \hspace{2pt} h'(x'_k)
  \end{split}\\
  J_{x_k}(W_k x_k) & = W_k^T
\end{align}
That means that we can write $\vec{\bigtriangledown}_{x_k} = (\widetilde{h}_k \circ \widetilde{g}_k)(\vec{\bigtriangledown}_{x_{k+1}})$ such that the connectivity matrix $\widetilde{W}_k$ is obtained by transposition. This can be interpreted as gradient calculation being a \emph{back-propagation} on the same neural network, in opposition of the \emph{forward-propagation} done to compute the output.
\end{remark}

\todo{Overfitting remark}

\subsection{Examples of layer}

\begin{definition}\textbf{Connections}\\
The set of \emph{connections} of a layer $(g,h)$, denoted $C_g$, is defined as:
\begin{gather*}
  C_g = \{(i,j), \exists p, W_g[i,j] := \theta_g[p]\}
\end{gather*}
We have $0 \leq |C_g| \leq n_1^{(W_g)} n_2^{(W_g)}$.
\end{definition}

\begin{definition}\textbf{Dense layer}\\
A \emph{dense layer} $(g,h)$ is a layer such that $|C_g| = n_1^{(W_g)} n_2^{(W_g)}$, \ie all possible connections exist. The map $(i,j) \mapsto p$ is usually a bijection, meaning that there is no weight sharing.
\end{definition}

\begin{definition}\textbf{Partially connected layer}\\
A \emph{partially connected layer} $(g,h)$ is a layer such that $|C_g| < n_1^{(W_g)} n_2^{(W_g)}$.

A \emph{sparsely connected layer} $(g,h)$ is a layer such that $|C_g| \ll n_1^{(W_g)} n_2^{(W_g)}$.
\end{definition}

\begin{definition}\textbf{Convolutional layer}\\
A \emph{$n$-dimensional convolutional layer} $(g,h)$ is such that the weight kernel~$\theta_g$ can be reshaped into a tensor $w$ of rank $n+2$, and such that
\begin{gather*}
\left\{
\begin{array}{l}
  I_g \mbox{ and } O_g \mbox{ are tensor spaces of rank }n+1 \\
  \forall x \in I_g, g(x) = (g(x)_q = \sum\limits_p{x_p \ast^n w_{p,q}})_{\forall q}
\end{array}
\right.
\end{gather*}
where $p$ and $q$ index slices along the last ranks.
\label{def:convlayer}
\end{definition}

\begin{definition}\textbf{Feature maps and input channels}\\
The slices $g(x)_q$ are typically called \textit{feature maps}, and the slices $x_p$ are called \textit{input channels}. Let's denote by $n_o = n_{n+1}^{(O_g)}$ and $n_i =n_{n+1}^{(I_g)}$ the number of feature maps and input channels.
In other words, \defref{def:convlayer} means that for each feature maps, a convolution layer computes $n_i$ convolutions and sums them, computing a total if $n_i \times n_o$ convolutions.
\end{definition}

\begin{remark}
Note that because they are simply summed, entries of two different input channels that have the same coordinates are assumed to share some sort of relationship. For instance on images, entries of each input channel (typically corresponding to Red, Green and Blue channels) that have the same coordinates share the same pixel location.
\end{remark}

\begin{remark}
Given a tensor input $x$, the $n$-dimensional convolutions between the inputs channels $x_p$ and slices of a weight tensor $w_{p,q}$ would result in outputs $y_q$ of shape $n_1^{(x)} - n_1^{(w)} + 1 \times \ldots \times n_n^{(x)} - n_n^{(w)} + 1$. So, in order to preserve shapes, a padding operation must pad $x$ with $n_1^{(w)} - 1 \times \ldots \times n_n^{(w)} - 1$ zeros beforehand. For example, the padding function of the library \emph{tensorflow}~\citep{tensorflow2015-whitepaper} pads each rank with a balanced number of zeros on the left and right indices (except if $n_t^{(w)} - 1$ is odd then there is one more zero on the left).
\end{remark}

\begin{definition}\textbf{Padding}\\
A convolutional layer with \emph{padding} $(g, h)$ is such that $g$ can be decomposed as $g = g_\text{pad} \circ g'$, where $g'$ is the linear part of a convolution layer as in \defref{def:convlayer}, and $g_\text{pad}$ is an operation that pads zeros to its inputs such that $g$ preserves tensor shapes.
\end{definition}

\begin{remark}
One asset of padding operations is that they limit the possible loss of information on the borders of the subsequent convolutions, as well as preventing a decrease in size. Moreover, preserving shape is needed to build some neural network architectures, especially for ones with branching operations \eg \remref{rem:branching_ex}. On the other hand, they increase memory and computational footprints.
\end{remark}

\begin{proposition}\textbf{Connectivity matrix of a convolution with padding}\\
A convolutional layer with padding $(g, h)$ is equivalently defined as $W_g$ being a $n_i \times n_o$ block matrix such that its blocks are Toeplitz matrices.
\end{proposition}

\begin{proof}
Let's consider the slices indexed by $p$ and $q$, and to simplify the notations, let's drop the subscripts $\hspace{0pt}_{p,q}$. We recall from \defref{def:convdef} that
\begin{align*}
  y &= (x \ast^n w)[j_1, \ldots, j_n] \\
 &= \displaystyle \sum_{k_1=1}^{n_1^{(w)}} \cdots \sum_{k_n=1}^{n_n^{(w)}}
    x[j_1 + n_1^{(w)} - k_1, \ldots, j_n + n_n^{(w)} - k_n] \hspace{2pt} w[k_1, \ldots, k_n] \\
 &= \displaystyle \sum_{i_1=j_1}^{j_1 + n_1^{(w)} - 1} \cdots \sum_{i_n=j_n}^{j_n + n_n^{(w)} - 1}
    x[i_1, \ldots, i_n] \hspace{2pt} w[j_1 + n_1^{(w)} - i_1, \ldots, j_n + n_n^{(w)} - i_n] \\
 &= \displaystyle \sum_{i_1=1}^{n_1^{(x)}} \cdots \sum_{i_n=1}^{n_n^{(x)}}
    x[i_1, \ldots, i_n] \hspace{2pt} \widetilde{w}[i_1, j_1, \ldots, i_n, j_n] \\
 & \text{ where } \widetilde{w}[i_1, j_1, \ldots, i_n, j_n] = \\
 & \quad \quad
 \begin{cases}
   w[j_1 + n_1^{(w)} - i_1, \ldots, j_n + n_n^{(w)} - i_n] & \text{if } \forall t, 0 \le i_t - j_t \le n_t^{(w)} - 1 \\
   0 & \text{otherwise}
 \end{cases}
\end{align*}
Using Einstein summation convention as in~\eqref{eq:indices} and permuting indices, we recognize the following tensor contraction
\begin{align}
y_{j_1 \cdots j_n} = x_{i_1 \cdots i_n} \widetilde{w} \hspace{1pt}^{i_1 \cdots i_n} \hspace{0pt}_{j_1 \cdots j_n} \label{eq:toep1}
\end{align}
Following \propref{prop:matprodeq}, we reshape~\eqref{eq:toep1} as a matrix product. To reshape $y \mapsto Y$, we use the row major order bijections $g_j$ as in~\eqref{eq:rowmajor} defined onto $\{(j_1, \ldots, j_n), \forall t, 1 \le j_t \le n_t^{(y)}\}$. To reshape $x \mapsto X$, we use the same row major order bijection $g_j$, however defined on the indices that support non zero-padded values, so that zero-padded values are lost after reshaping. That is, we use a bijection $g_i$ such that $g_i(i_1, i_2, \ldots, i_n) = g_j(i_1 - o_1, i_2 - o_2, \ldots, i_n - o_n)$ defined if and only if $\forall t, 1 + o_t \le i_t \le n_t^{(y)}$, where the $\{o_t\}$ are the starting offsets of the non zero-padded values. $\widetilde{w} \mapsto W$ is reshaped by using $g_j$ for its covariant indices, and $g_i$ for its contravariant indices. The entries lost by using $g_i$ do not matter because they would have been nullified by the resulting matrix product. We remark that $W$ is exactly the block $(p,q)$ of $W_g$ (and not of $W_{g'}$). Now let's prove that it is a Toeplitz matrix.

Thanks to the linearity of the expression~\eqref{eq:rowmajor} of $g_j$, by denoting $i'_t = i_t - o_t$, we obtain
\begin{gather}
  g_i(i_1, i_2, \ldots, i_n) - g_j(j_1, j_2, \ldots, j_n) = g_j(i'_1 - j_1, i'_2 - j_2, \ldots, i'_n - j_n)
\label{eq:toep2}
\end{gather}

To simplify the notations, let's drop the arguments of $g_i$ and $g_j$. By bijectivity of $g_j$, \eqref{eq:toep2} tells us that $g_i - g_j$ remains constant if and only if $i'_t - j_t$ remains constant for all $t$. Recall that 
\begin{gather}
  W[g_i,g_j] =
 \begin{cases}
   w[j_1 + n_1^{(w)} - i'_1, \ldots, j_n + n_n^{(w)} - i'_n] & \text{if } \forall t, 0 \le i'_t - j_t \le n_t^{(w)} - 1 \\
   0 & \text{otherwise}
 \end{cases}
\label{eq:toep3}
\end{gather}
Hence, on a diagonal of $W$, $g_i - g_j$ remaining constant means that $W[g_i,g_j]$ also remains constants. So $W$ is a Toeplitz matrix.

The converse is also true as we used invertible functions in the index spaces through the proof.
\end{proof}

\begin{remark}
The former proof makes clear that the result doesn't hold in case there is no padding. This is due to border effects when the index of the $n$\powth rank resets in the definition of the row-major ordering function $g_j$ that would be used. Indeed, under appropriate definitions, the matrices could be seen as almost Toeplitz.
\end{remark}

\begin{remark}
Comparatively with dense layers, convolution layers enjoy a significant decrease in the number of weights. For example, an input $2 \times 2$ convolution on images with $3$-color input channels, would breed only $12$ weights per feature maps, independently of the numbers of input neurons. On image datasets, their usage also breeds a significant boost in performance compared with dense layers~\citep{krizhevsky2012imagenet}, for they allow to take advantage of the topology of the inputs while dense layers don't~\citep{lecun1995convolutional}. A more thorough comparison and explanation of their assets will be discussed in \secref{sec:gnn}.
\end{remark}

\begin{definition}\textbf{Stride}\\
A convolutional layer with \emph{stride} is a convolutional layer that computes strided convolutions (with stride $> 1$) instead of convolutions.
\end{definition}

\begin{definition}\textbf{Pooling}\\
A layer with \textit{pooling} $(g,h)$ is such that $g$ can be decomposed as $g = g' \circ g_\text{pool}$, where $g_\text{pool}$ is a pooling operation.
\end{definition}

\begin{remark}\textbf{Downscaling}\\
Layers with stride or pooling downscale the signals that passes through the layer. These types of layers allows to compute features at a coarser level, giving the intuition that the deeper a layer is in the network, the more abstract is the information captured by the weights of the layer.
\end{remark}

\todo{below}

\subsection{Examples of regularization}

\begin{remark}\textbf{Overfitting}
\todo{}
\end{remark}

A layer with \textit{dropout} $(g,h)$ is such that $h = h_1 \circ h_2$, where $(g,h_2)$ is a layer and $h_1$ is a dropout operation~\citep{srivastava2014dropout}. When dropout is used, a certain number of neurons are randomly set to zero during the training phase, compensated at test time by scaling down the whole layer. This is done to prevent overfitting.

\subsection{Examples of architecture}
\label{sec:nn_arch}

\todo{rephrase}

A multilayer perceptron (MLP)~\citep{hornik1989multilayer} is a neural network composed of only dense layers.
A convolutional neural network (CNN)~\citep{lecun1998gradient} is a neural network composed of convolutional layers.

Neural networks are commonly used for machine learning tasks. For example, to perform supervised classification, we usually add a dense output layer $s=(g_{L+1},h_{L+1})$ with as many neurons as classes. We measure the error between an output and its expected output with a discriminative loss function $\mathcal{L}$. During the training phase, the weights of the network are adapted for the classification task based on the errors that are back-propagated~\citep{hornik1989multilayer} via the chain rule and according to a chosen optimization algorithm (\eg \cite{bottou2010large}).

\newpage

\section{Graphs and signals}

\subsection{Basic definitions}

\subsubsection{Graphs}

\begin{definition}\textbf{Graph}\\
A \emph{graph} $G$ is defined as a couple of sets $\langle V,E \rangle$ where $V$ is the set of \emph{vertices}, also called \emph{nodes}, and $E \subseteq\binom{V}{2}$ is the set of \emph{edges}. For all $u, v \in V$ we define the relation $u \sim v \Leftrightarrow \{u,v\} \in E$. Unless stated otherwise, we will consider only \emph{weighted} graphs \ie each graph $G$ is associated with a weight mapping $w: E \rightarrow \bbr^*$.
\end{definition}

\figref{fig:graph} illustrates an example of a graph. Note that we employ interchangeably the terms \emph{vertex} and \emph{node}.

\begin{figure}[H]
\centering
\begin{tikzpicture}
\draw (0,0) -- (4,0) -- (4,4) -- (0,4) -- (0,0);
\node at (2,2){placeholder};
\end{tikzpicture}
\caption{Example of a graph}
\label{fig:graph}
\end{figure}

\begin{remark}According to this definition, we consider simple graphs \ie no two edges share the same set of vertices and there is no self-loop.
\end{remark}

\begin{definition}\textbf{Path}\\
A \emph{path} of length $n \in \bbn$ in a graph $\gve$ is a sequence $v_1 v_2 \cdots v_n$ in $V$ such that $\forall i, v_i \sim v_{i+1}$.
\label{def:path}
\end{definition}

\begin{remark}Our definition of graphs admit no self-loop so $\forall i, v_i \neq v_{i+1}$
\end{remark}

\begin{definition}\textbf{Order}\\
The order of a graph $\gve$ is define as $\order(G) = |V| \in \bbn \cup \{+\infty\}$
\end{definition}

\begin{definition}\textbf{Adjacency matrix}\\
The \emph{adjacency matrix} of a finite graph $\gve$ of order $n$, is a $n \times n$ real-valued matrix $A$ associated to an indexing of $V = \{v_1, v_2, \ldots v_n\}$, such that
\begin{gather*}
A[i,j] =
 \begin{cases}
   w\left(\{v_i, v_j\}\right) & \text{if } v_i \sim v_j \\
   0 & \text{otherwise}
 \end{cases}
\end{gather*}
 \end{definition}

\begin{definition}\textbf{Degree}\\
The \emph{degree} of a vertex $v \in V$ of a graph $\gve$ is defined as $\deg(v) = |\{u \in V, u \sim v\}| \in \bbn \cup \{+\infty\}$.

The \emph{degree} of the graph $G$ is defined as $\deg(G) = \displaystyle \max_{v \in V}\{\deg(v)\}$.

A graph is said to be \emph{regular} if $\deg$ is constant on the vertices.
\end{definition}

\begin{definition}\textbf{Degree matrix}\\
The \emph{degree matrix} of a finite graph $\gve$ of order $n$, is the diagonal matrix $D$, associated to an indexing of $V = \{v_1, v_2, \ldots v_n\}$, such that $D = \diag(\deg(v_1), \deg(v_2), \ldots, \deg(v_n))$.
\end{definition}

\begin{definition}\textbf{Laplacian matrix}\\
The \emph{laplacian matrix} of a graph $\gve$ of order $n$, associated to an indexing of $V = \{v_1, v_2, \ldots v_n\}$, is defined as $L = D - A$, where $D$ is the degree matrix and $A$ is the adjacency matrix.
\end{definition}

\begin{definition}\textbf{Digraph}\\
A digraph is an oriented graph \ie $E \subseteq V \times V - \{(v,v), v \in V\}$. Contrary to a graph, the weight mapping $w$, the relation $\sim$, the adjacency matrix $A$, and the laplacian matrix $L$ are not symmetric. Notions defined on graphs naturally extends to digraphs where possible.
\end{definition}

\begin{definition}\textbf{Bipartite graph}\\
A \emph{bipartite graph} is a triplet of sets $\langle V^{(1)}, V^{(2)}, E \rangle$, where $V^{(1)}$ and $V^{(2)}$ are sets of vertices, $V^{(1)} \cap V^{(2)} \neq \emptyset$, and $E \subseteq V^{(1)} \times V^{(2)}$. It is associated with a weight mapping $w: E \rightarrow \bbr^*$. Its adjacency matrix $A$ is associated to indexings of $V^{(1)} = \{v^{(1)}_1, v^{(1)}_2, \ldots v^{(1)}_n\}$ and $V^{(2)} = \{v^{(2)}_1, v^{(2)}_2, \ldots v^{(2)}_n\}$, such that
\begin{gather*}
A[i,j] =
 \begin{cases}
   w\left((v^{(1)}_i,v^{(2)}_j)\right) & \text{if } (v^{(1)}_i,v^{(2)}_j) \in E \\
   0 & \text{otherwise}
 \end{cases}
\end{gather*}
\end{definition}

\begin{definition}\textbf{Induced subgraph}\\
The \emph{subgraph} $\widetilde{G} = \langle \widetilde{V},\widetilde{E} \rangle$ of a graph $\gve$, \emph{induced} by $\widetilde{V} \subseteq V$, is such that $\forall (u,v) \in \widetilde{V}^2, u \overset{\widetilde{G}}{\sim} v \Leftrightarrow u \overset{G}{\sim} v$.
\end{definition}

\begin{definition}\textbf{Grid graph}\\
A \emph{grid graph} $\gve$ is 
such that $V \cong \bbz^2$, $v_1 \sim v_2 \Rightarrow \|v_2 -v_1\|_\infty \in \{0, 1\}$ and either one of the following is true:
\begin{gather*}
\left\{
  \begin{array}{l}
    (i_1,j_1) \sim (i_2,j_2) \Leftrightarrow |i_2 - i_1| \XOR |j_2 - j_1| \quad \text{($4$ neighbours)}\\
    (i_1,j_1) \sim (i_2,j_2) \Leftrightarrow |i_2 - i_1| \AND |j_2 - j_1| \quad \text{($4$ neighbours)}\\
    (i_1,j_1) \sim (i_2,j_2) \Leftrightarrow |i_2 - i_1| \OR |j_2 - j_1| \quad \text{($8$ neighbours)}
  \end{array}
\right.
\end{gather*}

A \emph{(rectangular) grid graph} of size $n \times m$ is the subgraph of a grid graph induced by $\llbracket 1, n \rrbracket \times \llbracket 1, m \rrbracket$.

A \emph{square grid graph} is a rectangular grid graph of square size.
\end{definition}

\subsubsection{(Real-valued) Signals}

\begin{definition}\textbf{Signal space}\\
The \emph{signal space} $\cs(V)$, over the set $V$, is the linear space of real-valued functions defined on $V$.
\end{definition}

We have $\dim(\cs(V)) = |V| \in \bbn \cup \{+\infty\}$.

\begin{remark}
In particular, a vector space, and more generally a tensor space, are finite-dimensional signal spaces over any of their bases.
\end{remark}

\begin{definition}\textbf{Signal}\\
A \emph{signal} over $V$, $s \in \cs(V)$, is a function $s: V \rightarrow \bbr$.

An \emph{entry} of a signal $s$ is an image by $s$ of some $v \in V$ and we denote $s[v]$. If~$v$~is represented by a $n$-tuple, we can also write $s[v_1, v_2, \ldots, v_n]$.

The \emph{support} of a signal $s \in \cs(V)$ is $\supp(s) = \{ v \in V, s[v] \neq 0 \}$.
\end{definition}

\begin{definition}\textbf{Graph signal}\\ 
A \emph{graph signal} over $G$ is a signal over its vertex set. We denote by $\cs(G)$ the graph signal space.
\end{definition}

We have $\dim(\cs(G)) = \order(G) \in \bbn \cup \{+\infty\}$.

\begin{definition}\textbf{Underlying structure}\\
An \emph{(underlying) structure} of a signal $s$ over a set $V$, is a graph $G$ with vertex set $V$.
\end{definition}

\begin{remark}\textbf{Example of images, time series and graph signals}\\
When their is a unique clear underlying structure, we say it is \emph{the} underlying structure. For example, images are compactly supported signals over $\bbz^2$ and their underlying structure is a rectangular grid graph. Time series are signals over $\bbn$ and their underlying structure is a line graph. The underlying structure of a graph signal is obviously the graph itself.
\end{remark}

\subsection{Graphs in deep learning}

\todo{below}

We come across the notion of graphs several times in deep learning:
\begin{itemize}
\item Connections between two layers of a deep learning model can be represented as a bipartite graph, the \emph{connectivity graph}. It encodes how the information is propagated through a layer to another. See \secref{con_graph}.
\item Neural architectures can be represented by a graph. In particular, a computation graph is used by deep learning programming languages to keep track of the dependencies between layers of a deep learning model, in order to compute forward and back-propagation. See \secref{comp_graph}.
\item A graph can represent the underlying structure of an object (often a vector or a signal). The nodes represent its features, and the edges represent some structural property. See \secref{inductive_graph}.
\item Datasets can also be graph-structured. The nodes represent the objects of the dataset, and its edge represent some sort of relation between them. See \secref{transductive_graph}.
\end{itemize}

%\subsection{Graphs related to models}

\subsubsection{Connectivity graph}
\label{con_graph}

A Connectivity graph is the bipartite graph whose adjacency matrix is the connectivity matrix of a layer of neurons.
%$U = \{u_1, u_2, \ldots, u_n\}$
Formally, given a linear part of a layer, let $\textbf{x}$ and $\textbf{y}$ be the input and output signals, $n$ the size of the set of input neurons $N = \{u_1, u_2, \ldots, u_n\}$, and $m$ the size of the set of output neurons $M = \{v_1, v_2, \ldots, v_m\}$. This layer implements the equation $y = \Theta x$ where $\Theta$ is a $n \times m$ matrix.

\begin{definition}
{The \emph{connectivity graph} $G = (V,E)$ is defined such that $V = N \cup M$ and $E = \{(u_i,v_j) \in  N \times M, \Theta_{ij} \neq 0 \} $.}
\end{definition}

I.e. the connectivity graph is obtained by drawing an edge between neurons for which $\Theta_{ij} \neq 0$.
For instance, in the special case of a complete bipartite graph, we would obtain a dense layer. 
Connectivity graphs are especially useful to represent partially connected layers, for which most of the $\Theta_{ij}$ are $0$. 
For example, in the case of layers characterized by a small local receptive field, the connectivity graph would be sparse, and output neurons would be connected to a set of input neurons that corresponds to features that are close together in the input space. \figref{con_ex} depicts some examples.

\begin{figure}[h]
  \begin{center}
    \begin{tikzpicture}
      \tikzstyle{every node} = [draw, circle, thick, inner sep = 2pt]
      \foreach \y in {0,...,4}{
        \pgfmathtruncatemacro{\yplusone}{5 - \y}
        \node(a\y) at (0,.6*\y) {\footnotesize\yplusone};
      }
      \foreach \y in {0,...,4}{
        \pgfmathtruncatemacro{\yplusone}{5 - \y}
        \node(\y) at (2,.6*\y) {\footnotesize\yplusone};
      }

      \foreach \x in {0,...,4}{
        \foreach \y in {0,...,4}{
          \path[opacity=0.5] (a\x) edge (\y);
        }
      }
    \end{tikzpicture}
  \end{center}
  \caption{Examples}
  \label{con_ex}
\end{figure}

\todo{\figref{con_ex}. It's just a placeholder right now}


Connectivity graphs also allow to graphically modelize how weights are tied in a neural layer. Let's suppose the $\Theta_ij$ are taking their values only into the finite set $K = \{w_1, w_2, \ldots, w_\kappa\}$ of size $\kappa$, which we will refer to as the \emph{kernel} of \emph{weights}. Then we can define a labelling of the edges $s: E \rightarrow K$. $s$ is called the \emph{weight sharing scheme} of the layer. This layer can then be formulated as $\displaystyle \forall v \in M, y_v = \sum_{u \in N, (u,v) \in E} w_{s(u,v)} x_u$. \figref{cnn} depicts the connectivity graph of a 1-d convolution layer and its weight sharing scheme.

\begin{figure}[h]
  \begin{center}
    \begin{tikzpicture}
      \tikzstyle{every node} = [draw, circle, thick, inner sep = 2pt]
      \foreach \y in {0,...,4}{
        \pgfmathtruncatemacro{\yplusone}{5 - \y}
        \node(a\y) at (0,.6*\y) {\footnotesize\yplusone};
      }
      \foreach \y in {0,...,4}{
        \pgfmathtruncatemacro{\yplusone}{5 - \y}
        \node(\y) at (2,.6*\y) {\footnotesize\yplusone};
      }
      \path[opacity=0.5]
      (a0) edge (0);
      \path[dashed]
      (a0) edge (1);
      \path[dotted]
      (a1) edge (0);
      \path[opacity=0.5]
      (a1) edge (1);
      \path[dashed]
      (a1) edge (2);
      \path[dotted]
      (a2) edge (1);
      \path[opacity=0.5]
      (a2) edge (2);
      \path[dashed]
      (a2) edge (3);
      \path[dotted]
      (a3) edge (2);
      \path[opacity=0.5]
      (a3) edge (3);
      \path[dashed]
      (a3) edge (4);
      \path[dotted]
      (a4) edge (3);
      \path[opacity=0.5]
      (a4) edge (4);
    \end{tikzpicture}
  \end{center}
  \caption{Depiction of a 1D-convolutional layer and its weight sharing scheme.}
  \label{cnn}
\end{figure}


\todo{Add weight sharing scheme in \figref{cnn}}

\subsubsection{Computation graph}
\label{comp_graph}

%\subsection{Graphs related to data}

\subsubsection{Underlying graph structure and signals}
\label{inductive_graph}

\subsubsection{Graph-structured dataset}
\label{transductive_graph}

%transductive vs inductive
\newpage

%
% Chapter 3
%



%
% Chapter 4
%

%
% Chapter 5
%

%
% Chapter 6
%

%
% Bin
%

\setcounter{chapter}{-1}
\chapter{Drafts}
\textcolor{red}{TODO: Rework 1.1}

%\section{Disambiguations and definitions}

% This thesis manuscript is about deep learning on \emph{irregular domains}. So what does it mean exactly ?

%% start of <see below> comment

% The term \emph{deep learning}, as introduced in the previous chapter, refers to a family of learnable models based on deep neural networks. The inputs of these models are \emph{signals} of a specific type. Learning is made over a training dataset of such signals. Hence, the term \emph{domain} as in \emph{irregular domains} refers to the definition domain of these input signals.

%% Shoud be put later, must make disambiguation with "unstructured" as well

% In this section we recall the basic naming convention in~\ref{basic}, of some definitions in~\ref{regularity}, and categorize the models we will review by the tasks for what they are designed in~\ref{tasks}.

\subsection{Naming conventions}
\label{basic}

\subsubsection{Basic notions}

Let's recall the naming conventions of basic notions.

A \emph{function} $f: E \rightarrow F$ maps objects $x \in E$ to objects $y \in F$, as $y = f(x)$.\\
Its \emph{definition domain} $\cd_f = E$ is the set of objects onto which it is defined. We will often just use the term \emph{domain}.\\
%Objects of its domain $\cd_f$ are mapped to objects of its \emph{codomain} $\cd_f^c= F$.\\
We also say that $f$ is \emph{taking values} in its \emph{codomain} $F$.\\
The \emph{image per $f$} of the subset $U \subset E$, denoted $f(U)$, is $\{y \in F, \exists x \in E, y = f(x)\}$.\\
The \emph{image of $f$} is the image of its domain. We denote $\ci_f$.\\
% The \emph{fiber} of the object $y \in \ci_f$ is the object $x \in E$ such that $y = f(x)$.\\
% The \emph{inverse image per $f$} of the subset $V \subset F$, denoted $f^{-1}(V)$ is $\{x \in E, \exists y \in F, y = f(x)\}$.
A vector space $E$, which we will always assume to be finite-dimensional in our context, is defined as $\bbr^n$, and is equipped with pointwise addition and scalar multiplication.% TODO reword?

A \emph{signal} $s$ is a function taking values in a vector space. In other words, a signal can also be seen as a \emph{vector} with an \emph{underlying structure}, where the vector is composed from its image, and the underlying structure is defined by its \emph{domain}.\\

For example, images are signals defined on a set of pixels. Typically, an image~$s$ in RGB representation is a mapping from pixels~$p$ to a 3d vector space, as $s_p = (r,g,b)$.

\textcolor{red}{TODO?: figure}
% quadillage , arrow ->, quadrillage remplie en 3 images
\begin{figure}

\end{figure}

\subsubsection{Graphs and graph signals}

%
\textcolor{red}{TODO: more defs on grid graphs and other graphs}
% need to define covariance graph, nearest neighbour
% Need to define grid graphs, regular grids from geometry, etc ...
% A regular grid graph is a nearest neighbor graph of a regular geometric grid
%

A \emph{graph} $G = (V, E)$ is defined as a set of nodes $V$, and a set of edges $E \subseteq\binom{V}{2}$. The words \emph{node} and \emph{vertex} will be used equivalently, but we will rather use the first.

A \emph{graph signal}, or \emph{graph-structured signal} is a signal defined on the nodes of a graph, for which the underlying structure is the graph itself.
A \emph{node signal} is a signal defined on a node, in which case it is a \emph{node embedding} in a vector space.

Although this is rarely seen, a signal can also be defined on the edges of a graph, or on an edge. We then coin it respectively \emph{dual graph signal}, or \emph{edge signal} / \emph{edge embedding}.

\emph{Graph-structured data} can refer to any of these type of signals.

\subsubsection{Data and datasets}

% Adjacency matrix, laplacian, etc ...
A dataset of signals is said to be \emph{static} if all its signals share the same underlying structure, it is said to be \emph{non-static} otherwise.\\
For image datasets, being non-static would mean that the dataset contains images of different sizes or different scales. For graph signal datasets, it would mean thats the underlying graph structures of the signals are different.

The point in specifying that objects of a dataset of a machine learning task are signals is that we can hope to leverage their underlying structure.

\textcolor{red}{TODO: figure}

\subsection{Disambiguation of the subject}

This thesis is entitled \emph{Deep learning models for data without a regular structure}.
So either the data of interest in this manuscript do not have any structure, or either their structure is not regular.

\subsubsection{Irregularly structured data}

By structured data, we mean that there exists an underlying structure over which the data is defined. This kind of data are usually modelized as signals defined over a domain. These domains are then composed of objects that are related together by some sort of structural properties. For example, pixels of images can be seen as located on a grid with integer spatial coordinates (a 2d cartesian grid graph).

It then come in handy to define the notions of structure and regularity with the help of graph signals.

\begin{definition}{Structure}\\
  Let $s: D \rightarrow F$ be a signal defined over a finite domain.\\
  An \emph{underlying structure} of the signal $s$ is a graph $G$ that has the domain of $s$ for nodes.\\
  A dataset is said to be \emph{structured}, if its objects can be modelized as signals with an underlying structure.\\
  It is said to be \emph{static} if all its objects share the same underlying structure, and \emph{non-static} otherwise.
\end{definition}

In other words, we chose to define ``structured data'' as ``graph-structured data'' by some graph. Hence we need to specify for which graphs this structure would be said to be regular, and for which it would not.

\begin{definition}{Regularity}\\
An underlying structure is said to be \emph{regular}, if it is a regular grid graph.
It is said to be \emph{irregular} otherwise.\\
A dataset is said to be \emph{regularly structured}, if the underlying structures of its objects are regular.
It is said to be \emph{irregularly structured} otherwise.
\end{definition}


\textcolor{red}{TODO: examples}
%% example images , example time series, example graph signals, example manifolds

\subsubsection{Unstructured data}

Data can also be unstructured. If the data is not yet embedded into a finite dimensional vector space, then we will be interested in embedding techniques used in representation learning. In the other case, it is often possible to fall back to the case of irregularly structured data. For example, vectors can be seen as signals defined over the canonical basis of the vector space, and the vectors of this basis can be related together by their covariances through the dataset. It is typical to use the graph structure that has the canonical basis for nodes, with edges obtained by covariance thresholding.

\textcolor{red}{TODO: examples}
%% give examples of unstructured data, graphs, scramble image datasets, etc..

%%%%%%%%%%%%%%%%%%%%%%%%%%%%%%%%%%%%%%%%%%%%%%%%%%%%%%%%%% END %%%%%%%%%%%%%%%%%%%%%%%%%%%%%%%%%%%%%%%%%%%%%%%%

%
% DRAFTS
%
\textcolor{red}{What follows is a draft}


%\subsection{Theoretical results on regularity and convolutions}

% idea : regularity of a domain implies with poset
% prop: if a domain has a poset, define translations and convolutions
% proving the converse too

\subsection{Datasets}

\subsection{Tasks} %% probably too early
\label{tasks}



\subsection{Goals}

\subsection{Invariance}

In order to be observed, invariances must be defined relatively to an observation. Let's give a formal definition to support our discussion.

...

\subsection{Methods}
\label{methods}
\newpage

%
% Post body
%

% Temptative previsional plan
\addcontentsline{toc}{chapter}{Temptative previsional plan I}
\chapter*{}
\setcounter{chapter}{0}
\fakechapter{Introduction}
\fakesection{Plan, vision, etc}
\fakesection{Deep learning and history}
\fakesection{Regular deep learning}
\fakesection{Irregular deep learning}
\fakesection{Unstructured deep learning}
\fakesection{Propagational point of view}

\fakechapter{Presentation of the domain}
\fakesection{Typology of data}
\fakesection{Standardized terminology}
\fakesection{Motivation}
\fakesection{Datasets}
\fakesection{Unifying framework (tensorial product)}
\fakesection{Other Unifying frameworks}

\fakechapter{Review of models and propositions}
\fakesection{How to compare models}
\fakesection{Spectral models}
\fakesection{Non-spectral}
\fakesection{Non-convolutional}
%\fakesection{Our models I}
%\fakesection{Our models II}
\fakesection{Recap and (big) comparison table}
\fakesection{Explaining current SOA, current issues, and further work}

\fakechapter{Transposing the problem formulation: Structural learning}
\fakesection{Structural Representation}
\fakesection{Feature visualization (viz on input)}
\fakesection{Propagated Signal visualization (viz on S)}
\fakesection{Temptatives on learning S}
\fakesection{Temptatives on learning S (other)}
\fakesection{Covariance-based convolution}
\fakesection{Conclusion}

\fakechapter{Industrial applications}
\fakesection{Context}
\fakesection{The Warp 10 platform and Warpscript language}
\fakesection{Presentation of use cases: uni vs multi-variate, spatial vs geo, etc ..}
\fakesection{Review and application on regularly structured (spatial) time series}
\fakesection{Application to time series database (unstructured)}
\fakesection{Application to geo time series (unstructured)}
\fakesection{Application to visualization}
\fakesection{Market reality (what clients need, what they don't know that can be done ...)}
\fakesection{Conclusion}

\fakechapter{Conclusion}
\fakesection{Summary}
\fakesection{Lesson learned}
\fakesection{Further avenues}


% Temptative previsional plan
\addcontentsline{toc}{chapter}{Temptative previsional plan II}
\chapter*{}
\setcounter{chapter}{-1}
\fakechapter{Introduction}
\fakesection{Teaser}
\fakesection{Goals}
\fakesection{Difficulties}
\fakesection{Outline}

\fakechapter{Subject disambiguation}
\fakesection{Title}
\fakesection{Deep learning}
\fakesection{Signals, features, structure, underlying graph}
\fakesection{Regular, Irregular, Unstructured}
\fakesection{Motivation}

\fakechapter{Presentation of the field}
\fakesection{Tensors}
\fakesection{Neural netowrks}
\fakesection{Graphs}

\fakechapter{Supervised learning}
\fakesection{SotA review}
\fakesection{ours}
\fakesection{Analysis and discussion}

\fakechapter{Semi-supervised learning}
\fakesection{SotA review}
\fakesection{ours}
\fakesection{Analysis and discussion}

\fakechapter{Representation learning} %and unsupervised learning ?
\fakesection{SotA review}
\fakesection{ours}
\fakesection{Analysis and discussion}

\fakechapter{Industrial applications}
\fakesection{Context, Warp10, (Geo) Time series}
\fakesection{Supervised learning}
\fakesection{Semi-supervised learning}
\fakesection{Representation learning}
\fakesection{Market reality and perspectives}

\fakechapter{Conclusion}
\fakesection{Summary}
\fakesection{Discussion}
\fakesection{Further avenues}


% keywords and temptative titles
\begin{keywords}
Deep learning,
representation learning,
propagation learning,
visualization,
structured,
unstructured regular,
irregular,
covariant,
invariant,
equivariant,
tensor,
scheme,
weight sharing,
graphs,
manifold,
euclidean,
signal processing,
graph signal processing,
time series,
time series database,
distributed application,
spatial-time series,
geo time series,
industrial applications,
warp 10,
warpscript,
...
\end{keywords} % to be put somewhere else
\section*{Temptative titles}

\begin{itemize}
\item Learning propagational representations of irregular and unstructured data
\item Learning representations of unstructured or irregularly structured datasets
\item Propagational learning of unstructured or irregularly structured datasets
\item Learning tensorial representation of irregular and unstructured data
\item Tensorial representation of propagation in deep learning for irregular and unstructured dataset
\item Structural representation learning for irregular or unstructured data
\item Word for both ``irregularly structured'' + ``unstructured'' = ? (maybe ``unorthodox'' ?)
\item Unorthdox deep learning
\item ...
\item Deep learning of unstructured or irregularly structured datasets
\item Deep learning models for data without a regular structure
\item On structures in deep learning
\item On deep learning for when data is lacking a regular structure
\item Deep learning for non regularly structured data
\end{itemize}


% Bibliography
\printbibliography[heading=bibintoc]

\end{document}