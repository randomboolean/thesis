\section{Tensors}

Intuitively, tensors in the field of deep learning are defined as a generalization of vectors and matrices, as if vectors were tensors of rank $1$ and matrices were tensors of rank $2$. That is, they are objects in a vector space and their dimensions are indexed using as many indices as their rank, so that they can be represented by multidimensional arrays. In mathematics, a tensor can be defined as a special type of multilinear function~\citep{bass1968cours, marcus1975finite, williamson2015tensor}, which image of a basis can be represented by a multidimensional array. Alternatively, Hackbush propose a mathematical construction of a tensor space as a quotient set of the span of an appropriately defined tensor product \citep{hackbusch2012tensor}, which coordinates in a basis can also be represented by a multidimensional array. In particular in the field of mathematics, tensors enjoy an intrinsic definition that neither depend on a representation nor would change the underlying object after a change of basis, whereas in our domain, tensors are confounded with their representation.

\subsection{Definition}

Our definition of tensors is such that they are a bit more than multidimensional arrays but not as much as mathematical tensors, for that they are embedded in a vector space so that deep learning objects can be later defined rigorously.

Given canonical bases, we first define a tensor space, then we relate it to the definition of the tensor product of vector spaces.

\begin{definition}\textbf{Tensor space}\\
\label{def:tensor}
We define a \emph{tensor space} $\bbt$ of rank $r$ as a vector space such that its canonical basis is a cartesian product of the canonical bases of $r$ other vector spaces.

Its shape is denoted $n_1 \times n_2 \times \cdots \times n_r$, where the $\{n_k\}$ are the dimensions of the vector spaces.
\end{definition}

\begin{definition}\textbf{Tensor product of vector spaces}\\
Given $r$ vector spaces $\bbv_1, \bbv_2, \ldots, \bbv_r$, their \emph{tensor product} is the tensor space~$\bbt$ spanned by the cartesian product of their canonical bases under coordinate-wise sum and outer product.

We use the notation $\bbt = \displaystyle \bigotimes_{k=1}^r \bbv_k$.
\end{definition}

\begin{remark}
This simpler definition is indeed equivalent with the definition of the tensor product given in~(\cite{hackbusch2012tensor}, p. 51). The drawback of our definition is that it depends on the canonical bases, which at first can seem limiting as being canon implies that they are bounded to a certain system of coordinates. However this is not a concern in our domain as we need not distinguish tensors from their representation.
\end{remark}

\paragraph{Naming convention}
For naming convenience, from now on, we will distinguish between the terms \emph{linear space} and \emph{vector space} \ie we will abusively use the term \emph{vector space} only to refer to a linear space that is seen as a tensor space of rank $1$. If we don't know its rank, we rather use the term \emph{linear space}.
We also make a clear distinction between the terms \emph{dimension} (that is, for a tensor space it is equal to $\displaystyle \prod_{k=1}^r n_k$) and the term \emph{rank} (equal to $r$). Note that some authors use the term \emph{order} instead of \emph{rank} (\eg \cite{hackbusch2012tensor}) as the latter is affected to another notion.

\begin{definition}\textbf{Tensor}\\
A \emph{tensor} $t$ is an object of a tensor space. The \emph{shape} of $t$, which is the same as the shape of the tensor space it belongs to, is denoted $n_1^{(t)} \times n_2^{(t)} \times \cdots \times n_r^{(t)}$.
\end{definition}

%\subsection{Comparison with other definitions} % todo ?
%
\subsection{Manipulation}

In this subsection, we describe notations and operators used to manipulate data stored in tensors.

\begin{definition}\textbf{Indexing}\\
An \emph{entry} of a tensor $t \in \bbt$ is one of its scalar coordinates in the canonical basis, denoted $t[i_1, i_2, \ldots, i_r]$.

More precisely, if $\bbt = \displaystyle \bigotimes_{k=1}^r \bbv_k$, with bases $((e_k^i)_{i=1,\ldots,n_k})_{k=1,\ldots,r}$, then we have
\begin{gather*}
t =  \displaystyle \sum_{i_1=1}^{n_1} \cdots \sum_{i_r=1}^{n_r} t[i_1, i_2, \ldots, i_r] (e_1^{i_1}, \ldots, e_r^{i_r})
\end{gather*}

The cartesian product $\bbi=\displaystyle \prod_{k=1}^r \llbracket 1, n_k \rrbracket$ is called the \emph{index space} of~$\bbt$
\end{definition}

\begin{remark}
When using an index $i_k$ for an entry of a tensor $t$, we implicitly assume that $i_k \in \llbracket 1, n_k^{(t)} \rrbracket$ if nothing is specified.
\end{remark}

\begin{definition}\textbf{Subtensor}\\
A \emph{subtensor} $t'$ is a tensor of same rank composed of entries of $t$ that are contiguous in the indexing, with at least one entry per rank. We denote $t' = t[l_1{:}u_1, l_2{:}u_2, \ldots, l_r{:}u_r]$, where the $\{l_k\}$ and the $\{u_k\}$ are the lower and upper bounds of the indices used by the entries that compose~$t'$.
\end{definition}

\begin{remark}
We don't necessarily write the lower bound index if it is equal to $1$, neither the upper bound index if it is equal to $n_k^{(t)}$.
\end{remark}

\begin{definition}\textbf{Slicing}\\
A \emph{slice} operation, along the last ranks $\{r_1, r_2, \ldots, r_s\}$, and indexed by $(i_{r_1}, i_{r_2}, \ldots, i_{r_s})$, is a morphism $s: \bbt = \displaystyle \bigotimes_{k=1}^r \bbv_k \rightarrow \displaystyle \bigotimes_{k=1}^{r-s} \bbv_k$, such that:
\begin{align*}
s(t)[i'_1, i'_2, \ldots, i'_{r-s}] &= t[i'_1, i'_2, \ldots, i'_{r-s}, i_{r_1}, i_{r_2}, \ldots, i_{r_s}] \\
\text{ \ie } \quad s(t) :&= t[:,:, \ldots, :, i_{r_1}, i_{r_2}, \ldots, i_{r_s}]
\end{align*}
where $:=$ means that entries of the right operand are assigned to the left operand.
We denote $t_{i_{r_1}, i_{r_2}, \ldots i_{r_s}}$ and call it the \emph{slice} of $t$. 
Slicing along a subset of ranks that are not the lasts is defined similarly.
$s(\bbt)$ is called a \emph{slice subspace}.
\end{definition}

\begin{definition}\textbf{Flattening}\\
A \emph{flatten} operation is an isomorphism $f: \bbt \rightarrow \bbv$, between a tensor space $\bbt$ of rank~$r$ and an $n$-dimensional vector space $\bbv$, where $n =\displaystyle \prod_{k=1}^r n_k$. It is characterized by a bijection in the index spaces $g: \displaystyle \prod_{k=1}^r \llbracket 1, n_k \rrbracket \rightarrow \llbracket 1, n \rrbracket$ such that
\begin{gather*}
  \forall t \in \bbt, f(t)[g(i_1, i_2, \ldots, i_r)] = f(t[i_1, i_2, \ldots, i_r])
\end{gather*}

We call an inverse operation a \emph{de-flatten} operation.
\end{definition}

\paragraph{Row major ordering}
The choice of $g$ determines in which order the indexing is made. $g$ is reminiscent of how data of multidimensional arrays or tensors are stored internally by programming languages. In most tensor manipulation languages, incrementing the memory address (\ie the output of $g$) will first increment the last index $i_r$ if $i_r < n_r$ (and if else $i_r = n_r$, then $i_r := 1$ and ranks are ordered in reverse lexicographic order to decide what indices are incremented). This is called \emph{row major ordering}, as opposed to \emph{column major ordering}. That is, in row major, $g$ is defined as
\begin{align}
  g(i_1, i_2, \ldots, i_r) = \displaystyle \sum_{p=1}^r \left( \prod_{k=p+1}^r n_k \right) i_p \label{eq:rowmajor}
\end{align}

\begin{definition}\textbf{Reshaping}\\
A \emph{reshape} operation is an isomorphism defined on a tensor space $\bbt = \displaystyle \bigotimes_{k=1}^r \bbv_k$ such that some of its basis vector spaces $\{\bbv_k\}$ are de-flattened and some of its slice subspaces are flattened.
\end{definition}

\subsection{Binary operations}

We define binary operations on tensors that we'll later have use for. In particular, we define  \emph{tensor contraction} which is sometimes called \emph{tensor multiplication}, \emph{tensor product} or \emph{tensor dotproduct} by other sources. We also define \emph{convolution} and \emph{pooling} which serve as the common building blocks of convolution neural network architectures (see \secref{sec:nn_arch}).

\begin{definition}\textbf{Contraction}\\
A \emph{tensor contraction} between two tensors, along ranks of same dimensions, is defined by natural extension of the dot product operation to tensors.

More precisely, let $\bbt_1$ a tensor space of shape $n_1^{(1)} \times n_2^{(1)} \times \cdots \times n_{r_1}^{(1)}$, and $\bbt_2$ a tensor space of shape $n_1^{(2)} \times n_2^{(2)} \times \cdots \times n_{r_2}^{(2)}$, such that $\forall k \in \llbracket 1, s \rrbracket, n_{r_1-(s-k)}^{(1)} = n_k^{(2)}$, then the tensor contraction between $t_1 \in \bbt_1$ and $t_2 \in \bbt_2$ is defined as:
\begin{gather*}
\left\{
  \begin{array}{l}
    t_1 \otimes t_2 = t_3 \in \bbt_3 \text{ of shape } n_1^{(1)} \times \cdots \times n_{r_1-s}^{(1)} \times n_{s+1}^{(2)} \times \cdots \times n_{r_2}^{(2)}
    \text{ where} \\
    t_3[i_1^{(1)}, \ldots, i_{r_1-s}^{(1)}, i_{s+1}^{(2)}, \ldots, i_{r_2}^{(2)}] = \\
    %\displaystyle \sum_{k_1, \ldots, k_s}
    \displaystyle \sum_{k_1=1}^{n_1^{(2)}} \cdots \sum_{k_s=1}^{n_s^{(2)}}
    t_1[i_1^{(1)}, \ldots, i_{r_1-s}^{(1)}, k_1, \ldots, k_s] \hspace{2pt}
    t_2[k_1, \ldots, k_s, i_{s+1}^{(2)}, \ldots, i_{r_2}^{(2)}]
  \end{array}
\right.
\end{gather*}
\label{def:contr}
\end{definition}

For the sake of simplicity, we omit the case where the contracted ranks are not the last ones for $t_1$ and the first ones for $t_2$. But this definition still holds in the general case subject to a permutation of the indices.

\begin{definition}\textbf{Covariant and contravariant indices}\\
Given a tensor contraction $t_1 \otimes t_2$, indices of the left hand operand $t_1$ that are not contracted are called \emph{covariant} indices. Those that are contracted are called \emph{contravariant} indices. For the right operand $t_2$, the naming convention is the opposite. 
The set of covariant and contravariant indices of both operands are called the \emph{transformation laws} of the tensor contraction.
\end{definition}

\begin{remark}
Contrary to most mathematical definitions, tensors in deep learning are independent of any transformation law, so that they must be specified for tensor contractions.
\end{remark}

\paragraph{Einstein summation convention}
The Einstein summation convention is a notational convention to write a sum-product expression as a product expression. The summation indices are those that appear simultaneously in the superscript of the left operand and in the subscript of the right one, if subscripts precede superscripts in the notation, or else vice-versa. For example, a dot product is written $u_k v^k = \lambda $ and a matrix product is written $A_i\hspace{0pt}^k B_k\hspace{0pt}^j = C_i\hspace{0pt}^j$.

The tensor contraction of \defref{def:contr} can be rewritten using this convention:
\begin{gather}
t_1 \hspace{0pt}_{i_1^{(1)} \cdots i_{r_1-s}^{(1)} } \hspace{0pt}^{ k_1 \cdots k_s} 
t_2 \hspace{0pt}_{ k_1^{\phantom{(}} \cdots k_s^{\phantom{(}}} \hspace{0pt}^{i_{s+1}^{(2)} \cdots i_{r_2}^{(2)}} =
t_3 \hspace{0pt}_ {i_1^{(1)} \cdots i_{r_1-s}^{(1)} } \hspace{0pt}^{i_{s+1}^{(2)} \cdots i_{r_2}^{(2)}}
\label{eq:indices}
\end{gather}
%Dot product $u_k v^k = \lambda $ and matrix product $A_i\hspace{0pt}^k B_k\hspace{0pt}^j = C_i\hspace{0pt}^j$ are common examples of tensor contractions.

\begin{proposition}%\textbf{Matrix product equivalence I}\\
A contraction can be rewritten as a matrix product.
\label{prop:matprodeq}
\end{proposition}
\begin{proof}
Using notation of \eqref{eq:indices}, with the reshapings $t_1 \mapsto T_1$, $t_2 \mapsto T_2$ and $t_3 \mapsto T_3$ defined by grouping all covariant indices into a single index and all contravariant indices into another single index, we can rewrite
\begin{gather*}
T_1 \hspace{0pt}_{g_i(i_1^{(1)}, \ldots, i_{r_1-s}^{(1)})} \hspace{0pt}^{g_k(k_1, \ldots, k_s)} 
T_2 \hspace{0pt}_{g_k(k_1^{\phantom{(}}, \ldots, k_s^{\phantom{(}})} \hspace{0pt}^{g_j(i_{s+1}^{(2)}, \ldots, i_{r_2}^{(2)})} =
T_3 \hspace{0pt}_ {g_i(i_1^{(1)}, \ldots, i_{r_1-s}^{(1)})} \hspace{0pt}^{g_j(i_{s+1}^{(2)}, \ldots, i_{r_2}^{(2)})}
\end{gather*}
where $g_i$, $g_k$ and $g_j$ are bijections defined similarly as in \eqref{eq:rowmajor}.
\end{proof}

\begin{definition}\textbf{Convolution}\\
The \emph{$n$-dimensional convolution}, denoted $\ast^n$, between $t_1 \in \bbt_1$ and $t_2 \in \bbt_2$, where $\bbt_1$ and $\bbt_2$ are of the same rank $n$ such that $\forall p \in \llbracket 1, n \rrbracket, n_p^{(1)} \ge n_p^{(2)}$, is defined as:
\begin{gather*}
\left\{
  \begin{array}{l}
    t_1 \ast^n t_2 = t_3 \in  \bbt_3 \text{ of shape } n_1^{(3)} \times \cdots \times n_n^{(3)}
    \text{ where} \\
    \forall p \in \llbracket 1, n \rrbracket, n_p^{(3)} = n_p^{(1)} - n_p^{(2)} + 1 \\
    t_3[i_1, \ldots, i_n] =
    \displaystyle \sum_{k_1=1}^{n_1^{(2)}} \cdots \sum_{k_n=1}^{n_n^{(2)}}
    t_1[i_1 + n_1^{(2)} - k_1, \ldots, i_n + n_n^{(2)} - k_n] \hspace{2pt} t_2[k_1, \ldots, k_n] \\
  \end{array}
\right.
\end{gather*}
\label{def:convdef}
\end{definition}

\begin{proposition}%\textbf{Matrix product equivalence II}\\
A convolution can be rewritten as a matrix product.
\label{prop:matprodeq2}
\end{proposition}

\begin{proof}
Let $t_1 \ast^n t_2 = t_3$ defined as previously with $\bbt_1 = \displaystyle \bigotimes_{k=1}^r \bbv_k^{(1)}$, $\bbt_2 = \displaystyle \bigotimes_{k=1}^r \bbv_k^{(2)}$. Let $t'_1 \in \displaystyle \bigotimes_{k=1}^r \bbv_k^{(1)} \otimes \displaystyle \bigotimes_{k=1}^r \bbv_k^{(2)}$ such that $t'_1[i_1, \ldots, i_n, k_1, \ldots, k_n] = t_1[i_1 + n_1^{(2)} - k_1, \ldots, i_n + n_n^{(2)} - k_n]$, then
\begin{gather*}
t_3[i_1, \ldots, i_n] =
    \displaystyle \sum_{k_1=1}^{n_1^{(2)}} \cdots \sum_{k_n=1}^{n_n^{(2)}}
    t'_1[i_1, \ldots, i_n, k_1, \ldots, k_n] \hspace{2pt} t_2[k_1, \ldots, k_n]
\end{gather*}
where we recognize a tensor contraction. \propref{prop:matprodeq} concludes.
\end{proof}

The two following operations are meant to further decrease the shape of the resulting output.

\begin{definition}\textbf{Strided convolution}\\
The $n$-dimensional \emph{strided} convolution, with strides $s = (s_1, s_2, \ldots, s_n)$, denoted $\ast^n_s$, between $t_1 \in \bbt_1$ and $t_2 \in \bbt_2$, where $\bbt_1$ and $\bbt_2$ are of the same rank $n$ such that $\forall p \in \llbracket 1, n \rrbracket, n_p^{(1)} \ge n_p^{(2)}$, is defined as:
\begin{gather*}
\left\{
  \begin{array}{l}
    t_1 \ast^n_s t_2 = t_4 \in  \bbt_4 \text{ of shape } n_1^{(4)} \times \cdots \times n_n^{(4)}
    \text{ where} \\
    \forall p \in \llbracket 1, n \rrbracket, n_p^{(4)} = \lfloor\frac{n_p^{(1)} - n_p^{(2)} + 1}{s_p}\rfloor \\
    t_4[i_1, \ldots, i_n] = (t_1 \ast^n t_2)[(i_1 - 1)s_n + 1, \ldots, (i_n - 1)s_n + 1]\\
  \end{array}
\right.
\end{gather*}
\end{definition}

\begin{remark}
Informally, a strided convolution is defined as if it were a regular subsampling of a convolution. They match if $s = (1,1,\ldots,1)$.
\end{remark}

\begin{definition}\textbf{Pooling}\\
Let a real-valued function $f$ defined on all tensor spaces of any shape, \eg the \emph{max} or \emph{average} function.
An $f$-pooling operation is a mapping $t \mapsto t'$ such that each entry of $t'$ is an image by $f$ of a subtensor of $t$.
\end{definition}

\begin{remark}
Usually, the set of subtensors that are reduced by $f$ into entries of $t'$ are defined by a regular partition of the entries of $t$.
\end{remark}