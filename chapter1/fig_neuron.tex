\begin{figure}[h!tb]
\centering
\begin{tikzpicture}[
init/.style={
  draw,
  circle,
  inner sep=2pt,
  font=\Huge,
  join = by o-latex
},
squa/.style={
  draw,
  inner sep=2pt,
  font=\Large,
  join = by -latex
},
start chain=2,node distance=13mm
]
\node[on chain=2] 
  (x2) {$x_2$};
%\node[on chain=2,join=by o-latex] 
%  {$w_2$};
\node[on chain=2,init] (sigma)
  {$\displaystyle\Sigma$};
\node[on chain=2,squa,label=above:{\parbox{2cm}{\centering activation}}]   
  {$h$};
\node[on chain=2,join=by -o] 
  {$y$};
\begin{scope}[start chain=1]
\node[on chain=1] at (0,1.5cm) 
  (x1) {$x_1$};
%\node[on chain=1,join=by o-latex] 
%  (w1) {$w_1$};
\end{scope}
\begin{scope}[start chain=3]
\node[on chain=3] at (0,-1.5cm) 
  (x3) {$x_3$};
%\node[on chain=3,label=below:Weights,join=by o-latex] 
%  (w3) {$w_3$};
\end{scope}
%\node[label=above:{\centering 1}] at (sigma|-x1) (b) {};
\node[label={[label distance=-0.3cm]above:{\parbox{2cm}{\centering 1}}}] at (sigma|-x1) (b) {};


\draw[o-latex] (x1) -- (sigma);
\draw[o-latex] (x3) -- (sigma);
\draw[o-latex] (b) -- (sigma);

\node at (1cm,1.2cm) {$w_1$};
\node at (1cm,0.2cm) {$w_2$};
\node at (1cm,-0.5cm) {$w_3$};
\node at (2.37cm,0.9cm) {$b$};

%\draw[decorate,decoration={brace,mirror}] (x1.north west) -- node[left=10pt] {Inputs} (x3.south west);
\end{tikzpicture}
\caption{A neuron}
\label{fig:neuron}
\end{figure}