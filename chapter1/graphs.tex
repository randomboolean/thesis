\section{Deep learning on graphs}

\subsection{Graph and signals}

We present the vocabulary, notation and conventions we will employ for graphs and signals.

\begin{definition}\textbf{Graph}\\
A \emph{graph} $G$ is defined as a couple of vertex and edge sets $\langle V,E \rangle$ \st $E \subset V^2$.
\end{definition}

The terms \emph{vertex} and \emph{node} are used interchangeably. Additionaly, we consider that a graph is always \emph{simple} \ie no two edges share the same set of vertices.
Unless stated otherwise, a graph is undirected, \ie $(u,v)$ and $(v,u)$ refer to the same edge. When it's not the case, it is called a \emph{digraph}.
We define the relation $u \sim v \Leftrightarrow (u,v) \in E$. We precise the graph if needed over the symbol $\overset{G}\sim$.
A \emph{path} is a sequence $v_1 \sim \cdots \sim v_r$. A graph is said to be \emph{connected} if there exists a path from any vertex to any other vertex.
We define the \emph{neighborhood} of a vertex as $\cn_u = \{v \in V, u \sim v\}$. For digraphs, it is equal to the union of the \emph{in}- and \emph{out}-neighborhoods. We only consider graphs without isolated vertex (a vertex with an empty neighborhood).
We also only consider \emph{weighted} graphs. That is, a graph $\gve$ is associated with a weight mapping $w: V^2 \to \bbr+$ \st $w(u,v) = 0 \Leftrightarrow u \nsim v$.
If $G$ is finite, its \emph{adjacency matrix} $A \in \bbr^{V \times V}$ is defined \wrt to a vertex ordering $V = \{v_1, \ldots, v_n\}$ as $A[i,j] = w(v_i,v_j)$. \figref{fig:graph} illustrates an example of a graph and its adjacency matrix.

\begin{figure}[h!tp]
\centering
\begin{tikzpicture}
\draw (0,0) -- (4,0) -- (4,4) -- (0,4) -- (0,0);
\node at (2,2){placeholder};
\end{tikzpicture}
\caption{Example of a graph}
\label{fig:graph}
\end{figure}

The \emph{order} of $G$ is equal to its number of vertices, possibly infinite.
The \emph{degree} of a vertex $v$ is equal to the number of edges it is attached to.
For digraphs the degree is the sum of the \emph{in}- and \emph{out}-degrees.
The \emph{degree} of $G$ refers to its max degree.
$G$ is said to be \emph{degree-regular} if all its vertices have the same degree.
If it is finite, its \emph{degree matrix} $D$ (\wrt to a vertex ordering $V = \{v_1, \ldots, v_n\}$) is the diagonal matrix for which the diagonal entry corresponding to a vertex is the sum of the weights of the edges it is part of.
Its \emph{laplacian matrix} $L$ is the substraction $L = D-A$, which can be \emph{normalized} $L = I - D^{-\frac{1}2}AD^{-\frac{1}2}$, \emph{left-normalized} $L = I - D^{-1}A$, or \emph{right-normalized} $L = I - AD^{-1}$.
A subgraph of $G$ induced by a subset $U \subset V$ is the graph with vertex and edge set restricted by $U$. The \emph{complement} graph $G^C$ shares the same vertex set but $u \overset{G^C}\sim v \Leftrightarrow u \overset{G}\nsim v$.
A \emph{complete} graph is such that there exists an edge between any two vertices.

% \begin{definition}\textbf{Grid graph}\\
% A \emph{grid graph} $\gve$ is 
% such that $V \cong \bbz^2$, $v_1 \sim v_2 \Rightarrow \|v_2 -v_1\|_\infty \in \{0, 1\}$ and either one of the following is true:
% \begin{gather*}
% \left\{
%   \begin{array}{l}
%     (i_1,j_1) \sim (i_2,j_2) \Leftrightarrow |i_2 - i_1| \XOR |j_2 - j_1| \quad \text{($4$ neighbours)}\\
%     (i_1,j_1) \sim (i_2,j_2) \Leftrightarrow |i_2 - i_1| \AND |j_2 - j_1| \quad \text{($4$ neighbours)}\\
%     (i_1,j_1) \sim (i_2,j_2) \Leftrightarrow |i_2 - i_1| \OR |j_2 - j_1| \quad \text{($8$ neighbours)}
%   \end{array}
% \right.
% \end{gather*}

% A \emph{(rectangular) grid graph} of size $n \times m$ is the subgraph of a grid graph induced by $\llbracket 1, n \rrbracket \times \llbracket 1, m \rrbracket$. A \emph{square grid graph} is a rectangular grid graph of square size.
% \end{definition}

\begin{definition}\textbf{Grid graph}\\
Let a graph $\gve$ such that the expression $u \sim v \Leftrightarrow \|u-v\|_1 = 1$ makes sense. $G$ can be called:
\begin{itemize}[nolistsep,noitemsep]
\item a \emph{grid graph} if $V = \bbz^2$
\item a \emph{finite grid graph} if $\exists (n,m) \in \bbz^2, V = \llbracket 1, n \rrbracket \times \llbracket 1, m \rrbracket$
\item a \emph{circulant grid graph} if $\exists (n,m) \in \bbz^2, V = \bbz /n \bbz \times \bbz /m \bbz$
\end{itemize}
\end{definition}

\begin{definition}\textbf{Bipartite graph}\\
A graph is called \emph{bipartite} if its vertex set is a disjoint union $V = V_1 \cup V_2$ \st $$u \sim v \Rightarrow (u,v) \in V_1 \times V_2 \vee (u,v) \in V_2 \times V_1$$
\end{definition}

If it is finite, its \emph{bipartite-adjacency} matrix $A \in \bbr^{V_1 \times V_2}$ is a rectangular matrix defined \wrt to a vertex ordering $V_1 = \{u_1, \ldots, u_n\}$, $V_2 = \{v_1, \ldots, v_n\}$ and weight mapping $w$ as $A[i,j] = w(u_i,v_j)$.

\begin{definition}\textbf{Signal}\\
A \emph{signal} on $V$, $s \in \cs(V)$, is a function $s: V \rightarrow \bbr$.
The \emph{signal space} $\cs(V)$ is the linear space of signals on $V$.
\end{definition}

\begin{remark}
In particular, a vector space, and more generally a tensor space, are finite-dimensional signal spaces on any of their bases.
\end{remark}

A \emph{graph signal} on a graph $\gve$ is a signal on its vertex set $V$. We denote by $\cs(G)$ or $\cs(V)$ the graph signal space. $G$ can be referred as the \emph{underlying structure} of $\cs(V)$.
An \emph{entry} of a signal $s$ is an image by $s$ of some $v \in V$ and we denote $s[v]$. If~$v$~is represented by an $n$-tuple, we can also write $s[v_1, v_2, \ldots, v_n]$.
The \emph{support} of a signal $s \in \cs(V)$ is the subset $\supp(s) \subset V$ on which $s \neq 0$.
For spaces of signals that aren't real-valued, their codomain~$\bbe$ is precised in the subscript~$\cs_{\bbe}(V)$.

\subsection{Learning tasks}

\todo{add example of datasets descriptions}

There are many tasks related to deep learning on graphs.%:
% \begin{itemize}[nolistsep,noitemsep]
%   \item supervised classification of graph signals
%   \item supervised classification of graphs
%   \item semi-supervised classification of node signals
%   \item semi-supervised representation learning of nodes
% \end{itemize}

\paragraph{Supervised classification of graph signals}
This is the classical application of deep learning transposed to graph signals, rather than image or audio signals. It is the principled targeted task we will have in mind in the course of the remainder of this manuscript. Given a graph $\gve$ and an input signal $x \in \cs(G)$ the goal is to classify~$x$. If there are~$c$ possible classes, a neural network~$f$ outputs a vector $y = f(x)$ of dimension~$c$, and its dimension with the biggest weight determines the predicted class. Indeed, a standard MLP can be trained on a dataset of graph signals. However, an MLP wouldn't take the graph structure $G$ into consideration. By similarity with CNNs that leverage the grid structure of images to achieve better performances than MLPs, a challenge is to define a neural network on graph signals that can leverage~$G$. We review some models from the litterature in \secref{sec:spec} and in \secref{sec:vert}. We develop an algebraic understanding in \chapref{chap:2} of why and how they should work, and also propose our own models and point of view in \chapref{chap:3}.

\paragraph{Semi-supervised classification of nodes}
This task is in some way obtained from a transposed perspective of the previous one. Given a dataset of graph signals, represented as a matrix $X \in \bbr^{n \times N}$, where the rows represent the nodes, and the columns represent the signals, the goal is to classify the nodes. This amounts to classify the rows, whereas the previous task amounts to classify the columns. As opposed to the previous one, this task is \emph{transductive} \ie node data from the test set are available during training (but their labels are not), and it is \emph{semi-supervised} \ie some node labels of the train set are unknown. This allows to learn on much more data than if we were restricted to labeled data. In this task, the edges connect learning samples, however in the previous one, the edges were connecting features of learning samples. This is this edge relationship between learning samples that renders the semi-supervised approach possible. This task have received much more attention than the previous one in the recent litterature. We explain why in \secref{sec:spec}.

\paragraph{Other learning tasks}
In this manuscipt, we are less interested in other deep learning tasks related to graphs, so we briefly discuss them here. One is supervised classification of graphs, which is different than classifying graph signals. Examples include \citep{niepert2016learning,tixier2017classifying}. Another interesting task is the semi-supervised representation learning of nodes, which tackles the challenge to learn a linear representation of nodes. A common approach, derived from word2vec \citep{mikolov2013efficient,mikolov2013distributed}, is called node2vec \citep{grover2016node2vec}, and was later improved in graphSAGE \citep{hamilton2017inductive}. A review on this subject is done by \cite{hamilton2017representation}.

\subsection{Spectral methods}
\label{sec:spec}

Spectral methods are based on spectral graph theory \citep{chung1996spectral} which aims at characterizing structral properties of a graph $\gve$ through the eigenvalues of the laplacian matrix $L$. In particular, since it is hermitian, it admits a complete set of normalized eigenvectors. By fixing a normalized eigenvector basis ordered in the rows of $U$ (by ascending eigenvalues), $U$ is used to define the \emph{Graph Fourier Transform} (GFT) of a signal $s \in \cs(G)$ \citep{shuman2013emerging}, and the conjugate-transpose $U^*$ defines the inverse GFT. We write
\begin{align}
\widehat{s} &= Us\\
\widetilde{s} &= U^*s
\end{align}

\begin{remark}
The GFT extends the notion of \emph{Discrete Fourier Transform} (DFT) to general graphs, since that for circulant grid graphs $U$ can be the DFT matrix.
\end{remark}

By analogy with the convolution theorem, a convolution can be defined as pointwise multiplication, denoted $\cdot$, in the spectral domain of the graph \citep{hammond2011wavelets}. For $s, g \in \cs(G)$, we have:
\begin{gather}
s \ast g = \widetilde{\widehat{s} \cdot \widehat{g}} \label{eq:sc}
\end{gather}

This expression can be used to define convolutional layers and spectral CNNs on graphs. However, \cite{bruna2013spectral} pointed out that \eqref{eq:sc} would generate filters with $\co(n)$ weights, where $n$ is the order of $G$. So they proposed to learn filters $\theta$ with only $\co(1)$ weights and then to smoothly interpolate the remaining weights as $g = K \theta$, where $K$ is a linear smoother matrix. They motivate their construction by the fact that smooth multipliers in the spectral domain should simulate local operations in the vertex domain. To elaborate a bit on this, note that we have:
\begin{align}
Ls[u] &= \displaystyle\sum_{v \in V} w(u,v)(s[u] - s[v])
\end{align}
And so,
\begin{align}
s^TLs &= \displaystyle\sum_{u \in V}\sum_{v \in V} w(u,v)s[u](s[u] - s[v])\nonumber\\
&= \displaystyle \frac{1}2\sum_{u \in V}\sum_{v \in V} w(u,v)s[u](s[u] - s[v]) + \frac{1}2\sum_{v \in V}\sum_{u \in V} w(v,u)s[v](s[v] - s[u])\nonumber\\
&=  \displaystyle\sum_{u \in V}\sum_{v \in V} \frac{w(u,v)}2(s[u] - s[v])^2 \label{eq:smooth}
\end{align}
That is, $s^TLs$ is some sort of measure of \emph{smoothness} of the signal $s$, penalized by the weights $w$. The bigger is $w(u,v)$, the closest $s(u)$ and $s(v)$ must be to lower the smoothness \eqref{eq:smooth}. Since $L$ is symmetric, its eigenvalues are non-negative real numbers, and $U$ diagonalizes $L$ as $\Lambda = ULU^*$. Denote $(\lambda_i)_i$ the eigenvalues, the smoothness measure rewrites:
\begin{align}
s^TLs = \widehat{s}^*\Lambda\widehat{s} = \displaystyle\sum_{i=1}^n \lambda_i \widehat{s}[i]^2
\end{align}
Therefore, as they pointed out, smoothness of $s$ can be read off the coordinates of $\hat{s}$, like for the DFT. Moreover, spectral multipliers modulate its smoothness, and decay in the spectral domain is related to smoothness in the vertex domain. But contrary to their conjecture, smoothness in the spectral domain is not necessary related to decay is the vertex domain (and so to some form of locality). For instance, since the laplacian $L^C$ of the complement graph $G^C$ commutes with $L$, it can share the same eigenvector basis $U$, and thus define the same GFT, but their notion of locality in the vertex domain are opposed. Another drawback is that this method requires computing the GFT which complexity is at least $\co(n^2)$ as there is no equivalent of the Fast Fourier Transform (FFT) on graphs, so the authors suggest to use a lower number of eigenvectors $d < n$ from the laplacian eigenbasis.

Then, \cite{defferrard2016convolutional} remedy to these issues by proposing an approximate formulation based on the Chebychev polynomials, denoted by $(T_i)_i$, where $i$ is the polynomial order.
That is, their proposed approximate filters are in the form
\begin{gather}
g_\theta(L) = \sum_{i=0}^k \theta[i] \h{2} T_i(\widetilde{L})
\end{gather}
where $\widetilde{L} = \frac{\lambda_{\max}}2L - I_n$ is the scaled normalized laplacian with eigenvalues lying in the range $[-1,1]$. $g_\theta(L)$ are spectral multipliers since we have:
\begin{align}
g_\theta(L)s &= g_\theta(U^*\Lambda U)s
= U^* g_\theta(\Lambda) Us\nonumber\\
&= \widetilde{g_\theta(\Lambda) \textbf1} \ast s
\end{align}

These filters enjoy locality properties, they contain $\co(1)$ weights, and their complexity is $\co(n)$ when rows of $L$ are sparse. The use of truncated Chebychev exapansion \citep{hammond2011wavelets} ensures that in theory any set of spectral multipliers can be approximated. Also, since they are laplacian polynomials, some authors would argue that these filters are transferable from one graph to another. From a combinatorial point of view this is true. However there is no reason that spectral multipliers from a spectral domain make sense in another one, and there are no experiment in the literature to support the hypothesis. On the other hand, \citep{yi2016syncspeccnn} (which don't use polynomial filters) fix a canonical spectral base in order to synchronize every spectral domains. Their idea is to learn a warping from any eigenbasis to the canonical one, prior to performing spectral multiplication, in the manner of spatial transformer networks (STN) \cite{jaderberg2015spatial}).

However, it is hard to evaluate if a model performs well on the task of supervised classification of graph signals, because there are not much known datasets in the litterature for which the given graph domain holds enough information.

For example, \citeauthor{defferrard2016convolutional} built a graph signal dataset from a text categorization dataset called 20NEWS \citep{joachims1996probabilistic}. Each text is represented as a word2vec vector, and features are linked by edges with their nearest neighbors. However, their model (ChebNet32) fails to surpass Multinomial Naive Bayes (MNB). Moreover, even though they report that their model beat MLPs, our experiments show the contrary. In results we report in \tabref{tab:20}, we see that a lighter MLP, composed of a single Fully-Connected~(FC) layer with ReLU and 20\% dropout outperforms ChebNet32. We replicated their preprocessing phase from the code on their github repository and averaged our results on 10 runs of 20 epochs.

\begin{table}[H]
  \caption{Accuracies on 20NEWS}
  \begin{center}
    \bgroup
    \def\arraystretch{1.5}%  1 is the default, change whatever you need
    \begin{tabular}{|c|c|c|c|c|}
      \hline
      MNB & FC2500 & FC2500-FC500 & ChebNet32 & FC500\\
      \hline
      68.51\%$^a$ & 64.64\%$^a$ & 65.76\%$^a$ & 68.26\%$^a$ & \textbf{71.46}$\pm$0.08\%$^b$\\
      \hline
    \end{tabular}
    \egroup
  \end{center}
\begin{flushleft}
\footnotesize{
$^a$ As reported in \cite{defferrard2016convolutional}\\
$^b$ From our experiments.
}
\end{flushleft}
  \label{tab:20}
\end{table}

Despite the significant theoretical contribution, this negative result stresses out the importance of the practical graph used to support the convolution, a point that they also discussed. \cite{henaff2015deep} proposed supervised graph estimation techniques, but a better graph signal dataset would be one that come with an already suitable graph, that of current literature is still lacking.

On the other hand, attention in the domain has shifted toward the task of semi-supervised classification of nodes, where good datasets are not lacking. For example, \cite{levie2017cayleynets} mainly demonstrate the usefulness of their model on these type of tasks. They define polynomial filters, for which Chebychev filters are a special case, that are capable to specialize in narrow bands of frequency in the spectral domain.

Another spectral avenue consists in using wavelets defined in the graph spectral domain \citep{hammond2011wavelets}, in order to build a scattering network \citep{bruna2013invariant,chen2014unsupervised}. This idea have been exploited recently by \cite{zou2018Graph} then by \cite{gama2018Diffusion}.

\subsection{Vertex-domain methods}
\label{sec:vert}

As their name suggests, vertex-domain methods operates directly on the vertices of the graph. These works were originally motivated by chemistry datasets~\citep{duvenaud2015convolutional,kearnes2016molecular}. Convolution is defined as a function $f$ of the kernel weights $\theta$ and neighboring vertices (contained in the local receptive field $\ccr(v)$), usually based on dot products. That is
\begin{gather}
y[v] = f_\theta\left(\{u \in \ccr(v)\}\right)
\end{gather}
As such, it retains the property of being localized and of sharing weights. But there remains the need to specify how the shared weights are allocated in this local receptive field~\citep{vialatte2016generalizing}. This allocation can depend on \eg an arbitrary order~\citep{niepert2016learning}, on the number of hops~\citep{atwood2016diffusion,du2017topology}, on both vertices and their neighbors~\citep{monti2016geometric,simonovsky2017dynamic}, on a random walk \citep{hechtlinger2017generalization}, on another learned kernel~\citep{vialatte2017learning}, on an attention mechanism~\citep{velickovic2017graph,lee2018attention}, on pattern identification~\citep{sankar2017motif}, or on translation identification~\citep{pasdeloup2017convolutional}. All these methods differ in the function~$f$, but in the end, their definition highly overlap. That is why some authors have proposed unified frameworks~\citep{gilmer2017neural}.

In particular, \cite{kipf2016semi} were first to transpose ChebNet to the task of semi-supervised node classification. Chebychev filters then take a form that is interpretable in the vertex domain, which is
\begin{gather}
Y = \displaystyle\sum_{i=0}^k T_i(\widetilde{L}) X \Theta
\end{gather}
where $X \in \bbr^{n \times N}$, $\Theta \in \bbr^{N \times M}$, $n$ is the number of nodes, $N$ is the number of input channels (features per node), and $M$ is the number of output feature maps. On the left, powers of $\widetilde{L}$ diffuse the graph signal $X$ to share node information. On the right, $\Theta$ maps the diffused signals to another representation. So in essence, this formulation is more a vertex-domain method. They found that the best performing filters were expressed in a simplified form
\begin{gather}
Y = \widetilde{A} X \Theta
\end{gather}
where $\widetilde{A}$ is the normalized adjacency matrix of the graph to which self-loops are added. They called the architecture composed with these simple filters a Graph Convolution Network (GCN). Similiarly, $AX$ shares node information via the edges of the graph and $\Theta$ makes the model learns. This fomulation attracted a lot of research attention and was, in particular, extended with attention mechanism (no pun intended), inspired from the field of neural machine translation \citep{bahdanau2014neural}. Attention can be learned toward which input feature map is most useful \citep{velickovic2017graph}, or which neighboring vertex is \citep{lee2018attention}. Works extending GCN are numerous in recent days (\eg \cite{niepert2018towards}). We don't cover them since their novelty compared to GCN is limited.

%todo{extend ?}