\section{Deep learning on graphs}

\subsection{Graph and signals}

We present the vocabulary, notation and conventions we will employ for graphs and signals.

\begin{definition}\textbf{Graph}\\
A \emph{graph} $G$ is defined as a couple of vertex and edge sets $\langle V,E \rangle$ \st $E \subset V^2$.
\end{definition}

The terms \emph{vertex} and \emph{node} are used interchangeably. Additionaly, we consider that a graph is always \emph{simple} \ie no two edges share the same set of vertices.
Unless stated otherwise, a graph is undirected, \ie $(u,v)$ and $(v,u)$ refer to the same edge. When it's not the case, it is called a \emph{digraph}.
We define the relation $u \sim v \Leftrightarrow (u,v) \in E$. We precise the graph if needed over the symbol $\overset{G}\sim$.
We define the \emph{neighborhood} of a vertex as $\cn_u = \{v \in V, u \sim v\}$. For digraphs, it is equal to the union of the \emph{in}- and \emph{out}-neighborhoods. We only consider graphs without isolated vertex (a vertex with an empty neighborhood).
We also only consider \emph{weighted} graphs. That is, a graph $\gve$ is associated with a weight mapping $w: V^2 \to \bbr$ \st $w(u,v) = 0 \Leftrightarrow u \nsim v$.
Its \emph{adjacency matrix} $A \in \bbr^{V \times V}$ is defined \wrt to a vertex ordering $V = \{v_1, \ldots, v_n\}$ as $A[i,j] = w(v_i,v_j)$. \figref{fig:graph} illustrates an example of a graph and its adjacency matrix.

\begin{figure}[H]
\centering
\begin{tikzpicture}
\draw (0,0) -- (4,0) -- (4,4) -- (0,4) -- (0,0);
\node at (2,2){placeholder};
\end{tikzpicture}
\caption{Example of a graph}
\label{fig:graph}
\end{figure}

The \emph{order} of $G$ is equal to its number of vertices, possibly infinite.
The \emph{degree} of a vertex $v$ is equal to the number of edges it is attached to.
For digraphs the degree is the sum of the \emph{in}- and \emph{out}-degrees.
The \emph{degree} of $G$ refers to its max degree.
$G$ is said to be \emph{degree-regular} if all its vertices have the same degree.
Its \emph{degree matrix} $D$ is the diagonal matrix that contains the degree of its vertices on its diagonal (\wrt to a vertex ordering $V = \{v_1, \ldots, v_n\}$).
Its \emph{laplacian matrix} $L$ is the substraction $L = D-A$, which can be \emph{normalized} $L = I - D^{-\frac{1}2}AD^{-\frac{1}2}$, \emph{left-normalized} $L = I - D^{-1}A$, or \emph{right-normalized} $L = I - AD^{-1}$.
A subgraph of $G$ induced by a subset $U \subset V$ is the graph with vertex and edge set restricted by $U$.

\begin{definition}\textbf{Grid graph}\\
A \emph{grid graph} $\gve$ is 
such that $V \cong \bbz^2$, $v_1 \sim v_2 \Rightarrow \|v_2 -v_1\|_\infty \in \{0, 1\}$ and either one of the following is true:
\begin{gather*}
\left\{
  \begin{array}{l}
    (i_1,j_1) \sim (i_2,j_2) \Leftrightarrow |i_2 - i_1| \XOR |j_2 - j_1| \quad \text{($4$ neighbours)}\\
    (i_1,j_1) \sim (i_2,j_2) \Leftrightarrow |i_2 - i_1| \AND |j_2 - j_1| \quad \text{($4$ neighbours)}\\
    (i_1,j_1) \sim (i_2,j_2) \Leftrightarrow |i_2 - i_1| \OR |j_2 - j_1| \quad \text{($8$ neighbours)}
  \end{array}
\right.
\end{gather*}

A \emph{(rectangular) grid graph} of size $n \times m$ is the subgraph of a grid graph induced by $\llbracket 1, n \rrbracket \times \llbracket 1, m \rrbracket$. A \emph{square grid graph} is a rectangular grid graph of square size.
\end{definition}

\begin{definition}\textbf{Bipartite graph}\\
A graph is said to be \emph{bipartite} if its vertex set is a disjoint union of two sets $V = V_1 \cup V_2$ \st $$u \sim v \Rightarrow (u,v) \in V_1 \times V_2 \vee (u,v) \in V_2 \times V_1$$
\end{definition}

Its \emph{bipartite-adjacency} matrix $A \in \bbr^{V_1 \times V_2}$ is a rectangular matrix defined \wrt to a vertex ordering $V_1 = \{u_1, \ldots, u_n\}$, $V_2 = \{v_1, \ldots, v_n\}$ and weight mapping $w$ as $A[i,j] = w(u_i,v_j)$.

\begin{definition}\textbf{Signal}\\
A \emph{signal} on $V$, $s \in \cs(V)$, is a function $s: V \rightarrow \bbr$.
The \emph{signal space} $\cs(V)$ is the linear space of signals on $V$.
\end{definition}

\begin{remark}
In particular, a vector space, and more generally a tensor space, are finite-dimensional signal spaces on any of their bases.
\end{remark}

A \emph{graph signal} on a graph $\gve$ is a signal on its vertex set $V$. We denote by $\cs(G)$ or $\cs(V)$ the graph signal space. $G$ can be referred as the \emph{underlying structure} of $\cs(V)$.
An \emph{entry} of a signal $s$ is an image by $s$ of some $v \in V$ and we denote $s[v]$. If~$v$~is represented by a $n$-tuple, we can also write $s[v_1, v_2, \ldots, v_n]$.
The \emph{support} of a signal $s \in \cs(V)$ is the subset $\supp(s) \subset V$ on which $s \neq 0$.
For space of signals that aren't real-valued, their codomain~$\bbe$ is precised in the subscript~$\cs_{\bbe}(V)$.

\subsection{Learning tasks}

\subsection{Spectral methods}

\subsection{Vertex-domain methods}


% \subsection{Graphs in deep learning}

% \todo{below}

% We come across the notion of graphs several times in deep learning:
% \begin{itemize}
% \item Connections between two layers of a deep learning model can be represented as a bipartite graph, the \emph{connectivity graph}. It encodes how the information is propagated through a layer to another. See \secref{con_graph}.
% \item Neural architectures can be represented by a graph. In particular, a computation graph is used by deep learning programming languages to keep track of the dependencies between layers of a deep learning model, in order to compute forward and back-propagation. See \secref{comp_graph}.
% \item A graph can represent the underlying structure of an object (often a vector or a signal). The nodes represent its features, and the edges represent some structural property. See \secref{inductive_graph}.
% \item Datasets can also be graph-structured. The nodes represent the objects of the dataset, and its edge represent some sort of relation between them. See \secref{transductive_graph}.
% \end{itemize}

% %\subsection{Graphs related to models}

% \subsubsection{Connectivity graph}
% \label{con_graph}

% A Connectivity graph is the bipartite graph whose adjacency matrix is the connectivity matrix of a layer of neurons.
% %$U = \{u_1, u_2, \ldots, u_n\}$
% Formally, given a linear part of a layer, let $\textbf{x}$ and $\textbf{y}$ be the input and output signals, $n$ the size of the set of input neurons $N = \{u_1, u_2, \ldots, u_n\}$, and $m$ the size of the set of output neurons $M = \{v_1, v_2, \ldots, v_m\}$. This layer implements the equation $y = \Theta x$ where $\Theta$ is a $n \times m$ matrix.

% \begin{definition}
% {The \emph{connectivity graph} $G = (V,E)$ is defined such that $V = N \cup M$ and $E = \{(u_i,v_j) \in  N \times M, \Theta_{ij} \neq 0 \} $.}
% \end{definition}

% I.e. the connectivity graph is obtained by drawing an edge between neurons for which $\Theta_{ij} \neq 0$.
% For instance, in the special case of a complete bipartite graph, we would obtain a dense layer. 
% Connectivity graphs are especially useful to represent partially connected layers, for which most of the $\Theta_{ij}$ are $0$. 
% For example, in the case of layers characterized by a small local receptive field, the connectivity graph would be sparse, and output neurons would be connected to a set of input neurons that corresponds to features that are close together in the input space. \figref{con_ex} depicts some examples.

% \begin{figure}[h]
%   \begin{center}
%     \begin{tikzpicture}
%       \tikzstyle{every node} = [draw, circle, thick, inner sep = 2pt]
%       \foreach \y in {0,...,4}{
%         \pgfmathtruncatemacro{\yplusone}{5 - \y}
%         \node(a\y) at (0,.6*\y) {\footnotesize\yplusone};
%       }
%       \foreach \y in {0,...,4}{
%         \pgfmathtruncatemacro{\yplusone}{5 - \y}
%         \node(\y) at (2,.6*\y) {\footnotesize\yplusone};
%       }

%       \foreach \x in {0,...,4}{
%         \foreach \y in {0,...,4}{
%           \path[opacity=0.5] (a\x) edge (\y);
%         }
%       }
%     \end{tikzpicture}
%   \end{center}
%   \caption{Examples}
%   \label{con_ex}
% \end{figure}

% \todo{\figref{con_ex}. It's just a placeholder right now}


% Connectivity graphs also allow to graphically modelize how weights are tied in a neural layer. Let's suppose the $\Theta_ij$ are taking their values only into the finite set $K = \{w_1, w_2, \ldots, w_\kappa\}$ of size $\kappa$, which we will refer to as the \emph{kernel} of \emph{weights}. Then we can define a labelling of the edges $s: E \rightarrow K$. $s$ is called the \emph{weight sharing scheme} of the layer. This layer can then be formulated as $\displaystyle \forall v \in M, y_v = \sum_{u \in N, (u,v) \in E} w_{s(u,v)} x_u$. \figref{cnn} depicts the connectivity graph of a 1-d convolution layer and its weight sharing scheme.

% \begin{figure}[h]
%   \begin{center}
%     \begin{tikzpicture}
%       \tikzstyle{every node} = [draw, circle, thick, inner sep = 2pt]
%       \foreach \y in {0,...,4}{
%         \pgfmathtruncatemacro{\yplusone}{5 - \y}
%         \node(a\y) at (0,.6*\y) {\footnotesize\yplusone};
%       }
%       \foreach \y in {0,...,4}{
%         \pgfmathtruncatemacro{\yplusone}{5 - \y}
%         \node(\y) at (2,.6*\y) {\footnotesize\yplusone};
%       }
%       \path[opacity=0.5]
%       (a0) edge (0);
%       \path[dashed]
%       (a0) edge (1);
%       \path[dotted]
%       (a1) edge (0);
%       \path[opacity=0.5]
%       (a1) edge (1);
%       \path[dashed]
%       (a1) edge (2);
%       \path[dotted]
%       (a2) edge (1);
%       \path[opacity=0.5]
%       (a2) edge (2);
%       \path[dashed]
%       (a2) edge (3);
%       \path[dotted]
%       (a3) edge (2);
%       \path[opacity=0.5]
%       (a3) edge (3);
%       \path[dashed]
%       (a3) edge (4);
%       \path[dotted]
%       (a4) edge (3);
%       \path[opacity=0.5]
%       (a4) edge (4);
%     \end{tikzpicture}
%   \end{center}
%   \caption{Depiction of a 1D-convolutional layer and its weight sharing scheme.}
%   \label{cnn}
% \end{figure}


% \todo{Add weight sharing scheme in \figref{cnn}}

% \subsubsection{Computation graph}
% \label{comp_graph}

% %\subsection{Graphs related to data}

% \subsubsection{Underlying graph structure and signals}
% \label{inductive_graph}

% \subsubsection{Graph-structured dataset}
% \label{transductive_graph}

% %transductive vs inductive